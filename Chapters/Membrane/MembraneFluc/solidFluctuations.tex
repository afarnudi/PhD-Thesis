کانون توجه این رساله بررسی اُفت و خیز سطح غشا‌های جامد (مانند شبکه‌ی پلیمری گلبول‌های قرمز) و پوسته‌های الاستیکی

نیست. اما جهت تکمیل مطالعه‌ی افت و خیز، لازم می‌دانم که به شکل شدت افت و خیز سطح پوسته‌ی کروی الاستیک به شعاع
$R$
، مدول یانگ دو بعدی
$Y_{2d}$
، خمش ذاتی 
$r_s=R$
، سختی خمش
$\kappa$
، و تحت اختلاف فشار 
$p$
در دمای 
$k_BT$
را معرفی کنم. افت و خیز چنین سطحی تا مرتبه‌ی دوم به شکل 
\begin{equation}
\langle|u_{\ell,m}|^2\rangle=\frac{k_BT}{\kappa(\ell+2)^2(\ell-1)^2-pR^3\left[1+\frac{1}{2}\ell(\ell+1)\right]+R^2\frac{Y_{2D}}{1+\frac{Y_{2D}}{2\mu(\ell^2+\ell-2)}}}
\label{eq:ulmSolid}
\end{equation}


تعریف می‌شود. و تعریف مدول یانگ دو بعدی بر حسب ضرایب لمه
\LTRfootnote{lam\'e coefficients}
$\lambda,\mu$
به شکل
\begin{equation}
Y_{2D}=\frac{4\mu(\mu+\lambda)}{2\mu+\lambda}
\label{eq:youngLame}
\end{equation}

است. همچنین می‌توان معادله‌ی
\ref{eq:ulmSolid}
را برای ماده‌ای که دارای تنش بُرشی 
$\mu=3Y_{2D}/8$
باشد (مثلا موادی که اتصالاتشان شبکه‌ی مثلثی تشکیل می‌دهد) به شکل زیر تقریب زد،
\begin{equation}
\langle|u_{\ell,m}|^2\rangle=\frac{k_BT}{\kappa(\ell+2)^2(\ell-1)^2-pR^3\left[1+\frac{1}{2}\ell(\ell+1)\right]+Y_{2D}R^2\frac{3(\ell^2+\ell-2)}{3(\ell^2+\ell)-2}}
\label{eq:ulmSolidTri}
\end{equation}
جزئیات محاسبات بخش خمش طبق محاسبات ارائه شده در این رساله‌است و محاسبات مربوط به اثر فشار و خاصیت الاستیکی سطح با جزئیات کامل در مرجع
\cite{gomppernelson2012}
یافت می‌شود.
