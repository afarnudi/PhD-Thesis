
 کُره‌ بر اساس تعریف آن حجم کاهیده‌ی 
 $\nu=1$
 دارد. غشا با حجم کاهیده‌ی واحد نمی‌تواند افت و خیز کند، زیراکه برای افت و خیز مساحت آن باید کمی بیشتر از مساحت کُره با همان حجم باشد. می‌توانیم مساحت شکلی که با انرژی خمش هلفریش افت و خیز می‌کند را به شکل زیر محاسبه کنیم. از بخش قبل می‌دانیم که
\begin{equation}
dA=r_{0}^2(1+g^2+\frac{1}{2}g\mathcal{L}^2g)d\Omega
\label{eq:areaPatchDifferential}
\end{equation}
 و با انتگرال گیری روی تمام زوایای فضایی،
 \begin{equation}
\begin{aligned}
A&=\int dA=r_{0}^2\int(1+g^2+\frac{1}{2}g\mathcal{L}^2g)d\Omega\\
&=r_{0}^2(4\pi+\sum_{\ell}|u_{\ell,m}|^2[1+\frac{1}{2}\ell(\ell+1)])
\label{eq:AreaGL}
\end{aligned}
\end{equation}
 حجم غشای در حال افت و خیز نیز به همین ترتیب قابل محاسبه‌است. ابتدا عنصر حجمی را تعریف کرده،
 \begin{equation}
\begin{aligned}
dV&=\frac{1}{3}(r\hat r\cdot\hat n)dA\\
&\approx\frac{1}{3}\frac{r_{0}(1+g)}{\sqrt{1+\frac{1}{2}g\mathcal{L}^2g}}r_{0}^2(1+g^2+\frac{1}{2}g\mathcal{L}^2g)d\Omega\\
&\approx\frac{1}{3}r_{0}^3(1+g^2)d\Omega
\label{eq:volumeDifferential}
\end{aligned}
\end{equation}
 که در معادله‌ی بالا بردار عمود بر سطح (مطابق بخش قبل)به شکل
 \begin{equation}
\vec n =\left(1+g\mathcal{L}^2g\right)^{-\frac{1}{2}}(\hat r-\frac{r_0}{r}\frac{\partial}{\partial\theta}g\hat\theta-\frac{r_{eq}}{r}\frac{1}{\sin\theta}\frac{\partial}{\partial\phi}\hat\phi)
\label{eq:surfaceNormal}
\end{equation}
تعریف شده است. حال می‌توانیم با انتگرال گیر بر روی تمام زوایای فضای حجم غشا را محاسبه کنیم،
\begin{equation}
\begin{aligned}
V&=\int dV=\frac{1}{3}r_{eq}^3\int(1+g^2)d\Omega\\
&=\frac{1}{3}r_{0}^3(4\pi+\sum_{\ell}|u_{\ell,m}|^2)
\label{eq:VolumeGL}
\end{aligned}
\end{equation}
حالا با استفاده از معادلات
\ref{eq:AreaGL}
و
\ref{eq:VolumeGL}
و جایگذاری در معادله‌ی
\ref{eq:reducedVolume}
حجم کاهیده لازم را محاسبه می‌کنیم،
\begin{equation}
\begin{aligned}
\nu=\frac{\sqrt{4\pi}(4\pi+\sum|u_{\ell,m}|^2)}{(4\pi+\sum|u_{\ell,m}|^2[1+\frac{1}{2}\ell(\ell+1)])^{3/2}}
\label{eq:nuUndulated}
\end{aligned}
\end{equation}
با فرض اینکه ضریب سختی خمش یک غشا حدود
$\kappa=20k_BT$
است، با جمع روی مُدها می‌توان حجم کاهیده‌ی لازم را حدود
$\nu\approx0.9626$
تخمین زد.
