
 کُره‌ بر اساس تعریف آن حجم کاهیده‌ی 
 $\nu=1$
 دارد. غشا با حجم کاهیده‌ی واحد نمی‌تواند افت و خیز کند، زیراکه برای افت و خیز مساحت آن باید کمی بیشتر از مساحت کُره با همان حجم باشد. می‌توانیم مساحت شکلی که در حال افت و خیز است را به شکل زیر محاسبه کنیم. از بخش قبل می‌دانیم که
\begin{equation}
dA=r_{0}^2(1+g^2+\frac{1}{2}g\mathcal{L}^2g)d\Omega
\label{eq:areaPatchDifferential}
\end{equation}
 و با انتگرال گیری روی تمام زوایای فضایی،
 \begin{equation}
\begin{aligned}
A&=\int dA=r_{0}^2\int(1+g^2+\frac{1}{2}g\mathcal{L}^2g)d\Omega\\
&=r_{0}^2(4\pi+\sum_{\ell}|u_{\ell,m}|^2[1+\frac{1}{2}\ell(\ell+1)])
\label{eq:AreaGL}
\end{aligned}
\end{equation}
با استفاده از معادله‌ی فوق می‌توان انرژی حاصل از تغییر مساحت را تا مرتبه‌ی دوم تخمین زد،
\begin{equation}
E_A=\frac{1}{2}k_A\frac{(A-4\pi r_{0}^2)^2}{4\pi r_{0}^2}=\frac{1}{8\pi}k_A r_{0}^2(\sum|u_{\ell,m}|^2[1+\frac{1}{2}\ell(\ell+1)])^2.
\label{eq:AreaGLFluctuationAmplitude}
\end{equation}


 حجم غشای در حال افت و خیز نیز به همین ترتیب قابل محاسبه‌است. با انتگرال گیری بر روی تمام زوایای فضای حجم غشا را محاسبه می‌کنیم،
\begin{equation}
\begin{aligned}
V&=\int dV=\frac{1}{3}\int r^3d\Omega\\
&=\frac{1}{3}r_{0}^3\int(1+g)^3d\Omega=\frac{1}{3}r_{0}^3\int1+3g+3g^2d\Omega\\
&=\frac{1}{3}r_{0}^3(4\pi+3\sum_{\ell}|u_{\ell,m}|^2)
\label{eq:VolumeGL}
\end{aligned}
\end{equation}
مشابه با پاراگراف قبل، انرژی ناشی از تغییر حجم پوسته تا مرتبه‌ی دوم برابر است با
\begin{equation}
E_V=\frac{1}{2}k_V\frac{(V-\frac{4}{3}\pi r_{0}^3)^2}{\frac{4}{3}\pi r_{0}^3}= \frac{3}{8\pi}k_Vr_{0}^3(\sum|u_{\ell,m}|^2)^2
\label{eq:VolumeGLFluctuationAmplitude}
\end{equation}



همانطور که قبلا اشاره شد. شکلی که حجم کاهیده‌ی 

داشته باشد قادر به افت و خیز نیست.  با استفاده از معادلات
\ref{eq:AreaGL}
و
\ref{eq:VolumeGL}
و جایگذاری در معادله‌ی
\ref{eq:reducedVolume}
می‌توان حجم کاهیده شکل در حال افت و خیز  را محاسبه می‌کنیم،
\begin{equation}
\begin{aligned}
\nu=\frac{\sqrt{4\pi}(4\pi+\sum|u_{\ell,m}|^2)}{(4\pi+3\sum|u_{\ell,m}|^2[1+\frac{1}{2}\ell(\ell+1)])^{3/2}}
\label{eq:nuUndulated}
\end{aligned}
\end{equation}
%با فرض اینکه ضریب سختی خمش یک غشا حدود
%$\kappa=20k_BT$
%است، با جمع روی مُدها می‌توان حجم کاهیده‌ی لازم را حدود
%$\nu\approx0.9626$
%تخمین زد.
