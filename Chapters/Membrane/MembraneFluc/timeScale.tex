
در این بخش به محاسبه‌ی زمان دوره‌ی تناوبی بسامد نوسان‌‌های هماهنگ‌های کروی برای یک غشای سیال‌گون می‌پردازیم. 
در بخش 
\ref{sec:bendingFluctuations}
انرژی انحنای یک غشای بر اساس مُدهای نرمال آن (هماهنگ‌های کُروی) محاسبه شد. معادله‌ی لاگرانژ، دینامیک درجه‌های آزادی را به تغییرات انرژی درجات آزادی، 
$q$
مرتبط می‌کند،

\begin{equation}
\frac{d}{dt}\left(\frac{\partial K}{\partial \dot q}\right)=-\frac{\partial U}{\partial q}.
\label{eq:LagrangeEquation}
\end{equation}
در معادله‌ی فوق
$K$
انرژی جنبشی،
$U$
انرژی پتانسیل سیستم مورد مطالعه‌است. این معادله مستقل از شکل معادله‌ی دینامیک سیستم بر قرار است. در صورتی که غشا در محیط خلا در حال نوسان باشد، معادله‌ی حرکت سیستم توسط معادله‌ی نیوتن تعیین می‌شود. در صورتی که غشا در محیط سیال در حال نوسان باشد، معادله‌ی حرکت با حل معادله‌ی ناویر-استوکس
\LTRfootnote{Navier-Stokes }
تعیین می‌شود. برای غشایی که در خلا حرکت می‌کند، انرژی جنبشی آن به سادگی با جمع روی انرژی جنبشی تکه‌های جرمی کوچک برای سطح آن محاسبه می‌شود،
\begin{equation}
\begin{aligned}
K&=\frac{1}{2}\rho_m\int\d\Omega~\dot r^2(\theta,\phi) dm=\frac{1}{2} \left\{\frac{\partial}{\partial t}\left[r_0(1+\sum_{\ell,m}u_{\ell,m}Y_{\ell,m})\right\}\right)^2\\
&=\frac{1}{2} \rho_mr_0^2\int d\Omega\left[\sum_{\ell,m}\frac{\partial}{\partial t}u_{\ell,m}Y_{\ell,m}\right]^2\\
&=\frac{1}{2} \rho_mr_0^2\sum_{\ell,m}\dot u_{\ell,m}^2
\end{aligned}
\label{eq:kineticEnergyNewton}
\end{equation}
در محاسبات فوق فرض شده که غشایی به جرم 
$M$
با چگالی یکنواخت در تمام جهت فضا 
$\rho_m=M/4\pi$
در حال حرکت است. حالا می‌توانیم معادله‌ی لاگرانژ را برای یک غشای در حال نوسان در خلا محاسبه کنیم،
\begin{equation}
\begin{aligned}
\frac{d}{dt}\left(\frac{\partial}{\partial \dot u_{\ell,m}}\frac{1}{2}\frac{Mr_0^2}{4\pi}\sum_{\ell',m'}\dot u_{\ell',m'}^2\right)&=-\frac{\partial}{\partial u_{\ell,m}}\left[ 8\pi\kappa +\frac{1}{2}\kappa\sum_{\ell',m'}|u_{\ell',m'}|^2(\ell'+2)(\ell'+1)\ell'(\ell'-1)\right]\\
\frac{Mr_0^2}{4\pi}\ddot u_{\ell,m}&=-\kappa u_{\ell,m}(\ell+2)(\ell+1)\ell(\ell-1)\\
\ddot u_{\ell,m}&=-\frac{4\pi\kappa}{Mr_0^2}(\ell+2)(\ell+1)\ell(\ell-1)~u_{\ell,m}
\end{aligned}
\label{eq:LagrangeNewton}
\end{equation}
که شکل معادله‌ی نوسانگر هارمونیک را دارد. در نتیجه بسامد زاویه‌ای هر مُد برابر با،
\begin{equation}
\omega_{\ell,m}^2=\frac{4\pi\kappa}{Mr_0^2}(\ell+2)(\ell+1)\ell(\ell-1)
\label{eq:Omegalm}
\end{equation}
است. در نتیجه مقیاس زمانی برای نوسان هر مُد مشخص شده‌است. رابطه‌ی 
$\omega_{\ell,m}$
و
$\ell$
یک رابطه‌ی مستقیم است، یعنی مُدهای بالا بسامد بزرگتری خواهند داشت. با توجه به اینکه 
\begin{equation}
T=\frac{2\pi}{\omega}
\end{equation}

کوچکترین مُد سیستم که تغییر شکل ایجاد می‌کند،
$\ell=2$
طولانی ترین دوره‌ی تناوب را دارد و هر جه مُد بالاتر باشد، دوره‌ی تنواب سریع‌تری خواهد داشت.
\begin{equation}
T_{\ell,m}=\sqrt{\frac{M\pi r_0^2}{\kappa}\frac{1}{(\ell+2)(\ell+1)\ell(\ell-1)}}
\end{equation}
این نکته‌ی کلیدی در انتخاب قدم زمانی مناسب برای پیاده‌سازی محاسبات دینامیک ملکولی است.

جهت تکمیل کردن بحث، انرژی جنبشی سطح غشا بین دو سیال تراکم ناپذیر با چگالی 
$\rho_1$
(سیال داخل غشا) و 
$\rho_2$
شکل زیر را خواهد داشت،
\cite{Christer1984}
\begin{equation}
K=\frac{1}{2}r_0^5\sum_{\ell,m}\left(\frac{\rho_1}{\ell}+\frac{\rho_2}{\ell+1}\right)\dot u_{\ell,m}^2
\end{equation}
که با حل معادله ناویر-استکس به دست می‌آید.







