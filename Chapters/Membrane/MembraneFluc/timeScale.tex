برای یافتن دینامیک سطح یک غشای نوسانگر در خلا، معادله‌ی نیوتن را برای نقاط روی سطح می‌نویسیم،
\begin{equation}
\begin{aligned}
\frac{M}{4\pi r_0^2}\frac{dA}{d\Omega}\ddot r(\theta,\phi,t)&=F(\theta,\phi)\\
\frac{Mr_0}{4\pi}\sum_{\ell,m}\ddot u_{\ell,m,t}Y_{\ell,m}(\theta,\phi)&=F(\theta,\phi)\\
\end{aligned}
\label{eq:NewtonEqRaw}
\end{equation}
در معادله‌ی بالا
$F(\theta,\phi)$
نیرویی است که در زاویه‌ی فضایی مشخصی بر سطح غشا عمل می‌کند. با تبدیل فوریه‌ی کروی معادله‌ی فوق می‌توان معادلات دینامیک مد‌های سطح را محاسبه‌کرد. تبدیل فوریه‌ی سمت چپ معادله به شکل زیر محاسبه می‌شود،
\begin{equation}
\begin{aligned}
\frac{Mr_0}{4\pi}\int d\Omega Y_{\ell,m}^*(\Omega)\sum_{\ell',m'}\ddot u_{\ell',m'}(\Omega,t)Y_{\ell',m'}(\Omega)\\
=\frac{Mr_0}{4\pi}\ddot u_{\ell,m}(t)
\end{aligned}
\label{eq:NewtonEqLHS}
\end{equation}
می‌توان از معادله‌ی 
\ref{eq:bendingFluctuations}
برای محاسبه‌ی نیروهای ناشی از هر مد استفاده کرد،
\begin{equation}
\begin{aligned}
f_{\ell,m}(t)&=-\frac{1}{r_0}\frac{\partial}{\partial u_{\ell,m}}E_b\\
&=-\frac{\kappa}{r_0}u_{\ell,m}(t)(\ell+2)(\ell+1)\ell(\ell-1)
\end{aligned}
\label{eq:NormalModeForce}
\end{equation}
که با نتیجه‌ی محاسبات ما در معادله‌ی
\ref{eq:normal_force_lm}
یکسان است. در حالتی که خمش ذاتی سطح غیر صفر باشد، رابطه‌ی مفصل تری برای نیرو باید محاسبه شود. جزئیات این محاسبات در مرجع
\cite{milnersafranPRA1987}
شرح داده شده‌است.

با برابر قرار دادن معدلات 
\ref{eq:NewtonEqLHS}
و
\ref{eq:NormalModeForce}
می‌توان شکل معادلات دینامیک مد‌ها را بدست آورد،
\begin{equation}
\begin{aligned} 
\ddot u_{\ell,m}&=-\frac{4\pi\kappa}{Mr_0^2}u_{\ell,m}(t)(\ell+2)(\ell+1)\ell(\ell-1)\\
\ddot u_{\ell,m}&=-\omega_{\ell,m}^2u_{\ell,m}
\end{aligned}
\label{eq:NewtonModes}
\end{equation}
در نتیجه فرکانس زاویه‌ای هر مد
$\omega_{\ell,m}$
به این شکل رفتار می‌کند،
\begin{equation}
\omega_{\ell,m}=\sqrt{\frac{4\pi\kappa}{Mr_0^2}(\ell+2)(\ell+1)\ell(\ell-1)}.
\label{eq:NewtonModesFreq}
\end{equation}
از آنجایی که دوره‌ی تنواب هر مد 
$\tau_{\ell,m}=2\pi/\omega_{\ell,m}$
با فرکانس رابطه‌ی معکوس دارد، کوچک‌ترین عدد مد
$\ell=2$
کند‌ترین مد را بیان می‌کند.


معادلات لانژون دینامیک قادر به توصیف دینامیک غشایی‌است که با اتلاف محیطی
$\zeta$
ناشی از در تماس بودن با حمام گرمایی
$T$
است،
\begin{equation}
\begin{aligned}
m\ddot{x}+\zeta\dot{x}+kx&=F_{\eta}\\
F_{\eta}&=\sqrt{2k_BT\zeta}\eta(t).
\end{aligned}
\label{eq:LangevinEq}
\end{equation}
در معادله‌ی بالا 
$\eta(t)$
یک فرایند وینر
\LTRfootnote{Wiener process}
با مشخصات زیر را بیان می‌کند.
\begin{equation}
\begin{aligned}
\langle\eta(t)\rangle&=0\\
\langle\eta(t)\eta(t')\rangle&=\delta(t-t').
\end{aligned}
\label{eq:Wiener}
\end{equation}
همانند عملکردمان در مورد معادلات نیوتن، می‌توان معادلات حرکت مد‌ها را برای معادلات لانژون محاسبه کرد،
\begin{equation}
\begin{aligned} 
\frac{Mr_0}{4\pi}\ddot{u}_{\ell,m}+\frac{\zeta r_0}{4\pi}\dot{u}_{\ell,m}+\frac{Mr_0}{4\pi}\omega_{\ell,m}u_{\ell,m}&=F_{\eta}\\
\ddot{u}_{\ell,m}+\gamma \dot{u}_{\ell,m}+\frac{k_{\ell,m}}{M}u_{\ell,m}&=\frac{4\pi}{Mr_0}F_{\eta}
\end{aligned}
\label{eq:LangevinMode}
\end{equation}
در معادلات فوق، 
$k_{\ell,m}=M\omega_{\ell,m}^2$
و جمله‌ی اتلاف را با 
$\zeta=\gamma M$
جایگزین کرده‌ایم. معادله‌ی 
\ref{eq:LangevinMode}
یک نوسانگر میرا با فرکانس
$\omega_{\ell,m} ^{\prime}=\sqrt{\omega_{\ell,m}^{2}-\gamma^2/4}$
را توصیف می‌کند. رفتار نوسانگر میرا به 
$\omega_{\ell,m}^{\prime}$
بستگی خواهد داشات. در صورتی که 
$\omega_{\ell,m}^{\prime}$
مقادیل حقیقی داشته باشد، رفتار نوسانگر زیرمیرا
\LTRfootnote{under damped}
است. در صورتی که 
$\omega_{\ell,m}^\prime=0$
شاهد میرایی بحرانی 
\LTRfootnote{critically damped}
خواهیم بود و در صورتی که 
$\omega_{\ell,m}^{\prime}$
مقادیر مختلط داشته باشد رفتار نوسانگر فوق میرا 
\LTRfootnote{over damped}
خواهد بود. با توجه به معدله‌ی
\ref{eq:NewtonModesFreq}
می‌توان عدد مدی که در آن رفتار میرایی بحرانی مشاهده می‌شود را محاسبه کرد،
\begin{equation}
\ell_{crit}\approx\frac{1}{2}\left(\frac{Mr_0^2}{\pi\kappa}\right)^\frac{1}{4}\sqrt\gamma
\label{eq:criticalMode}
\end{equation}
در سیستمی با گامای مشخص، اعداد مد بزرگتر از مد بحرانی (موج‌های کوتاه‌تر) 
$\ell>\ell_{crit}$
رفتار زیر میرایی و موج‌های بلندتر روفتار فوق میرا خواهند داشت.


رابطه‌ی خودهمبستگی
\LTRfootnote{auto-correlation}
دامنه‌ی مدها برای یک نوسانگر زیر میرا
\begin{equation}
\begin{aligned}
\langle u_{\ell,m}&(t)u_{\ell,m}(0)\rangle=\langle |u_{\ell,m}|^2\rangle e^{-\frac{\gamma}{2}t}\left( \cos(\omega_{\ell,m}^\prime t)+\frac{\gamma}{2\omega_{\ell,m}^\prime}\sin(\omega_{\ell,m}^\prime t)\right)
\end{aligned}
\label{eq:autoUnderDamped}
\end{equation}
در این معادله، زمان واحلش سیستم 
$1/\gamma$
است. خودهمبستگی دامنه‌ی مدها برای یک غشای با میرایی بحرانی
\begin{equation}
\langle u_{\ell,m}(t)u_{\ell,m}(0)\rangle=\langle |u_{\ell,m}|^2\rangle e^{-\frac{\gamma}{2}t}\left(1+\frac{\gamma}{2}t\right)
\label{eq:autoCriticallyDamped}
\end{equation}

و در نهاتی رابطه‌ی خودهمبستگی برای یک غشا با دینامیک فوق میرا
\begin{equation}
\begin{aligned}
\langle u_{\ell,m}&(t)u_{\ell,m}(0)\rangle=\langle |u_{\ell,m}|^2\rangle e^{-\frac{\gamma}{2}t}\Bigl[\sinh(|\omega_{\ell,m}^\prime|t)+\frac{2}{\gamma}|\omega_{\ell,m}^\prime|\cosh(|\omega_{\ell,m}^\prime|t)\Bigr]
\end{aligned}
\label{eq:autoOverDamped}
\end{equation}
در صورتی که اتلاف در سیستم بسیار زیاد باشد، در معادله‌ی
\ref{eq:LangevinEq}
می‌توان از تاثیر جملات اینرسی صرف نظر کرد. معادلات حرکت در این حالت حدی با معادلات براونین توصیف می‌شوند،
\begin{equation}
\frac{r_0}{4\pi}\zeta \dot{u}_{\ell,m}(t)+\frac{r_0}{4\pi}k_{\ell,m}u_{\ell,m}(t)= F_{\eta}.
\label{eq:BrowninaMode}
\end{equation}
رابطه‌ی خود همبستگی دامنه‌ی مد‌ها برای این نوع دینامیک به شکل 
\begin{equation}
\langle u_{\ell,m}(t)u_{\ell,m}(0)\rangle= \langle|u_{\ell,m}|^2\rangle e^{-t/\tau_\zeta}.
\label{eq:BrowninaModeAuto}
\end{equation}
است. زمان واحلش سیستم در این حالت 
$\zeta/k_{\ell,m}=\gamma/\omega_{\ell,m}^2$
است که تابع نسبت سختی نیروی هماهنگ ساده و نیرو‌های اتلافی است.

می‌توان از روش کلی پیاده شده در این بخش برای مطالعه‌ی دینامیک سطح غشای قوطه ور در یک شارع استفاده کرد. در چنین سیستمی نیرو‌های ناشی از انحنا، مانند این بخش، همچنان به خواص غشا وابسته است. اما سرعت نقاط روی سطح غشا کاملا توسط دینامیک شارع داخل و خارج غشا تعیین خواهد شد. تابع همبستگی مد‌های سطحی در این سیستم 
\cite{milnersafranPRA1987}
با محاسبه‌ی میدان سرعت بر سطح غشا با حل معادلات استوکس قابل محاسبه خواهد بود
\cite{Christer1984}
.





