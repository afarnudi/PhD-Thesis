\setRL
%\pagenumbering{arabic} 



.
 
 
 
 
 
\section{محاسبه‌ی اندازه افت و خیز روی کره}
در این بخش نحوه‌ی اندازه‌ی دامنه‌ی افت و خیزهایی که بر روی سطح کره اینجاد می‌شود را اندازه‌گیری می‌کنیم. 
\cite{safran1983}
فرض می‌کنیم که انرژی خمش کُره به شکل زیر تعریف می‌شود
\begin{equation}
E_{b}=\frac{1}{2}\kappa\int dS\left(H-H_0\right)^2
\end{equation}
 که در اینجا 
 $\kappa$
 سختی خمش و
 $H$
و 
$H_0$
به ترتیب خمش و خمش ذاتی سطح کروی است. خمش ذاتی به شکل 
$H_0=2/r_s$
و خمش سطح به شکل
\begin{equation}
H=\left(\frac{1}{r_1}+\frac{1}{r_2}\right)=\frac{\nabla\cdot\hat n}{2}
\end{equation}
تعریف می‌شود. در معادلات بالا 
$r_s$
شعاع خمش ذاتی و 
$r_1$ و $r_2$
شعاع‌های پایه‌ای خمش
\LTRfootnote{princople curvature radii}
و $\hat n$ بردار عمود بر سطح است. در نتیجه انرژی خمش را به شکل زیر باز نویسی می‌کنیم
\begin{equation}
E_{b}=\frac{1}{8}\kappa\int dS\left(\nabla\cdot\hat n-\frac{2}{r_s}\right)^2
\label{eq:ebforsubstitution}
\end{equation}
سطح تقریبا کروی که مرکز آن در مبدا مختصات وجود دارد را به شکل زیر تعریف می‌کنیم
\begin{equation}
R(r)= r-r_0\left[1+g(\theta,\phi)\right]=0
\label{eq:radiusdef}
\end{equation}
. در این معادله $r_0$ شعاع متوسط کره است که $g$
\begin{equation}
g(\theta,\phi)=\sum_{\ell,m}u_{\ell m}Y_{\ell m} (\theta,\phi)
\label{eq:gdef}
\end{equation}
اختلاف شعاع هر نقطه از شعاع متوسط است که با هماهنگ‌های کروی، $Y_{\ell m}$، نشان داده شده. بردار عمود در هر نقطه از سطح کره را می‌توانیم به شکل زیر محاسبه کنیم
\begin{equation}
\hat n = \frac{\nabla R(r)}{|\nabla R(r)|}= \frac{\hat r-\frac{r_0}{r}g_\theta \hat\theta-\frac{r_0}{r\sin\theta}g_\phi\hat\phi }{\sqrt{1+\left(\frac{r_0}{r}g_\theta\right)^2+\left(\frac{r_0}{r\sin\theta}g_\phi\right)^2 }}
\end{equation}
که در اینجا برای ساده سازی از
$g_\theta=\partial/\partial\theta g$
و
$g_\phi=\partial/\partial\phi g$
استفاده شده‌است و محاسبات در دستگاه مختصات کروی انجام شده‌است.
جهت یادآوری،
\begin{equation}
\begin{aligned}
&\nabla f =\frac{\partial}{\partial r}f\hat r + \frac{1}{r} \frac{\partial}{\partial\theta}f\hat\theta+ \frac{1}{r\sin\theta} \frac{\partial}{\partial\phi}f\hat\phi\\
&\nabla\cdot \vec A =\frac{1}{r^2}\frac{\partial}{\partial r}(r^2A_r)+ \frac{1}{r\sin\theta} \frac{\partial}{\partial\theta}(A_\theta\sin\theta)+ \frac{1}{r\sin\theta} \frac{\partial}{\partial\phi}A_\phi\\
&\nabla^2f =\frac{1}{r^2}\frac{\partial}{\partial r}\left(r^2\frac{\partial}{\partial r}f\right)+ \frac{1}{r^2\sin\theta} \frac{\partial}{\partial\theta}\left(\sin\theta\frac{\partial}{\partial\theta}f\right)+ \frac{1}{r^2\sin^2\theta} \frac{\partial^2}{\partial\phi^2}f
\end{aligned}
\end{equation}
از این پس محاسبات را تنها تا مرتبه‌ی دوم نسبت به $g$ 
انجام خواهیم داد. در نتیجه بردار عمود بر سطح را به این شکل بازنویسی می‌کنیم،
\begin{equation}
\hat n \simeq\left\{1-\frac{1}{2}\left[\left(\frac{r_0}{r}g_\theta\right)^2+\left(\frac{r_0}{r\sin\theta}g_\phi\right)^2 \right]\right\}^{-\frac{1}{2}}\left( \hat r-\frac{r_0}{r}g_\theta \hat\theta-\frac{r_0}{r\sin\theta}g_\phi\hat\phi \right)
\end{equation}
حالا می‌توان دیورژانس را به ترتیب زیر محاسبه کرد،
\begin{equation}
\begin{aligned}
&\nabla\cdot\hat n \simeq \frac{2}{r}+\frac{1}{r\sin\theta}\frac{\partial}{\partial\theta}\left(-\frac{r_0}{r}g_\theta\sin\theta\right)+\frac{1}{r\sin\theta}\left(-\frac{r_0}{r\sin\theta}g_\phi\right)\\
&=\frac{2}{r}\left[1-\frac{r_0}{2r}\left(\frac{1}{\sin\theta}\frac{\partial}{\partial\theta}g_\theta\sin\theta+\frac{1}{\sin^2\theta}g_{\phi\phi}\right)\right]
\label{eq:divn}
\end{aligned}
\end{equation}
با در نظر گرفتن تعریف عملگر اندازه حرکت زاویه‌ای 
\begin{equation}
L^2=-\frac{1}{\sin\theta}\frac{\partial}{\partial\theta}\left(\sin\theta\frac{\partial}{\partial\theta}\right)-\frac{1}{\sin^2\theta}\left(\frac{\partial^2}{\partial\phi^2}\right)
\end{equation}
می‌توانیم معادله‌ی
\ref{eq:divn}
را به شکل زیر بازنویسی کنیم،
\begin{equation}
\nabla\cdot\hat n =\frac{2}{r}\left[1+\frac{r_0}{2r}L^2g\right]
\label{eq:divnL2}
\end{equation}
همچنین انتگرال عنصر سطحی را نیز می‌توان به ترتیب زیر تعریف کرد و تا حد تقریب مرتبه‌ی دوم رد $g$ جلو رفتچ
\begin{equation}
dS=r^2d\Omega\left[1+\frac{r_0^2}{2r^2}\left(g_\theta^2+\frac{g_\phi^2}{\sin^2\theta}\right)\right]=r^2d\Omega\left(1+\frac{r_0^2}{2r^2}gL^2g\right)
\label{eq:dsL2}
\end{equation}
حال کافی است که جملات بالا را در معادله‌ی \ref{eq:ebforsubstitution} جایگذاری کنیم،
\begin{equation}
E_b=\frac{1}{8}\kappa\int r^2d\Omega\left(1+\frac{r_0^2}{2r^2}gL^2g\right)\left[\frac{2}{r}\left(1+\frac{r_0}{2r}L^2g\right)-\frac{2}{r_s}\right]^2
\label{eq:ebcalc1}
\end{equation}
 $2/r$ را از جملات داخل کروشه فاکتور گرفته سپس جملات 
 \ref{eq:ebsubs}
 را در معادله‌ی
 \ref{eq:ebcalc1}
 جایگذاری می‌کنیم‌. در نتیجه انرژی به شکل زیر تعریف می‌شود،
\begin{equation}
\begin{aligned}
&\tilde{g}=\frac{1}{2}L^2g\\
&r=r_0(1+g)\\
&\frac{r_0}{r}\simeq 1-g
\label{eq:ebsubs}
\end{aligned}
\end{equation}


\begin{equation}
E_b=\frac{1}{2}\kappa\int d\Omega\left[1+\tilde gg(1-g)^2\right]\left[1+\tilde g(1-g)-\frac{r_0}{r_s}(1+g)\right]^2
\end{equation}
پس از انجام عملیات جبری و حفظ جملات تا مرتبه‌ی دوم نسبت به $g$
معادله‌ی انرژی به شکل زیر در می‌آید
\begin{equation}
\begin{aligned}
&E_b=\frac{1}{2}\kappa\int d\Omega\left[1-2\frac{r_0}{r_s}+\left(\frac{r_0}{r_s}\right)^2+\tilde gg -2\frac{r_0}{r_s}\tilde gg+\left(\frac{r_0}{r_s}\right)^2\tilde gg\right.\\
&\left.+\tilde g^2-2\tilde gg +\left(\frac{r_0}{r_s}\right)^2g^2+2\left(\frac{r_0}{r_s}\right)^2g+2\tilde g-2\frac{r_0}{r_s}g-2\frac{r_0}{r_s}\tilde g\right]
\end{aligned}
\end{equation}
پس از فاکتورگیری و مرتب سازی شکل در می‌آید
\begin{equation}
E_b=\frac{1}{2}\kappa\int d\Omega\left[\left(1-\frac{r_0}{r_s}\right)^2(1+\tilde gg)+\tilde g(\tilde g-2g)+\left(\frac{r_0}{r_s}\right)^2g^2\right]
\label{eq:ebfinal}
\end{equation}
از آنجایی که در معادله‌ی
\ref{eq:radiusdef}
افت و خیز شعاع را نسبت به شعاع متوسط تعریف کردیم، انتگرال جملات مرتبه‌ی اول $g$ روی سطح کره برابر با صفر خواهد شد، این جملات از معادله‌ی 
\ref{eq:ebfinal}
حذف شده‌اند. فرض کنیم که سطح مورد بررسی ترجیه خمش نداد و حالت کمینه انرژی آن زمانی است که سطح تخت باشد، در این صورت با $r_s\rightarrow\infty$ انرژی به شکل زیر تغییر خواهد کرد:
\begin{equation}
E_b=\frac{1}{2}\kappa\int d\Omega\left(1-g\tilde g+\tilde g^2\right)
\label{eq:ebfinalnors}
\end{equation}
حال با جایگذاری معادله‌ی
\ref{eq:gdef}
و $\tilde g=(1/2)L^2g$ در معادله‌ی بالا انرژی را محاسبه می‌کنیم.
\begin{equation}
\begin{aligned}
&E_b=\frac{1}{2}\kappa\int d\Omega\left[1-g\frac{1}{2}L^2g+\frac{1}{4}\left(L^2g\right)^2\right]\\
&=\frac{1}{2}\kappa\int d\Omega\left[1-\frac{1}{2}\sum_{\ell',m'}u_{\ell' m'}Y_{\ell' m'} (\theta,\phi)\sum_{\ell,m}u_{\ell m}L^2Y_{\ell m} (\theta,\phi)+\frac{1}{4}\left(\sum_{\ell,m}u_{\ell m}L^2Y_{\ell m} (\theta,\phi)\right)^2\right]\\
&=2\pi \kappa+\frac{1}{8}\kappa\sum_{\ell,m}|u_{\ell m}|^2\left[\ell^2(1+\ell)^2-2\ell(1+\ell)\right]\\
&=2\pi\kappa+\frac{1}{8}\kappa\sum_{\ell,m}|u_{\ell m}|^2\ell(\ell+1)(\ell-1)(\ell+2)
\end{aligned}
\end{equation}
با توجه به اصل همپاری انرژی می‌توانیم انرژی هر مد را محاسبه کنیم،
\begin{equation}
\begin{aligned}
\frac{1}{2}k_BT&=\frac{1}{8}\kappa\sum_{\ell,m}|u_{\ell m}|^2\ell(\ell+1)(\ell-1)(\ell+2)
\end{aligned}
\end{equation



 
 
 
 
 
 
 
 
 
 