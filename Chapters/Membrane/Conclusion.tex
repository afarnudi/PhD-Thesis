\setRL
\clearpage
%\def \MemRes {\Mempath /MembraneResults}
در این رساله با هدف فهمیدن فیزیک حاکم بر غشا‌ها و رویه‌های نازک، مدل‌های پیوسته‌ای پوسته‌های نازک مطالعه شد. این مطالعه شامل بررسی مدل‌سازی رفتار کششی، خمش و اثر ویژگی‌های هندسی پوسته‌ها بود. جهت بررسی عمیق این مدل‌ها رفتار آماری غشا‌ها با ویژگی‌های فیزیکی مختلف در فضای مُد کاملا بررسی شد. با هدف شبیه‌سازی این سامانه‌ها روش‌های مختلف شبیه‌سازی این سامانه‌ها در طول تاریخ مطالعه شد.

 مدل‌سازی غشا‌های سیال هنگام استفاده از مدل‌های درشت دانه‌ی خیلی بزرگ که خاصیت دو قطبی ملکلول‌های لیپیدی در آن گم می‌شود بسیار دشوار است. در این مدل‌ها خمش دیگر از برهمکنش دوگانه‌ی ذرات ایجاد نمی‌شود. برای حل این مشکل معمولا غشا با یک شبکه‌ نمایش داده می‌شود و خمش بر حسب عناصر مش تعریف می‌شود. گامپر و کرول خمش غشا را برای شبکه‌های مثلثی با توزیع رئوس یکنواخت تعریف کردند. روش دینامیک مثلثی بهترین روشی است که با استفاده از آن شرایط مورد نیاز برای محاسبه‌ی خمش گامپر حفظ شده و می‌توان رفتار سیال‌گون غشا را شبیه‌سازی کرد. ماهیت الگوریتم دینامیک مثلثی به نحوی است که فقط می‌توان از روش مانتی کارلو برای شبیه‌سازی آن استفاده کرد. 
 
ما متوجه شدیم که  هیچ روش دینامیک ملکولی بسیار درشت دانه‌ای برای مطالعه‌ی «دینامیک» و افت و خیز غشا‌های سیال وجود ندارد. در جهت پُر کردن این جای خالی روش محاسبه‌ی خمش در شبکه‌های مثلثی که توزیع رئوس آن یکنواخت نیست بررسی شد. مش‌های نرم مش‌هایی هستند که شبکه‌ی مثلثی دارند ولی طول اضلاع آن می‌تواند آزادانه تغییر کند. پس از تایید صحت محاسبات خمش بر مش‌های نرم، روش جدیدی به نام توزیع دینامیک مساحت جهت شبیه‌سازی رفتار سیالگون غشا ابداع شد. در این روش توپولوژی شبکه تغییر نمی‌کند و در نتیجه با روش شبیه‌سازی دینامیک ملکولی مطابقت دارد. در این روش با تغییر مساحت مثلث‌های شبکه، چیدمان مختصات مش کاملا تغییر می‌کند ولی مساحت کل مش  ثابت است. مطالعه‌ی ما نشان داد که مدل خمش یولیشر به لحاظ دینامیکی بسیار پایدار است ولی معروف‌ترین مدل محاسبه‌ی خمش، مدل گامپر و کرول، به لحاظ دینامیکی پایدار نیست. با شناسایی ناپایداری‌های مدل گامپر، مدل جدیدی با نام گامپر-بریسنتریک معرفی شد که به لحاظ دینامیکی بسیار پایدار است و همچنین نیروهای خمش صحیحی تولید می‌کند.

در این رساله روش توزیع دینامیک مساحت برای مش‌های مثلثی بررسی شد. این روش را می‌توان برای هر مشی که مساحت، حجم، و خمش در آن تعریف مشخصی داشته باشد، استفاده کرد. روش شبیه سازی دینامیک ملکولی بهترین روش برای مطالعه‌ی دینامیک و افت و خیز است. با محاسبات زمانی ارائه شده در این رساله می‌توان دینامیک شبیه‌سازی را با مطالعات آزمایشگاهی مقایسه کرد. مثلا می‌توان رفتار مش‌های ترکیبی در شرایط هیدرودینامیکی مختلف یا هنگام عبور از مجاری نازک بررسی کرد. همچنین می‌توان این مدل را با مدل‌های درشت‌دانه‌ی دینامیک ملکولی سلول ترکیب و رفتار جمعی آن را مطالعه کرد.

در روش توزیع دینامیک مساحت، تا زمانی که مساحت کل و انرژی خمش تغییر نکند، رئوس مش شبیه به مُدهای گلدستون تقریبا آزادانه حرکت می‌کنند. در نتیجه تغییرات مساحت مثلث‌ها بسیار سریع‌تر از تغییر شکل مش است. ما نشان دادیم که برای یک شبیه‌سازی معمولی با نسبت مسحاتی
$\Phi=0.037$
زمان واحلش تغییرات مساحت مثلث‌ها حدود 
$6\tau_h$
است. دوره‌ی تناوب بزرگترین تغییر شکل یک وسیکل که ضریب سختی خمش
$\kappa=20k_BT$
دارد حدود 
$\tau_{2,0}\approx250\tau_h$
است. در نتیجه تغییرات چیدمان مش نسبت به تغییر شکل مش حدود ۴۰ برابر سریع‌تر است. 

از آنجایی که در روش توزیع دینامیک مساحت اتصالات مش تغییر نمی‌کند،‌ این روش تغییر شکل‌هایی که نیاز به تغییرات توپولوژی دارد (مانند ایجاد اشکال لوله‌ای شکل بر سطح مش یا جوانه زدن)  را نمی‌تواند نشان دهد. پیشنهاد ما برای مطالعه‌ی چنین فرآیندهایی استفاده از یک قدم مش سازی نزدیک به نقاط حساس تغییر شکل جهت تغییر تپولوژی مش به توپولوژی منطبق‌تر است. همچنین لازم است به این نکته اشاره کرد که با وجود اینکه روش توزیع مثلث برای دینامیک ملکولی طراحی شده، در روش مانتی کارلو نیز قابل استفاده است.

\subsection{
پیشنهاد برای مطالعات آینده
}

افت و خیز غشا معمولا در آزمایشگاه با اندازه‌گیری افت و خیز مدار استوایی غشا انجام می‌شود. افراد با مطالعه‌ی شدت‌ بسامد مد‌های مختلف مدول‌های الاستیک نمونه‌ها  را اندازه‌گیری می‌کنند. با وجود مدل‌های دینامیک ملکولی برای غشاها‌ی جامد و سیال می‌توان افت‌ و خیز‌های ۳ بعدی و مداری اجسام کروی و شبه کروی را بررسی کرد. سوال اول این است که آیا شدت افت و خیز‌های مدار استوایی اطلاعات کافی برای باز سازی مدل ۳ بعدی یک غشا را در اختیار ما قرار می‌دهد یا خیر. غشا‌های زیستی به ندرت شکل شبه کروی دارند. اما روش‌های اندازه‌گیری افت و خیز همگی بر اساس نظریه‌های افت و خیز اجسام کروی است. سول دوم این است که تا چه اندازه مطالعات افت و خیز مدار برای اشکال غیر کروی اعتبار دارد.

به علت نبود مدل دینامیک ملکولی صحیح برای گلبول‌های قرمز، افت و خیز این سلول‌ها تنها با روش‌های آزمایشگاهی مطالعه شده‌است. پیشنهاد می‌شود که از مدل توزیع دینامیک مساحت برای این مطالعه استفاده شود. 

تغییر شکل گلبول‌های قرمز در جریان‌های برشی مطالعه‌ی زیادی شده‌است. می‌توان از روش توزیع دینامیک مساحت برای مطالعه‌ی مشابه رفتار تغییر شکل گلبول قرمز در این جریان‌ها و همچنین هنگام عبور از مجاری تنگ استفاده کرد. 








