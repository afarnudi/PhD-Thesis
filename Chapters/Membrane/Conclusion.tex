\setRL
\clearpage
%\def \MemRes {\Mempath /MembraneResults}

\section{
بحث
}
\subsection{مِش‌های نرم و سخت}
مش‌های مثلثی همگن مشخصات هندسی رویه‌ها را بسیار دقیقتر از مش‌هایی با تعداد نقاط برابر ولی با مثلث‌‌های با اندازه‌ی مختلف نمایش می‌دهند. به همین علت در بیشتر اندازه‌گیری‌ها و شبیه‌سازی‌های عددی که غشا را مورد مطالعه قرار می‌دهند، از مش‌های سخت با توزیع نقاط یکنواخت استفاده می‌شود. اما زمانی‌ که شکل مش‌ نیاز به تغییر دارد، برای استفاده از مش‌های سخت باید دائما از روش‌های بازسازی مش استفاده کرد تا تنش و کرنش حاصل از تغییر شکل، آزاد شود.

در این رساله ما پیشنهاد کردیم که در ازای از دست دادن دقت، از انعطاف مش‌های نرم برای شبیه‌سازی و مدل‌سازی تغییر شکل‌ رویه‌ها استفاده شود. هنگام شبیه‌سازی با مش‌های نرم، در صورتی که اتصالات مش با شکل‌ نهایی مطابقت لازم را داشته باشد، ثابت بودن اتصالات میان نقاط مش در طول شبیه‌سازی مانع ایجاد نخواهد کرد.

\subsection{
روش گسسته‌سازی مستحکم انرژی انحنا
}
هنگام استفاده از مش‌های سخت، تفاوت خاصی میان استفاده از روش گامپر و کرول
\cite{gompper1996},
که بر اساس روش گسسته سازی ایتزیکسن 
\cite{Itzykson1986}
بنا شده، و روش گسسته سازی یولیشر
\cite{Julicher1996}
وجود ندارد. اما روش گامپر و کرول برای اندازه‌گیری انحنا بر مش‌های نرم (بر عکس روش یولیشر) تقریبا به خوبی مش‌های سخت است. در بخش 
\ref{sec:curvatureCalculation}
نشان داده شد که در مدل گسسته سازی یولیشر خطا‌های زیادی در اندازه‌گیری انحنا (شکل
\ref{fig:unitsphereAll})
مشاهده شد، در صورتی که این اختلاف هنگام محاسبه‌ی نیرو‌های ناشی از ایجاد انحنا بر سطح دیده نمی‌شود (شکل
\ref{fig:AllULM20}).
  دلیل اصلی دقت بالا در روش گسسته سازی ایتزیکسون
\cite{Itzykson1986},
انتخاب مساحت ورونی به جای بریسنتر برای وزن نقاط روی مش است، زیراکه صورت معادلات یولیشر و ایتزیکسون پاسخ تقریبا یکسانی برای انحنا در نظر می‌گیرند (شکل‌های
\ref{fig:unitsphereBendingScatter}
و
\ref{fig:ULM20BendingScatter}).

در بخش
\ref{sec:Results DAR MD}
نشان دادیم که در ازای از دست دادن دقت هنگام استفاده از مساحت بریسنتر، پایداری عددی بسیار برتر و مستحکمی در محاسبه‌ی انحنا مش‌های نرم بدست می‌آید (شکل
\ref{fig:timeSteps}).
با اینکه پتانسیل‌های تکمیلی طراحی شده در این رساله (شکل
\ref{fig:PlacketEnergyArea})
هر دو روش محاسبه‌ی انحنا را بسیار پاید می‌دهد (شکل
\ref{fig:PlacketEnergyAll}),
 روش محاسبه‌ی انحنای بر حسب مساحت ورنوی شرایطی را ایجاد می‌کند که خطا‌های بزرگی (شکل
\ref{fig:energyLandscapeZoomed})
 در محاسبات پدید می‌آید.

\subsection{
شبیه‌سازی دینامیکِ مِش
}
دینامیک مش روشی بر اساس دینامیک ملکولی است که می‌توان با آن مدل‌های پیوسته‌ی غشا‌ها را شبیه‌سازی کرد. در این روش درجات آزادی، نقاط بر سطح مش دو بُعدی (در فضای سه بُعدی) است که اتصالات میان آنها تغییر نمی‌کند. علاوه بر انرژی انحنا (بخش
\ref{sec:curvatureD}),
روشی برای پیاده‌سازی پتانسیل‌های چند ذره‌ای که مساحت و حجم مش را کنترل می‌کنند، یافته شد (بخش
 \ref{sec:areaVolumeD}).
  برخلاف روش‌های مانتی کارلو که قدم‌های آن تنها بر یک ذره تمرکز می‌کند، با حل معادلات حرکت، با جفت شدگی میان درجات آزادی سیستم، حرکت دسته‌جمعی ذرات به شکل غیر موضعی مساحت‌ و حجم‌ مثلث‌های روی مش را باز پخش می‌کند.

ارزیابی‌های اولیه‌ی ما که در این رساله ارائه شد، نشان می‌دهد که روش دینامیک مش بهره‌وری بالایی دارد (بخش
\ref{sec:larger shape changes})
 و هنگامی که بر مش‌های نرم پیاده‌سازی می‌شود، رفتاری‌های نمایی بلند مقیاس غشا‌های مایع را بدون نیاز به بازسازی اتصالات مش تولید می‌کند (بخش
\ref{sec:resultsBendingFluctuations}).
همچنین دینامیک اتلافی ناچیزی که در این روش تولید می‌شود مانع شبیه‌سازی سامانه‌های در رژیم عدد رنولدز پایین نیست (بخش
\ref{sec:dynamics}).

\subsection{
دینامیکِ مِش یا دینامیک مثلثی؟
}
دینامیک مش نشان می‌دهد که دینامیک یک سطح می‌تواند مستقل از اتصالات مش رفتار کند. در بحث مکانیک آماری سطوح نشان دادیم که متوسط گیری آماری روی اشکال سطح را می‌توان با متوسط گیری بر مختصات و اتصالات مش تقریب زد. تا زمانی که اتصالات مش با شکل کلی سطح مطابقت داشته باشد، با تقریب خوبی می‌توان مشاهده پذیر‌های سطح را با تنها انتگرال ‌گیری بر فضای فاز چیدمان‌ها با اتصالات ثابت اندازه‌گیری کرد  (بخش
\ref{sec:MeshObservables}).
 در شکل
\ref{fig:nuShapes}
نشان دادیم که با استفاده از مش‌های نرم می‌توان اشکال مربوط به سطوح با حجم کاهیده‌ی
$\nu\ll1$
را با چنین شبیه‌سازی‌هایی تولید کرد.

التبه این استدلال به این معنی نیست که همه جا باید از مش‌های نرم با اتصالات ثابت استفاده کرد. روش مثلث بندی دینامیک
\cite{BoalPRA1992, Gompper1992Science}
به وضوح تنها روش برای از میان بردن محدودیت‌هایی است که هنگام دینامیک مش سخت با آن مواجه می‌شویم (شکل
\ref{fig:nuShapes}).
 همانطور که در شکل
\ref{fig:UhvsNu}
نشان داده شد، اختلاف ایجاد شده در انرژی خمش و پتانسیل تکمیلی
 $\langle U_h\rangle$,
که وظیفه‌ی آن کنترل نرمی مش است، با اجرای الگوریتم  دینامیک مثلثی رفع خواهد شد(در این رساله انجام نشد). در این رساله تلاش نشد که روش دینامیک مش با روش مانتی کارلوی  دینامیک مثلثی ترکیب شود. در صورتی که این دو روش با هم ترکیب شوند، پیش‌بینی می‌شود که  دینامیک مش وظیفه‌ی باز پخش غیر موضعی جرم و دینامیک سطح را به عهده خواهد داشت، و الگوریتم  دینامیک مثلثی انرژی ذخیره شده در پتانسیل‌های تکمیلی را آزاد خواهد کرد تا شکل سطح بدون محدودیت، چیدمان‌های مختلف را به خود بگیرد. واضح است که هرچقدر سختی مش بیشتر باشد، فرکانس استفاده از روش مثلث بندی دینامیک  باید بالا رود (شکل
\ref{fig:UhvsNu}).


\section{
خلاصه و نتیجه‌گیری
}
بیشتر شبیه‌سازیهای مدل‌های پیوسته‌ی غشاها از روش گامپر و کرول
\cite{gompper1996}
که  روش مانتی کارلو مثلث بندی دینامیک 
\cite{Boal1992PRA, Gompper1992Science}
را همراه با روش گسسته‌سازی مدل انرژی انحنا که برگرفته از کار ایتزیکسون
\cite{Itzykson1986}
است استفاده می‌کنند. در این رساله نشان دادیم که روش یولیشر
\cite{Julicher1996}
به لحاظ پایداری عددی بسیار مستحکم‌تر می‌تواند انحنای مش‌های مثلثی که در آن مثلث‌ها اندازه‌های مختلف دارند را محاسبه کند. خیلی جالب است که منشا استحکام پایدار روش یولیشر از استفاده کردن مساحت بریسنتر مجموعه‌ی مثلث‌ها به عنوان وزن هر نقطه روی مش است. درواقع استفاده از مساحت ورنوی که در روش ایتزیکسون است باعث ناکامی این روش می‌شود (شکل
\ref{fig:PlacketEnergyAll}).

با مِش‌های نرم می‌توان شکل‌های مختلفی را بدون نیاز به باز سازی اتصالات میان نقاط نمایش داد. در نتیجه این مِش‌ها برای مطالعه‌ی دینامیک رویه‌ها بسیار مناسب هستند. در این رساله برای شبیه‌سازی غشا‌های سیال‌گون با استفاده از روش دینامیک مِش، بجای شبیه‌سازی دینامیک مثلثی، از مدل‌های محاسبه‌ی انحنا بر پای مساحت بریسنتر استفاده شد. شبیه‌سازی دینامیکِ مِش بر پایه‌ی دینامیک ملکولی است و در آن درجات آزادی، نقاط روی مش دو بعدی است که حرکت درون صفحه‌ای نقاط تحت قید بسیار محدودی است. به طور مشخص در این شبیه‌سازی، مساحت‌ مش به شکل غیر موضعی کنترل می‌شود. در این روش با حرکت درون صفحه‌ای نقاط، مساحت مثلث‌های مش به شکل غیر موضعی بر سطح مش باز پخش می‌شوند. 

ما روش دینامیکِ مِش را از طریق نرم افزار مدل سلول مجازی
\cite{VCMgit}
در موتور محاسباتی دینامیک ملکلوی 
OpenMM \cite{OpenMM2017}
پیاده‌سازی کردیم. در نتیجه توانستیم از قابلیت 
OpenMM 
برای پخش محاسبات بر منابع محاسباتی 
CPU
و یا
GPU
بهره ببریم.  همچنین با استفاده از این موتور محاسباتی قادر به محاسبه‌ی نیروهای حاصل از پتانسیل‌های چند ذره‌ای هستیم.

به عنوان اولین قدم، در جهت معرفی مِش‌های نرم، پتانسیل‌های گسسته (بخش‌های
\ref{sec:areaVolumeD}
و
\ref{sec:curvatureD})
و پتانسیل‌ تکمیلی مناسب (بخش
\ref{sec:auxPotentials})
 تعریف و مطالعه شد (بخش
\ref{sec:Soft meshes}).
پس از ساخته شدن پایه‌ی محاسبات عددی روش دینامیکِ مِش (بخش
\ref{sec:MeshObservables})
 قدم زمانی مورد نیاز برای شبیه‌سازی دینامیک ملکولی محاسبه و تعیین شد (بخش
\ref{sec:Results DAR MD}).
در نهایت نشان داده شد که این شبیه‌سازی‌ها بسیار پایدار هستند (بخش
\ref{sec:ResultsStability}).

در قدم دوم، بر تایید فیزیک تولید شده  توسط شبیه‌سازی دینامیکِ مِش، برای مش‌هایی متناظر با غشاهایی سیال گون با قطر 
$1 \mu m$
تا
$3 \mu m$
و سختی خمش 
$10 k_BT$
تا
$80 k_BT$
پرداختیم (بخش
\ref{sec:resultsBendingFluctuations}).
ملاک ارزیابی کمی این روش، مطالعه‌ی میانگین مربع دامنه‌ی افت و خز مُد‌های سطحی
($\langle|u_{\ell,m}|^2\rangle$)
 غشا‌های تقریبا کروی بود
\cite{milnersafranPRA1987, gomppernelson2012}.
با توجه به طیف دامنه‌ی افت و خیز مد‌های سطحی مش، نشان دادیم که رفتار صحیح سیال‌گون غشا‌ها برای مد‌های کوچک یا طول موج‌های بلند، که با رفتار 
 $\langle|u_{\ell,m}|^2\rangle\propto \frac{1}{\ell^4}$
مشخص می‌شود، با پیش‌بینی‌های نظری مطابقت کامل دارد (شکل
\ref{fig:kappaULMS}).
 علاوه بر این، بررسی ما نشان داد که در شبیه‌سازی دینامیک مش به علت وجود پتانسیل تکمیلی
$U_h$
و ثابت بودن اتصالات مش، مدول یانگ دو بعدی موثری 
$Y_{2d}(\Phi)$
در رفتار مش دیده‌ می‌شود که با تنظیم نرمی مش
 $\Phi$ 
قابل کنترل است (شکل
\ref{fig:kappaULMS}$(d)$
و
$(e)$).

پس از اطمینان از سلامت فیزیک تولید شده توسط شبیه‌ سازی دینامیکِ مِش، تغییر شکل‌های بزرگ مقیاس غشا (شکل
\ref{fig:shapeChanges})
در بخش 
\ref{sec:larger shape changes}
مطالعه‌ شد. نتایج اولیه‌ی ما نشان داد که با استفاده از این شبیه‌سازی می‌توان شکل‌های معروف هامیلتونی هلفریش را مطالعه‌کرد (شکل
\ref{fig:UhvsNu}).
از آنجایی که دینامیک مش بر اساس روش دینامیک ملکولی بنا شده‌است، پتانسیل‌های استخراج شده از این روش می‌تواند در کنار پتانسیل‌های دینامیک ملکولی استفاده شود. با ترکیب کردن مدل غشا‌های سیال و جامد می‌توان غشا‌های ترکیبی مانند غشای گلبول قرمز را شبیه‌سازی و مطالعه کرد (بخش 
\ref{sec:hybrid}).
بررسی زمان لازم برای تشکیل اشکال حاصل از حجم کاهیده و گلبول‌های قرمز به روش شبیه‌سازی دینامیک مش نشان داد که این اتفاق در کسری از  زمان لازم برای تکمیل بزرگ‌ترین دوره تناوب غشا،
$\tau_{2,m}$
اتفاق می‌افتد (شکل
\ref{fig:deformationEvo}).
 سرعت بالای به تعادل رسیدن سیستم، خود را در زمان و میزان مصرف انرژی برای  شبیه‌سازی‌ این سیستم‌ها نیز نشان می‌دهد (بخش 
\ref{sec:computational efficiency})

در قدم آخر،  با بررسی رفتار تابع خودهمبستگی دامنه‌ی افت و خیز‌ مُدهای سطحیِ مِش‌،  دینامیک رویه‌ی غشا را در رژیم‌های دینامیکی مختلف (نیوتنی، لانژون، و براونی) مطالعه کردیم (بخش
\ref{sec:dynamics}).
مطالعه‌ی دینامیک مش نشان داد که قادر به فهم کمی زمان واهلش مد‌های سطحی در رژیم‌های نیوتنی، لانژون، و براونی هستیم (شکل
\ref{fig:autoAnalysis}).
 همچنین این مطالعه نشان داد که دینامیک مش یک رفتار اتلافی شبیه‌ به دینامیک لانژون با ضریب اتلاف وابسته به عدد موج در مش‌ها ایجاد می‌کند (شکل
 \ref{fig:gammaEll}).
  مهم‌تر از همه اتلاف داخلی مش، حتی برای کند‌ترین تغییر شکل مش، زیر-میرا است. این مشخصه‌ی مهم، تولید دینامیک صحیح در رژیم‌های عدد رنولدز پایین که در آن حرکت نقاط سطح غشا توسط هیدرودینامیک سیال درون و بیرون غشا کنترل می‌شود را تضمین می‌کند
\cite{schneider1984,milnersafranPRA1987}.

 

دو محدودیت اصلی این رساله را محاصره کرده‌است.
\subsection{
محدودیت اول
}

پتانسیل‌های مساحت و حجم استفاده شده در  شبیه‌سازی‌های ما به شکل موضعی
\cite{Vutukuri2020}
مساحت و حجم عناصر مش را تنظیم نمی‌کند بلکه به شکل سراسری و لحظه‌ای عمل می‌کند. در نتیجه مثلث‌های تشکیل شده بر سطح مش، دارای اندازه و شکل‌‌های مختلف است (شکل
\ref{fig:meshTypesMesthod})
و در هر قدم از شبیه‌سازی نماینده میزان مختلفی از رویه‌ی غشا هستند. از طرف همیلتونی عمومیِ مِش (معادله‌ی
\ref{eq:HamiltonianGeneral with explicit areas})
با ایجاد برهمکنشی موثر (معادله‌ی
\ref{eq:microStateProbabilityExpansion})
میل به یکسان‌سازی مساحت اختصاص داده شده به تمام نقاط روی مش دارد.  باید اعتراف کرد که حل معادلات حرکت (معادله
\ref{eq:HamiltonianGeneralEOMMomentum})
 و پیاده‌سازی آنها در موتور 
OpenMM
نیاز به تلاش و مطالعه‌ی بیشتری دارد. ممکن است در نسخه‌های آینده‌ی موتور
OpenMM
از پیاده‌سازی این معادلات حمایت شود. حل دقیق معادلات حرکت مطالعه‌ی سیستم‌هایی همچون دینامیک حباب صابون را امکان پذیر خواهد کرد. در سیستم‌های شبیه‌ به حباب صابون حرکت نقاط مش به آرایش توزیع جرم بر سطح مش حساس است. ما باور داریم که برای غشا‌هایی که درون سیال در حال حرکت هستند (و جملات اینرسی از معادلات حذف می‌شود) استفاده از تقریب جرم یکسان (معادله
\ref{eq:EoM for HamiltonianFixedMass})
 برای نقاط کافی است. البته ما در عمل غشایمان را با موتورهای حل هیدرودینامیک
\cite{MPCD2008JCP}
 ترکیب نکردیم تا از صحت این جمله کاملا اطمینان پیدا کنیم.

\subsection{
محدودیت دوم
}


به طور عمومی برای پیاده‌سازی مدل‌های پیوسته غشا و مطالعه‌ی تمامی تغییر شکل‌هایی که یک غشای سیال می‌تواند انجام دهد نیاز به بازسازی اتصالات میان نقاط مش است. در نتیجه باید روش دینامیک مش به عنوان یک روش تکمیلی برای روش دینامیک مثلثی در نظر گرفته شود و نه یک روش جایگزین. از آنجایی که تمام کار ما در این رساله بر اساس هامیلتونین بیان شده‌است، هیچ محدودیت مهمی برای ترکیب این روش با روش‌های مانتی‌کارلو دینامیک مثلثی یا هر روشِ بازسازیِ مِشِ دیگیری دیده نمی‌شود.


%
%
%در این رساله با هدف فهمیدن فیزیک حاکم بر غشا‌ها و رویه‌های نازک، مدل‌های پیوسته‌ای پوسته‌های نازک مطالعه شد. این مطالعه شامل بررسی مدل‌سازی رفتار کششی، خمش و اثر ویژگی‌های هندسی پوسته‌ها بود. جهت بررسی عمیق این مدل‌ها رفتار آماری غشا‌ها با ویژگی‌های فیزیکی مختلف در فضای مُد کاملا بررسی شد. با هدف شبیه‌سازی این سامانه‌ها روش‌های مختلف شبیه‌سازی این سامانه‌ها در طول تاریخ مطالعه شد.
%
% مدل‌سازی غشا‌های سیال هنگام استفاده از مدل‌های درشت دانه‌ی خیلی بزرگ که خاصیت دو قطبی ملکلول‌های لیپیدی در آن گم می‌شود بسیار دشوار است. در این مدل‌ها خمش دیگر از برهمکنش دوگانه‌ی ذرات ایجاد نمی‌شود. برای حل این مشکل معمولا غشا با یک شبکه‌ نمایش داده می‌شود و خمش بر حسب عناصر مش تعریف می‌شود. گامپر و کرول خمش غشا را برای شبکه‌های مثلثی با توزیع رئوس یکنواخت تعریف کردند. روش دینامیک مثلثی بهترین روشی است که با استفاده از آن شرایط مورد نیاز برای محاسبه‌ی خمش گامپر حفظ شده و می‌توان رفتار سیال‌گون غشا را شبیه‌سازی کرد. ماهیت الگوریتم دینامیک مثلثی به نحوی است که فقط می‌توان از روش مانتی کارلو برای شبیه‌سازی آن استفاده کرد. 
% 
%ما متوجه شدیم که  هیچ روش دینامیک ملکولی بسیار درشت دانه‌ای برای مطالعه‌ی «دینامیک» و افت و خیز غشا‌های سیال وجود ندارد. در جهت پُر کردن این جای خالی روش محاسبه‌ی خمش در شبکه‌های مثلثی که توزیع رئوس آن یکنواخت نیست بررسی شد. مش‌های نرم مش‌هایی هستند که شبکه‌ی مثلثی دارند ولی طول اضلاع آن می‌تواند آزادانه تغییر کند. پس از تایید صحت محاسبات خمش بر مش‌های نرم، روش جدیدی به نام توزیع دینامیک مساحت جهت شبیه‌سازی رفتار سیالگون غشا ابداع شد. در این روش توپولوژی شبکه تغییر نمی‌کند و در نتیجه با روش شبیه‌سازی دینامیک ملکولی مطابقت دارد. در این روش با تغییر مساحت مثلث‌های شبکه، چیدمان مختصات مش کاملا تغییر می‌کند ولی مساحت کل مش  ثابت است. مطالعه‌ی ما نشان داد که مدل خمش یولیشر به لحاظ دینامیکی بسیار پایدار است ولی معروف‌ترین مدل محاسبه‌ی خمش، مدل گامپر و کرول، به لحاظ دینامیکی پایدار نیست. با شناسایی ناپایداری‌های مدل گامپر، مدل جدیدی با نام گامپر-بریسنتریک معرفی شد که به لحاظ دینامیکی بسیار پایدار است و همچنین نیروهای خمش صحیحی تولید می‌کند.
%
%در این رساله روش توزیع دینامیک مساحت برای مش‌های مثلثی بررسی شد. این روش را می‌توان برای هر مشی که مساحت، حجم، و خمش در آن تعریف مشخصی داشته باشد، استفاده کرد. روش شبیه سازی دینامیک ملکولی بهترین روش برای مطالعه‌ی دینامیک و افت و خیز است. با محاسبات زمانی ارائه شده در این رساله می‌توان دینامیک شبیه‌سازی را با مطالعات آزمایشگاهی مقایسه کرد. مثلا می‌توان رفتار مش‌های ترکیبی در شرایط هیدرودینامیکی مختلف یا هنگام عبور از مجاری نازک بررسی کرد. همچنین می‌توان این مدل را با مدل‌های درشت‌دانه‌ی دینامیک ملکولی سلول ترکیب و رفتار جمعی آن را مطالعه کرد.
%
%در روش توزیع دینامیک مساحت، تا زمانی که مساحت کل و انرژی خمش تغییر نکند، رئوس مش شبیه به مُدهای گلدستون تقریبا آزادانه حرکت می‌کنند. در نتیجه تغییرات مساحت مثلث‌ها بسیار سریع‌تر از تغییر شکل مش است. ما نشان دادیم که برای یک شبیه‌سازی معمولی با نسبت مسحاتی
%$\Phi=0.037$
%زمان واحلش تغییرات مساحت مثلث‌ها حدود 
%$6\tau_h$
%است. دوره‌ی تناوب بزرگترین تغییر شکل یک وسیکل که ضریب سختی خمش
%$\kappa=20k_BT$
%دارد حدود 
%$\tau_{2,0}\approx250\tau_h$
%است. در نتیجه تغییرات چیدمان مش نسبت به تغییر شکل مش حدود ۴۰ برابر سریع‌تر است. 
%
%از آنجایی که در روش توزیع دینامیک مساحت اتصالات مش تغییر نمی‌کند،‌ این روش تغییر شکل‌هایی که نیاز به تغییرات توپولوژی دارد (مانند ایجاد اشکال لوله‌ای شکل بر سطح مش یا جوانه زدن)  را نمی‌تواند نشان دهد. پیشنهاد ما برای مطالعه‌ی چنین فرآیندهایی استفاده از یک قدم مش سازی نزدیک به نقاط حساس تغییر شکل جهت تغییر تپولوژی مش به توپولوژی منطبق‌تر است. همچنین لازم است به این نکته اشاره کرد که با وجود اینکه روش توزیع مثلث برای دینامیک ملکولی طراحی شده، در روش مانتی کارلو نیز قابل استفاده است.
%
\section{
پیشنهاد برای مطالعات آینده
}

یک و دو: پیشنهاد اصلی ما برای ادامه‌ی این پروژه رفع دو محدودیتی است که در بخش قبل عنوان شد. 

سه: افت و خیز غشا معمولا در آزمایشگاه با اندازه‌گیری افت و خیز مدار استوایی غشا انجام می‌شود. افراد با مطالعه‌ی شدت‌ در نتیجه مدول‌های الاستیک غشاها با اندازه‌گیری بسامد مد‌های استوایی اندازه‌گیری می‌شود. از طرفی، در حال حاضر با استفاده از مدل‌های دینامیک ملکولی برای غشاها‌ی جامد و سیال قادر به بررسی دامنه‌ی افت‌ و خیز‌های سه بعدی و مداری اجسام کروی و شبه کروی هستیم. سوال اول این است که آیا شدت افت و خیز‌های مدار استوایی اطلاعات کافی برای باز سازی مدل سه بعدی یک غشا را در اختیار ما قرار می‌دهد یا خیر. غشا‌های زیستی به ندرت شکل شبه کروی دارند، اما روش‌های اندازه‌گیری افت و خیز همگی بر اساس نظریه‌های افت و خیز اجسام کروی است. سول دوم این است که تا چه اندازه مطالعات افت و خیز مدار غشا برای اشکال غیر کروی اعتبار دارد.

چهار: به علت نبود مدل دینامیک ملکولی درشت دانه‌ی مناسب برای مطالعه‌ی دینامیک گلبول‌های قرمز، افت و خیز این سلول‌ها تنها با روش‌های آزمایشگاهی مطالعه شده‌است. پیشنهاد می‌شود که از روش دینامیک مش برای این مطالعه استفاده شود. 

پنج: تغییر شکل گلبول‌های قرمز در جریان‌های بُرشی مطالعه‌ی زیادی شده‌است. می‌توان از روش دینامیک مش  برای مطالعه‌ی مشابه رفتار تغییر شکل گلبول قرمز در این جریان‌ها و همچنین هنگام عبور از مجاری تنگ استفاده کرد. مطالعه‌ی مفصل تغییر شکل‌هایی که در نمودار
\ref{fig:deformationEvo}
بررسی شده‌است. 








