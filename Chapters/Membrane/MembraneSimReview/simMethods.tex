بیشتر اجزای تشکیل دهنده‌ی غشا ملکول‌های لیپیدی دو قطبی‌است. یک غشای زیستی معمولا از انواع مختلف لیپید‌ها ساخته شده‌است. سلول با تنظیم غلظت این ملکول‌ها می‌تواند مشخصاتی همچون انحنای ذاتی، سختی خمش غشا را کنترل کند. حالت‌هایی نیز وجود دارد که در بخش‌هایی از غشای سلول جدایی فاز اتفاق می‌افتد که سلول را برای ایجاد غشاهای جوانه شکل برای تبادل پروتئین با محیط آماده می‌کند. همچنین در غشاهای سلولی پروتئین‌ها و کانال‌های یونی زیادی وجود دارد که جابجایی آب، یون، و ملکول‌های کوچک را ممکن می‌سازد.


وِسیکِل‌ها
\LTRfootnote{vesicles}
 غشاهایی هستند که فقط از یک نوع ملکول‌ لپیدی ساخته ‌شده‌اند و یک سیال را درون خود بسته بندی کرده‌اند. در نتیجه وسیکل‌ها کاندید خوبی برای بررسی رفتارهای ساده‌ و مدل سازیی غشاهای زیستی هستند. با اینکه در اینجا ما از کلمه‌ی «ساده» برای توصیف غشاها استفاده کرده‌ایم، باید به خاطر داشته باشیم که این سیستم همچنان یک سیستم چند ذره‌ایست که برهمکنش‌های دوقطبی بین اجزای آن و سیال اطراف تغییر شکل‌  نسبتا پیچیده‌ و تغییر فاز ایجاد می‌کند که برای غشاهای با مشخصه‌های توپولیژیکی مختلف (غشاهای سوراخ‌دار) متفاوت است. در این بخش به معرفی روش‌های شبیه‌سازی مختلف که در مقیاس‌های مختلف انجام می‌شود می‌پردازیم. این مقیاس‌ها از مقیاس ملکلولی است که به مشخصات ملکول‌های بیپیدی و پروتئین‌ها می‌پردازد شروع شده، تا مقیاس‌های اَبَرملکولی
 \LTRfootnote{super molecule scale}
 که معمولا به رفتار خودساماندهی شده و رفتار فازی لیپید‌ها می‌پردازد، تا مقیاس وسیکل که به شکل، تغییر شکل، اثر جدایی فاز در غشا و سیال داخل آن، و همچنین تغییر شکل وسیکل ناشی از نیروهای خارجی مطالعه می‌شود. شبیه‌سازی هم نقش مهمی در مطالعه سیستم‌های پیچیده دارد. گلبول‌های قزمز مثال بسیار مهمی‌ است زیرا که غشای آن از اتصال یک غشای لیپیدی به یک شبکه‌ی پلیمری متصل است که با اسکلت سلولی در ارتباط است. در نتیجه این غشاها بر خلاف غشاهای لیپیدی مدول بُرشی
 \LTRfootnote{shear modulus}
 دارند. همچنین گلبول‌های قرمز در کنار افت و خیز ترمودینامیکی محیط، افت و خیز فعّال نیز دارند.
 
 
 فیزیک غشا‌ها و وسیکل‌ها رفتار‌های مقیاسی و زمانی مختلفی را شامل می‌شود. از مقیاس چند آنگستروم که رفتارهای کوانتم مکانیکی تک ملکولی و پیوندهای هیدروژنی میان آن را شامل می‌شود، تا رفتار نانومتری که شامل رفتار دولایه‌ی لیپیدی می‌شود تا مقیاس چند صد نانومتر (وسیکل‌های کوچک) تا ۱۰ میکرومتر که شامل رفتار وسیکل‌های بزرگ و برهمکنش‌های هیدرودینامیکی آن‌هاست.
 
 
 \subsection{
 مدل‌های اتمی
 }
 
\begin{figure}[h]
\begin{center}
\includegraphics[width=4.5in]{\MemSimRev /Pics/allAtom}
\caption{
یک لحظه از یک شبیه‌سازی تمام-اتم به روش دینامیک ملکولی را نشان می‌دهد. در این شبیه‌سازی
$128$
ملکول لیپیدی
$DPPC$
در کنار
$3910$
ملکول آب نمایش داده شده‌است. ملکول‌های آب به شکل استوانه‌های نازک و اتم‌های ملکول‌های لیپیدی به شکل کُره‌ نمایش داده شده‌است
\cite{Tieleman1997}.
}
\label{fig:allAtom}
\end{center}
\end{figure}

برای شبیه‌سازی در مقیاس‌های چند آنگسترم تا چند نانومتر، از روش‌های شبیه‌سازی‌های تمام-اتُم
 \LTRfootnote{all atom}
استفاده می‌شود که در آن محل اتُم‌ها و نوع برهمکنش‌ میان آن‌ها به طور صریح در نظر گرفته می‌شود. برخی اوقات برهمکنش‌های میان اتم‌ها با محاسبات کوانتم مکانیکی اندازه‌گیری می‌شود ولی اکثر اوقات از میدان‌های نیروی کلاسیک استفاده می‌شود. شبیه‌سازی تمام-اتم هنگامی که ساختار شیمیایی نقش داشته باشد اجتناب ناپذیر است. مثلا نحوه‌ی پُمپ شدن یک یون توسط یک پروتئین در غشا تنها از طریق چنین روش‌های شبیه‌سازی قابل فهم خواهد بود. البته باید در نظر گرفت که با توان محاسباتی کامپیوتر‌های فعلی، تنها شبیه‌سازی تمام-اتم چند هزار ملکلول لیپید ممکن است . در شکل 
\ref{fig:allAtom}
نمونه‌ای از شبیه‌سازی ملکول‌های لیپید
$DPPC$
و ملکول‌های آب را نشان می‌دهد. 



 \subsection{
 مدل‌های درشت دانه‌ی غشا
 }
 \begin{figure}[h]
\begin{center}
\includegraphics[width=4.5in]{\MemSimRev /Pics/coarseGrained}
\caption{
مدل درشت دانه‌ی غشا که ملکول‌های آب با ذرات قرمز، سر آب دوست ملکول‌های لیپیدی با ذرات سفید و دُم آبگریز آن با ذرات آبی نمایش داده شده‌است.
\cite{AlGhoul2004}.
}
\label{fig:allAtom}
\end{center}
\end{figure}
 مدل‌های درشت دانه‌ی غشا
\LTRfootnote{coarse-grained membrane models}
زمانی استفاده می‌شود که جزئیات ساختار شیمیایی ملکلول‌ها اهمیت نداشته باشد و تنها خواص کلی ملکلول‌های دو قطبی (مانند تعداد دُم‌های هیدروکربنی که هر لیپید دارد، طول دُم هر لیپید یا ترکیبی از چند نوع ملکول لیپیدی) اهمیت دارد. در شبیه‌سازی درشت‌ دانه اثر چندین اتُم با یک ذره‌ جایگزین می‌شود. معمولا نحوه‌ی برهمکنش این ذرات درشت دانه با پتانسیل لنارد-جونز
\LTRfootnote{Lennard-Jones}
است. در این نوع روش شبیه‌ سازی یک تک ذره‌ی لنارد-جونز با برهمکنش جاذب نماینده‌ی هر ملکول آب خوهد بود. همچنین ملکلو‌ل‌های دو قطبی با زنجیری از ذراتی وارد شبیه‌سازی می‌شوند که یا با ذرات آب برهمکنش جاذب دارند یا برهم کنش دافع
\cite{Goetz1998, Goetz1999, Otter2003}.
به غیر از ذرات لنارد-جونز، شبیه‌سازی‌های درشت دانه با ذرات با برهمکنش نَرم هم وجود دارد که اغلب در شبیه‌سازی به روش دینامیک پخشی ذرات 
\LTRfootnote{dissipative particle dynamics (DPD)}
(زیرمجموعه شبیه‌سازی دینامیک ملکولی) استفاده می‌شود
\cite{Shillcock2002, Laradji2004PRL, Ortiz2005}.


البته مدل‌های درشت‌ دانه‌ای هم وجود دارد که علاوه بر برهمکنش لنارد-جونز، برهمکنش‌های الکتریکی هم در نظر می‌گیرد
\cite{Marrink2003, Marrink2007}
 (تا تاثیر ساختار شیمیایی نیز شامل شود). البته به علت ماهیت درشت دانه‌ی مسئله با چنین مدل‌هایی رفتار کیفی غشا مورد بررسی قرار می‌گیرد. ولی دلیل استفاده از این مدل بررسی بهتر پدیده‌هایی است که در مقیاس چند هزار ملکلول لیپیدی است، مانند نحوه‌ی تشکیل، ساختار، و دینامیک وسیکل‌های فسفولیپیدی کوچک
 \cite{Marrink2003,Marrink2009}.
\subsection{
 مدل‌های غشای بدون حلال
 }
 \begin{figure}[h]
\begin{center}
\includegraphics[width=4.5in]{\MemSimRev /Pics/solventFree}
\caption{
استفاده از مدل‌ غشای بدون حلال برای بررسی جوش دو غشای لیپیدی ساخته شده از
$500$
ذره. سر آب دوست با رنگ قرمز و سر آب گریز با رنگ زرد نمایش داده شده‌است 
\cite{Noguchi2001}.
}
\label{fig:solventFree}
\end{center}
\end{figure}

 در مدل‌های درشت دانه، نقش حلال به دو دلیل بسیار مهم است. اول اینکه به علت برهم کنش دافعه‌ای که با دُم آبگریز ملکلول‌های دو قطبی دارد، ساختار دو لایه را پایدار می‌کند. در درجه‌ی دوم بستر برهمکنش هیدرودینامیک غشا را فراهم می‌کند. ولی شبیه‌سازی قسمت ذرات حلال کسر بسیار بزرگی از زمان شبیه‌سازی را به خود اختصاص می‌دهد. در نتیجه برای بررسی ساختار و خواص ترمودینامیکی غشاها، مدل‌های غشای بدون حلال
 \LTRfootnote{solvent-free membrane models}
 طراحی شده‌اند (شکل
 \ref{fig:solventFree}
 ). در این مدل‌های برهمکنش‌های جدید برای ذرات تعریف می‌شود تا تاثیر حلال را جایگزین کند
 \cite{Noguchi2001, Noguchi2001PRE, Brannigan2003, Cooke2005}
 . با اینکه استفاده از این روش شبیه‌سازی شما را به مقیاس ملکول لیپید محدود خواهد کرد، ولی از آنجایی که تعداد ذراتی که باید شبیه‌سازی شود را یک مرتبه‌ی بزرگی کاهش می‌دهد، بسیار جذاب است. از طرفی دیگر از آنجایی که  ذرات درشت دانه همچنان هر ملکول لیپید را به صورت مجزا بیان می‌کنند، مقیاس شبیه‌سازی همچنان در مقیاس ملکول‌های دوقطبی خواهد بود.

\subsection{
 مدل غشا بر شبکه‌ی مثلثی
 }
  \begin{figure}[h]
\begin{center}
\includegraphics[width=4.5in]{\MemSimRev /Pics/snapShotstomato_disco}
\caption{
نمونه‌ای از شبکه‌های مثلثی استفاده شده برای مدل‌کردن رفتار گلبول‌های قرمز.
}
\label{fig:RBCmeshRep}
\end{center}
\end{figure}
 
روش‌هایی که تا به اینجا معرفی شد غشا را در مقیاس 
$1nm$
ملکول لیپیدی تشکیل دهنده‌ی آن شبیه‌سازی می‌کنند. این مقیاس برای شبیه‌سازی غشایی که قطر آن حدود چند میکرون است از لحاظ منابع کامپیوتری لازم بسیار پر هزینه خواهد بود. رفتار چنین غشا‌هایی با استفاده از مدل‌های پیوسته سطوح الاستیک به خوبی قابل توصیف است. 

برای پیاد‌ه سازی مدل‌های الاستیک پیوسته در شبیه‌سازی معمولا از شبکه‌های مثلثی استفاده می‌شود 
\cite{Gompper1997, NelsonBook2004}.
در این نمایش  غشا با توزیع تقریبا یکنواختی از نقاط نمایش داده می‌شود. با اتصال نقاط به همسایه‌های نزدیکشان شبکه‌ی مثلثی تشکیل خواهد شد. سطح هر مثلث نماینده‌ی صدها و گاهی هزاران ملکول لیپید خواهد بود. به علت محدودیت در تعریف مدل خمش بر روی این سطوح، همیشه به توزیع یکنواختی از نقاط نیاز است. برای شبیه‌سازی غشاهای سیال گون  از الگوریتم مثلث‌بندی دینامیکی
\LTRfootnote{dynamic triangulation}
\cite{Boal1992PRA, Gompper1992Science}
برای تغییر شبکه‌بندی در طول شبیه‌سازی استفاده می‌شود. حاصل استفاده از این الگوریتم حرکت پخشی
\LTRfootnote{diffusion}
 نقاط بر روی سطح با حفظ توزیع یکسانی از شکل و اندازه‌ی مثلث‌های روی سطح است. برای شبیه‌سازی غشاهای پلیمری مانند کپسیدها
 \LTRfootnote{capsids}
 یا شبکه‌ی پلیمری گلبول‌های قرمز از شبکه‌های مثلثی با مثلث بندی ثابت استفاده می‌شود (شکل 
 \ref{fig:RBCmeshRep}
).

\subsection{
 مدل غشا بدون شبکه‌
 }
\begin{figure}[h]
\begin{center}
\includegraphics[width=5.5in]{\MemSimRev /Pics/meshless}
\caption{
نمونه‌ای از غشای تشکیل شده در مدل بدون شبکه. به ترتیب از بالا به پایین شاهد تشکیل غشا با حفره، غشای بدون تنش، و غشای خم شده هستیم
\cite{Noguchi2006PRE}.
}
\label{fig:meshless}
\end{center}
\end{figure}
در این روش شبیه‌سازی غشا با مجموعه‌ای از نقاط مدل می‌شود که هیچ شبکه‌ای برا آنها تعریف نشده است. در این روش برهمکنش‌های دوتایی دافع و جاذب میان ذرات تعریف می‌شود تا ذرات در فاصله‌ی تعادلی نسبت به یکدیگر قرار گیرند و رفتار سیال گون نشان دهند. سپس با روش‌هایی همچون کمینه-مربع متحرک
\LTRfootnote{moving least-square}
با برازش
\LTRfootnote{fit}
 موضعی سطح به مختصات نقاط انرژی خمش چیدمان تعریف می‌شود. هدف این روش‌ها ایجاد  توزیع کم و بیش یکنواخت نقاط بر سطح و همچنین و اینجاد سطوح با خمش پیوسته
\LTRfootnote{smooth surface}
است
\cite{Noguchi2006PRE}.
بَرتَری این روش در شبیه‌سازی سطوح با شرایط مرزی باز و همچنین شبیه‌سازی تغییر شکل‌های حاصل از تغییر توپولوژی غشاست.








