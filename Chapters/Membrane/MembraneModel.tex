\setRL
\clearpage
\def \MemModel {\Mempath /MembraneModel}

\section{
چکیده
}
در بخش
\ref{sec:cellmembrane}
مروری کوتاهی بر تارخچه‌ی تحقیقات انجام شده در جهت کشف و بررسی ساختار غشای سلول‌های زنده در قرن‌های ۱۹ و ۲۰ میلادی شده‌است. در بخش
\ref{sec:nuclearenvelope}
ساختار بسته‌ی هسته به صورت مختصر توضیح داده شده‌است.

\section{
مقدمه
}
برای مدل‌سازی غشاها ابتدا پوسته‌های نازک را بررسی خواهیم کرد. توصیف پوسته‌های نازک را با بررسی انرژی خمشی و انرژی کششی یک محیط پیوسته آغاز خواهیم کرد. پس حل معادلات پیوسته، این معادلات را برای مجموعه نقاط در فضا که بر روی شبکه‌ی مثلثی قرار دارند حل می‌کنیم. سپس تغییرات انرژی در صورت ایجاد نقاط نقص بر روی شبکه را محاسبه خواهیم کرد. 


مقاله الکساندرا مرجع ۷ در مقده تيوری خیلی خوب راجع به آنالیز فرکانس نوشته


\begin{figure}[h]
\begin{center}
\includegraphics[width=4in]{\Mempath/Pics/Membrane_fluctuations}
\caption{
مجموعه تصاویر پست سر هم از تغییر شکل یک غشای لیپیدی را با تصویر برداری فلورسانت در بازه‌های ۵ ثانیه‌ای نشان می‌دهد. خط مقیاس سفید رنگ اندازه‌ی ۵ میکرومتر را نشان می‌دهد. 
\cite{ParthasarathyMembraneMeasurement}
}
\label{fig:flucmem}
\end{center}
\end{figure}

شگل 
\ref{fig:flucmem}
تغییر شکل یک غشای لیپیدی غول آسا (قطر حدود ۱۰ میکرون) در بازه‌های زمانی ۵ ثانیه نشان می‌دهد
\cite{ParthasarathyMembraneMeasurement}
. از نظر انرژی تغییر شکل این غشا را می‌توان به دو بخش کلی تقسیم کرد. تغییر انرژی ناشی از خم شدن و تغییر حاصل از کشش سطح غشا. شکل
\ref{fig:elasticdeformation}
الف، تغییر شکل یک عنصر سطحی بر اثر خمش را نشان می‌دهد. خمش سطح را می‌توان با اندازه‌ی شعاع دو دایره که بر عنصر سطح مماس هست، توصیف کرد. همچنین شکل 
\ref{fig:elasticdeformation}
ب، تغییر شکل عنصر سطح به علت ایجاد کشش در سطح نشان می‌دهد. تغییر سطح با اختلاف مساحت عنصر سطح با حالت کشیده نشده توصیف می‌شود.
\begin{figure}[h]
\begin{center}
\includegraphics[width=4in]{\Mempath/Pics/surface elemnts.pages.pdf}
\caption{
تغییر شکل عنصر سطحی بر اثر الف، خمش و ب، کشش.
}
\label{fig:elasticdeformation}
\end{center}
\end{figure}


\section{
انرژی آزاد غشا
}
\setRL
%\pagenumbering{arabic} 


\subsection{
انرژی کشش در سطح
}
اگر فرض کنیم جابجایی روی یک عنصر سطحی حاصل از کشیده‌ یا فشرده شدن سطح با بردار 
$u$
توصیف شود، با فرض خطی بودن عکس العمل ماده، انرژی پتانسیل حاصل از تغییر شکل سطح را می‌توان با معادله‌ی زیر بررسی کنیم.

\begin{equation}
E_{stretching}=\frac{1}{2}Y_{2D}A\varepsilon^2
\end{equation}
که اینجا 
$Y_2D$
مدول دو بعدی یانگ،
$A$
سطح عنصر در حالت کشیده نشده، و
$\varepsilon$
تانسور کرنش است. تانسور کرنش برای سطح دو بعدی به شکل زیر تعریف می‌شود:
\begin{equation}
\varepsilon_{ij} = \frac{1}{2}(u_{ij}+u_{ji})
\end{equation}


.
 
 
 
 
 
 
 
 
 
 
 
 
 
 
 
\setRL
%\pagenumbering{arabic} 


\subsection{
انرژی خمش سطح
}
انرژی خمش یک سطح را می‌توان با انرژي هلفریش
\cite{Helfrich1973}
 کمی کرد،
\begin{equation}
E_{bending}=\int dS\left\{\frac{1}{2}\kappa (H-H_s)^2 +\tilde \kappa K_0\right\}
\label{eq:helfrish}
\end{equation}
در اینجا
\begin{equation}
H = \frac{1}{R_1}+\frac{1}{R_2}
\end{equation}
خمش سطح است که با شعاع دو دایره‌ی مماس بر عنصر سطح بیان می‌شوند (
\ref{fig:elasticdeformation}
). 
$H_s$
خمش زاتی سطح را مشخص می‌کند که همانند خمش،
$H$
تعریف می‌شود. برای مثال خمش زاتی سطحی که در تمام جهت‌ها علاقه دارد شعاع 
$R_s$
داشته باشد، 
\begin{equation}
H_s = \frac{2}{R_s}
\end{equation}
است. 
$K_0$
خمش گاووسی است که به شکل 
\begin{equation}
H_s = \frac{2}{R_s}
\end{equation}
تعریف می‌شود. همچنین 
$\kappa$
و
$\tilde\kappa$
به ترتیب سختی خمشی و سختی خمش گاووسی است. بنا به قضیه گاووس-بونت
\LTRfootnote{Gauss–Bonnet}
انتگرال روی سطح خمش گاووسی پاسخی ساده دارد،
\begin{equation}
\int dS \tilde \kappa K_0=4\pi\tilde\kappa(1-g)
\end{equation}
که در معادله‌ی بالا 
$g$
جینوس 
\LTRfootnote{genus}
سطح، یا تعداد سوراخ یا تعداد دسته‌
\LTRfootnote{handle}
است. اگر پوسته‌ی مورد نظر در طول مطالعه تغییر توپولوژی ندهد حاصل این انتگرال همیشه ‌یک عدد ثابت خواهد بود. در صورتی علاقه‌ی ما محاسبه‌ی نیرو‌های خمشی (مشتق جمله انرژي) یا اختلاف انرژی خمشی باشد، جمله‌ی ثابت خمش گاووسی در محاسبات اهمیت نخواهد داشت.
\subsubsection{
محاسبه‌ی انرژی خمش کره
}
برای مثال انرژی خمش یک کره به شعاع 
$R$
را با رابطه‌ی هلفریش محاسبه می‌کنیم. معادله‌ی 
\ref{eq:helfrish}
به شکل زیر در می‌آید:
\begin{equation}
\begin{aligned}
E_{bending}&=\int dS\left\{2\kappa \left(\frac{1}{R}-\frac{1}{R_s}\right)^2 +\tilde \kappa K_0\right\} \\
&=2\kappa\int \left(\frac{1}{R}-\frac{1}{R_s}\right)^2dS +4\pi\tilde \kappa
\end{aligned}
\end{equation}
در صورتی که کره را از ماده‌ای ساخته باشیم که به طور ذاتی علاقه داشته باشد که یک سطح تخت باشد،‌
$R_s\rightarrow\infty$
انرژی خمش مقدار ثابت خواهد بود:
\begin{equation}
E_{bending}|_{R_s\rightarrow\infty}=4\pi(2\kappa+\tilde\kappa)
\end{equation} 


.
 
 
 
 
 
 
 
 
 
 
 
 
 
 
 

\section{
انرژی آزاد یک شبکه‌ی مثلثی
}
%\setRL
%\pagenumbering{arabic} 

\subsection{
انرژی آزاد کشش
}

در نظریه‌ی الاستیک سطح هر تغییر شکل با یک میدان بردار جابجایی 
$u(r)=(u_1,u_2)$
نشان داده می‌شود نقطه‌ی 
$r(x,y)$
را به نقطه‌ی 
$r+u$
نگاشت می‌کند. اگر در شبکه نقص وجود نداشته باشد این نگاشت یک به یک خواهد بود. در صورتی که فرض کنیم که ماده مورد مطالعه یکنواخت و همسانگرد است، برای جابجایی‌های کوچک (رژیم خطی) قانون هوک را به شکل توان دوم تانسور کرنش نوشت
\LTRfootnote{Cauchy, 1822; Lam ́e, 1852}
،
\begin{equation}
E_s=\frac{1}{2}\int d^2r(2\mu u_{ij}^2+\lambda u_{kk}^2)
\label{eq:energylame}
\end{equation}
که در اینجا $\lambda$
و $\mu$
ثابت‌های لم
\LTRfootnote{Lamé Coefficients}
است. ما می‌دانیم که تانسور کرنش به شکل زیر تعریف می‌شود،
\begin{equation}
u_{ij}=\frac{1}{2}(\partial_i u_j+\partial_j u_i+\partial_i u_k\partial_j u_k)
\end{equation}
اما برای جابجایی کوچک از جمله‌ی غیر خطی صرف نظر می‌کنیم و تانسور کرنش را به این شکل تعریف می‌کنیم.
\begin{equation}
u_{ij}=\frac{1}{2}(\partial_i u_j+\partial_j u_i)
\label{eq:simplestrain}
\end{equation}
می‌توانیم  از انرژی کششی گرادیان بگیریم و مقدار کمینه‌ی آن را بررسی کنیم، در نتیجه
\begin{equation}
\begin{aligned}
&\partial_i\sigma_{ij}=0\\
&\sigma_{ij}=2\mu u_{ij}+\lambda u_{kk}\delta_{ij}
\label{eq:stress}
\end{aligned}
\end{equation}
که در این معادله 
$\sigma_{ij}$
تانسور تنش است. معادله‌ی 
\ref{eq:stress}
را به تنهایی می‌توان حل کرد ولی از آنجایی که دیورژانس تنش صفر است معمول است که این معادله را به شکل یک پتانسیل اسکالر بنویسیم،
\begin{equation}
\sigma_{xx}=\frac{\partial^2\chi}{\partial y^2},\quad\sigma_{yy}=\frac{\partial^2\chi}{\partial x^2},\quad\sigma_{xy}=\frac{\partial^2\chi}{\partial_x\partial_y} 
\end{equation}
انتخاب‌های خیلی زیادی می‌توانند معادله‌ی بالا را ارضاء خواهد کرد، ولی جواب‌هایی که به لحاظ فیزیک قابل قبول هستند باید بتوانند رابطه‌ی بین میدان جابجایی و 
$\chi$
را رعایت کنند،
\begin{equation}
\begin{aligned}
\frac{1}{2}(\partial_iu_j+\partial_ju_i)&=u_{ij}\\
&=\frac{1+\nu}{Y}\sigma_{ij}-\frac{\nu}{Y}\sigma_{ll}\sigma_{ij}\\
&=\frac{1+\nu}{Y}\epsilon_{im}\epsilon_{jn}\partial_{m}\partial_{n}\chi-\frac{\nu}{Y}\nabla^2\chi\delta_{ij}
\label{eq:constraint}
\end{aligned}
\end{equation}
در اینجا $Y$
و $\nu$
به ترتیب مدول ۲ بعدی یانگ
\LTRfootnote{2D Young Modulus}
 و نسبت پواسون
\LTRfootnote{Poisson ratio}
است که بر حسب ضرایب لم به شکل زیر بیان می‌شوند،
\begin{equation}
\begin{aligned}
Y&=\frac{4\mu(\mu+\lambda)}{2\mu+\lambda}\\
\nu&=\frac{\lambda}{2\mu+\lambda}
\label{eq:younglame}
\end{aligned}
\end{equation}
فرض می‌کنیم که شبکه‌ای را بررسی می‌کنیم که فاصله‌ی متوسط بین تمام نقاط به اندازه‌ی $a$ باشد. هر گونه تغییر شکل در شبکه یک نقطه از شبکه را از $r_a$ به $r_a'$ جابجا خواهد کرد. در نتیجه می‌توان انرژی کشش را به شکل زیر تعریف کرد (شکل
\ref{fig:mesh_def}
).
\begin{figure}[h]
\begin{center}
\includegraphics[width=6in]{\MemModel/Pics/mesh_def.pages.pdf}
\caption{
تغییر شکل مش
}
\label{fig:mesh_def}
\end{center}
\end{figure}

\begin{equation}
E_s^{discrete}=\frac{1}{2}\epsilon_s\sum_{\langle a,b\rangle}\left(|r_a'-r_b'|-a\right)^2
\label{eq:stretchdiscrete}
\end{equation}
که جمع روی تمام جفت‌های $a$ و $b$ است که شامل تغییر شکل شده‌اند.  همچنین می‌توان جمع بالا را به شکل چگالی موضعی انرژی حول نقاط شبکه و جمع روی همسایگی‌ آن نقاط تعریف کرد،

\begin{equation}
\begin{aligned}
&E_s^{discrete}=\frac{1}{2}\epsilon_s\sum_aU_a\\
&U_a=\frac{1}{2}\sum_b\left(|r_a'-r_b'|-a\right)^2
\end{aligned}
\end{equation}
برای محاسبه‌ی حد پیوستگی فرض می‌کنیم که نقشه‌‌ی تغییر شکل پیوسته‌ای وجود دارد که نقاط 
$r\rightarrow r'$
که معادل نقشه‌ی گسسته‌ی شبکه‌ی ماست
$r_a\rightarrow r_a'=r_a+u_a(r_a)$
. اگر تانسور متریک این تغییر شکل به شکل زیر تعریف شده باشد،
\begin{equation}
g_{ij}=\partial_i r'\cdot\partial_jr'
\end{equation}
در نتیجه می‌توانیم تغییر شکل گسسته را به شکل زیر تخمین بزنیم،

\begin{equation}
\begin{aligned}
|r_a'-r_b'|&\approx \left[g_{ij}(r_a)r_{ab}^ir_{ab}^j\right]^{1/2}\\
&= \left[g_{ij}(r_a)r_{ab}^ir_{ab}^j\right]^{1/2}\\
&= \left\{\left[\delta_{ij}+2u_{ij}(r_a)\right]r_{ab}^ir_{ab}^j\right\}^{1/2}\\
&= a\left[1+2u_{ij}(r_a)\frac{r_{ab}^ir_{ab}^j}{a^2}\right]^\frac{1}{2}\\
&\approx a\left[1+u_{ij}(r_a)\frac{r_{ab}^ir_{ab}^j}{a^2}\right]
\label{eq:gstrain1}
\end{aligned}
\end{equation}
که در رابطه‌ی بالا تانسور متریک را با تانسور تنش جاگذاری کردیم،
$g_{ij}=\delta_{ij}+2u_{ij}$
از آنجایی که اندیس $b$ بین تمامی همسایه‌ی $a$ تعریف می‌شود و همچنین بردار فاصله‌
$r_{ab}=r_a-r_b$
روی بردار‌های
$d_\beta$
 شبکه‌ی شش ضلعی تعریف می‌شود می‌توانیم انرژی موضعی را به این ترتیب محاسبه‌ کنیم،

\begin{equation}
\begin{aligned}
U_a&=\frac{1}{2}\sum_{\beta=1}^6(u_{ij}\frac{d_\beta^id_\beta^j}{a})^2\\
&=\frac{1}{2a^2}\sum_{\beta=1}^6u_{ij}u_{kl}d_\beta^id_\beta^jd_\beta^kd_\beta^l\\
&=\frac{1}{2a^2}a^2u_{ij}u_{kl}(\delta_{ij}\delta_{kl}+\delta_{ik}\delta_{jl}+\delta_{il}\delta_{jk})\cos^2(\pi/3)\\
&=\frac{3}{8}(2u_{ij}^2+u_{kk}^2)
\label{eq:gstrain1}
\end{aligned}
\end{equation}
در نتیجه‌ حد پیوسته انرژی کشسانی را می‌توان به شکل زیر نوشت
\begin{equation}
\begin{aligned}
E_s^{discrete}=\frac{1}{2}\epsilon_s\sum_\alpha U_a&\approx\frac{1}{\sqrt3}\epsilon\int d^2rU(r)\\
&\approx\frac{\sqrt3}{8}\epsilon_s\int d^2r(2u_{ij}^2+u_{kk}^2)
\end{aligned}
\end{equation}
با مقایسه با معادله‌ی 
\ref{eq:energylame}
می‌توانیم ضرایب لم را بخوانیم
\begin{equation}
\lambda=\mu=\frac{\sqrt3}{4}\epsilon_s
\end{equation}
با داشتن ضرایب لم می‌توانیم با توجه به معادله‌ی 
\ref{eq:younglame}
مدول ۲ بعدی یانگ و ضریب پواسون را برای این شبکه محاسبه کنیم،
\begin{equation}
\begin{aligned}
Y&=\frac{4\mu(\mu+\lambda)}{2\mu+\lambda}=\frac{2}{\sqrt3}\epsilon_s\\
\nu&=\frac{\lambda}{2\mu+\lambda}=\frac{1}{3}
\end{aligned}
\end{equation}
همانطور که می‌بینیم برای مش‌های مثلثی ۶ ضلعی، مدول یانگ و نسبت پواسون به اندازه‌ی مش بستگی ندارد. محاسبات عددی
\cite{springnetworkPRE2011}
نیز این نتایج را تایید می‌کنند.




.
 
 
 
 
 
 
 
 
 
 
 
 
 
 
 
%\setRL
%\pagenumbering{arabic} 



\subsection{
انرژی انحنای متوسط
}


انحنای متوسط یک ناحیه روی مِش 
$H=C_1+C_2$
جمع خمش‌های اصلی در آن نقطه است. در این صورت می‌توان انرژی انحنا را با جمع انحنای میانگین در هر نقطه تعریف کرد،
\begin{eqnarray}
E_{b}=\frac{1}{2}\kappa\int dA \left[H-C_0\right]^2\equiv\frac{1}{2}\kappa\sum_i a_i \left[H_i-C_0\right]^2.
\label{eq:bendingDiscretisation}
\end{eqnarray}
در اینجا 
$H_i$,
انحنای متوسط در هر نقطه  
$v_i$
است، 
$C_0$
عکس شعاع انحنا،  و 
$a_i$
سهم مساحتی است که هر نقطه روی سطح دارد. شعاع انحنای متوسط در هر نقطه را می‌توان بر اساس مساحت هر نقطه به شکل 
$H_i=\frac{h_i}{a_i}$
 در نظر گرفت، و معادله‌ی بالا را بازنویسی کرد،
\begin{eqnarray}
\begin{aligned}
E_{b}&=\frac{1}{2}\kappa\sum_i a_i \left[\frac{h_i}{a_i}-C_0\right]^2\\
&=\frac{1}{2}\kappa\sum_i a_i \left[\frac{h_i^2}{a_i^2}-2\frac{h_i}{a_i}C_0+C_0^2\right]\\
&=\frac{1}{2}\kappa\sum_i \left[\frac{h_i^2}{a_i}-2h_iC_0+a_iC_0^2\right]
\end{aligned}
\label{eq:bendingDiscretisationSpontaneous}
\end{eqnarray}
در صورتی که انحنای ذاتی برابر صفر باشد، 
\begin{equation}
E_{b}=\frac{1}{2}\kappa\sum_i \frac{h_i^2}{a_i}
\end{equation}


\subsubsection{
روش 
}
در بخش قبل نحوه‌ی محاسبه‌ی انرژی انحنا به روش دو سطحی\LTRfootnote{dihedral}
معرفی شد. این روش در اصل توسط نلسون و کانتور
\cite{NelsonPRL1987}
معرفی شده بود. گامپر و کرول در سال ۱۹۹۶ به طور مفصل این روش را نقد کرده‌اند
\cite{gompper1996}.
 این روش مشکلات زیادی دارد که در واقع انحنای شکل را غلط پیشبینی می‌کند. مشکل اساسی این است که رابطه‌ی 
$\epsilon_b$
و 
$\kappa$
(معادله‌ی  
\ref{eq:HelfrichCurvatureEnergy})
 تابع شکل سطح است.  مثلا برای کره
$\epsilon_b\approx\frac{\sqrt{3}}{2}\kappa$
و برای استوانه
$\epsilon_b\approx\sqrt{3}\kappa$
است. پس نمی‌توان از این رابطه برای محاسبه‌ی سطحی که در حال تغییر شکل است و یا شکل خوش تعریفی ندارد استفاده کرد. 
از طرف دیگر از آنجایی که این انرژی تنها میان یک جفت مثلث تعریف می‌شود و از هندسه‌ی اطرافش بی‌خبر است، نمی‌تواند انرژی نقاط زین اسبی را به درستی محاسبه کند. و در نهایت اندازه‌ی مثلث‌ها در اندازه‌ی انحنا نقشی ندارند. این نکته از طرفی مهم است زیرا انرژی انحنا مستقل از اندازه‌ی هندسی شکل است ولی در حالتی که مش مثلث‌های با اندازه‌های مختلف داشته باشد، یک جفت مثلث غول‌آسا و یک جفت مثلث ریز به یک میزان انرژی انحنا خواهند داشت.

\begin{figure}[h]
\begin{center}
\includegraphics[width=4.5in]{\MemDiscr /Pics/tringlePairBoth}
\caption{
سمت چپ زاویه‌های 
$\theta_1^{ij}$
و
$\theta_2^{ij}$
را نشان می‌دهد که زاویه‌هایی است که در شبکه‌ی دوگان به ضلع
$\ell_{ij}$
نسبت داده می‌شود
\cite{Meyer2003}.
 سمت راست جفت مثلثی همراه بردار‌های عمود بر سطوح آن،
$n_\alpha$
و
$n_\beta$
و زاویه‌ی دوسطحی میان آن دو
$\phi_{ij}$
نمایش داده شده ‌است.
}
\label{fig:trianglePairAngle}
\end{center}
\end{figure}

گامپر\LTRfootnote{Itzykson}
در سال ۱۹۸۶ لاپلاسین میدان اسکالر بر روی یک شبکه‌ی مثلثی تصادفی را ب محاسبه کرد
\cite{Itzykson1986}.
از طرفی طبق هندسه‌ی دیفرانسیلی  انحنای متوسط در هر نقطه 
$\vec r$
که بردار عمود بر سطح 
$\vec n$
را دارد به شکل 
$H=\vec n\cdot\Delta \vec R$
تعریف می‌شود 
\cite{gompper1996}
و 
$\Delta$
عملگر لاپلاس-بلترامی\LTRfootnote{Laplace–Beltrami}
است. 


گامپر و کرول در سالت ۱۹۹۲ از رابطه‌ی ایتزیکسون برای محاسبه‌ی انحنا بر روی یک شبکه‌ی مثلثی استفاده کردند،
\begin{eqnarray}
E_{b}^\text{GK}=\frac{1}{2}\kappa\sum_{i}\frac{1}{\sigma_i}\left[\sum_{j(i)}\frac{\tilde\ell_{ji}}{\ell_{ij}}(\vec r_i-\vec r_j)\right]^2.
\label{eq:ItzyksonPotential}
\end{eqnarray}

در اینجا 
$\ell_{ij}$
طول ضلع تعریف شده میان نقاط 
$i$
و
$j$
است، 
$\vec r_i$
و
$\vec r_j$
بردار‌های مکان نمای این دو نقطه‌ است (شکل
\ref{fig:trianglePairAngle}
). 
$\tilde\ell_{ij}$
طول ضلع 
$\ell_{ij}$
در شبکه‌ی دوگانه‌\LTRfootnote{dual lattice}
است و با استفاده از زوایا‌ی روبرو آن به شکل 
\begin{eqnarray}
\tilde\ell_{ij}=\frac{1}{2}\ell_{ij}(\cot\theta_1^{ij}+\cot\theta_2^{ij})
\label{eq:dualLattice}
\end{eqnarray}
تعریف می‌شود.

\begin{figure}[htbp]
\begin{center}
\includegraphics[width=9cm]{\MemDiscr /Pics/Voronoi_Barycentric}

\caption{
سمت چپ مساحت بریسنتریک (مرکز جرمی) و سمت راست مساحت وُرُنُوی برای یک پلاکت را نمایش می‌دهد.
}
\label{fig:voronoiBarycentric}
\end{center}
\end{figure}
مساحت وُرُنُوی\LTRfootnote{Voronoi}
یک نقطه به اندیس 
$i$
با استفاده از طول اضلاع در شبکه‌ی دوگانی قابل محاسبه است
\begin{eqnarray}
\sigma_i=\frac{1}{4}\sum_{j(i)}\tilde\ell_{ij}\ell_{ij}.
\label{eq:voronoiArea}
\end{eqnarray}
. در معادله‌ی بالا جمع روی تمام اندیس‌های همسایه‌ی نقطه‌ی 
$i$
است. با توجه به این تعاریف در هر نقطه می‌توان انحنا را به شکل زیر تعریف کرد،
\begin{eqnarray}
H_i=\vec n\cdot\Delta \vec r\equiv\frac{1}{\sigma_i}\vec n \cdot\left[\frac{\sum_{j(i)}\tilde\ell_{ji}}{\ell_{ij}}(\vec r_i-\vec r_j)\right],
\label{eq:meanCurvatureDiscreteSingleVertex}
\end{eqnarray}
. تعریف بردار عمود در هر نقطه به شکل زیر تعریف می‌شود
\cite{Thurrner1998NormalVec}
\begin{eqnarray}
\vec n_i=\frac{\sum_{tri(i)} \eta_{tri}^i~\vec n_{tri}^i}{|\sum_{tri(i)} \eta_{tri}^i~\vec n_{tri}^i|},
\label{eq:noramlVector}
\end{eqnarray}
که جمع روی تمام مثلث‌های عضو پلاکت\LTRfootnote{plaqette}
است (تمام مثلث‌هایی که نقطه‌ی 
$i$
بین آنها مشترک است). 
$\eta_{tri}^i$
و
$\eta_{tri}^i$
به ترتیب زاویه‌ی راس مثلث در نقطه‌ی 
$i$
و بردار عمود بر مثلث است. از آنجایی که در ۳ بُعد بردار عمود بر سطح و لاپلاسین هم‌جهت هستند
\cite{gompper1996}
معادله‌ی 
\ref{eq:ItzyksonPotential}
تعریف صحیحی از انحنا است. تعریف انحنا در هر نقطه در صورتی که نیاز به  اضافه کردن انحنای ذاتی به معادله انحنا باشد، اهمیت دارد. 

معادله‌ی
\ref{eq:ItzyksonPotential}
برای شبکه‌های مثلثی در نظر گرفته شده که مثلثی با زاویه‌ی منفرجه نداشته باشد و همچنین شکل و اندازه تمام مثلث‌ها تقریبا یکسان باشد
\cite{Itzykson1986}.
 همانطور که گامپر و کرول هم اشاره کرده‌اند
\cite{gompper1996}
در این روش داشتن زوایای منفرجه ناپایداری‌های عددی در محاسبات انحنا (به خصوص در علامت پارامتر‌های 
$\sigma_i$
یا
$\tilde\ell_{ij}$
) ایجاد خواهد کرد. به این علت مهم، این روش تنها  در مطالعاتی به کار برده می‌شود  که مِش‌های  مثلثی  توزیع یکنواختی از نقاط داشته و توزیع طول اضلاع کنترل شده باشد تا تمام مثلث‌های تشکیل شده اندازه و شکل کم و بیش یکسان داشته باشند.



\subsubsection{
روش یولیشِر
}
در سال ۱۹۹۶ فرنک یولیشِر 
\cite{Julicher1996}
روش دیگری برای تخمین انحنا بر نقاط شبکه‌های مثلثی استفاده کرد. در روش یولیشِر انحنای متوسط در هر نقطه با محاسبه‌ی  میانگین تصویر تانسور انحنا برای هر دوسطحی (جفت مثلث‌) بر صفحه‌ی مماس بر پلاکت  تخمین زده می‌شود
\cite{Ramakrishnan2011}
مساحت بَرییسنتریک\LTRfootnote{Barycentric}
(مرکز جرمی) سهم هر نقطه را در انحنا تعیین می‌کند،
\begin{eqnarray}
E_{b}^{J}=2\kappa\sum_{i}\frac{1}{a_i}\left[\sum_{j(i)}\frac{1}{4}(\ell_{ij}\phi_{ij})\right]^2.
\label{eq:JulicherPotential}
\end{eqnarray}
در معادله‌ی بالا 
$\ell_{ij}$
و
$\phi_{ij}$
طول ضلع و زاویه‌ی دوسطحی آن (شکل
\ref{fig:trianglePairAngle})
 است. با فرض اینکه توپولوژی سطح تغییر نکند، مشخصه‌ی اویلری سطح ثابت باشد، و سطح انحنای ذاتی نداشته باشد، انحنای ذاتی میانگین سطح با جمع زیر محاسبه می‌شود، 
\begin{eqnarray}
M=\frac{1}{2}\sum_{<i,j>)}\ell_{ij}\phi_{ij} = \frac{1}{4}\sum_i\sum_{j(i)}\ell_{ij}\phi_{ij}.
\label{eq:JulicherTotalMeanCurvature}
\end{eqnarray}
مساحتی که به هر نقطه نسبت داده می‌شود،
$a_i$
مساحت بریسنتریک (مرکز جرمی) پلاکت است (شکل
\ref{fig:voronoiBarycentric})
 که برابر یک سوم مساحت تمام مثلث‌های پلاکت است، 
\begin{eqnarray}
a_i=\frac{1}{3}\sum_{tri (i)}a_{tri}.
\label{eq:BarycentricArea}
\end{eqnarray}
در این مدل، در صورتی که انحنای ذاتی در سطح وجود داشته باشد، انحنا در هر نقطه به شکل،
\begin{eqnarray}
H_i^J=\vec n\cdot\frac{1}{4}\frac{1}{a_i}\sum_{j(i)}\ell_{ij}\phi_{ij},
\label{eq:meanCurvatureDiscreteSingleVertexJulicher}
\end{eqnarray}
تعریف می‌شود. در اینجا تعریف بردار عمود بر سطح مطابق معادله‌ی
\ref{eq:noramlVector}
است. رابطه‌ی یولیشر را با یک فاکتورگیری ساده می‌توان مشابه با رابطه‌ی  بازنویسی کرد،
\begin{eqnarray}
E_{b}^{J}=\frac{1}{2}\kappa\sum_{i}\frac{1}{a_i}\left[\sum_{j(i)}\frac{1}{2}(\ell_{ij}\phi_{ij})\right]^2.
\label{eq:JulicherPotentialHalf}
\end{eqnarray}
در قسمت نتایج نشان خواهیم داد که اختلاف انحنای میانگین محاسبه شده توسط  و یولیشر در وزنی‌است که به هر پلاکت نسبت می‌دهند و برای عموم چیدمان‌ پلاکت‌ها و انحنای سطح کوچک،
\begin{eqnarray}
\left[\sum_{j(i)}\frac{1}{2}(\ell_{ij}\phi_{ij})\right]^2\approx\left[\sum_{j(i)}\frac{\sigma_{ij}}{\ell_{ij}}(\vec r_i-\vec r_j)\right]^2.
\label{eq:JulicherItzyksonNumerator}
\end{eqnarray}


\subsubsection{
روش‌های -بریسنتریک و یولیشر-ورنوی
}
با توجه به رابطه‌ی 
\ref{eq:JulicherItzyksonNumerator}
با جابجایی وزن نسبت داده شده به هر پلاکت می‌توان دو نوع روش جدید برای محاسبه‌ی انحنا در شبکه‌های مثلثی طراحی کرد. یکی محاسبه‌ی انحنا به روش  ولی با وزن بریسنتریک،
\begin{eqnarray}
E_{b}^{GKB}=\frac{1}{2}\kappa\sum_{i}\frac{1}{a_i}\left[\sum_{j(i)}\frac{\sigma_{ij}}{\ell_{ij}}(\vec r_i-\vec r_j)\right]^2,
\label{eq:ItzyksonBarycentricPotential}
\end{eqnarray}
و دیگری محاسبه‌ی انحنا با روش یولیشر ولی با وزن ورنوی است،
\begin{eqnarray}
E_{b}^{JV}=\frac{1}{2}\kappa\sum_{i}\frac{1}{\sigma_i}\left[\sum_{j(i)}\frac{1}{2}(\ell_{ij}\phi_{ij})\right]^2.
\label{eq:JulicherVoronoiPotential}
\end{eqnarray}
انگیزه‌ی اصلی برای پیشنهاد این دو روش جدید بررسی پایداری عددی روش‌های مختلف محاسبه‌ی انحنا برای محاسبات دینامیک ملکولی است.  در بخش نتایج مفصل راجع به پایداری عددی این روش‌ها صحبت خواهد شد.








 
\section{
تغییر انرژی آزاد با افزودن نقطه‌ی نقص به شبکه
}
\setRL
%\pagenumbering{arabic} 


\subsection{
تغییر انرژی کششی
}
در نظریه‌ی الاستیک سطح هر تغییر شکل با یک میدان بردار جابجایی 
$u(r)=(u_1,u_2)$
نشان داده می‌شود نقطه‌ی 
$r(x,y)$
را به نقطه‌ی 
$r+u$
نگاشت می‌کند. اگر در شبکه نقص وجود نداشته باشد این نگاشت یک به یک خواهد بود. در صورتی که در شبکه دررفتگی
\LTRfootnote{dislocation}
یا نقص وجود داشته باشد هر انتگرال بسته پاد ساعتگرد که محل نقص داخل آن قرار گیرد با بردار ثابت برگر
\LTRfootnote{Burger}
برابر خواهد بود.
\cite{mitchell1961}
از آنجایی هم که بردار برگر همیشه با یکی از بردارهای شبکه برابر است، یک به یک نبودن نگاشت در حضور نقص مشکلی در فیزیک مسئله ایجاد نخواهد کرد. این بحث به زبان ریاضی شکل زیر را به خود می‌گیرد،
\begin{equation}
\begin{aligned}
&\oint_Ldu_k=\oint_L\partial_iu_kdx_i=b_k\\
&\epsilon_{li}\partial_l\partial_iu_j=b_j\delta(r-r_0)
\end{aligned}
\end{equation}
که در بالا 
$r_0$
محل نقص، و 
$b$
 بردار برگر است. در رفتگی  بر حسب میدان زاویه‌ی بین پیوندهای شبکه مشخص می‌شود، که جهت گیری در پیرامون هر اتم را مشخص می‌کند. صراحت هر نقص،
 $s$
 حول هر مسیر بسته دور نقص تعریف می‌شود. در شبکه‌ای که تقارن $n$
 تایی داشته باشد، 
 $s$ حتما ضریبی از 
 $2\pi/n$ خواهد بود.
 در این بخش شبکه‌های شش ضلعی با تقارن 
 $n=6$
و لغزش‌های کوچک
$s=\pm2\pi/6$
مورد توجه ماست. به زبان ریاضی می‌توان این جملات را به این شکل نشان داد،

 \begin{equation}
\begin{aligned}
&\oint_Ld\theta=\oint_L\partial_i\theta dx_i=s\\
&\epsilon_{ij}\partial_i\partial_i\theta=s\delta(r-r_0)
\label{eq:thetauij}
\end{aligned}
\end{equation}
با جایگذاری
\begin{equation}
\theta=\frac{1}{2}\epsilon_{ij}\partial_iu_j
\end{equation}
حال می‌خواهیم شرایط معادله‌ی 
\ref{eq:constraint}
را به صورت قید برای $\chi$
تعریف کنیم تا تضمین کند که همیشه می‌توانیم $\chi$
را به صورت جابجایی‌ها بنویسیم. برای اینکار طرفین معادله‌ی 
\ref{eq:constraint}
را در 
$\epsilon_{ik}\epsilon_{jl}\partial_k\partial_l$
ضرب می‌کنیم که نتیجه‌ی آن،
\begin{equation}
\frac{1}{Y}\nabla^4\chi=\epsilon_{ik}\epsilon_{jl}\partial_k\partial_lu_{ij}=\epsilon_{ik}\epsilon_{jl}\partial_k\partial_l\frac{1}{2}(\partial_iu_j+\partial_ju_i)
\label{eq:incompatibility}
\end{equation}
در صورتی که سمت راست معادله‌ی فوق برابر با صفر شود، می‌توان گفت که $u_{ij}$ 
سازگار است و تنها یک جواب برای میدان جابجایی وجود دارد که جواب معادله‌ی 
\ref{eq:constraint}
است. در غیر این صورت معدله‌ی 
\ref{eq:constraint}
بیش از یک جواب دارد. در نتیجه رایج است که به نام سمت راست معادله‌ی
\ref{eq:incompatibility}
را ناسازگاری
\LTRfootnote{incompatibility}
و $\epsilon_{ik}\epsilon_{jl}\partial_k\partial_l$
را عملگر ناسازگاری بنامند. می‌توانیم محاسبات معادله‌ی 
\ref{eq:incompatibility}
را به این شکل ادامه دهیم،

\begin{equation}
\begin{aligned}
\frac{1}{Y}\nabla^4\chi&=\epsilon_{ik}\epsilon_{jl}\partial_k\partial_l\frac{1}{2}(\partial_iu_j-\partial_ju_i)+\epsilon_{ik}\epsilon_{jl}\partial_k\partial_l\partial_ju_i\\
&=\epsilon_{kl}\partial_k\partial_l\theta+ \epsilon_{ik}\partial_k(\epsilon_{jl}\partial_l\partial_ju_i)\\
&=\sum_{\alpha}s_\alpha\delta(r-r_\alpha)+\sum_\beta b_i^\beta\epsilon_{ik}\partial_k\delta(r-r_\beta)
\label{eq:disclination}
\end{aligned}
\end{equation}
که $s_\alpha$
بار نقص در محل $r_\alpha$
و $b^\beta$
بردار برگر لغزش در محل $r_\beta$
را مشخص می‌کند. خط آخر معادله‌ی 
\ref{eq:disclinationX}
چگالی نقصان
$s(r)$
 را در شبکه مشخص می‌کند. در نتیجه نظریه کشسانی ۲ بعدی به معادله‌ی زیر خلاصه می‌شود،
\begin{equation}
\frac{1}{Y}\nabla^4\chi=s(r)
\label{eq:masterstretch}
\end{equation}
بدون در نظر گرفتن شرایط مرزی معادله‌ی فوق جواب یکه نخواهد داشت. فرض کنیم که یک غشای دایروی را بررسی می‌کنیم که در مرز‌ها آزاد است. در نتیجه جمع نیرو‌ها روی مرز باید صفر باشد، یعنی 
$\sigma_{rr},\sigma_{r\phi}=0$
. اگر فرض کنیم که لغزش در مرکز مختصات است، معادله‌ی 
\ref{eq:masterstretch}
به شکل زیر در می‌آید،
\begin{equation}
\frac{1}{Y}\nabla^4\chi=b_i\epsilon_{ij}\partial_j\delta(r)
\end{equation}
که به پاسخ
\begin{equation}
\chi=\frac{Y}{4\pi}b_i\epsilon_{ij}r_j\ln r
\label{eq:masterstretchsol}
\end{equation}
منجر می‌شود. البته که اگر قرار بود معدله‌ی 
\ref{eq:masterstretch}
را برای شرایط مرزی محدود حل کنیم،‌ باید جملات دیگری نیز به پاسخ 
\ref{eq:masterstretchsol}
اضافه می‌کردیم، ولی از آنجایی که این جملات در حد 
$r\rightarrow\infty$
صفر می‌شوند با این پاسخ مسئله‌ را جلو می‌بریم. حالا معادله‌ی 
\ref{eq:stress}
را بر حسب تنش می‌نویسیم،
\begin{equation}
F_s=\frac{1}{2Y}\int d^r(\nabla^2\chi)^2-\frac{1+\nu}{2Y}\int d^r\epsilon_{ik}\epsilon_{jl}\partial_k\partial_l(\partial_i\chi\partial_j\chi)
%\label{eq:masterstretchsol}
\end{equation}
با جایگذاری $\chi$ از معادله‌ی
\ref{eq:masterstretchsol}
و انتگرال گیری خواهیم داشت،
\begin{equation}
F_s=\frac{Yb^2}{8\pi}\ln\left[\frac{R}{a}\right]
%\label{eq:masterstretchsol}
\end{equation}
که انرژی حاصل از لغزش در محدوده‌ی 
$a\leq r\leq R$
در یک غشا با اندازه‌ی محدود را مشخص می‌کند. حالا معادله‌ی 
\ref{eq:masterstretch}
را برای وجود نقص در  مرکز  شبکه جلو می‌بریم.
\begin{equation}
\begin{aligned}
&\frac{1}{Y}\nabla^4\chi=s\delta(r)\\
&\chi=\frac{Ys}{8\pi}(Ar^2+r^2\ln r)
%\label{eq:disclination}
\end{aligned}
\end{equation}
بدون وجود جمله‌ی $Ar^2$
حاصل معادله کرنش بی‌نهایت در مرز خواهد بود که با آهنگ $\ln R$ بزرگ می‌شود. از آنجایی که تمام تقریب‌هایی که تا به الان استفاده شد هارمونیک بودند، این رفتار غیر قابل قبول خواهد بود زیرا که در این صورت ماده به علت کرنش زیاد از هم گسسته خواهد شد. به علت تقارن چرخش در مسئله نیز مؤلفه‌ی تنش زاویه‌دار نیز صفر  خواهد بود
\begin{equation}
\sigma_{r\phi}=-\frac{\partial}{\partial r}\left[\frac{1}{r}\frac{\partial\chi}{\partial\phi}\right]
\end{equation}
در این صورت نیاز است که در مرز مؤلفه‌ی تنش،
\begin{equation}
\sigma_{rr}=\frac{1}{r}\frac{\partial\chi}{\partial r}+\frac{1}{r^2}\frac{\partial^2\chi}{\partial \phi^2}
\end{equation}
یعنی هنگامی که $r=R$ جمله‌ی بالا صفر شود که حاصل آن تعیین کمیت $A$
است،
\begin{equation}
A=-\frac{1}{2}-\ln R
\end{equation}
حالا می‌توانیم تعریف مناسبی از تنش و انرژي سیستم را بنویسیم،
\begin{equation}
\begin{aligned}
&\chi=\frac{Ys}{8\pi}r^2\left[\ln \left(\frac{r}{R}\right)-\frac{1}{2}\right]\\
&E_s=\frac{Ys^2}{32\pi}R^2
%\label{eq:disclination}
\end{aligned}
\label{eq:stretchdiscenergy}
\end{equation}





 
 
 
 
 
 
 
 
 
 
 
 
 
 
 
\setRL
%\pagenumbering{arabic} 



\subsection{
تغییر انرژی خمش
}
برای اینکه جابجایی خارج از صفحه را توصیف کنیم علاوه بر میدان جابجایی 
$u(r)=(u_1,u_2)$
نیاز به تابع جدید 
$f(r)$
داریم که انحراف
\LTRfootnote{deflection}
 نقاط شبکه را توصیف می‌کند یعنی تغییرات نقطه‌ی 
$(x_1,x_2,0)$
را به نقطه‌ی 
$(x_1+u_1,x_2+u_2,f)$
نگاشت می‌کند. در نتیجه انرژي کل سیستم حاصل جمع انرژی کشسانی و انرژي خمشی خواهد بود. انرژی کشسانی همچنان طبق معادله‌ی
\ref{eq:energylame}
با این تفاوت که به جای تعریف کرنش در معادله‌ی 
\ref{eq:simplestrain}
از رابطه‌ی زیر استفاده می‌کنیم
\begin{equation}
u_{ij}=\frac{1}{2}(\partial_iu_j+\partial_ju_i+\partial_if\partial_jf)
\label{eq:nonlinearstrain}
\end{equation}
در اینجا نیز همانند بخش قبلی از جملات مرتبه‌ی ۲ به بالای جابجایی صرف نظر کرده‌ایم. معمولا هنگام  مدل‌سازی صفحات تخت در حالت تغییر شکل کوچک همچنان استفاده از معدله‌ی 
\ref{eq:simplestrain}
رایج است که حاصل آن یک نظریه‌ی کاملا خطی است. در اینجا ما قصد داریم تغییر شکل‌هایی را بررسی کنیم که در آن $f$ مهم است و کمترین مرتبه‌ای که $f$ 
تاثیر خود را نشان می‌دهد مرتبه‌ی دوم است، در نتیحه کرنش را به شکل  معادله‌ی 
\ref{eq:nonlinearstrain}
قابل قبول است. انرژی خمش را طبق نظریه‌ی هلفریش
\cite{Helfrich1973}
با خمش سطح $H$
و خمش گاووسی $K$
تعریف می‌کنیم، 
\begin{equation}
F_b=\int dS\left(\frac{1}{2}\kappa H^2+\kappa_GK\right)
\end{equation}
که در اینجا 
$\kappa$
سختی خمش، 
$\kappa_G$
سختی گاووسی، و 
$dS$
عنصر سطح است. خمش بر حسب 
$f$
 به شکل زیر محاسبه می‌شوند،
\begin{equation}
\begin{aligned}
H&=\nabla\cdot\left[\frac{\nabla f}{\sqrt{1+|\nabla f|^2}}\right],\\
K&=\frac{\det(\partial_i\partial_jf)}{\left(1+|\nabla f|^2\right)^2}
\end{aligned}
\end{equation}
برای تغییر شکل‌های کوچک می‌توانی از تقریب زیر استفاده کنیم،
\begin{equation}
\begin{aligned}
H&\approx\nabla^2f\\
K&\approx \det(\partial_i\partial_jf)=-\frac{1}{2}\epsilon_{ik}\epsilon_{jl}\partial_k\partial_l(\partial_if\partial_jf)
\end{aligned}
\end{equation}
با جایگذاری روابط بالا می‌توانیم انرژي خمش را بازنویسی کنیم،
\begin{equation}
F_b\approx\frac{1}{2}\kappa\int d^2r(\nabla^2 f)^2+\frac{1}{2}\kappa_G\int d^2r\epsilon_{ik}\epsilon_{jl}\partial_k\partial_l(\partial_if\partial_jf)
\label{eq:bendingenergyequ}
\end{equation}
حالا با مشتق‌گیری نسبت به $u$ و $f$
می‌توانیم مانند بخش قبل معادلاتی که به تعریف تنش می‌انجامد را تعریف کنیم
\begin{equation}
\begin{aligned}
\kappa\nabla^4f&=\partial_i(\sigma_{ij}\partial_jf)\\
\partial_i\sigma_{ij}&=0
\end{aligned}
\end{equation}
که در بالا رابطه‌ی بین تانسور تنش و تانسور غیر خطی کرنی مشابه معادله‌ی 
\ref{eq:stress}
تعریف شده است. حالا مشابه مراحلی که منجر به معادله‌ی 
\ref{eq:disclination}
شد عمل کرده و به رابطه‌ی زیر می‌رسیم،

\begin{equation}
\frac{1}{Y}\nabla^4\chi-\frac{1}{2}\epsilon_{ik}\epsilon_{jl}=\sum_\alpha s_\alpha\delta(r-r_\alpha)+\sum_\beta b_i^\beta\epsilon_{ik}\partial_k\delta(r-r_\beta)
\end{equation}
و در نهایت می‌توانیم یک سیستم معادلا کامل بنویسیم،
\begin{equation}
\begin{aligned}
&\kappa\nabla^4f+\epsilon_{ik}\epsilon_{jl}\partial_k\partial_l(\partial_i\chi\partial_jf)=0\\
&\frac{1}{Y}\nabla^4\chi=s(r)-K(r)
\end{aligned}
\end{equation}
و همانند قسمت قبل $s(r)$ چگالی نقص و 
$k(r)$
خمش گاووسی است. نقش خمش گاووسی به صورت کم کردن تنش در اینجا ظاهر می‌شود. از آنجایی که انتگرال خمش گاووسی به انتگرال روی محیط می‌تواند کاهش پیدا کند بر روی فیزیک روی سطح مسئله تاثیر نمی‌گذارد بلکه تاثیر خود را روی شرایط مرزی نشان می‌دهد. پس به قیود 
$\sigma_{rr},\sigma_{r\phi}=0$
باید قیود زیر را نیز اضافه کنیم،
\begin{equation}
\begin{aligned}
&\frac{\kappa}{\kappa_G}\nabla^2f+\left[\frac{1}{r}\frac{\partial f}{\partial r}+\frac{1}{r^2}\frac{\partial^2 f}{\partial\phi^2}\right]=0\\
&\frac{\kappa}{\kappa_G}\frac{\partial}{\partial r}\nabla^2f-\frac{1}{r}\frac{\partial}{\partial r}\frac{1}{r}\frac{\partial^2 f}{\partial\phi^2}=0
\end{aligned}
\end{equation}
که بر روی مرز دایروی ارضاء می‌شوند. اگر بسط بالا را باز کنیم معادلات شکل زیر را به خود می‌گیرند،

\begin{equation}
\begin{aligned}
&\kappa\nabla^4f=\frac{\partial^2\chi}{\partial y^2}\frac{\partial^2f}{\partial x^2}+\frac{\partial^2\chi}{\partial x^2}\frac{\partial^2f}{\partial y^2}-\frac{\partial^2\chi}{\partial x\partial y}\frac{\partial^2f}{\partial x\partial y},\\
&\frac{1}{Y}\nabla^4\chi+\frac{\partial^2f}{\partial x^2}\frac{\partial^2f}{\partial y^2}-\left[\frac{\partial^2f}{\partial x\partial y}\right]^2=\sum_\alpha s_\alpha \delta(r-r_\alpha)+\sum_\beta b_i^\beta \epsilon_{ik}\partial_k\delta(r-r_\beta)
\end{aligned}
\end{equation}
در صورتی که هیچ نقصی در شبکه وجود نداشته باشد و جملات شامل دلتای دیراک را برابر با صفر قرار دهیم همان معادله‌ی کارمن
\LTRfootnote{Kármán}
 را بدس می‌آوریم. این معدلات غیر خطی به راحتی قابل حل نیستند. سعی می‌کنیم این معادلات را برای حالت خیلی ساده شده‌ای که شامل یک نقص در مرکز شبکه‌ای که نسبت به مرکز تقارن دایره‌ای داشته باشد، حل کنیم. برای فواصل دور از نقطه‌ی نقص،‌ معادلات به شکل زیر در می‌آید،

\begin{equation}
\begin{aligned}
&\kappa\nabla^4f=\frac{1}{r}\frac{d}{dr}\left[\frac{d\chi}{dr}\frac{df}{dr}\right],\\
&\frac{1}{Y}\nabla^4\chi+\frac{1}{2r}\frac{d}{dr}\left[\frac{df}{dr}\right]^2=0
\end{aligned}
\end{equation}

که گرادیان به شکل زیر در نظر گرفته شده،
\begin{equation}
\nabla^2=\frac{1}{r}\frac{d}{dr}r\frac{d}{dr}
\end{equation}
. حدس می‌زنیم جواب معادلات به شکل زیر باشد،


\begin{equation}
\begin{aligned}
&\chi=-\kappa\ln\left[\frac{r}{a}\right]\\
&f=\pm\left[\frac{s}{\pi}\right]^{\frac{1}{2}}r
\end{aligned}
\end{equation}
. پس می‌توانیم بردار جابجایی را با کمک معادله‌ی 
\ref{eq:nonlinearstrain}
بنویسیم. 

\begin{equation}
\begin{aligned}
&u_x=-\frac{s}{2\pi}y\phi-\frac{s}{2\pi}x+\frac{\kappa(1+\sigma)}{Y}\frac{x}{r^2}\\
&u_y=\frac{s}{2\pi}x\phi-\frac{s}{2\pi}y+\frac{\kappa(1+\sigma)}{Y}\frac{y}{r^2}
\end{aligned}
\end{equation}
که در اینجا
$\frac{y}{x}=\tan\phi$
. در نهایت با جایگذاری پاسخ حدسی در معادله‌ی
\ref{eq:bendingenergyequ}
به فرم انرژی زیر می‌رسیم.
\begin{equation}
E_{bending}= s\kappa\ln\left[\frac{R}{a}\right]
\label{eq:bendingdiscenergy}
\end{equation}





 و حالا باید با مقایسه‌ی این دو انرژی بگی و ارجاع بدی که بسته به گاما ما شکل‌های کروی و ۲۰ وجهی خواهیم دید.






