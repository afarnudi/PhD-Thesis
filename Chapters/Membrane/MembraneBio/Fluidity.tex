%\setRL
%\pagenumbering{arabic} 

دیگر خاصیت مشترک میان تمام غشا‌ها این است که به علت سرعت بالای پخش مولکول‌ها بر روی سطح آن، حالت سیال بودن خود را حفظ می‌کند. سیال بودن غشا تقریبا از ده‌ی ۱۹۷۰ توسط عموم پذیرفته شد. در آن زمان آزمایش‌هایی به طور همزمان و مستقل شواهدی ارائه دادند که خاصیت سیال‌گون غشا را نشان می‌داد. اولین مشاهده در این تحقیقات مربوط به اندازه‌گیری سرعت پخش مولکول‌های لیپیدی  با برچسب اسپینی\LTRfootnote{spin-labelled lipids}
\cite{Kornberg1971DiffusionPhospholipids, Devaux1972LateralDiffusion}
و مولکول‌های استروید\LTRfootnote{steroids}  
\cite{Sackmann1972, Traeubl1972}
بر سطح غشا بود که حدود 
$1 \mu m^2\cdot sec^{-1}$
اندازه‌گیری شد. روش‌های جدید اندازه‌گیری  حرکت مولکول‌های فلورسانت\LTRfootnote{fluorescence recovery after photobleaching (FRAP)}  
\cite{almeida1992lateral}
و همچنین روش‌های شناسایی تک ذره‌ای\LTRfootnote{single particle tracking}  
\cite{Sako1994, Saxton1997, Fujiwara2002, Kusumi2005}
این ضریب پخش را در مورد غشاها تایید می‌کنند.

دومین سری آزمایش‌هایی که ماهیت سیال‌گون غشا را تایید کرد  مطالعات بر روی نحوه‌ی تغییر شکل گلبول‌های قرمز 
\cite{Canham1970, Evans1974}
و غشاهای لیپیدی
\cite{Helfrich1973, Helfrich1976}
انجام شده ماهیت سیال‌گونه‌ی غشا را نیز تایید می‌کند. در اثر تغییر شکل، خمش سطح غشا  به طور پیوسته و بدون شکستگی تغییر می‌کند و در صورتی که جنس غشا جامد و یا پلیمری باشد چنین تغییر شکلی امکان پذیر نخواهد بود. به طور مشخص این نوع تغییر شکل زمانی مشاهده می‌شود که حباب‌های جانبی بر روی سطح غشا تشکیل می‌شود (شکل 
\ref{fig:budding}
).

\begin{figure}[t]
\begin{center}
\includegraphics[width=\columnwidth]{\MemBio /Pics/budding.pdf}
\caption{
تشکیل یک غشای حبابی بر سطح یک غشای غول‌آسا در تصویربرداری میکروسکوپی اختلاف فازی. این حباب ظرف پنج ثانیه تشکیل  شده که نشان از جنس سیال‌گون غشا است
\cite{Dimova2006}.
}
\label{fig:budding}
\end{center}
\end{figure}

این نوع جوانه‌زدن\LTRfootnote{budding}  
یکی از مهم‌ترین ساز‌ و کار‌های مبادله‌ی مولکول با محیط در سلول‌های زیستی‌ است. در یک سری از فرآیند‌های زیستی، سلول پروتئین‌ها را به سطح داخلی غشا منتقل می‌کند و با ایجاد جوانه می‌تواند مولکول‌های خود را به محیط صادر کند. عکس این فرآیند هم انجام می‌شود که بر اثر آن سلول اطلاعاتی از سلول‌های حاضر در محیط اطراف خود دریافت می‌کند.






