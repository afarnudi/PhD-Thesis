\setRL
%\pagenumbering{arabic} 


%\section{\label{sec:cellmembrane}
%غشای سلولی
%}
\section{
مقدمه
}
مهم‌ترین نقش غشاهای زیستی ایجاد یک دیوار یا مانع برای مشخص کردن مرز داخل و خارج سلول است که نیاز اولیه‌ی وجود حیات است
\cite{Boyle2008Biology}.
 علاوه بر در بر گرفتن تمام اعضای یک سلول، در سلول‌های هسته‌دار، غشای هسته اورگان‌های خیلی مهم سلول را نیز در خود جا داده و محافظت می‌کند (شکل
\ref{fig:cellparts}
). به عنوان مرز سلول با دینای خارج، غشا نقش مهمی در مدیریتِ نقل و انتقال مواد بین سلول و محیط پرامون دارد. درون غشا پروتئین‌ها، لیگاند‌ها
\LTRfootnote{ligand} 
،
و ملکول‌های درشت خیلی زیادی وجود دارد که در طیف‌ گسترده‌ای از فرآیندها نقش دارد. برای مثال غشا نقل و انتقال یون‌ها به درون و خارج سلول را از طریق کانال‌های پروتئینی مدیریت می‌کند
\cite{NEHER1976ProteinChannel}،
نسبت به تغییر فشار اُسمزی محیط واکنش نشان می‌دهد
\cite{Perozo2006Osmotic,Vasquez2009Osmotic,Haswell2011Osmotic}،
و همچنین در ساز و کار‌های بزرگ مقیاس مانند تقسیم سلولی، حرکت و جابجایی سلول، و چسبیدن به سطح نقش دارد.
\begin{figure}[h]
\begin{center}
\includegraphics[width=4in]{\MemBio /Pics/CellParts}
\caption{
شکل طراحی شده از سلول پستانداران که غشا و اعضای سلول را نشان می‌دهد
\cite{CellParts}
.
}
\label{fig:cellparts}
\end{center}
\end{figure}


غشای سلول در خیلی از مسائل مهم پزشکی نیز مطرح می‌شود مانند نقش آن در انتقال پالس الکتریکی در سلول‌های عصبی هنگام بیهوشی عمومی 
\cite{BioMemBook2007}.
همچنین بیشتر تلاش صنعت داروسازی، طراحی داروی مناسب برای اتصال به پروتئین‌های درون غشاست. تقریبا یک سوم پروتئین‌ها درون غشای سلول قرار دارند که  مورد هدف ۶۰ در صد از داروی‌های موجود در بازار هستند
\cite{DrugDelivery2007}.
همچ دانش یافته شده از مطالعه‌ی غشا کاربرد در تکنولوژی‌های مدرن مانند، بسته‌بندی و انتقال دارو
\cite{Torchilin2006Drugdelivery}،
ایجاد اتاقک‌های کوچک برای انجام واکنش‌های شیمیایی
\cite{Karlsson2001MemChamber}،
 و حس‌گرهای زیستی که دستگاه‌های الکتریکی را با غشا ترکیب می‌کند
\cite{MemeElctronics2012}.


\section{
تاریخچه کوتاه
}
حاصل زحمات افراد در قرن ۱۹ و ۲۰ میلادی در راستای پی بردن به ساز و کار واحد‌های سازنده‌ی موجودات زنده، تصویری با جزئیات زیاد از سلول‌های زنده است. در سال ۱۹۸۵ ارنست اُورتن 
\LTRfootnote{Ernest Overton}  
سلول‌های گیاهی را در محلول‌های مختلف (قند،‌ الکل، اتر، فنول، و استن
\LTRfootnote{sugar, alcohols, ether, phenol, and acetone}
) قرار داد. مشاهدات وی نشان داد که (تحت اختلاف فشار اسمزی یکسان) محلول‌هایی مانند قند که در آب به راحتی حل می‌شوند نمی‌توانند وارد سلول شوند در صورتی که محلول‌های دیگری که حل شونده‌ی خوبی در آب نیستند، می‌توانند وارد سلول شوند. او نتیجه گرفت که جنس مرز سلول با سیتوپلاسم درون آن متفاوت است و به احتمال زیاد از ملکول‌های چربی گون تشکیل شده است
\cite{overton1985}.
در سال ۱۹۱۷ اِروین لَنگموئر
 \LTRfootnote{Irving Langmuir}
 مقاله‌ای چاپ کرد که در آن روشی برای اندازه‌گیری فشار جانبی
  \LTRfootnote{lateral}
 غشا در سطح مقطع مشترک هوا و آب پیشنهاد داد
 \cite{Langmuir1917}.
 او با استفاده از روش خود نشان داد که غشا در این سطح مقطع مشترک یک تک لایه‌ی ملکول تشکیل می‌دهد و مساحت هر ملکول چربی را 
 $S_{lipid} = 0.7 nm^2$
 گزارش کرد. او همچنین در گزارش خود پیشنهاد داد که غشا از ملکول‌های دوزیست
   \LTRfootnote{Amphiphiles}
 تشکیل شده. ملکول‌ها دوزیست ملکول‌هایی هستند که یک گروه سَر دو قطبی آب دوست و یک یا چند دُم هیدروکربنی آب‌گریز دارند.
 
 با به کارگیری روش اندازه‌گیری لنگموئر، در سال ۱۹۲۵، گُرتِر
  \LTRfootnote{E. Gorter}
 و گرِندِل
   \LTRfootnote{F. Grendel}
 نشان دادند که  غشای گلبول‌های قرمز از یک دو-لایه لیپید تشکیل شده
 \cite{Gorter1925}.
 آن‌ها غشای گلبول را در اَسِتُن حل کرده و با روش لنگموئر سطح آن را اندازه‌گیری کردند. سپس این عدد را با سطح گلبول قرمز خشک شده مقایسه کردند. در آزمایش آن‌ها دو خطاوجود داشت؛ اول اینکه اَسِتُن نمی‌تواند تمام ملکول‌های چربی را از گلبول جدا کند، و دوم در اندازه‌گیری سطح گلبول قرمز خطا داشتند
 \cite{BiomembranesBook1989,BioMemBook2007}.
 ولی خوشبختانه این دو خطای اندازه‌گیری همدیگر را تکمیل کردند و آن‌ها نسبت این دو عدد را با تقریب خوبی نزدیک به ۲ اندازه‌گیری کردند که با اطلاعات فعلی ما که ساختار غشا از ملکول‌های چربی دو-لایه تشکیل شده، سازگار است
 \cite{Edidin2003}.
 
  
 
 
 در سال ۱۹۳۲ (حدود صد سال پیش) کِنِت کول
 \LTRfootnote{Kenneth Cole}
 تخم جوجه‌تیغی دریایی
 \LTRfootnote{sea urchin (arbacia) egg} 
را بر سطحی قرار داد و  با کمک یک فیبرِ از جنس طلا به تخم  فشار وارد کرد. با اندازه‌گیری  کشش سطحی و مقایسه‌ی آن با حد تحمل فشار تخم، استدلال کرد که لایه‌ی نازکی که در اطراف سلول وجود دارد تنها از ملکول‌های چربی درست نشده است
 \cite{Cole1932}. 

در دهه‌ی ۱۹۳۰ داوسن
 \LTRfootnote{H. Davson} 
و دنیِلی
 \LTRfootnote{J. F. Danielli} 
مدل جدید برای غشا پیشنهاد دادند. آن‌ها با آزمایش بر روی غشا‌های مصنوعی و سلولی، اختلاف در تراوایی دو روی غشا نسبت به مواد یونی و دوقطبی را با وجود لایه‌ای از پروتئین بر روی غشا توصیف کردند
\cite{Danielli1935}. 
این مدل غشا اولین مدلی بود که به طور عمومی پذیرفته  و تا سال‌ها  در تحقیقات استفاده شد. تنها در زمان پدید آمدن ابزار‌های تصویر‌برداری دقیق از غشا با کمی تغییر جزئی مواجه شد.

در دهه‌ی ۱۹۵۰ با فرا رسیدن تکنولوژی میکروسکوپ الکترونی، رابرتسون
\LTRfootnote{J. D. Robertson}
 اولین تصاویر از غشای سلول را رونمایی کرد و با استفاده از روش‌های رنگ آمیزی، وجود لیپید‌ها
\LTRfootnote{lipid}
 در غشا را تایید و ضخامت غشای سلولی را بین ۵ تا ۱۰ نانومتر اندازه‌گیری کرد
\cite{ROBERTSON1959aa}.
او همچنین نشان داد که غشای پلاسمایی و تمامی اعضای غشاگون مثل غشای هسته‌ی سلول و غشای دو-لایه میتوکندریا
RevTeX 4.2 Template and Sample
ساختار مشترکی دارند که مدل داوسن-دنیلی و مشاهدات گُرتِر و گرِندِل را تایید می‌کرد.
همچنین درنتیجه‌ی اندازه‌گیری با میکروسکوپ و پراش اشعه‌ی X ضخامت غشا 
 $5-8nm$
و ضخامت لایه‌ی مرکزی آبگریز آن
 $3-4nm$
گزارش شد
\cite{NelsonBook2004}.
با پیشرفت سریع روش‌های اندازه‌گیری و تصویربرداری بر پایه‌ی تشدید مغناطیسی (مانند تشدید مغناطیسی‌ هسته‌ای یا NMR
\LTRfootnote{nuclear magnetic resonance}
و تشدید اسپین الکترون
\LTRfootnote{electron spin resonance}
) آزمایش‌های زیادی نشان دادند که غشا  خواصی شبیه به مایع دارد
\cite{Edidin2003}
و لیپید‌ها می‌توانند بر سطح غشا با ضریب پخش
$D_{lipid}\sim 10^{-8}cm^2/s\approx 10^6S_{lipid}/s$
 حرکت کنند
\cite{NelsonBook2004,Chapman1975}.
همچنین عدم تقارن بین دولایه‌ی غشا تایید شد و برای اولین بار با علامت گذاری پروتئین‌های بر روی دو سطح غشا، حضور پروتئین‌ها در درون غشا نشان داده شد
\cite{Bretscher1973}.




\begin{figure}[h]
\begin{center}
\includegraphics[width=4in]{\MemBio /Pics/Cell_membrane_detailed_diagram}
\caption{
شکل از سایت ویکیپیدیا گرفته شده است
\cite{wikiCellMembrane}
. این یک نقاشی از غشا بر اساس مدل غشای مایع موزایکی است. بیشتر غشا از ملکول‌های چربی تشکیل شده ولی پروتئین‌های خیلی زیادی نیز در غشا قرار دارد.  غشا از طریق این پروتئین‌ها به اسکلت سلولی و اجزای دیگر متصل است. دایره‌های قرمز سر آب دوست و رشته‌های زرد دم‌های آب گریز لیپید‌ها را نشان می‌دهد.
}
\label{fig:fluidmembranemodel}
\end{center}
\end{figure}



افراد زیادی در تکمیل تصویری که از غشای سلول هست نقش داشتند، ولی تصویر مدرنی که امروز از غشاهای سلول‌ها داریم، بیشتر بر پایه‌ی مدل غشای مایع موزایکی‌
\LTRfootnote{the mosaic fluid model of membranes}
 است که در سال ۱۹۷۲ توسط سینگر
 \LTRfootnote{Singer}
  و نیکلسون
  \LTRfootnote{Nicholson}
 ارائه شد
\cite{Singer1972}
(شکل 
\ref{fig:fluidmembranemodel}
). بنا بر این مدل، غشا را می‌توان یک مایع همگن دو بعدی وُشکسان از چربی‌ها و کُلِسترول فرض کرد که ماکرو-ملکول‌های پروتئینی به کمک برهمکنش‌های آبگریز  بر روی سطح یا درون آن قرار گرفته و کم و بیش آزادانه حرکت کنند (مانند دریای قطب شمالو تکه‌های یخ  شناور در آن).  در نتیجه هیچ نظم بلند بردی در غشا دیده نمی‌شود. 

در نتیجه‌ی توسعه‌ی تحقیقات، می‌دانیم غشا‌های زیستی بیشتر ساختار موزائیکی دارند تا مایع. یعنی در سلول‌های مختلف غشا خیلی همگن نیست و در یک سلول قسمت‌هایی از غشا ممکن است  ترکیب پروتئینی متفاوتی از بخش‌های دیگر همان سلول داشته باشد. همچنین ضخامت برخی از بخش‌های غشا ممکن است از چند ۱۰ نانومتر تا ۱۰۰ نانومتر (که با سفینگولیپید
  \LTRfootnote{sphingolipids}
و کلسترول غنی شده
) تغییر کند
\cite{Engelman:2005aa}
. یکی از دلایل غیر همگن بودن غشا عدم تقارنی است که قسمت میانی آب‌گریز غشا در حضور پروتئین‌ها یا پلی‌پپتیدها 
\LTRfootnote{polypeptide}
القا می‌کند.  مرکز غشا از نواری از دم‌های آب‌گریز تشکیل شده. پروتئینی هم که درون غشا قرار گرفته قسمت آبگریز خود را درون غشا جا داده و قسمت آب‌گریزش را بیرون. کافی‌است که قسمت آب‌گریز پروتئین  کمی از ضخامت نوار آب‌گریز لیپید‌های  بیشتر یا کمتر باشد که ضخامت غشا را تغییر دهد
\cite{Mouritsen1984}. 
در این حالت یا پروتئین باید تغییر شکل دهد که خود را با ضخامت غشا تنظیم کند، یا هر دو (شکل 
\ref{fig:BilayerPlusProtein}
) که در هر حالت باعث بروز برهمکنش‌‌های پروتئین-لیپید و لیپید-لیپید می‌شود
\cite{Huang1986,Aranda-Espinoza1996,Safran2000,Haselwandter2013,Haselwandter-Christoph2013}
.

\begin{figure}[h]
\begin{center}
\includegraphics[width=4in]{\MemBio /Pics/BilayerPlusProtein}
\caption{
نقاشی از برش یک غشای لیپیدی که حاوی نوعی ناخالصی (مانند پروتئین یا پلی‌پپتید) است. ملکول‌های لیپید با دایره‌های سفید (سر آب‌دوست) و دم‌های آب‌گریز، و  ناخالصی‌ها به شکل مستطیل‌های دارای سر‌های آب‌دوست و ناحیه‌ی میانی هاشور خورده‌ی آب‌گریز نمایش داده شده‌است. اگر ناحیه‌ی آب‌گریز ناخالصی نسبت به غشا ضخیم‌تر (الف) یا نازک‌تر (ب) باشد، ضخامت غشا تحت تاثیر قرار می‌گیرد
\cite{Mouritsen1984}
.
}
\label{fig:BilayerPlusProtein}
\end{center}
\end{figure}




مدل موزائیکی مدل خوبی است که مشکلات حرکتی لیپید‌ها و پروتئین‌های چسبیده به غشا را توضیح می‌دهد
\cite{Simons2000,Simons1997}.


 غشاهایی هم می‌توان یافت که درصد پروتئین و کربوهیدارت در ساختار غشای آن به ترتیب بین ۱۸ تا ۷۵ درصد و  ۳ تا ۱۰ درصد باشد
\cite{MembraneProteins1972}.




با وجود اینکه غشای چربی به طور خود سامانده تشکیل می‌شود ولی علاوه بر ملکول‌های چربی پروتئین‌های زیادی نیز به غشا ساختار می‌بخشد
\cite{wikiCellMembrane}
. غشا از طریق این پروتئین‌ها به اجزای پیچیده‌تر داخل سلول (مانند اسکلت سلولی) متصل است. خاصیت تراوایی فوسفولیپید دو لایه و کانال‌های پروتئینی درون غشا، ارتباط سلول با محیط اطراف را کنترل و مدیریت می‌کند. 






\section{
ساختارهای خودسامانده غشا
}
 
ملکول‌های لیپید یا چربی یکی از ۴ عناصری است که در کنار آمینو اسید‌ها، نوکلئیک اسید‌ها، و ملکول‌های قندی ساختار موجودات زنده را تشکیل می‌دهد که از این میان فقط لیپید‌ها پلیمر نیستند
\cite{Membraneasamatteroffat}
. بیش از هزار نوع ملکول چربی در گونه‌های زیستی وجود دارد ولی ساختار کلی آنها بسیار مشابه است. در سلول‌ پستانداران بیشتر فسفولیپید و گلیسرول یافت می‌شود. ملکول‌ فوسفولیپید از یک سر آب دوست
\LTRfootnote{hydrophilic}  
 و یک دُم آب گریز
 \LTRfootnote{hydrophobic}  
 ساخته شده‌است (شکل
\ref{fig:bilayer}
). فرق بین ملکول‌های لیپید مختلف در ساختار شیمایی سر آب دوست و دُم آب‌ گریز آنهاست. این ملکول‌ها در محلول‌های آبی
\textbf{بدون ایجاد پیوندها شیمیایی}
، به طور خود سامانده
\LTRfootnote{self-assembly}  
 ساختار‌های بسیار متنوعی تشکیل می‌دهند. اساس این ساختار‌ها کمینه کردن انرژی آزاد سیستم از طریق محافظت کردن دُم‌های آبگریز از آب است که با افزایش غلظت ملکول‌های لیپید ساختار‌هایی مختلفی تشکیل می‌دهند
 (شکل
\ref{fig:bilayer}
 )
. برای مثال مایسِل‌ها 
 \LTRfootnote{Micelle}
 در محلول‌های لیپدیدی با غلظت‌ پایین (ولی بالاتر از یک غلظت حدی) تشکیل می‌شود
 \cite{Lipowskyb1995ook}
 . در این ساختار انتهای تمام دُم‌های آب‌گریز در کنار هم قرار گرفته و سَر‌های آب‌دوست کُره تشکیل می‌دهند. هنگامی‌ که غلظت لیپید‌ها بیشتر شود یک تغییر حالت از مایسِل به ساختارهای دیگر مانند استوانه خواهیم دید. همچنین ساختار‌های ترکیبی مانند در کنارهم قرار گرفتن استوانه‌ها هندسه‌های شش ضلعی با فاصله‌های 
 $1-5nm$
 بسته به غلظت نسبی. 
 
 فرآوان‌ترین ساختار لیپید‌ها ساختار دو-لایه ‌است (شکل
 \ref{fig:bilayer}
 ب).  در ساختار‌های دو-لایه دُم‌های  آبگریز در مرکز لایه (به دور از آب) و سر آب دوست به سمت محلول جهت‌گیری می‌کند. از آنجایی که لبه‌های این سطوح شامل دُم‌های آب‌گریز است، به لحاظ انرژی هزینه‌بر است و به همین علت این سطوح هندسه‌های بسته تشکیل می‌دهند، مانند لیپوزم
  \LTRfootnote{Liposome}
 یا ساختار‌های ترکیبی مانند غشا‌هایی که از چندین دو-لایه تشکیل شده‌اند
\cite{LifeAsaMatterofFat2005}.

لازم به ذکر است که عوامل دیگری مانند دما و ترکیبات شیمیایی محلول بر نحوه‌ی تشکیل ساختار‌های لیپیدی تاثیر می‌گذارند. از آنجایی که دُم لیپید‌ها شکل ثابتی ندارد نمی‌توان حجم مشخصی برای این ملکول معین کرد. ولی تاثیر ترکیبات محلول، دما، و سایر عوامل موثر بر رفتار این ملکول را می‌توان با در نظر گرفتن حجم متوسط،
$v$
، سهم سطح، 
$a$
، و عمق اشغال شده ملکول درون ساختار،
$\ell$
، مُدل کرد. این پارامتر‌ها با اندازه‌گیری روی اندازه‌ی سر دوقطبی، طول د‌ُم اسید چرب و حل‌شوندگی دُم در محلول تنظیم کرد
\cite{LifeAsaMatterofFat2005}
. مثلا بالا رفتن دما  مُد‌های چرخشی زنجیر کربنی حول محورش را افزاریش داده و در نتیجه سهم مساحت اشغالی آن بالا می‌رود
\cite{BiomembranesBook1989}
و همین امر می‌تواند منجر به ذوب شدن غشا شود
\cite{BioMemBook2007}
. اسرائیلاچویلی
\LTRfootnote{LiposomeIsraelachvili}
و همکارانش در یک مقاله‌ی معروف در سال ۱۹۷۶
\cite{Israelachvili1976}
، با یک ضرب و تقسیم سر انگشتی تاثیر شکل ملکول‌های لیپیدی را توصیف می‌کنند. امکان قرار گرفتن یک ملکول لیپید در ساختاری مشخص با عدد بسته‌بندی یا پَکینگ
\LTRfootnote{packing}
مشخص می‌شود:
\begin{equation}
%\begin\centering
P=\frac{v}{a\ell}
%\end\centering
\end{equation}
مثلا برای مایسِلی به شعاع، 
$R_m$
، حجم، 
$v=\frac{1}{N}\frac{4\pi R_m^3}{3}$
، و سطح 
$a=\frac{1}{N}4\pi R_m^2$ 
تخمین زده می‌شود. که در اینجا،
$N$
، تعداد ملکول‌ها در ساختار مایسِل است. از آنجایی که طول دُم ملکول نمی‌تواند از شعاع مایسِل بزرگتر باشد (
$\ell\leq R_m$
):
\begin{equation}
%\begin\centering
P_m=\frac{v}{a\ell}=\frac{\frac{1}{N}\frac{4\pi R_m^3}{3}}{\frac{1}{N}4\pi R_m^2\ell}=\frac{1}{3}\frac{R_m}{\ell}\leq\frac{1}{3}
%\end\centering
\end{equation}
پس عدد پَکینگ باید کمتر از 
$\frac{1}{3}$
باشد تا ساختار‌‌های مایسِل پایدار داشته باشیم. این محاسبات برای  ساختارهای استوانه‌ای به شعاع، 
$R_c$
، و طول بلند، 
$L\gg R_c$
به این شکل انجام می‌شود،
\begin{equation}
%\begin\centering
P_c=\frac{v}{a\ell}=\frac{\frac{1}{N}\pi LR_c^2}{\frac{1}{N}2\pi LR_c\ell}=\frac{1}{2}\frac{R_c}{\ell}\leq\frac{1}{2}
%\end\centering
\end{equation}
در نتیجه ساختار استوانه‌ای پایدار در عدد پَکینگ 
$\frac{1}{3}<P<\frac{1}{2}$
ایجاد می‌شود. با تکرار محاسبات مشابه عدد عدد پَکینگ برای ساختار‌های دو لایه
$\frac{1}{2}<P<1$
تخمین زده می شود. لیپید‌هایی که از غشاهای زیستی استخراج می‌شود بیشتر 
$P>1$
دارند، در نتیجه‌ شکل کلی آن‌ها شبیه به یک مخروط وارونه است. در نتیجه دو-لایه‌هایی که فقط از ملکول‌های لیپید زیستی ساخته شده باشد تنش‌ها خمشی زیادی خواهد داشت
\cite{Mouritsen2011,Membraneasamatteroffat}
. این تنش‌ها با اضافه شدن پروتئین‌ها و تشکیل حباب‌های کوچک (هم رو به داخل
\LTRfootnote{endocytosis}
  هم رو به بیرون
 \LTRfootnote{exocytosis}
  ( بر روی غشای سلول آزاد می‌شود.



\begin{figure}[h]
\begin{center}
\includegraphics[width=4in]{\MemBio /Pics/Bilayer}
\caption{
الف) ساختار شیمیایی یک ملکول فوسفولیپید. سر آب دوست (دایره‌ی آبی) و  انتهای آبگریز مشخص شده است. ب) ساختار‌های معمول ملکول‌های چربی در آب. به ترتیب از چپ به راست، ساختار سطوح بزرگ دو لایه، کره‌های دو لایه (لیپوزوم)، و کره‌های کوچک تک لایه، مایسِل.
}
\label{fig:bilayer}
\end{center}
\end{figure}




\begin{figure}[h]
\begin{center}
\includegraphics[width=4in]{\MemBio /Pics/GUVTempChange}
\caption{
تغییر ساختار یک غشای غول ‌آسا به علت تغییر دما از ۲۷/۲ تا ۴۱ درجه‌ی سانتیگراد. در دمای ۳۶ درجه حالت بیضی شکل، بالاتر از دمای ۳۶ شکل گلابی، و با ماندن در دمای ۴۱ درجه یک حباب بر روی آن جدا شده.
}
\label{fig:GUVTempChange}
\end{center}
\end{figure}






با وجود پیچیدگی‌ که غشای سلول‌های زیستی دارند، باید از قوانین فیزیکی حاکم بر اجسام بی‌جان پیروی کنند. از طرفی غشا‌ها بیشتر دارای ساختارهای متنوع و خاص هستند تا ساختارهای مشترک که بتوان عمومیت داد
\cite{NelsonBook2004}.
ولی می‌توان خواص فیزیکی غشاها را با قوانین ترمودینامیک به صورت درشت دانه مدل کرد و رفتار‌ آنها را در طیفی از هندسه‌ها و مقیاس‌ها مشخص کرد
\cite{Seifert1997}
. مثلا انرژی مشخصه اندازه‌گیری شده برای تجمع واحد‌های تشکیل‌ دهنده‌ی غشاهای زیستی یک مرتبه‌ی بزرگی از انرژی گرمایی محیط 
$k_BT$
بیشتر گزارش شده. در اینجا، 
$k_B$،
ثابت بُلتسمَن و 
$T\approx300K$،
دمای محیط است. در نتیجه ساختار غشاهای زیستی در مقابل افت و خیز گرمایی در نزدیکی دماهای فیزیولوژیکی پایدار است. همچنین آنقدر نرم  هستند که توسط فرآیندهای زیستی مانند اتصال پروتئین‌ها
\cite{NelsonBook2004,Seifert1997,Deserno2015}،
ATP
هیدرولسیز
\LTRfootnote{ATP hydrolysis}
\cite{Boyle2008Biology,Lipowskyb1995ook}.
یکی از خواص مهم و جالب غشاها مقاومت خیلی کمِ خم شدنِ آن‌ها تحت نیروی خارجی است. این پدیده، افت و خیز دیواره‌ی سلولی، با میکروسکوپ نوری به راحتی قابل مشاهده است
\cite{NelsonBook2004}.
جذابیت دیگر غشا‌ها برای فیزیکدان‌ها ضخامت خیلی کم آن‌هاست که در نتیجه می‌توان آن را با یک صفحه‌ی ۲ بعدی مدل‌سازی کرد و ابعاد مسئله‌ را کاهش داد
\cite{Seifert1997,Deserno2015}.
درنتیجه‌ی مطالعه‌ی غشاها با استفاده از این نوع ساده‌سازی و مدل سازی می‌توانیم  رفتار آن‌ها را در مقیاس‌های مختلف پیش‌بینی کرده و نقش‌ آن‌ها را در فرآیند‌های زیستی توصیف کنیم.



محققان با مطالعه‌ی غشا‌های غول‌آسا
\LTRfootnote{Giant Unilamellar vesicle}  
 یا 
 GUV اطلاعات  زیادی راجع به ساماندهی غشاهای چربی جمع‌آوری کرده‌اند. این غشا‌ها را معمولا می‌توان با  مخلوط کردن 
\LTRfootnote{mixing}  
 غشا‌ و ترکیب‌های چربی در آزمایشگاه ساخت
 \cite{GUVmaking2009}
و اندازه‌ی آن  از چند میلی‌متر تا چند میکرون  است. 
GUV نسبت به تغییرات ترمودینامیکی محیط واکنش‌های بسیار جالبی نشان می‌دهد. برای مثال شکل
\ref{fig:GUVTempChange}
ایجاد یک حباب کوچک بر روی یک GUV را با تغییر دمای محیط از ۲۷ تا ۴۱ درجه در قالب ۶ سری عکس پشت سر هم نشان می‌دهد
\cite{MemReviewRamakrishnan2014}.
 
\begin{figure}[h]
\begin{center}
\includegraphics[width=4in]{\MemBio /Pics/GUVPresChange}
\caption{
مسیر‌های مختلفی که یک غشای غول آسای آب نباتی شکل (ستون سمت چپ، زاویه‌ی دوربین از بالا و کنار) که بر اثر تغییر غلظت نمک محیط طی می‌کند تا به شکل خیلی کشیده (ستون سمت راست) در بیاید، را نشان می‌دهد.
}
\label{fig:GUVPresChange}
\end{center}
\end{figure}

GUV همچنین نسبت به تغییرات فشار اسمزی محیط نیز واکنش نشان می‌دهد. مثلا در شکل 
\ref{fig:GUVPresChange}
می‌بینیم که با تغییر غلظت نمک در محیط یک غشایِ لیپیدیِ دارای کلسترول، از شکل اولیه آب نباتیِ 
\LTRfootnote{biconcave}  
شبیه‌ به گلبول قرمز به حالت کشیده و لوله‌ای در می‌آید. در شکل 
\ref{fig:GUVPresChange}
ساختار‌های هندسی میاینی (و در مواردی ناپایدار) مختلفی که غشا طی می‌کند تا از هندسه‌های سمت چپ به هندسه‌های سمت راست برسد، را می‌توان مشاهده کرد.
  
 
 
 
 
 
 
 
 
 
 