\setRL
%\pagenumbering{arabic} 


\section{\label{sec:cellmembrane}
غشای سلولی
}
غشاها ساختارهای زیستی هستند که در سامانه‌های مختلف زیستی سلول‌های هسته‌دار و بی هسته  نقش دارند. غشا‌ها در واقع دیوار یا مانعی‌است که مرز داخل و خارج سلول را مشخص می‌کند و در سلول‌ها هسته‌دار اورگان‌های خیلی مهم سلول را در خود جا می‌دهد. به عنوان مرز سلول با دینای خارج، غشا نقش مهمی در جابجایی و مدیریت نقل و انتقال مواد بین سلول و دنیای بیرون از سلول را دارد. درون غشا پروتئین‌ها، لیگاند‌ها
\LTRfootnote{ligand} 
،
و ملکول‌های درشت خیلی زیادی وجود دارد که در طیف‌ گسترده‌ای از فرآیندها مانند سیگنال‌دادن سلول تا تقسیم سلولی نقش دارند.
\begin{figure}[h]
\begin{center}
\includegraphics[width=4in]{\MemBio /Pics/Cell_membrane_detailed_diagram}
\caption{
شکل از سایت ویکیپیدیا گرفته شده است
\cite{wikiCellMembrane}
. نقاشی غشا بر اساس مدل غشای مایع موزایکی است. بیشتر غشا از ملکول‌های چربی تشکیل شده ولی پروتئین‌های خیلی زیادی در غشا قرار دارند. همچنین غشا از طریق این پروتئین‌ها به اسکلت سلولی و اجزای دیگر متصل است. دایره‌های قرمز سر آب دوست و رشته‌های زرد دم‌های آب گریز لیپید‌ها را نشان می‌دهد.
}
\label{fig:fluidmembranemodel}
\end{center}
\end{figure}

زحمات افراد در قرن ۱۹ و ۲۰ میلادی برای پی بردن به ساز و کار واحد‌های سازنده‌ی موجودات زنده، تصویری پر از جزئیات از سلول زنده ساخته است. در سال ۱۹۸۵ ارنست اُورتن 
\LTRfootnote{Ernest Overton}  
سلول‌های گیاهی را در محلول‌های مختلف (قند،‌ الکل، اتر، فنول، و استن
\LTRfootnote{sugar, alcohols, ether, phenol, and acetone}
) قرار داد. مشاهدات وی نشان داد که (تحت اختلاف فشار اسمزی یکسان) محلول‌هایی مانند قند که در آب به راحتی حل می‌شوند نمی‌توانند وارد سلول شوند در صورتی که محلول‌هایی که در آب حل شوندگی خوبی ندارند، می‌توانند وارد سلول شوند. او نتیجه گرفت که جنس مرز سلول با سیتوپلاسم درون آن متفاوت است و به احتمال زیاد از ملکول‌های چربی گون تشکیل شده است
\cite{overton1985}
. در سال ۱۹۳۲ (حدود صد سال پیش) کِنِت کول
 \LTRfootnote{Kenneth Cole}
 تخم جوجه‌تیغی دریایی
 \LTRfootnote{sea urchin (arbacia) egg} 
 را با یک فیبر از جنس طلا بر روی یک سطح فشار داد. با اندازه گیری  کشش سطحی و حد تحمل فشار تخم، استدلال کرد که لایه‌ی نازک اطراف سلول فقط از ملکول‌های چربی درست نشده است
 \cite{Cole1932}
. در دهه‌ی ۱۹۵۰ با تکنولوژی میکروسکوپ الکترونی، اولین تصاویر از غشای سلول رونمایی شد و با استفاده از روش‌های رنگ آمیزی، وجود لیپید‌ها
\LTRfootnote{lipid}
 در غشا تایید شد. 
ضخامت غشای سلولی بین ۵ تا ۱۰ نانومتر اندازه‌گیری شد
\cite{ROBERTSON1959aa}
. افراد زیادی در تکمیل‌ کردن تصویر غشای سلول  نقش داشتند، ولی تصویر مدرنی که ما از غشاهای سلول‌ها زنده داریم بیشتر بر پایه‌ی مدل غشای مایع موزایکی‌
\LTRfootnote{the mosaic fluid model of membranes}
 است که در سال ۱۹۷۲ توسط سینگر
 \LTRfootnote{Singer}
  و نیکلسون
  \LTRfootnote{Nicholson}
 ارائه شد
\cite{Singer1972}
(شکل 
\ref{fig:fluidmembranemodel}
). بنا بر این مدل غشا را می‌توان یک مایع دو بعدی فرض کرد که در آن چربی‌ها و پروتئین‌ها می‌توانند کم و بیش آزادانه حرکت کنند. در نتیجه هیچ نظم بلند بردی در غشا دیده نمی‌شود. با وجود اینکه غشای چربی به طور خود سامانده تشکیل می‌شود ولی علاوه بر آن پروتئین‌های زیادی نیز وجود دارد که به غشا ساختار می‌بخشند
\cite{wikiCellMembrane}
. غشا از طریق این پروتئین‌ها به اجزای پیچیده‌تر داخل سلول (مانند اسکلت سلولی) متصل است. خاصیت تراوایی فوسفولیپید دو لایه و کانال‌های پروتئینی درون غشا، ارتباط سلول با محیط اطراف را کنترل و مدیریت می‌کند. 







 
% \begin{figure}[h]
%\begin{center}
%\includegraphics[width=4in]{\MemBio /Pics/Phospholipid}
%\caption{
%ساختار شیمیایی یک ملکول فوسفولیپید. سر آب دوست (دایره‌ی آبی) و  انتهای آبگریز مشخص شده است.
%}
%\label{fig:phospholipid}
%\end{center}
%\end{figure}
 
 
ملکول‌های لیپید یا چربی یکی از ۴ عناصری است که در کنار آمینو اسید‌ها، نوکلئیک اسید‌ها، و ملکول‌های قندی ساختار موجودات زنده را تشکیل می‌دهد
\cite{Membraneasamatteroffat}
.  بیش از هزار نوع ملکول چربی در گونه‌های زیستی وجود دارد ولی ساختار کلی آنها بسیار مشابه است. در سلول‌ پستانداران بیشتر فسفولیپید و گلیسرول یافت می‌شود. ملکول‌ فوسفولیپید از یک سر آب دوست
\LTRfootnote{hydrophilic}  
 و یک دُم آب گریز
 \LTRfootnote{hydrophobic}  
 ساخته شده‌است (شکل
\ref{fig:phospholipid}
). فرق ملکول‌های لیپید در ساختار شیمایی سر آب دوست و دُم آب‌ گریز آنهاست. این ملکول‌ها در محلول‌های آبی
\textbf{بدون ایجاد پیوندها شیمیایی}
، به طور خود سامانده
  \LTRfootnote{self-assembly}  
 سطوح بزرگ دولایه تشکیل دهند.  دُم‌های  آبگریز در مرکز لایه (به دور از آب) و سر آب دوست به سمت محلول جهت‌گیری می‌کنند (شکل
\ref{fig:bilayer}
 ).
\begin{figure}[h]
\begin{center}
\includegraphics[width=4in]{\MemBio /Pics/Bilayer}
\caption{
الف) ساختار شیمیایی یک ملکول فوسفولیپید. سر آب دوست (دایره‌ی آبی) و  انتهای آبگریز مشخص شده است. ب) ساختار‌های معمول ملکول‌های چربی در آب. به ترتیب از چپ به راست، ساختار سطوح بزرگ دو لایه، کره‌های دو لایه (لیپوزوم)، و کره‌های کوچک تک لایه، مایسِل.
}
\label{fig:bilayer}
\end{center}
\end{figure}
البته با گسترش تحقیقات ما می‌دانیم غشا‌های زیستی بیشتر شکل موزائیکی دارند تا مایع. غشاها موزائیکی هستند به این معنی که در سلول‌های مختلف غشا خیلی همگن نیست و در سلول‌ها قسمت‌هایی از غشا دیده‌ می‌شود که ترکیب پروتئینی متفاوتی از بخش‌های دیگر داشته باشند و یا ضخامت برخی از بخش‌های غشا ممکن است از چند ۱۰ نانومتر تا ۱۰۰ نانومتر تغییر کند
\cite{Engelman:2005aa}
. همچنین بسته به نوع سلول غشا می‌تواند از ۱۸ تا ۷۵ درصد از پروتئین ساخته شده باشد همچنین میزان کربوهیدرات در ساختار آن بین ۳ تا ۱۰ درصد تغییر می‌کند
\cite{MembraneProteins1972}
. محققان با مطالعه‌ی غشا‌های غول‌آسا
\LTRfootnote{Giant Unilamellar vesicle}  
 یا 
 GUVها اطلاعات خیلی زیادی راجع به ساماندهی غشاهای چربی جمع‌آوری کرده‌اند. این غشا‌ها معمولا با هم‌زدن 
\LTRfootnote{mixing}  
 غشا‌ها و ترکیب‌های چربی در آزمایشگاه ساخته می‌شوند
 \cite{GUVmaking2009}
و اندازه‌ی آنها از چند میلی‌متر تا چند میکرون می‌تواند باشد. 
GUVها نسبت به تغییرات ترمودینامیکی محیط واکنش‌های بسیار جالبی نشان می‌دهند. برای مثال شکل
\ref{fig:GUVTempChange}
ایجاد یک حباب کوچک بر روی یک GUV را با تغییر دمای محیط از ۲۷ تا ۴۱ درجه به شکل ۶ سری عکس نشان می‌دهد
\cite{MemReviewRamakrishnan2014}
.
 \begin{figure}[h]
\begin{center}
\includegraphics[width=4in]{\MemBio /Pics/GUVTempChange}
\caption{
تغییر ساختار یک غشای غول ‌آسا به علت تغییر دما از ۲۷.۲ تا ۴۱ درجه‌ی سانتیگراد. در دمای ۳۶ درجه حالت بیضی شکل به خود می‌گیرد. بالاتر از دمای ۳۶، شکل گلابی به خود می‌گیرد. و با ماندن در دمای ۴۱ درجه شاهد ایجاد یک حباب هستیم.
}
\label{fig:GUVTempChange}
\end{center}
\end{figure}
GUVها همچنین نسبت به تغییرات فشار اسمزی محیط نیز واکنش نشان می‌دهند. مثلا در شکل 
\ref{fig:GUVPresChange}
می‌بینیم که با تغییر غلظت نمک در محیط یک غشای لیپیدی بسته دارای کلسترول، از شکل اولیه آب نباتی شکل 
\LTRfootnote{biconcave}  
شبیه‌ به گلبول قرمز به حالت کشیده و لوله‌ای در می‌آید. در این شکل ساختار‌های هندسی میاینی مختلفی که غشا طی می‌کند دیده می‌شود.
 \begin{figure}[h]
\begin{center}
\includegraphics[width=4in]{\MemBio /Pics/GUVPresChange}
\caption{
مسیر‌های مختلفی که یک غشای غول آسای آب نبات شکل (ستون سمت چپ نگاه از بالا و کنار) بر اثر تغییر غلظت نمک محیط طی می‌کند تا به شکل خیلی کشیده (ستون سمت راست) در بیاید.
}
\label{fig:GUVPresChange}
\end{center}
\end{figure}
 
 
 
 
 
 
 
 
 
 
 