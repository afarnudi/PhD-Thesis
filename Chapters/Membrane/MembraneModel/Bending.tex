\setRL
%\pagenumbering{arabic} 


\subsection{
انرژی خمش سطح
}
انرژی خمش یک سطح را می‌توان با انرژي هلفریش
\cite{Helfrich1973}
 کمی کرد،
\begin{equation}
E_{bending}=\int dS\left\{\frac{1}{2}\kappa (H-H_s)^2 +\tilde \kappa K_0\right\}
\label{eq:helfrish}
\end{equation}
در اینجا
\begin{equation}
H = \frac{1}{R_1}+\frac{1}{R_2}
\end{equation}
خمش سطح است که با شعاع دو دایره‌ی مماس بر عنصر سطح بیان می‌شوند (
\ref{fig:elasticdeformation}
). 
$H_s$
خمش زاتی سطح را مشخص می‌کند که همانند خمش،
$H$
تعریف می‌شود. برای مثال خمش زاتی سطحی که در تمام جهت‌ها علاقه دارد شعاع 
$R_s$
داشته باشد، 
\begin{equation}
H_s = \frac{2}{R_s}
\end{equation}
است. 
$K_0$
خمش گاووسی است که به شکل 
\begin{equation}
H_s = \frac{2}{R_s}
\end{equation}
تعریف می‌شود. همچنین 
$\kappa$
و
$\tilde\kappa$
به ترتیب سختی خمشی و سختی خمش گاووسی است. بنا به قضیه گاووس-بونت
\LTRfootnote{Gauss–Bonnet}
انتگرال روی سطح خمش گاووسی پاسخی ساده دارد،
\begin{equation}
\int dS \tilde \kappa K_0=4\pi\tilde\kappa(1-g)
\end{equation}
که در معادله‌ی بالا 
$g$
جینوس 
\LTRfootnote{genus}
سطح، یا تعداد سوراخ یا تعداد دسته‌
\LTRfootnote{handle}
است. اگر پوسته‌ی مورد نظر در طول مطالعه تغییر توپولوژی ندهد حاصل این انتگرال همیشه ‌یک عدد ثابت خواهد بود. در صورتی علاقه‌ی ما محاسبه‌ی نیرو‌های خمشی (مشتق جمله انرژي) یا اختلاف انرژی خمشی باشد، جمله‌ی ثابت خمش گاووسی در محاسبات اهمیت نخواهد داشت.
\subsubsection{
محاسبه‌ی انرژی خمش کره
}
برای مثال انرژی خمش یک کره به شعاع 
$R$
را با رابطه‌ی هلفریش محاسبه می‌کنیم. معادله‌ی 
\ref{eq:helfrish}
به شکل زیر در می‌آید:
\begin{equation}
\begin{aligned}
E_{bending}&=\int dS\left\{2\kappa \left(\frac{1}{R}-\frac{1}{R_s}\right)^2 +\tilde \kappa K_0\right\} \\
&=2\kappa\int \left(\frac{1}{R}-\frac{1}{R_s}\right)^2dS +4\pi\tilde \kappa
\end{aligned}
\end{equation}
در صورتی که کره را از ماده‌ای ساخته باشیم که به طور ذاتی علاقه داشته باشد که یک سطح تخت باشد،‌
$R_s\rightarrow\infty$
انرژی خمش مقدار ثابت خواهد بود:
\begin{equation}
E_{bending}|_{R_s\rightarrow\infty}=4\pi(2\kappa+\tilde\kappa)
\end{equation} 


.
 
 
 
 
 
 
 
 
 
 
 
 
 
 
 