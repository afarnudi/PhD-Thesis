\setRL
%\pagenumbering{arabic} 


\subsection{
انرژی کشش در سطح
}
اگر فرض کنیم جابجایی روی یک عنصر سطحی حاصل از کشیده‌ یا فشرده شدن سطح با بردار 
$u$
توصیف شود، با فرض خطی بودن عکس العمل ماده، انرژی پتانسیل حاصل از تغییر شکل سطح را می‌توان با معادله‌ی زیر بررسی کنیم.

\begin{equation}
E_{stretching}=\frac{1}{2}Y_{2D}A\varepsilon^2
\end{equation}
که اینجا 
$Y_2D$
مدول دو بعدی یانگ،
$A$
سطح عنصر در حالت کشیده نشده، و
$\varepsilon$
تانسور کرنش است. تانسور کرنش برای سطح دو بعدی به شکل زیر تعریف می‌شود:
\begin{equation}
\varepsilon_{ij} = \frac{1}{2}(u_{ij}+u_{ji})
\end{equation}


.
 
 
 
 
 
 
 
 
 
 
 
 
 
 
 