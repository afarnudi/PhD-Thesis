در حالت تعادلی، مولکول‌های لیپیدی، با توجه به شرایط ترمودینامیکی محیط، سطح غشا را با چیدمانی بهینه‌ می‌پوشانند.  به علت وجود نیروهای خارجی یا قیود مختلف،  سطح غشا از سطح تعادلی
$A_0$
تغییر کرده، و در سطح غشا تنش  
$\gamma_{st}(A)$
ایجاد خواهد شد. از آنجایی که غشا ماهیت سیال‌گون دارد،  تنشی که حاصل آن تنها جابجایی مولکول‌ها روی سطح باشد (تنش برشی\LTRfootnote{shear})
 انرژی سطح را تغییر نخواهد داد. تنها تنشی که مساحت کل غشا را تغییر دهد با مقاومت روبرو خواهد شد. تنش غشا تا مرتبه‌ی اول در جمله‌ی 
$(A-A_0)$
به شکل زیر تعریف می‌شود،
\begin{equation}
\gamma_{st}(A)=k_A\frac{A-A_0}{A_0}.
\end{equation}
در این معادله
$k_A$
مدول فشردگی سطحی\LTRfootnote{area compressibility modulus}
است
\cite{thegiantvesiclebook2019}.
 بدیهی‌است که این رابطه تا زمانی که تنش ایجاد شده از  آستانه‌ی پاره شدن غشا کمتر باشد قابل استفاده است. برای غشاهای لیپیدی، آستانه‌ی پاره شدن دو مرتبه‌ی بزرگی کوچک‌تر از مدول فشردگی سطحی و حدود چند
$mN/m$
است. انرژی تنش ایجاد شده حاصل از تغییر مساحت غشا، معادل کار انجام شده برای تغییر سطح است،
\begin{equation}
E_{st}(A)=\int_{A_0}^A dA~\gamma_{st}(A)=\frac{1}{2}k_A\frac{(A-A_0)^2}{A_0}.
\label{eq:surfaceTension}
\end{equation}
مشابه به این بحث می‌توان هزینه‌ی انرژی برای تغییر حجم سیال درون غشا را نیز به صورت زیر تعریف کرد
\cite{discher1998biophysicaljournal},
\begin{equation}
E_{v}(V)=\frac{1}{2}k_V\frac{(V-V_0)^2}{V_0}.
\label{eq:volumeEnergy}
\end{equation}
  در اینجا
$k_V$
مدول فشردگی حجمی\LTRfootnote{volume compressibility modulus}
 و 
$V_0$
حجم تعادلی سیال درون غشا است.







