همانطور که قبل‌تر توضیح داده شد سطح غشا بسته به شرایط ترمودینامیکی محیط یک مقدار تعادلی 

دارد. هر پدیده‌ای که سطح غشا را تغییر دهد در آن تنش 
$\gamma_{st}(A)$
ایجاد خواهد کرد. باید توجه ویژه داشت که غشا ماهیت سیال‌گون دارد، در نتیجه تنش‌هایی که تنها باعث جابجایی ملکول‌ها روی سطح شود (تنش برشی
\LTRfootnote{shear}
) انرژی سطح غشا را تغییر نمی‌دهد. تنها تنش‌هایی که مساحت کل غشا را تغییر دهد با مقاومت روبرو خواهد شد. تنش غشا تا مرتبه‌ی اول در جمله‌ی 
$(A-A_0)$
به شکل زیر تعریف می‌شود،
\begin{equation}
\gamma_{st}=k_A\frac{A-A_0}{A_0}
\end{equation}
. در این معادله
$k_A$
مدول فشردگی سطحی
\LTRfootnote{area compressibility modulus}
است. بدیحی‌است که  تنش ایجاد شده باید از مقدار تنش آستانه‌ی پاره شدن غشا کمتر باشد. برای غشاهای لیپیدی مقدار تنش آستانه‌ی پاره شدن دو مرتبه‌ی بزرگی کوچک‌تر از مدول فشردگی سطحی و حدود چند
$mN/m$
است. انرژی تنش ایجاد شده معدل کار انجام شده برای تغییر سطح است،
\begin{equation}
E_{st}(A)=\int \gamma_{st}()=\frac{1}{2}k_A\frac{(A-A_0)^2}{A_0}
\label{eq:surfaceTension}
\end{equation}
مشابه به این بحث می‌توان هزینه‌ی انرژی برای تغییر حجم سیال درون غشا را نیز به صورت زیر تعریف کرد،
\begin{equation}
E_{v}(V)=\frac{1}{2}k_V\frac{(V-V_0)^2}{V_0}
\label{eq:volumeEnergy}
\end{equation}
. که در اینجا
$k_V$
مدول فشردگی حجمی
\LTRfootnote{volume compressibility modulus}
، و 
$V_0$
حجم تعادلی سیال است.







