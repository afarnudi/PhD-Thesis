تا به اینجا فرض شده که تک لایه‌های تشکیل دهنده‌ی غشا تمایلی به خم شدن در جهت مشخصی ندارند. شکل تعادلی موضعی چنین غشایی یک سطح تخت خواهد بود. کمتر غشایی در طبیعت با چنین تقارنی مشاهده می‌شود ولی دلیل مطالعه‌ی چنین سیستم از نظر مدل‌سازی بسیار پر بهره است چرا که تمام ویژگی‌های الاستیک آن توسط پارامتر سختی خمش
$\kappa$
تعیین می‌شود که مقیاس خوبی برای انرژی یک غشاست. برای غشاهای فسفولیپیدی در دمای اتاق مرتبه‌ی انرژی سختی خمش حدود
$10^{19}J$،
یا
$20k_BT$
است. سختی خمش برای غشاهای از جنس دیگر ممکن است تا حدود یک مرتبه‌ی بزرگی متفاوت باشد. به دلیل اختلاف در ترکیبات تک لایه‌های سازنده‌ی غشاهای زیستی
\cite{Meer2008}
، غشاهای دولایه معمولا نامتقارن هستند. معروف‌ترین مثال غشاهای تشکیل شده از گنگلیوساید 
$GM1$
\LTRfootnote{ganglioside GM1}  
است که از نوع گلایکولیپید‌هاست
\LTRfootnote{glycolopids}  
و در غشاهای سلول‌های عصبی پستانداران به طور فراوان یافت می‌شود. 
$GM1$
نقش لنگر را برای مواد سمی، باکتری‌ها، و ویروس‌ها بازی می‌کند
\cite{Ewers2010}
. نحوه‌ی تغییر خمش موضعی با تغییر غلظت 
$GM1$
به طور مفصل به شکل تجربی
\cite{Bhatia2018, Raktim2018}
و هم به شکل شبیه‌سازی
\cite{Raktim2018, Sreekumari2018}
مطالعه شده‌است. همچنین جهت‌گیری پروتئین‌های درون غشا، چسبیدن پروتئین‌های محیط بر سطح غشا معمولا خمش را تغییر می‌دهد. در نتیجه اگر ذرات زیادی به سطح غشا بچسبند، خمش ذاتی در سطح غشا،
$C_0$
، ایجاد خواهد شد
\cite{Lipowsky2002}
که تابع تعداد ذراتی‌ است که به تک‌لایه‌های مختلف غشا متصل شده باشند
\cite{Breidenich2000}
. مقدار خمش ذاتی بسیار متغییر است و از مقیاس موضعی 
$1/10nm$
تا مقیاس خود غشا
$1/50\mu m$
گزارش شده است.
