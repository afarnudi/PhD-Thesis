\begin{figure}[h]
\begin{center}
\includegraphics[width=4.5in]{\MemTB /Pics/polymorphism}
\caption{
تغییر شکل یک غشا حاصل تغییر دمای محیط. شکل غشا ابتدا دمبلی است 
$(D)$
سپس دو شکل میانی ستومتوسایت
$S_1$
و
$S_2$
تشکیل شده و در نهایت پس از اتصال دو بازو، دو کُره‌ی تو در تو 
$L^{sto}$
تشکیل می‌دهد 
\cite{Berndl1990EPL}.
}
\label{fig:allAtom}
\end{center}
\end{figure}

درسته که غشاها رفتار سیال‌گون دارند ولی رفتار آنها با یک قطره‌ی مایع متفاوت است. در غیاب نیروی خارجی و قیود، قطره‌ی مایع به منظور کاهش انرژی سطح مشترک آن با محیط به شکل کُره در می‌آید. برخلاف یک قطره، غشا می‌توانند اشکال مختلفی همچون دیسکوسایت
\LTRfootnote{discocyte}
، ستومَتوسایت
\LTRfootnote{stomatocyte}
، و  دَمبِلی داشته باشد. همچنین شرایط ترمودینامیکی محیط (فشار اسمزی و دما) می‌تواند باعث تغییر شکل آن شود. از آنجایی که ملکلول‌های لیپیدی حل‌شوندگی ناچیزی دارند می‌توان فرض کرد که تعداد ملکلول‌های لیپیدی هنگام تغییر شکل غشا ثابت است. علاوه بر آن فاصله‌ی تعادلی ملکلول‌های لیپیدی تابع دمای محیط است. از طرفی تغییر مساحت غشا بر اثر نیرو‌های خارجی یا قیود مختلف بیسار ناچیز است و تغییر مساحت زیاد باعث پاره شدن غشا می‌شود. در نتیجه هنگام تغییر فشار اسمزی (کاهش یا افزایش سیال داخل آن)  در دمای ثابت، مساحت غشا با تقریب بسیار خوبی ثابت است. به طور عمومی حجم غشا می‌تواند به مقدار خیلی زیادی کاهش یابد ولی هرگز نمی‌تواند از حجم یک کُره بیشتر شود. 

در صورتی که دمای محیط تغییر کند، هم مساحت غشا هم حجم سیال درون آن تغییر خواهد کرد. نسبت سطح به حجم حاصل از دمای جدید شکل تعادلی غشا را مشخص خواهد کرد. در صورتی که دمای محیط به میزان 
$\Delta T$
تغییر کند، سطح و حجم غشا از حالت تعادلی
$A_0$
و
$V_0$
به مقدار 
$\Delta A=\alpha_A\Delta T A_0$
و حجم سیال درون آن 
$\Delta V=\alpha_V\Delta T V_0$
تغییر می‌کند. برای یک غشای لیپیدی که سیال درون آن آب باشد،
$\alpha_A\approx2\times 10^{-3}/K$
و
$\alpha_V\approx2\times 10^{-4}/K$
. یک محاسبه‌ی سر انگشتی نشان می‌دهد که در صورت افزایش دمای محیط نسبت حجم به سطح غشا کاهش پیدا می‌کند. نمونه‌ای از تغییر شکل غشا حاصل از افزایش دمای محیط در شکل 

نمایش داده شده‌است.

