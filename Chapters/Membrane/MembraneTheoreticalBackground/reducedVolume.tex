همانطور که در فصل اول توضیح داده شد. غشاهای زیستی نمیه‌تراوا هستند. یعنی حجم آب درون آنها توسط دمای محیط، غلظت ملکول‌های محیط، و ملکول‌های درون آن یا به طور عمومی شرایط اسمزی محیط مشخص می‌شود. در صورتی که اختلاف فشار اسمزی از بیرون بالا باشد، غشا ممکن است مچاله شود، بر روی خود تا شود، و یا غشا‌های کچک‌تری تشکیل دهد. ملکول‌های لیپید بسته به شرایط محیطی (مثلا دما) با چیدمان مشخصی فضا را اشغال می‌کنند. با فرض اینکه تعداد ملکول‌های غشا تغییر نکند، سطح غشا در طول عمر آن ثابت خواهد بود. از طرفی تحت اختلاف فشار اسمزی منفی، غشا متورم خواهد شد. البته میزان تغییر سطح غشا فقط در حدود چند درصد است و در صورتی که نیاز به تغییر سطح بیشتری باشد، پاره خواهد شد. برای غشایی با مساحت
$A$
شعاع کُره‌ای معادل که همان مساحت را داشته باشد،
\begin{equation}
R_{ve}=\sqrt{\frac{A}{4\pi}}
\end{equation}
است. از آنجایی که بیشترین حجمی که غشا می‌تواند داشته باشد حجم یک کُره‌ است، حجم غشا همیشه کمتر مساوی این مقدار خواهد بود،
\begin{equation}
V\leq\frac{4\pi}{3}R_{ve}^3=\frac{4\pi}{3}\left(\frac{A}{4\pi}\right)^{\frac{3}{2}}
\end{equation}
در نتیجه می‌توان حجم کاهیده به شکل زیر تعریف کرد،
\begin{equation}
\nu=\frac{V}{\frac{4\pi}{3}R_{ve}^3}=6\sqrt{\pi}VA^{-\frac{3}{2}}
\label{eq:reducedVolume}
\end{equation}
که همیشه کوچک‌تر از یک است و در صورتی که غشا شکل کُره‌ای بی نقص داشته باشد برابر یک خواهد بود.
