\begin{figure}[h]
\begin{center}
\includegraphics[width=\columnwidth]{\MemMethod/Pics/dynamicTri}
\caption{
تغییر مثلث بندی مِش با تغییر موضعی جفت مثلث‌ها میان چهار نقطه. در حالت اولیه (سمت چپ) دو مثلث با رئوس
$ABC$
و
$DBC$
تعریف شده‌است. با تغییر ضلع مشترک بین دو مثلث از 
$BC$
به
$AD$
مثلث بندی جدید با رئوس
$BAD$
و 
$CAD$
تشکیل خواهد شد (سمت راست).
}
\label{fig:dynamicTri}
\end{center}
\end{figure}



روش  مثلث بندی دینامیک\LTRfootnote{dynamic triangulation}
ابتدا توسط دیوید بول\LTRfootnote{David Boal}
و همکارش 
\cite{Boal1992PRA}
در سال ۱۹۹۲برای مِش‌های مثلثی طراحی شد. هدف اصلی این روش، شبیه‌سازی رفتار سیال گون غشا با استفاده از شبکه‌های مثلثی بود. کمی‌ بعد در همان سال، گامپر و کرول با  شبیه‌سازی موفق غشاهای سیال گون، سبب محبوبیت این روش شدند 
\cite{Gompper1992Science}.
این روش را بسیار محبوب کرد. گامپر و کرول در این مطالعه از روش انحنای دو سطحی برای محاسبه‌ی انرژی انحنا استفاده کردند. همانطور که در بخش
\ref{sec:curvatureDiscDef}
توضیح داده شد، این روش برای محاسبه‌ی انحنای غلط است. در سال ۱۹۹۶ گامپر و کرول روش محاسبه‌ی ایتزیکسون را با روش مثلث بندی دینامیکی ترکیب کردند و با موفقیت رفتار سیال‌گون غشا را شبیه‌سازی کردند
\cite{gompper1996}.

در روش مثلث بندی دینامیک دو مثلث مجاور در نظر گرفته می‌شود. رئوس این دو مثلث مانند شکل 
\ref{fig:dynamicTri}
سمت چپ، چهار وجهی 
$ABCD$
را تشکیل خواهد داد. مثلث بندی در ابتدا دو مثلث 
$ABC$
و
$DBC$
را تعریف می‌کند. با حذف ضلع
$BC$
و ایجاد ضلع
$AD$
مثلث بندی جدید با مثلث‌های
$CAD$
و
$BAD$
تشکیل خواهد شد (شکل
\ref{fig:dynamicTri}
سمت راست). تغییر مثلث بندی، انرژی انحنا و کششی مِش را تغییر خواهد داد. همانطور که در شکل با رنگ‌های قرمز و سبز نمایش داده شده، همسایه‌های مثلث‌های آبی در این فرآیند تغییر خواهد کرد. در نتیجه علاوه بر تغییر زاویه‌ی میان مثلث‌های آبی در دو مثلث بندی، زاویه میان همسایه‌ها نیز تغییر خواهد کرد. به لحاظ انرژی کششی، در صورتی که طول ضلع 
$BC$
و
$AD$
متفاوت باشد، انرژی کششی نیز تغییر خواهد کرد. انتخاب مثلث بندی با یک وزن متروپلیس\LTRfootnote{Metropolis}
 انجام می‌شود. در این روش، ابتدا انرژی مثلث بندی در حالت اولیه محاسبه می‌شود
($E_i$),
سپس مثلث بندی تغییر داده می‌شود و انرژی مش در چیدمان جدید محاسبه ‌می‌شود
($E_f$).
 در صورتی که انرژی با مثلث بندی جدید کاهش پیدا کند مثلث بندی جدید حتما پذیرفته می‌شود. در صورتی که انرژی مثلث بندی جدید بیشتر باشد این چیدمان با یک وزن بولتزمن\LTRfootnote{Boltzman}
انتخاب یا رَد خواهد شد. 

به علت ماهیت این الگوریتم بهترین روش برای شبیه‌سازی آن استفاده از روش مانتی کارلو\LTRfootnote{Monte Carlo}
است. گامپر و کرول هنگام ترکیب الگوریتم مثلث بندی دینامیک با روش اندازه‌گیری انحنای ایتزیکسون، محدودیت‌های زیادی در انتخاب طول اضلاع قرار داد. نتیجه‌ی این محدودیت‌ها ایجاد توزیع یکنواخت نقاط بر سطح شبکه‌ی مثلثی و کنترل شکل و اندازه مثلث‌ها در جهت پایدار کردن روش ایتزیکسون بود.

\begin{figure}[h]
\begin{center}
\includegraphics[width=13cm]{\MemMethod/Pics/DT.pdf}
\caption{
نمایش یک مربع (a)، یک مستطیل (b)، و دو شکل غیر مرتبط (c) که همگلی از ۳۴۰ مثلث متساوی الاضلاع تشکیل شده‌اند.
}  
\label{fig:meshDT}
\end{center}
\end{figure} 


همچنین با استفاده از الگوریتم مثلث‌بندی دینامیک 
\cite{Boal1992PRA, Gompper1992Science},
می‌توان اتصالات مختلف
 $\cal G$
را نمونه‌گیری کرد. این الگوریتم مهم‌ترین و پر کاربرد‌ترین روش برای بازسازی مش است که برای شبیه‌سازی اشکال غشا‌ها استفاده می‌شود. با تعمیم این روش
\cite{Kohyama2003PRE},
می‌توان رفتارهای پیچیده‌ی سطوح مایع‌گون (شکل 
\ref{fig:meshDT}c)
 را نیز شبیه‌سازی کرد. با وجود آزادی زیادی که این روش برای مدل سازی غشا‌ها در اختیار ما قرار می‌دهد، این روش محدودیت‌هایی نیز دارد. قدرت اصلی این الگوریتم در تغییر اتصالات مش‌های سخت است. یعنی برای یک تغییر شکل ساده مربعی به مستطیلی (شکل 
\ref{fig:meshDT})
 باید چیدمان نقاط (و اتصالات میانشان) را با دینامیک موضعیِ پخشی تغییر داد. این کار بسیار زمانبر است.













