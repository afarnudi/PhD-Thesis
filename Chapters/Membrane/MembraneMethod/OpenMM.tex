در تحقیقات این رساله از موتور محاسباتی دینامیک ملکولی
OpenMM\cite{OpenMM2017}
جهت حل معادلات حرکت استفاده شده‌است. تمامی پتانسیل‌ها به شکلی بازنویسی شده‌اند که در این بسته‌ی نرم‌افزاری قابل پیاده‌سازی باشد. در این فصل نحوه‌ی محاسبه‌ی نیروی حاصل از پتانسیل‌ها ارائه نشده زیراکه 
OpenMM
با دانستن پتانسیل حاکم بر ذرات قادر به محاسبه‌ی نیرو است. 


شبیه‌سازی ذراتی که تحت معادله‌ی نیوتن حرکت می‌کنند، 
\begin{equation}
m_i\frac{d\vec v_i}{d_t}=\vec f_i.
\label{eq:newton}
\end{equation}
از روش انتگرال گیری پرش قورباغه‌‌ای ورله موجود در بسته‌ی نرم افزاری 
OpenMM
استفاده شد. جهت بررسی افت و خیز سطح، معادله‌ی حرکت لانژون برای محاسبه‌ی نیرو استفاده شد
 \begin{equation}
m_i\frac{dv_i}{d_t}=\vec f_i -\gamma m_i\vec v_i+R_i,
\label{eq:newton}
\end{equation}
در این بسته‌ی نرم افزاری دینامیک ملکولی این معادله با الگوریتم پرش قورباغه‌ای 
\cite{IZAGUIRRE2009}
گسسته سازی شده. 
$\gamma$
ضریب اصطکاک و 
$R_i$
نیروی غیر همبسته‌ی تصادفی با مقدار میانگین صفر و واریانس
$2m_i\gamma k_BT$
است. 