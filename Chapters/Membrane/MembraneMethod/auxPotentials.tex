\subsection{
پتانیسل ارتفاع
}

فاصله‌ی میان نقاط و اضلاع مثلث‌ها در شبکه را می‌توان با پتانسیل 
Weeks-Chandler-Andersen
یا به اختصار
WCA
کنترل کرد. این پتانسیل منشا فیزیکی ندارد و برای ایجاد پایداری در شبیه‌سازی به کار برده می‌شود. با کنترل فاصله‌ی نقاط با اضلاع مثلث‌ می‌توان حد پایین برای اندازه‌ی پلاکت‌ها تعیین کرد و در نتیجه اندازه قدم شبیه‌ سازی دینامیک ملکولی را تعیین کرد. از طرفی محاسبات هندسی مختلفی که برای محاسبه‌ی سطح، حجم، و انحنای مش مورد نیاز است در صورتی دو نقطه از شبکه یا یک نقطه و یک ضلع مثلث دقیقا روی هم قرار بگیرند به پاسخ گنگی منجر خواهد شد. با استفاده از این پتانسیل احتمال ناپایداری در محاسبات را می‌توان از میان برد. از آنجایی که یک مثلث ۳ اتفاع دارد، برای هر مثلث در مش ۳ پتانسیل نیاز خواهیم داشت. برای نمونه، این پتانسیل برای یکی از ارتفاتعات مثلث به شکل زیر تعریف می‌شود،
\begin{equation}
U_{h}=\epsilon\left[\left(\frac{d_h}{h}\right)^8-\left(\frac{d_h}{h}\right)^4+\frac{1}{4}\right].
\label{eq:wcah}
\end{equation} 
پارامتر
$d_h$
فاصله‌ی کمینه ممکن برای ارتفاع را مشخص می‌کند،  عمق چاه
$\epsilon=4k_BT$,
 و فاصله‌ی قطع\LTRfootnote{cut off}
پتانیسل 
$h_{cutoff}=\sqrt[6]{2}d_h$
است. ارتفاع راس
$i$
از ضلع تعریف شده میان دو راس
$j$
و
$j'$
در مثلث
$ijj'$
به صورت 
\begin{equation}
h_i=\ell_{ij}\sin\theta_{ijj'}.
\end{equation} 
محاسبه می‌شود.

\subsection{
پتانیسل غیر خطی دوسطحی
}
به طول عمومی، تمام پتانسیل‌هایی که برای محاسبه‌ی انحنای رویه در این رساله معرفی شده برای حد خمش‌های کم صادق است. جهت محاسبه‌ی صحیح انرژی انحنا لازم است که در طول شبیه‌سازی از پدید آمدن لبه‌های نوک تیز روی مش جلوگیری کرد. از آنجایی که انرژی انحنا یک پتانسیل هارمونیک است، پتانسیلی که زوایای میان مثلث‌ها را کنترل کند باید در حد انحناهای کم، هزینه‌ی انرژی تقریبا صفر داشته باشد و تنها در  انحنای زیاد ظاهر شده و در برابر خم شدن مثلث‌ها مقاومت نشان دهد. با تعریف پتانسیل به صورت
\begin{equation}
U_{\phi}=\frac{1}{2}k_{\phi^4}\left[e^{2(1-\cos\phi_{ij})}-1-\phi_{ij}^2 \right],
\label{eq:theta4}
\end{equation}
میان تمام جفت مثلث‌ها می‌توان از ایجاد زوایای دوسطحی بزرگ جلوگیری کرد. قدرت پتانسیل با پارامتر 
$k_{\phi}$
قابل تنظیم است.










