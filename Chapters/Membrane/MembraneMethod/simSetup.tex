نتایج ارائه شده در این رساله  به طور پیش‌فرض با مِش‌های تصادفی دارای 
$N=1002$
نقطه و جرم کل
$M=N\times m$
ایجاد شده‌اند. جرم هر ذره 
$m=50 [m_0]$
مقدار دهی شده و در صورتی که مِش کُروی باشد اندازه‌ی شعاع آن 
$r_0=1000 [l]$
تنظیم شده‌است. سرعت اولیه ذرات از یک توزیع بولتزمن با دمای 
$k_BT=2.49 [\varepsilon]$
انتخاب شده.

پارامترهای مربوط به خواص ماده‌ی غشا به این ترتیب مقدار دهی شده، ضریب سختی خمش 
$\kappa =20k_BT$
ضریب فشردگی سطحی
$k_A=5.22\times10^{5}k_BT/r_0^2 [\varepsilon/l^2]$
مدول فشردگی حجمی
$k_V=1.6\times10^7k_BT/r_0^3 [\varepsilon/l^3]$
و مقادیر تعادلی مساحت
$A_0$
و حجم
$V_0$
مستقیم از هندسه‌ی اولیه مِش محاسبه‌ شده‌است. 

در مورد پتانیسل‌های تکمیلی برای تمامی مش‌ها پتانسیل 
WCAh
با اندازه‌ی کمینه ارتفاع 
$d_h=0.02r_0 [l]$
و عمق چاه
$\epsilon=4k_BT$
استفاده شد. در صورتی که نیاز به پتانسیل غیر خطی دو وجهی، ضریب سختی آن 
$k_{\phi^4}=20k_BT$
قرار داده شده‌است.

در شبیه‌سازی واحد زمان
$[\tau]$
بر حسب واحد طول 
$[l]$
واحد جرم
$[m_0]$
و واحد انرژی
$[\varepsilon]$
به صورت
\begin{equation}
\tau =\sqrt{\frac{m_0l^2}{\varepsilon}}
\end{equation}
تعیین می‌شود.









