توجه ما در این بخش به مکانیک آماری سطوح 
$\cal S$
است که شکل یک غشا را بیان می‌کنند. مقادیر قابل اندازه‌گیری که برای سطح غشا تعریف می‌شود با میانگین‌گیری آنسامبلی بر تمامی چیدمان‌های ممکن سطوح غشا به این شکل تعریف می‌شود
\begin{equation}\label{eq:<A(S)>}
\langle X \rangle = \frac{ \int {\cal D}{\cal S}\, X({\cal S}) \exp\left[-\beta E({\cal S})\right] } 
                                     { \int {\cal D}{\cal S}\,                  \exp\left[-\beta E({\cal S})\right] }\ ,
\end{equation}
در معادله‌ی فوق 
$E({\cal S})$
مجموع انرژی ناشی از سطح، حجم، و انحنای غشاست. محاسبات عددی مربوط به اشکال غشا وابسته به شبکه
$\cal M$
است که سطح توسط آن گسسته سازی شده‌است. یک شبکه توسط مختصات نقاط
$\bm q=\{\bm q_1,\bm q_2, \ldots, \bm q_N \} \in \cal S$
و اتصالات
$\cal G$
که میان نقاط است تعریف می‌شود. به طور مشخص می‌توان از شبکه‌های‌ مثلثی برای گسسته سازی استفاده کرد. در شبکه‌های مثلثی، زمانی که اتصالات نقاط میان همسایگان نزدیک تعریف شود، مثلث‌های که سطح را تشکیل می‌دهند همپوشانی نخواهند داشت.

میان‌گین گیری آنسامبلی که در معادله‌ی 
\ref{eq:<A(S)>}
برای غشا تعریف شد، برای سطوح گسسته سازی شده بر روی تمامی درجات آزادی مِش (شبکه) تعریف می‌شود،
\begin{equation}\label{eq:<A(M)>}
\langle X \rangle = \frac{ \sum_{\cal G}\int d {\bm q}\, X({\cal G}, {\bm q}) \exp\left[-\beta U({\cal G}, {\bm q})\right] } 
                                     { \sum_{\cal G}\int d {\bm q}\,                                \exp\left[-\beta U({\cal G}, {\bm q})\right] }\ ,
\end{equation}
در این معادله انرژی
\begin{equation}
U({\cal M}) = E({\cal M}) + E_{aux}({\cal M})
\end{equation}
یک چیدمان از مش
 ${\cal M} = ({\cal G},\bm q)$
مجموع انرژی‌های گسسته سازی شده‌ی مساحت، حجم، و انحنای سطحی است که مش بیان می‌کند
$\cal S(\cal M)$,  $E({\cal M}) \approx E(\cal S(\cal M))$
و تمام پتانسیل‌های تکمیلی است که شکل، مساحت، اندازه اضلاع، ... مثلث‌ها را کنترل می‌کند است.

وزن آماری مشی
${\cal M} = ({\cal G},\bm q)$
که نماینده‌ی سطح
 $\cal S$
است
\begin{equation}\label{eq:w(M) for given S}
w({\cal G}, {\bm q}|{\cal S}) = \exp\left[-\beta U({\cal G}, {\bm q})\right] \prod_i \delta({\bm q}_i\in {\cal S}).
\end{equation}
تعریف می‌کنیم. در معادله‌ی فوق، جمله‌ی
 $\delta({\bm q}_i\in {\cal S})$
مانند یک دلتای دیراک است که مش‌هایی که نقاط آن در فاصله‌ی مناسبی نسبت به سطح 
 $\cal S$
قرار دارد را قبول می‌کند. در طول این رساله مش‌ها را به دو دسته‌ی کلی سخت و نرم دسته بندی می‌کنیم. مش‌های سخت مش‌هایی هستند که در آن پتانسیل‌های تکمیلی شکل مثلث‌های را به شکل تقریبا یکسانی بر سطح مش حفظ می‌کنند. به شکل مشابه مش‌های نرم مش‌هایی هستند که در آن پتانسیل‌های تکمیلی اجازه‌ می‌دهند مثلث‌های سطح اشکال بسیار متنوعی داشته باشند. 

در اکثر شبیه‌سازی‌های غشا از مش‌های سخت استفاده می‌شود. مهم‌ترین دلیل این امر استفاده از روش گسسته‌سازی انحنایی است که گامپر و کرول است
\cite{gompperkroll1996}
. این روش با این فرض طراحی شده است که تمامی مثلث‌هایی که بر مش قرار دارند (کم و بیش) شکل یکسان دارند. در نتیجه برای یک مش سخت احتمال
 $\exp\left[-\beta E_{aux}({\cal G}, {\bm q})\right]$
برای مش‌هایی که مثلث‌های یکسان ندارند بسیار کوچک می‌شود. مثلا برای تغییر شکل مشی مربعی مانند شکل 
\ref{fig:meshDT}a
به یک مش مستطیلی (شکل
\ref{fig:meshDT}b
) اتصالات میان نقاط باید کاملا تغییر کند. در نتیجه اگر برای شبیه‌سازی سطوحی که دائم در حال تغییر هستند از مش‌های سخت استفاده شود اتصالات مش
$\cal G$
دائما نیاز به تغییر دارد. از طرفی، با استفاده از مش‌های نرم می‌توان شکل‌های مختلفی را با اتصال یکسان
$\cal G$
نمایش داد. برای مثال در شکل
\ref{fig:meshDAR}
تغییر شکل یک مش مربعی (شکل
\ref{fig:meshDAR}a
) به یک مش مستطیلی (شکل
\ref{fig:meshDAR}b
) را با مش نرم نشان دادیم. به طور کمی، به شرطی که 
${\cal G}$
برای نمایش سطح
${\cal S}$
تطابق لازم داشته باشد، بیشترین وزن آماری مربوط به 
$\exp\left[-\beta E({\cal S})\right]$
خواهد بود و مشاهده‌ پذیر‌ها را می‌توان به این شکل اندازه‌گیری کرد
\begin{equation}\label{eq:<A(M)> soft meshes}
\langle X \rangle \approx
\langle X \rangle_{\cal G} = \frac{\int d {\bm q}\, X({\cal G}, {\bm q}) \exp\left[-\beta U({\cal G}, {\bm q})\right] } 
                                                  { \int d {\bm q}\,                                \exp\left[-\beta U({\cal G}, {\bm q})\right] }\ ,
\end{equation}















