معادله‌ی انرژی هارمونیک مربعی که برای بیان رفتار تغییر مساحت غشا محاسبه شد (معادله‌ی
\ref{eq:stretchEnergySigma}
) را می‌توان بر اساس جمع بر روی لیست مثلث‌های یک مش به شکل زیر بازنویسی کرد
\begin{equation}
\begin{aligned}
U_{A}&=\frac{1}{2}\frac{k_A}{A_0}\left[\sum_{i=1}^{N_{tri}}a_i-A_0\right]^2\\
&=\frac{1}{2}\frac{k_A}{A_0}\left[\sum_{i=1}^{N_{tri}}a_i^2+A_0^2-2A_0\sum_{i=1}^{N_{tri}}a_i+2\sum_{i\neq j}a_ia_j\right]\\
&=\frac{1}{2}\frac{k_A}{A_0}\left[\sum_{i=1}^{N_{tri}}(a_i-A_0)^2-(N_{tri}-1)A_0^2+2\sum_{i\neq j}a_ia_j\right]\\
&\\
&=\sum_{i=1}^{N_{tri}}\frac{1}{2}\frac{k_A}{A_0}\left((a_i-A_0)^2-\frac{(N_{tri}-1)}{N_{tri}}A_0^2\right)\\
&~~+\sum_{i\neq j}\frac{k_A}{A_0}a_ia_j.
\label{eq:GlobalAreaPotentialExpansion}
\end{aligned}
\end{equation}
. در جمع بالا 
$A_0$
مساحت تعادلی مش
$a_i$
مساحت مثلث
$i$
ام در لیست مثلث‌های مش است. در صورتی که فرض کنیم مثلث
$i$
ام از سه راس 
$l,k$
و
$f$
تشکیل شده مساحت مثلث را می‌توان بر اساس طول دو ضلع
$\ell_{lk}$
و
$\ell_{lf}$
 ا و زاویه‌ میان این دو ضلع
 $\theta_{klf}$
  به شکل 
 \begin{equation}
a_i=\frac{1}{2}\ell_{lk}\ell_{lf}|\sin(\theta_{klf})|.
\label{eq:singleTriangleArea}
\end{equation}
 محاسبه شود. پارامتر 
 $N_{tri}$
تعداد مثلث‌های مش را مشخص می‌کند و 
$k_A$
ضریب فشردگی سطح غشا است که مقدار آن توسط شرایط فیزیکی مسئله مشخص می‌شود. تمام این پارامتر‌ها در ابتدای شبیه‌سازی تنظیم شده و تا انتهای شبیه‌سازی ثابت می‌مانند. جمع اول در معادله‌ی 
 \ref{eq:singleTriangleArea}
 را می‌توان به عنوان یک پتانسیل ۳ ذره‌ای برای تمام مثلث‌های مش تعریف کرد. جمع دوم جمله‌ای است فرآیند تبادل مساحت میان مثلث‌ها را ممکن می‌کند. این جمع به شکل یک پتانیسیل ۶ ذره‌ای میان انتخاب ۲ از تعداد تمام مثلث‌های مش
 ${N_{tri} \choose 2}$
 تعریف می‌شود. محاسبات مربوط به این بخش از پتانسیل به لحاظ منابع کامپیوتری مورد نیاز بسیار هزینه بر است.


به شکل مشابهی می‌توان انرژی حجم غشا (معادله‌ی 
\ref{eq:VolumeEnergy}
) را به شکل جمع روی لیستی از مثلث‌ها بازنویسی کرد،
\begin{equation}
\begin{aligned}
U_{V}&=\frac{1}{2}\frac{k_V}{V_0}\left[\sum_{i=1}^{N_{tri}}v_i-V_0\right]^2\\
&...\\
&=\sum_{i=1}^{N_{tri}}\frac{1}{2}\frac{k_V}{V_0}\left((v_i-V_0)^2-\frac{(N_{tri}-1)}{N_{tri}}V_0^2\right)\\
&~~+\sum_{i\neq j}\frac{k_V}{V_0}v_iv_j.
\label{eq:VolumePotentialExpansion}
\end{aligned}
\end{equation}
. در اینجا
$v_i$
حجم هرمی است که پایه‌ی آن مثلث‌
$i$
ام مش است. همانطور که در بخش
\ref{sec:areaVolumeDiscr}
توضیح داده شد، حجم غشا با یک ضرب سه گانه قابل محاسبه‌است و حجم حاصل مستقل از مختصات راس هرم است. جهت ساده شدن محاسبات راس مثلث ها  مرکز مختصات انتخاب شده‌است. در نتیجه با فرض اینکه مثلث 
$i$
ام از رئوس
$l,k$
و
$f$
تشکیل شده، حجم هر به شکل زیر در پتانیسیل فوق جایگذاری می‌شود،
\begin{equation}
\begin{aligned}
v_i=-\frac{1}{6}(&x_l(y_fz_k-z_fy_k)+\\
&x_k(y_lzf-z_ly_f)+\\
&x_f(y_kz_l-z_ky_l)).
\end{aligned}
\label{eq:VolumeTripleProductDef}
\end{equation}
. علامت منفی در معادله‌ی فوق به این دلیل است که ترتیب رئوس طوری در نظر گرفته شده که جهت ضرب برداری آن رو به خارج از غشا باشد. پارامتر 
 $N_{tri}$
تعداد مثلث‌های مش را مشخص می‌کند و 
$k_V$
مدول فشوردگی سیال درون غشا است که مقدار آن توسط شرایط فیزیکی مسئله مشخص می‌شود. تمام این پارامتر‌ها در ابتدای شبیه‌سازی تنظیم شده و تا انتهای شبیه‌سازی ثابت می‌مانند. همچنین مشابه به پتانسیل سطح، پتانسیل حجم را می‌توان به دو برهمکنش ۳ ذره‌ای و ۶ ذره‌ای تقسیم کرد. پتانسیل ۳ ذره‌ای برای تمامی مثلث‌های مش تعریف شده و پتانسیل ۶ ذره‌ای میان انتخاب ۲ از 
$N_{tri}$
برای تمام مثلث‌های متمایز روی مش است
 ${N_{tri} \choose 2}$
.

از آنجایی که هر دو پتانسیل بر لیست یکسانی از مثلث‌ها و انتخاب‌های میان آنها تعریف می‌شود، در صورتی در شبیه‌سازی هم حجم هم مساحت کنترل می‌شود، می‌توان این جمع دو پتانسیل را به شکل یک پتانسیل ۳ ذره‌ای و یک پتانسیل ۶ ذره‌ای سطح و حجم پیاده سازی کرد.










