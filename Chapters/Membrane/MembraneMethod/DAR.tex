روش توزیع دینامیک مساحت 
\LTRfootnote{Dynamic Area Redistribution}
با هدف  شبیه‌سازی خاصیت سیال‌گون غشا در محیط دینامیک ملکولی در گروه ماده چگال نرم دانشکده فیزیک دانشگاه صنعتی شریف طراحی شده‌است.  غشا سیال است و ملکول‌های لیپید در آن رفتار پخشی دارند. در صورتی که به نقطه‌ای از غشا تنش بُرشی توسط نیروی خارجی وارد شود، ملکول‌های لیپیدی بر سطح غشا حرکت می‌کنند. هنگامی که نیروی‌ خارجی متوقف شود، ملکول‌های لیپید دلیلی برای بازگشت به محل اولیه خود ندارند و مانند یک سیال به حرکت پخشی خود ادامه می‌دهند. 

\begin{figure}[h]
\begin{center}
\includegraphics[width=12cm]{\MemMethod/Pics/meshDiffusion.pdf}
\caption{
شبکه بندی سطح غشا تشکیل شده از 
$N$
ملکول لیپیدی را نمایش می‌دهد. در هنگام تعادل ترمودینامیکی چگالی ملکول‌ها در هر تکه از شبکه به طور متوسط
$\rho=N/A_0$
است.
}
\label{fig:cylindermesh}
\end{center}
\end{figure}

غشایی با مساحت 
$A_0$
را فرض می‌کنیم که از 
$N$
ملکلول لیپیدی تشکیل شده‌است. در صورتی که سطح غشا با 
$M$
قسمت با مساحت 
$A_0/M$
شبکه بندی شود، هنگام  تعادل ترمودینامیکی، به طور متوسط چگالی ملکول‌های لیپیدی در هر قسمت از شبکه برابر با 
$\rho=N/A_0$
است. اگر تنش  به قسمت
$i$
از شبکه وارد شود که باعث انبساط آن تکه شود، چگالی ملکول‌های آن کمی کمتر از متوسط خواهد شد،
$\rho_i=\rho^-$
. از آنجایی که تعداد ملکلو‌ل‌ها در سطح غشا یکسان است و غشای لیپیدی ضریب فشردگی بسیار بزرگی دارد، چگالی تمام قسمت‌های دیگر شبکه کمی افزایش می‌یابد (
$\rho_j=\rho^+$
برای هر قسمت که
$i\neq j$
). از آنجایی که غشا در تعادل ترمودینامیکی به سر می‌برد با گذشت زمان کم ملکلول‌های غشا (با دینامیک پخشی) در قسمت‌های مختلف شبکه جابجا شده و چگالی ملکولی در تمام قسمت‌ها را دوباره یکسان می‌کنند. در نتیجه تا زمانی که مساحت کل غشا تغییر نکند اندازه‌ی قسمت‌های مختلف شبکه می‌تواند با دمای محیط اُفت  و خیز کند. در این توصیف هر قسمت از شبکه نماینده‌ی تعداد ثابتی از ملکول‌های لیپیدی نیست ولی چگالی عددی ملکلول‌ها در سراسر غشا یکسان است.

روش توزیع دینامیک مساحت روی هر شبکه‌ بندی قابل پیاده‌سازی است. در این رساله، این روش بر روی شبکه‌های مثلثی پیاده‌سازی شده‌است. در دهه‌های گذشته مطالعات زیادی بر نحوه‌ی شبیه‌سازی با استفاده از شبکه‌های مثلثی شده‌است. در نتیجه نحوه‌ی گسسته سازی انرژی‌های مورد نیاز برای شبیه‌سازی یک غشا (انرژی خمش، مساحت، و حجم) بر بستر این شبکه‌ها از مطالعات گذشتگان در دسترس است. برای شبیه‌سازی رفتار سیال‌گون غشا کافی ‌است که انحنای غشا بر روی سطح به درستی تعریف شود و حجم و مساحت کل غشا در طول شبیه‌سازی قابل کنترل باشد. 

\begin{figure}[h]
\begin{center}
\includegraphics[width=12cm]{\MemMethod/Pics/DAR.pdf}
\caption{
شکل یک مش منظم
$a)$
درهم
$b)$
و مشی با حرکت غیر مجاز 
$c)$
برای یک مربع تخت را نشان می‌دهد
}
\label{fig:DARs}
\end{center}
\end{figure}


برای آشنایی با روش توزیع دینامیک مساحت یک مثال ساده را بررسی می‌کنیم. این روش را برای یک مربع تخت به مساحت 
$S$
و خمش صفر
$C_1=C_2=C_s=0$
تعریف می‌کنیم. مانند شکل
\ref{fig:DARs}a
می‌توان مربع را با یک مش منظم نمایش داد. از آنجایی که مثلث‌ها همپوشانی نداند، جمع مساحت آن‌ها برابر با 
$S$
است. زاویه‌ی دو سطحی میان تمام جفت مثلث‌ها صفر است در نتیجه خمش نیز همه‌جای مش برابر با صفر است. مانند شکل
\ref{fig:DARs}b
نقاط مش را روی سطح جابجا می‌کنیم. تا زمانی که نقاط مش هنگام جابجا شدن از روی ضلعی عبور نکنند، همچنان جمع مساحت آن‌ها برابر با 
$S$
خواهد بود و خمش همه‌ جا صفر. اما اگر مانند شکل
\ref{fig:DARs}c
نقطه‌ی 
p
از روی ضلعی عبور کند، حداقل یک مثلث پدید می‌آید که با مثلث‌های دیگر همپوشانی دارد (مساحت نارنجی رنگ). در این حالت جمع مساحت مثلث‌ها دیگر
$S$
نیست. همچنین زاویه‌ی میان جفت مثلثی که ضلع عبور شده میان آن‌ها مشترک است دیگر صفر نیست بکله
$\phi=\pi$
است. در صورتی که حرکت نقاط به شکلی محدود شود که هیچ نقطه‌ای نتواند از ضلعی عبور کند، تمام نقاط چیدمان‌های نقاط می‌توانند به درستی خمش و مساحت مربع را توصیف کنند.

این مفهوم را می‌توان برای سطوح ۳ بعدی نیز در نظر گرفت. در روش توزیع دینامیک مساحت، مساحت سلول‌های مش می‌توانند آزادانه حرکت کنند تا زمانی که،

۱- اتصالات یا توپولوژی مش تغییر نکند

۲- سلول‌های مش همپوشانی نداشته باشند

۳- تنها تغییرات مساحت کل مش هزینه‌ی انرژی داشته باشد.




