روش توزیع دینامیک مساحت 
\LTRfootnote{Dynamic Area Redistribution}
با هدف  شبیه‌سازی خاصیت سیال‌گون غشا در محیط دینامیک ملکولی در گروه ماده چگال نرم دانشکده فیزیک دانشگاه صنعتی شریف طراحی شده‌است.  غشا سیال است و ملکول‌های لیپید در آن رفتار پخشی دارند. در صورتی که به نقطه‌ای از غشا تنش بُرشی توسط نیروی خارجی وارد شود، ملکول‌های لیپیدی بر سطح غشا حرکت می‌کنند. هنگامی که نیروی‌ خارجی متوقف شود، ملکول‌های لیپید دلیلی برای بازگشت به محل اولیه خود ندارند و مانند یک سیال به حرکت پخشی خود ادامه می‌دهند. 

\begin{figure}[h]
\begin{center}
\includegraphics[width=12cm]{\MemMethod/Pics/meshDiffusion.pdf}
\caption{
شبکه بندی سطح غشا تشکیل شده از 
$N$
ملکول لیپیدی را نمایش می‌دهد. در هنگام تعادل ترمودینامیکی چگالی ملکول‌ها در هر تکه از شبکه به طور متوسط
$\rho=N/A_0$
است.
}
\label{fig:cylindermesh}
\end{center}
\end{figure}

غشایی با مساحت 
$A_0$
را فرض می‌کنیم که از 
$N$
ملکلول لیپیدی تشکیل شده‌است. در صورتی که سطح غشا با 
$M$
قسمت با مساحت 
$A_0/M$
شبکه بندی شود، هنگام  تعادل ترمودینامیکی، به طور متوسط چگالی ملکول‌های لیپیدی در هر قسمت از شبکه برابر با 
$\rho=N/A_0$
است. اگر تنش  به قسمت
$i$
از شبکه وارد شود که باعث انبساط آن تکه شود، چگالی ملکول‌های آن کمی کمتر از متوسط خواهد شد،
$\rho_i=\rho^-$
. از آنجایی که تعداد ملکلو‌ل‌ها در سطح غشا یکسان است و غشای لیپیدی ضریب فشردگی بسیار بزرگی دارد، چگالی تمام قسمت‌های دیگر شبکه کمی افزایش می‌یابد (
$\rho_j=\rho^+$
برای هر قسمت که
$i\neq j$
). از آنجایی که غشا در تعادل ترمودینامیکی به سر می‌برد با گذشت زمان کم ملکلول‌های غشا (با دینامیک پخشی) در قسمت‌های مختلف شبکه جابجا شده و چگالی ملکولی در تمام قسمت‌ها را دوباره یکسان می‌کنند. در نتیجه تا زمانی که مساحت کل غشا تغییر نکند اندازه‌ی قسمت‌های مختلف شبکه می‌تواند با دمای محیط اُفت  و خیز کند. در این توصیف هر قسمت از شبکه نماینده‌ی تعداد ثابتی از ملکول‌های لیپیدی نیست ولی چگالی عددی ملکلول‌ها در سراسر غشا یکسان است.

روش توزیع دینامیک مساحت روی هر شبکه‌ بندی قابل پیاده‌سازی است. در این رساله، این روش بر روی شبکه‌های مثلثی پیاده‌سازی شده‌است. در دهه‌های گذشته مطالعات زیادی بر نحوه‌ی شبیه‌سازی با استفاده از شبکه‌های مثلثی شده‌است. در نتیجه نحوه‌ی گسسته سازی انرژی‌های مورد نیاز برای شبیه‌سازی یک غشا (انرژی خمش، مساحت، و حجم) بر بستر این شبکه‌ها از مطالعات گذشتگان در دسترس است. برای شبیه‌سازی رفتار سیال‌گون غشا کافی ‌است که انحنای غشا بر روی سطح به درستی تعریف شود و حجم و مساحت کل غشا رد طول شبیه‌سازی قابل کنترل باشد. 

در آخر لازم است به این نکته اشاره شود که از آنجایی که غشا‌ مدول یانگ ندارد، تغییر شکل‌های کُره-بیست وجهی که حاصل از رقابت انرژی کشسانی و انرژی خمشی است (جزئیات بیشتر در بخش 
\ref{sec:gammaTransition}
) در آن‌ها دیده نخواهد شد.




