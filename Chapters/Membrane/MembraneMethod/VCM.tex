نرم افزار مدل سلول مجازی
\LTRfootnote{Virtual Cell Model}
در گروه ماده چگال نرم دانشگاه صنعتی شریف جهت مطالعه‌ی خواص مکانیکی سلول هنگام مهاجرت
\LTRfootnote{cell migration}
و چسبیدن به سطوح
\LTRfootnote{cell adhesion}
ایجاد شد
\cite{Tiam2017ACS, Tiam2018AFM}
. بخش غشای‌ این نرم افزار جهت انجام شبیه‌سازی‌های مورد نیاز این رساله توسعه یافت. در حال حاضر بخش غشا توانایی شبیه‌سازی غشا‌های جامد (دارای مدول یانگ) و غشاهای سیال را دارد. توسعه‌‌ی نرم افزار حدود ۳ سال طول کشید. در این زمان بخش‌هایی مانند مدل‌سازی شبکه‌ی اسکلت سلولی و کروماتین نیز توسعه داده شد. همچنین تمامی مقیاس‌های انرژی و تعاریف پتانسیل با واحد‌های قابل مقایسه با نتایج آزمایشگاهی کالیبره شد
\cite{VCMgit}
. 

به لحاظ عملکرد، موتور محاسباتی دینامیک ملکولی 
OpenMM
به عنوان مرکز محاسباتی اصلی این نرم افزار قرار داده شد که تمام قابلیت‌های این موتور محاسباتی را در دسترس کاربر قرار می‌دهد. از جمله قابلیتی که مورد توجه افراد مشغول در زمینه‌ی محاسباتی است، توانایی انجام محاسبات به شکل موازی بر پردازنده‌های مرکزی چند هسته‌ای
\LTRfootnote{multi-core CPU}
و پردازنده‌های گرافیکی
\LTRfootnote{GPU}
است. جهت کسب اطلاعات بیشتر به صفحه‌ی 
YouTube \cite{VCMYoutube}
و صفحه‌ی راهنمای این نرم افزار 
\cite{VCMhomepage}
مراجعه فرمایید.