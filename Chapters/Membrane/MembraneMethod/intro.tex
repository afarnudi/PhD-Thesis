در چند ده‌ی گذشته روش‌ مانتی‌ کارلو برای مطالعه‌ی غشا‌ها استفاده شده‌است
\cite{Canham1970, Evans1974, schneider1984, Nelson1987PRA, NelsonPRL1987, Gompper1991, Gompper1992Science, Lim2002PNAS, Gompper2005PRE}
. همانطور که در فصل 
\ref{ch:Discretisation}
توضیح داده شد، در اوایل ده‌ی ۱۹۹۰ به کمک روش مثلث بندی دینامیک
\cite{Boal1992PRA, Gompper1992Science}
، روش‌ بسیار خوبی بر پایه شبیه‌سازی مانتی کارلو برای مطالعه‌ی رفتار سیال‌گون غشا ابداع شد. در این روش اتصالات میان نقاط بر روی شبکه‌ی مثلثی دائم در حال تغییر است. در صورتی که به محلی بر روی شبکه تنش بُرشی وارد شود، نقاط بر سطح مِش شروع به حرکت کرده و با تغییر اتصالات و همسایه‌هاس خود تنش بُرشی را تعدیل و در نهایت از بین می‌برند. این یکی از مهم‌ترین دلیل‌های موفق بودن این روش است. از طرف دیگر روش محاسبه‌ی خمش متوسط ایتزیکسون
\cite{Gompper1992Science}
در کنار این الگوریتم امکان پذیر بوده و طبیعت بر پایه شبیه‌سازی مانتی‌کارلو این روش را تغییر نمی‌دهد. از آنجایی که با این الگوریتم می‌توان توپولوژی شبکه را تغییر داد، پدیده‌های همچون جوانه زدن
\LTRfootnote{budding}
، جدایی فاز ملکول‌های لیپیدی غشا
\LTRfootnote{phase seporation}
\cite{Kohyama2003PRE, Gompper2007PRL, Laradji2004PRL}
، و تولید بازو‌ و لوله‌ 
\LTRfootnote{phase seporation}
بر سطح غشا
\cite{Ramakrishnan2013BiophysJ}
با این روش مطالعه شده‌است. 

اساس روش شبیه‌سازی مانتی کارلو تولید آمار درست از یک پدیده‌است و برای مطالعه‌ی دینامیک طراحی نشده‌است. البته در این روش می‌توان حرکت‌هایی طراحی کرد که پیوستگی مکان و زمان را حفظ کند و در نتیجه دینامیک شبه  واقعی تولید کرد. اما میزان واقعی بودن دینامیک تولید شده از این روش تابع میزان ظرافت و هوشمندی  حرکت‌های طراحی شده است و همیشه تقریبی است از دینامیکی که از معادلات حرکت حاصل می‌شود.

روش شبیه‌سازی دینامیک ملکولی روش مناسبی برای مطالعه حرکت ذرات است. در این روش با گسسته‌سازی و حل معادلات نیرو‌های وارد بر ذرات محاسبه شده و  مکان ذرات پیش‌بینی  می‌شود. از روش دینامیک ملکولی برای مطالعه‌ی غشا نیز استفاده شده‌است. خواص الاستیک غشای گلبول‌های قرمز
\cite{Hale2009SoftMatter, Geekiyanage2019PLOS}
، تغییر شکل گلبول‌های قرمز تحت جریان‌های بُرشی هیدرودینامیکی 
\LTRfootnote{hydrodynamic shear}
\cite{Noguchi2005PNAS}
، و  خواص پوسته‌های الاستیک و کپسید‌ها
\LTRfootnote{capsids}
\cite{NelsonPRL1987, gomppernelson2012}
از جمله موضوعاتی است که با شبکه‌های مثلثی ثابت به کمک روش دینامیک ملکولی مطالعه شده‌است.

به علت محبوبیت زیاد روش دینامیک ملکولی میان دانشمندان، بسته‌های نرم‌ افزار‌ی بسیار زیادی همچون 
LAMMPS \cite{LAMMPS}
،
GROMACS \cite{GROMACS}
، و
OpenMM \cite{OpenMM2017}
ساخته‌ شده‌است. استفاده از بسته‌های نرم‌افزاری دینامیک ملکولی بسیار مفید است. این بسته‌ها معمولا شامل پتانسیل‌ها و ابزار‌های متداول و پایه‌ای مورد نیاز در تحقیقات هستند. در نتیجه آماده سازی شبیه‌سازی و محیط مجازی به کمک این نرم افزار‌ها با سرعت بالایی انجام می‌شود. به طور مثال نرم افزار‌های نام قادر به محاسبه‌ی نیروهای کوتاه بُرد و همچنین بلند بُرد بوده و قادر به حل معادلات حرکت پایه‌ای همچون معادله‌ی نیوتن، لانژون
\LTRfootnote{Langevin}
، و معادله‌ی ناویر-استوکس
\LTRfootnote{Navier-Stokes}
هستند. برخی از این بسته‌ها (مانند 
OpenMM
) حتی قادر به محاسبه‌ی نیرو از روی جملات پتانسیل نیز هستند. 


در نتیجه محققین می‌توانند بیشتر زمان‌شان را به طراحی مدل و بررسی ایده‌هایش بپردازند و زمان کمتری را برای برنامه نویسی و عیب یابی  و تست کُد بگذارند. از طرف دیگر محققانی که با شبیه‌سازی سر و کار دارند چه خواسته چه ناخواسته محاسبات خود را باید بر بستر کامپیوتر‌ی انجام دهند. سریع‌ترین  پردازنده‌های موجود در حال حاضر پردازنده‌های چند هسته‌ای
\LTRfootnote{multi core CPU}
 کلاسیک است. این پردازنده‌ها با معماری‌های مختلفی در دسترس عموم قرار می‌گیرد. مهم‌ترین پردازند‌ه‌های چند هسته‌ای که  نسبت سرعت پردازش به  قیمت بالایی دارد پردازنده‌های چند هسته‌ای گرافیکی
\LTRfootnote{GPU}
است. این پردازنده‌‌ها هسته‌های بسیار زیادی دارند که برای انجام محاسبات وابسته به مکان (مانند محاسبه‌ی فاصله‌ فضایی نقاط) 
و عملیات ماتریسی طراحی شده‌اند و در نتیجه برای انجام محاسبات مربوط به شبیه‌سازی دینامیک ملکولی بسیار مناسب هستند.  از طرفی یادگیری نحوه‌ی برنامه نویسی و استفاده از این پردازنده‌ها بسیار زمان‌بر است (حدود ۶ ماه تا یک سال) و زمان زیادی از انجام تحقیقات را اشغال می‌کند. اینجاست که مزیت استفاده از بسته‌های نزم افزاری برای انجام تحقیقات دوباره اهمیت پیدا می‌کند. بیشتر نرم‌افزار‌های مُدرن زحمت برنامه نویسی و انجام محاسبات بر این پردازنده‌ها را انجام داده‌اند. در نتیجه سرعت محقق هم در مدل‌سازی و هم در دریافت نتیایج چندین برابر می‌شود.

مدل غشا‌های ریسمانی روشی است که با شبیه‌سازی دینامیک ملکولی مطابقت دارد و برای مطالعه‌ی غشا‌ها استفاده شده‌است
\cite{Abraham1989PRL}
. غشای ریسمان بر روی یک شبکه‌ی مثلثی با توپولوژی ثابت تعریف می‌شود. در این روش طول اضلاع مثلث‌ها با یک حد بیشینه و کمینه کنترل شده و با استفاده از زاویه‌ی دوسطحی 
\LTRfootnote{dihedral angle}
انحنای غشا محاسبه می‌شود. همانطور که در فصل
\ref{ch:SimRev}
 اشاره شد، روش‌های مدل‌سازی ذرات لنارد جونز خود سامانده هم برای شبیه‌سازی غشا با استفاده از روش دینامیک ملکولی وجود دارد 
 \cite{Discher2004NatMat, Schindler2016BBA}
 اما این روش‌ها به منابع کامپیورتری خیلی زیادی برای شبیه‌‌سازی غشا‌های چند میکرونی دارند. می‌توان از روش‌های درشت دانه سازی استفاده کرد که غشا‌های بزرگ را با ذرات لنارد جونز شبیه‌سازی کرد
 \cite{Li2005}
 . در این روش‌ها غشا با یک تک لایه از این ذرات مدل‌سازی می‌شود. اما چالش اصلی در این روش محاسبه‌ی درست انرژی انحناست. راه حل معمول تعریف یک شبکه‌ی مثلثی موقت بر اساس مکان لحظه‌ای ذرات و محاسبه‌ی انرژی انحنای شبکه است. در نتیجه ذرات با روش دینامیک ملکولی حرکت می‌کنند ولی برای محاسبه‌ی نیروهای حاصل از انحنا دائم شبیه‌سازی باید متوقف شود، از محیط نرم افزار خارج شد و پس از محاسبه‌ی نیرو‌های خمش، دوباره وارد محیط شبیه‌سازی دینامیک ملکولی شد. در نتیجه تمام مزیت‌های استفاده از نرم‌افزار‌های دینامیک ملکولی از دست داده می‌شود.
 
 از طرفی مدل‌های ترکیبی دینامیک ملکولی و مانتی‌کارلو هم وجود استفاده شده‌است. در این روش‌ها رفتار سیال‌گون غشا توسط یک شبکه‌ی مثلثی مدل‌ می‌شود که توپولوژی آن با الگوریتم مثلث بندی دینامیک با روش مانتی کارلو شبیه‌سازی می‌شود و رفتار الاستیک غشا توسط یک شبکه‌ی مثلثی ثابت مدل می‌شود که با روش دینامیک ملکولی حرکت می‌کند
 \cite{Noguchi2005PNAS}
 . این روش‌ها بسیار موفق بوده‌اند ولی لازمه‌ی پیاده‌سازی آنها طراحی و ساخت نرم افزار شخصی است.
  













