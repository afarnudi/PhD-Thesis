\begin{figure}[h]
\begin{center}
\includegraphics[width=12cm]{\MemMethod/Pics/MeshTypes.pdf}
\caption{
عکس چهار نوع مِش استفاده شده در این رساله را نشان می‌دهد. مِش منظم (بالا سمت چپ)، تصادفی (بالا سمت راست)، دَرهَم منظم (پایین سمت چپ)، و مِش دَرهَم تصادفی (پایین سمت راست).
}
\label{fig:meshTypesMesthod}
\end{center}
\end{figure}
در شکل
\ref{fig:meshTypesMesthod}
نمونه‌ای از مِش‌هایی که در این رساله استفاده شده را می‌توان یافت. مِش‌ها به ترتیب زیر قابل تهیه هستند:
\subsection{
مِش منظم
}
مش منظم با شبکه‌ بندی کردن یک بیست وجهی منتظم ساخته می‌شود. در این شبکه‌ بندی ۱۲ نقطه‌ی نقص با درجه‌ی ۵ در ۱۲ گوشه‌ی بیست وجهی ایجاد خواهد شد و نقاط دیگر همگی از درجه‌ی ۶ خواهند بود. در نتیجه به ازای هندسه بسیار متقارن کُره تنها یک نمونه از چنین شبکه‌ای به ازای تعداد نقاط وجود دارد. همچنین این برای ساخت چنین مِشی تنها می‌توان از تعداد مشخصی نقطه که با رابطه‌ی 
\begin{equation}
N_{mesh}=10\times i^2+2,
\end{equation}
محاسبه می‌شود داشت.  که در اینجا 
$i$
مقادیر صحیح غیر صفر دارد. ‌شبکه‌ بندی با هر نرم‌افزار مش بندی قابل انجام است. در این مطالعه ما از نرم افزار 
Blender
برای تولید این مِش به خصوص استفاده کردیم. 

\subsection{
مِش تصادفی
}
مقدمه‌ی تهیه‌ی این نوع مش  قرار دادن 
$N$
نقطه به شکل تصادفی بر سطح یک کُره‌ی به شعاع ۱ است. سپس به کمک 
$N$
پتانسیل فنر هارمونیک با ضریب سختی زیاد، این نقاط را به مرکز کُره متصل کرده و بر سطح کُره مقید می‌کنیم. سپس میان این نقاط برهمکنش بلند بُرد دافعه تعریف می‌کنیم. هر برهمکنش دافعه‌ی بلند بُردی که  توزیع تقریبا یکنواخت نقاط بر سطح کُره ایجاد کند مورد قبول است. در این رساله از پتانیسل 
\begin{equation}
U_{EV}=10^{-3}\left(\frac{2\rho_0}{r}\right)^6,
\end{equation}
استفاده شد. در اینجا 
$r$
فاصله‌ی فضایی نقاط از یکدیگر بوده و 
$\rho_0$
شعاع متوسط هر نقطه روی کُره است. شعاع متوسط کُره با شعاع 
$N$
دایره تخمین زده شد که  سطح کُره را پوشش می‌دهند
\begin{equation}
N\pi\rho_0^2=4\pi\rightarrow \rho_0=\frac{2}{\sqrt{N}}.
\end{equation} 
. سپس ذرات تحت دینامیک لانژون شروع به حرکت کرده. با انتخاب دمای پایین برای ترموستات لانژون می‌توان سرعت ذرات را پله پله کم کرد تا جایی که ذرات در فاصله‌ی تعادلی نسبت به یکدیگر قرار گیرند. البته از هر روش کمینه کردن انرژی
\LTRfootnote{energy minimisation}
برای یافتن مختصات تعادلی نقاط می‌توان استفاده کرد. پس از یافت مختصات تعادلی نقاط با استفاده از الگوریتم مثلث‌ بندی دِلونی
\LTRfootnote{Delaunay triangulation algorithm}
\cite{DelaunayTriangulation1997}
یک شبکه‌ی مثلثی تصادفی ایجاد خواهد شد. با تکرار این دستورنامه با استفاده از بذر‌های
\LTRfootnote{seed}
 مختلف برای تولید اعداد تصادفی غیر یکسان می‌توان مِش‌های تصادفی متفاوتی تولید کرد.

\subsection{
مِش‌های درهم
}
برای تغییر توزیع مساحت مثلث‌های مِش‌های منظم و تصادفی‌  نقاط آن را با فنر‌های هارمونیک با ضریب سختی بالا مقید به حرکت بر سطح کُره کرده و سپس به نقاط سرعت و جهت تصادفی اختصاص داده شد. با محدود کردن طول ارتفاع مثلث‌های مش با پتانسیل 
$U_{h}$
(جزئیات این پتانسیل در بخش 
\ref{sec:auxPotentials}
قرار دارد) حد پایین برای اندازه‌ی مثلث‌ها تعیین شد. این پتانسیل تضمین می‌کند که اضلاع مثلث‌های بر اثر حرکت کاتوره‌ای نقاط همدیگر را قطع نکنند. ولی این پتانسیل برای جلوگیری از تا شدن مثلث‌ها بر روی سطح کافی نیست. برای پیشگیری از این اتفاق پتانسیل غیر خطی خمشی دوسطحی
$U_{\phi}$
(جزئیات این پتانسیل در بخش 
\ref{sec:auxPotentials}
قرار دارد) میان تمام جفت مثلث‌های مش تعریف شد. در نتیجه دینامیک چنین نقاطی تحت معادله‌ی حرکت نیوتن (انتگرال گیری وِرلِه سرعتی
\LTRfootnote{velocity Verlet}
) مساحت‌ مثلث‌ها افت و خیز می‌کند و توپولوژی اولیه مِش استفاده شده کاملا محفوظ می‌ماند.
 




