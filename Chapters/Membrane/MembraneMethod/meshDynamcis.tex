ما پیشنهاد می‌کنیم که به جای استفاده از روش دینامیک مثلثی (که هر بار بر اساس یک قدم مانتی کارلو مش را تغییر می‌دهد)، پیرو کارهایی که قبلا در این شاخه از فیزیک انجام شده بود

از مش‌های نرم دو بعدی با اتصالات ثابت برای شبیه سازی غشا استفاده شود. در این شبیه‌سازی‌ها حرکت نقاط بر روی سطح 
$\cal S$
با محدودیت بسیار کمی مواج است. در واقع ایده‌ی اصلی این است که نقاط مش تا زمانی باعث همپوشانی مثلث‌ها نشوند (شکل
\ref{fig:DARs}c
) می‌توانند آزادانه بر سطح دو بعدی غشا حرکت کنند. این شرط توسط پتانیسیل‌های تکمیلی
$E_{aux}({\cal M})$
کنترل می‌شود. یه طور عمومی مثلث‌ها یا پلاکت‌هایی که به هر نقطه اختصاص داده شده، نماینده‌ی یک مقدار ثابتی از غشا نیستند ولی نماینده‌ی کسری از جرم کل غشا هستند که با مساحت لحظه‌ای هر نقطه رابطه دارد
 $m_i = M a_i/A$
. برای غشایی که سطح آن تغییر نکند این رابطه را به شکل ساده‌تر
$m_i  \approx \rho a_i$
می‌توان نوشت. حرکت دسته‌ جمعی نقاط باعث می‌شود که سطح

در پاسخ به نیروهای خارجی سریع تغییر کند. در ۳ بُعد، می‌توان دینامیک مش را شبیه به یک توصیف شبه لاگرانژی از یک شارع دو بعدی درنظر گرفت. حرکت این رویه‌ی دوبعدی در جهات خراج از سطح لاگرانژی است ولی حرکت درون سطحی نقاط، که نماینده‌ی لیپید‌ها هستند، وابسته به پانسیل‌های تکمیلی مش است.

در این شرایط هامیلتونین سیستم را می‌توان به شکل زیر نوشت
\begin{eqnarray}
\mathcal H&=& \frac12 \sum_i^N \frac{\bm p_i^2}{m_i(\bm q)} + U(\bm q) 
\label{eq:HamiltonianGeneral}\\
 &=& \frac {A(\bm q)}{2M}  \sum_i^N \frac{\bm p_i^2}{a_i(\bm q)} + U(\bm q) 
 \label{eq:HamiltonianGeneral with explicit areas}\\
&\approx& \frac {1}{2\rho}  \sum_i^N \frac{\bm p_i^2}{a_i(\bm q)} + U(\bm q) \ ,
\label{eq:HamiltonianVariableMass}
\end{eqnarray}
که در اینجا
$\bm p=\{\bm p_1,\bm p_2, \ldots, \bm p_N \}$
و
$\bm q=\{\bm q_1,\bm q_2, \ldots, \bm q_N \}$ 
تکانه و مختصات است،
$m_i(\bm q) = Ma_i(\bm q)/A(\bm q)$
جرم نقاط،
$A(\bm q)$
مساحت لحظه‌ای مش، و 
$U(\bm q)$
مجموع انرژی‌های پتانسیل مش است که تابع مساحت، حجم، و انحنای مش است.

در این هامیلتونین، جرم نقاط تابع توزیع مختصات مش است
$m_i(\bm q)$
. اگر از تکانه‌ها انتگرال بگیریم، وزن آماری یک چیدمان مشخص از مش
$\bm q$
به شکل زیر تعریف می‌شود
\begin{equation}
\begin{aligned}
w(\bm q)&=\exp\left[-\beta U(\bm q)\right]\int d\bm p\exp\left(-\beta\frac {A(\bm q)}{2M}  \sum_i^N \frac{\bm p_i^2}{a_i(\bm q)}\right)\\
&=\prod_{i=1}^N\left[\frac{2\pi M}{\beta} \frac{a_i(\bm q)}{A(\bm q)}\right]^{\frac{3}{2}}\exp\left[-\beta U(\bm q)\right]\ .
\end{aligned}
\label{eq:microStateProbability1}
\end{equation}

همانطور که انتظار داشتیم،
$w(\bm q)$ 
به وزن بولتزمنی 
$\exp\left[-\beta U(\bm q)\right]$
وابسته است که به شکل سطح مربوط است. ضریب این جمله زمانی بیشینه است که تمامی پلاکت‌ها مساحت یکسانی داشته باشند
$a_i = A/N$
. در نتیجه می‌توانیم این وزن را به شکل وزن بولتزمن موثر بنویسیم
\begin{equation}
\begin{aligned}
w(\bm q)
&=\exp\left[\frac{3}{2}\sum_{i=1}^N\log\left\{\frac{2\pi M}{\beta} \frac{a_i(\bm q)}{A(\bm q)}\right\}-\beta U(\bm q)\right]\ .
\end{aligned}
\label{eq:microStateProbability}
\end{equation}
برای فهیمدن این وزن موثر، جمله‌ی بالا را بر اساس تغییرات مساحت پلاکت‌ها از مساحت میانگین پلاکت‌ها 
$a_i(\bm q)=A(\bm q)/N+\delta_i(\bm q)$
بازنویسی می‌کنیم
\begin{equation*}
\begin{aligned}
\log\left(\frac{2\pi M}{\beta} \frac{a_i(\bm q)}{A(\bm q)}\right)
&=\log\left(\frac{2\pi m}{\beta}\right)+\log\left(1+\frac{N\delta_i(\bm q)}{A(\bm q)}\right)\\
&\approx\log\left(\frac{2\pi m}{\beta}\right)+\frac{N\delta_i(\bm q)}{A(\bm q)}-\frac{1}{2}\left(\frac{N\delta_i\bm q)}{A(\bm q)}\right)^2.
\end{aligned}
\label{eq:logExpantion}
\end{equation*}
پس از جایگذاری این جمله در معادله‌ی 
\ref{eq:microStateProbability}
جمع جملات مرتبه‌ی صفرم یک ثابت عددی است، جمع جملات مرتبه‌ی اول بنان به تعریف صفر خواهد بود. در نتیجه جملات باقیمانده تنها از مربته‌ی دوم خواهد بود
\begin{equation}
\begin{aligned}
w(\bm q)\propto\exp\left[-\beta\left( \frac{3}{2}k_BT\sum_{i=1}^N \frac{1}{2}(\frac{N}{A_0}\delta_i(\bm q))^2+U(\bm q)\right)\right] \ ,
\end{aligned}
\label{eq:microStateProbabilityExpansion}
\end{equation}
این جمله شبیه به یک پتانسیل هارمونیک است که اندازه‌ی پلاکت‌ها را نزدیک به مساحت میانگین حفظ می‌کند.



اگر از تغییرات کوچک مساحت کل غشا صرفه نظر کنیم، معادلات حرکت نقاط مش را می‌توانیم با استفاده از معادلات حرکت هامیلتون محاسبه کنیم. از معادله‌ی اول هامیلتون 
$\dot {\bm q_i}=\frac{\partial}{\partial \bm p_i}\mathcal H$
می‌توانیم تکانه‌ی نقاط را حساب کنیم
$\bm p_i = \rho a_i(\bm q)\dot {\bm q_i}$
. و معادله‌ی دوم 
$\frac{d}{dt} \bm p_i = -\frac{\partial}{\partial \bm q_i}\mathcal H$
به شکل زیر ساده می‌شود
\begin{equation}
\frac{d}{dt}(\rho a_k(\bm q) \dot {\bm q}_k) = -\frac{\partial U(\bm q)}{\partial \bm q_k} - \frac{1}{2} \frac{\partial}{\partial \bm q_k}\sum_{i=1}^N\frac{\bm p_i^2}{\rho a_i(\bm q)} \end{equation}
یا
\begin{eqnarray}
\lefteqn{  \rho a_k(\bm q) \ddot{\bm q_k}  =}
\label{eq:HamiltonianGeneralEOMMomentum}\\
 &=& -\frac{\partial U(\bm q)}{\partial \bm q_k} + \frac{1}{2}\rho \sum_{i=1}^N\dot{\bm q_i}^2\frac{\partial a_i(\bm q)}{\partial \bm q_k}
 - \rho  \dot{\bm q_k}\sum_{i=1}^N\frac{\partial a_k(\bm q)}{\partial \bm q_i}\dot{\bm q_i}\nonumber
\end{eqnarray}
حل این معادلات دشوار است و برای بررسی دینامیک هر سیستمی که در آن تاثیر جرم مهم باشد (مانند حباب صابون) مورد نیاز است. در این رساله ما علاقمند به شبیه‌سازی رفتار غشایی هستیم که دینامیک سطح توسط جریان شارع درون و شارعی که غشا در آن قوطه‌ور شده حکم می‌شود
\cite{milnersafranPRA1987, schneider1984}
. از آنجایی که در این سیستم اینرسی نقش نخواهد داشت، برای ساده سازی معادلات فوق، جرم تمام نقاط مش را یکسان تقریب می‌زنیم
\begin{eqnarray}
\mathcal H&=& \frac12 \sum_i^N \frac{\bm p_i^2}{m_i} + U(\bm q) 
\label{eq:HamiltonianFixedMass} \ ,
\end{eqnarray}
در این حالت وزن بولتزمنی که اندازه‌گیری‌های آماری را تایین می‌کند
\begin{equation}
\begin{aligned}
w(\bm q)\propto\exp\left\{-\beta U(\bm q)\right\} \ ,
\end{aligned}
\label{eq:microStateProbability HamiltonianFixedMass}
\end{equation}
و معادلات حرکت به معادلات حرکت نیوتن ساده می‌شود
\begin{equation}
m_i \ddot{\bm q_k} = -\frac{\partial U(\bm q)}{\partial \bm q_k} \ .
\label{eq:EoM for HamiltonianFixedMass}
\end{equation}
برای حل دقیق‌تر این معادلات برای غشایی که در خلا حرکت می‌کند، می‌توان اندازه‌ی جرم تمام نقاط را با اندازه‌ی جرم متوسط هر نقطه جایگزین کرد (کاری که در این رساله انجام شده) 
$m_i = \rho \langle a_i \rangle$
یا با بازسازی مش، در هر مرحله از شبیه‌سازی از مشی‌ استفاده کرد که توزیع نقاط بر روی سطح همیشه یکنواخت باشد.





 










