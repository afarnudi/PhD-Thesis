\setRL
\clearpage
\def \MemMethod {\Mempath /MembraneMethod}

\section{
چکیده
}
در این بخش روش  دینامیکِ مِش  معرفی شده‌است. در مقدمه، انگیزه‌ی طراحی این روش را بیان شده‌است. سپس بحث کوتاهی در مورد مکانیک آماری سطوح مثلث بندی شده و نحوه‌ی اندازه‌گیری مشاهده پذیر‌ها در بخش 
\ref{sec:statMech}
بیان شده‌است. پس از یادآوری الگوریتم مثلث بندی دینامیک در بخش
\ref{sec:dynamicTriangulation}
 روش تکمیلی، دینامیک مش در بخش
 \ref{sec:meshDynamics}
و معادلات حرکت معرفی شده‌است. سپس در بخش
 \ref{sec:DAR}
به توزیع دینامیک مساحت پرداخته‌ایم که حاصل استفاده از روش دینامیک مش است. ‌تعریف مش‌های استفاده شده در این روش در بخش
\ref{sec:meshRecipe}
ارائه شده و روش‌پیاده سازی پتانسیل‌های مورد نیاز برای انجام شبیه‌سازی دینامیک مولکولی با جزئیات در بخش‌های
\ref{sec:areaVolumeD}
و
\ref{sec:curvatureD}
توضیح داده شده‌است. پس از تعریف پتانسیل‌های فیزیکی، در بخش
\ref{sec:auxPotentials}
پتانسیل‌های تکمیلی مورد نیاز دینامیک مش معرفی شده‌است. در نهایت نرم‌افزار‌ها و ابزارهای استفاده شده برای انجام شبیه‌سازی‌های مورد نیاز این مطالعه (بخش‌های
\ref{sec:OpenMM}
و
\ref{sec:VCM})
و شیوه‌ی آماده‌سازی شبیه‌سازی‌ها در بخش
\ref{sec:dataAcu}
توضیح داده شده‌است.

\section{
مقدمه
}
در چند دهه‌ی گذشته روش‌ مانتی‌ کارلو برای مطالعه‌ی غشا‌ها استفاده شده‌است
\cite{Canham1970, Evans1974, schneider1984, Nelson1987PRA, NelsonPRL1987, Gompper1991, Gompper1992Science, Lim2002PNAS, Gompper2005PRE}.
 همانطور که در فصل 
\ref{ch:Discretisation}
توضیح داده شد، در اوایل دهه‌ی ۱۹۹۰ به کمک الگوریتم مثلث بندی دینامیک
\cite{Boal1992PRA, Gompper1992Science},
 روش‌ بسیار خوبی بر پایه شبیه‌سازی مانتی کارلو برای مطالعه‌ی رفتار سیال‌گون غشا ابداع شد. در این روش اتصالات میان نقاط بر روی شبکه‌ی مثلثی دائم در حال تغییر است. در صورتی که به محلی بر روی شبکه تنش بُرشی وارد شود، نقاط بر سطح مِش شروع به حرکت کرده و با تغییر اتصالات و همسایه‌های خود تنش بُرشی را تعدیل و در نهایت از بین می‌برند. این یکی از مهم‌ترین دلیل‌های موفق بودن این روش است. از طرف دیگر روش محاسبه‌ی انحنای متوسط گامپر
\cite{Gompper1992Science}
در کنار این الگوریتم امکان پذیر بوده و ماهیت شبیه‌سازی مانتی‌کارلو را تغییر نمی‌دهد. از آنجایی که با این الگوریتم می‌توان توپولوژی شبکه را تغییر داد، پدیده‌های همچون جوانه زدن\LTRfootnote{budding},
 جدایی فاز مولکول‌های لیپیدی غشا\LTRfootnote{phase seporation}
\cite{Kohyama2003PRE, Gompper2007PRL, Laradji2004PRL},
 و تولید بازو‌ و لوله‌\LTRfootnote{tubulation}
بر سطح غشا
\cite{Ramakrishnan2013BiophysJ}
با این روش مطالعه شده‌است. 

اساس روش شبیه‌سازی مانتی کارلو تولید آمار درست از یک پدیده‌است و برای مطالعه‌ی دینامیک طراحی نشده‌است. البته در این روش می‌توان حرکت‌هایی طراحی کرد که پیوستگی مکان و زمان را حفظ کند و در نتیجه دینامیک شبه  واقعی تولید کرد. اما میزان واقعی بودن دینامیک تولید شده از این روش تابع میزان ظرافت و هوشمندی  حرکت‌های طراحی شده است و همیشه تقریبی است از دینامیکی که از معادلات حرکت حاصل می‌شود.

روش شبیه‌سازی دینامیک مولکولی روش مناسبی برای مطالعه حرکت ذرات است. در این روش با گسسته‌سازی و حل معادلات نیرو‌های وارد بر ذرات محاسبه شده و  مکان ذرات پیش‌بینی  می‌شود. از روش دینامیک مولکولی برای مطالعه‌ی غشا نیز استفاده شده‌است. خواص الاستیک غشای گلبول‌های قرمز
\cite{Hale2009SoftMatter, Geekiyanage2019PLOS},
 تغییر شکل گلبول‌های قرمز تحت جریان‌های بُرشی هیدرودینامیکی\LTRfootnote{hydrodynamic shear}
\cite{Noguchi2005PNAS},
 و  خواص پوسته‌های الاستیک و کپسید‌ها\LTRfootnote{capsids}
\cite{NelsonPRL1987, gomppernelson2012}
از جمله موضوعاتی است که با شبکه‌های مثلثی ثابت به کمک روش دینامیک مولکولی مطالعه شده‌است.

به علت محبوبیت زیاد روش دینامیک مولکولی میان دانشمندان، بسته‌های نرم‌ افزار‌ی بسیار زیادی همچون 
LAMMPS \cite{LAMMPS}, GROMACS \cite{GROMACS},
 و
OpenMM \cite{OpenMM2017}
ساخته‌ شده‌است.  این بسته‌ها معمولا شامل پتانسیل‌ها و ابزار‌های متداول و پایه‌ای مورد نیاز در تحقیقات هستند. در نتیجه آماده سازی شبیه‌سازی و محیط مجازی به کمک این نرم افزار‌ها با سرعت بالایی انجام می‌شود. به طور مثال نرم افزار‌های نام برده قادر به محاسبه‌ی نیروهای کوتاه بُرد و همچنین بلند بُرد بوده و قادر به حل معادلات حرکت پایه‌ای همچون معادله‌ی نیوتن، لانژون\LTRfootnote{Langevin},
 و معادله‌ی ناویر-استوکس\LTRfootnote{Navier-Stokes}
هستند. برخی از این بسته‌ها (مانند 
OpenMM)
 حتی قادر به محاسبه‌ی نیرو از روی جملات پتانسیل نیز هستند. 


در نتیجه، محققین می‌توانند بیشتر زمان‌شان را به طراحی مدل و بررسی ایده‌هایشان بپردازند و زمان کمتری را برای برنامه نویسی و عیب یابی  و تست کُد اختصاص دهند. از طرف دیگر محققانی که با شبیه‌سازی سر و کار دارند چه خواسته چه ناخواسته محاسبات خود را باید بر بستر کامپیوتر‌ی انجام دهند. سریع‌ترین  پردازنده‌های موجود در حال حاضر پردازنده‌های چند هسته‌ای\LTRfootnote{multi core CPU}
 کلاسیک است. این پردازنده‌ها با معماری‌های مختلفی در دسترس عموم قرار می‌گیرند. مهم‌ترین پردازند‌ه‌های چند هسته‌ای که  نسبت سرعت پردازش به  قیمت بالایی دارد پردازنده‌های چند هسته‌ای گرافیکی\LTRfootnote{GPU}
است. این پردازنده‌‌ها هسته‌های بسیار زیادی دارند که برای انجام محاسبات وابسته به مکان (مانند محاسبه‌ی فاصله‌ فضایی نقاط) 
و عملیات ماتریسی طراحی شده‌اند و در نتیجه برای انجام محاسبات مربوط به شبیه‌سازی دینامیک مولکولی بسیار مناسب هستند.  از طرفی یادگیری نحوه‌ی برنامه نویسی و استفاده از این پردازنده‌ها بسیار زمان‌بر است (حدود ۶ ماه تا یک سال). اینجاست که مزیت استفاده از بسته‌های نزم افزاری برای انجام تحقیقات دوباره اهمیت پیدا می‌کند. بیشتر نرم‌افزار‌های مُدرن زحمت برنامه نویسی و انجام محاسبات بر این پردازنده‌ها را انجام داده‌اند. در نتیجه سرعت محقق هم در مدل‌سازی و هم در دریافت نتیایج چندین برابر می‌شود.

مدل غشا‌های ریسمانی روشی است که با شبیه‌سازی دینامیک مولکولی مطابقت دارد و برای مطالعه‌ی غشا‌ها استفاده شده‌است
\cite{Abraham1989PRL}.
 غشای ریسمان بر روی یک شبکه‌ی مثلثی با توپولوژی ثابت تعریف می‌شود. در این روش طول اضلاع مثلث‌ها با یک حد بیشینه و کمینه کنترل شده و با استفاده از زاویه‌ی دوسطحی\LTRfootnote{dihedral angle}
انحنای غشا محاسبه می‌شود. همانطور که در فصل
\ref{ch:SimRev}
 اشاره شد، روش‌های مدل‌سازی ذرات لنارد جونز خود سامانده هم برای شبیه‌سازی غشا با استفاده از روش دینامیک مولکولی وجود دارد 
 \cite{Discher2004NatMat, Schindler2016BBA}
 اما این روش‌ها به منابع محاسباتی زیادی برای شبیه‌‌سازی غشا‌های چند میکرونی نیاز دارند. از طرف دیگر، می‌توان از روش‌های درشت دانه سازی استفاده کرد که غشا‌های بزرگ را با ذرات لنارد جونز شبیه‌سازی می‌کنند
 \cite{Li2005},
 استفاده‌ کرد. در این روش‌ها غشا با یک تک لایه از این ذرات مدل‌سازی می‌شود. اما چالش اصلی در این روش محاسبه‌ی درست انرژی انحناست. راه حل معمول تعریف یک شبکه‌ی مثلثی موقت بر اساس مکان لحظه‌ای ذرات و محاسبه‌ی انرژی انحنای شبکه است. در نتیجه ذرات با روش دینامیک مولکولی حرکت می‌کنند ولی برای محاسبه‌ی نیروهای حاصل از انحنا دائم شبیه‌سازی باید متوقف شود، از محیط نرم افزار خارج شد و پس از محاسبه‌ی نیرو‌های انحنا، دوباره وارد محیط شبیه‌سازی دینامیک مولکولی شد. در نتیجه تمام مزیت‌های استفاده از نرم‌افزار‌های دینامیک مولکولی از دست می‌رود.
 
 از طرفی مدل‌های ترکیبی دینامیک مولکولی و مانتی‌کارلو  وجود استفاده شده‌است
  \cite{Gompper2005PRE, Noguchi2005PNAS}.
  در این روش‌ها رفتار سیال‌گون غشا توسط یک شبکه‌ی مثلثی مدل‌ می‌شود که توپولوژی آن با الگوریتم مثلث بندی دینامیک با روش مانتی کارلو شبیه‌سازی می‌شود و رفتار الاستیک غشا توسط یک شبکه‌ی مثلثی ثابت مدل می‌شود که با روش دینامیک مولکولی حرکت می‌کند
 \cite{Gompper2005PRE, Noguchi2005PNAS}.
 این روش‌ها بسیار موفق بوده‌اند ولی لازمه‌ی پیاده‌سازی آنها طراحی و ساخت نرم افزار شخصی است.
  
















\section{\label{sec:statMech}
مکانیک آماری سطوح مثلث بندی شده
}
توجه ما در این بخش به مکانیک آماری سطوح 
$\cal S$
است که شکل یک غشا را بیان می‌کنند. مقادیر قابل اندازه‌گیری که برای سطح غشا تعریف می‌شود با میانگین‌گیری آنسامبلی بر تمامی چیدمان‌های ممکن سطوح غشا به این شکل تعریف می‌شود
\begin{equation}\label{eq:<A(S)>}
\langle X \rangle = \frac{ \int {\cal D}{\cal S}\, X({\cal S}) \exp\left[-\beta E({\cal S})\right] } 
                                     { \int {\cal D}{\cal S}\,                  \exp\left[-\beta E({\cal S})\right] }\ ,
\end{equation}
در معادله‌ی فوق 
$E({\cal S})$
مجموع انرژی ناشی از سطح، حجم، و انحنای غشاست. محاسبات عددی مربوط به اشکال غشا وابسته به شبکه
$\cal M$
است که سطح توسط آن گسسته سازی شده‌است. یک شبکه توسط مختصات نقاط
$\bm q=\{\bm q_1,\bm q_2, \ldots, \bm q_N \} \in \cal S$
و اتصالات
$\cal G$
که میان نقاط است تعریف می‌شود. به طور مشخص می‌توان از شبکه‌های‌ مثلثی برای گسسته سازی استفاده کرد. در شبکه‌های مثلثی، زمانی که اتصالات نقاط میان همسایگان نزدیک تعریف شود، مثلث‌های که سطح را تشکیل می‌دهند همپوشانی نخواهند داشت.

میان‌گین گیری آنسامبلی که در معادله‌ی 
\ref{eq:<A(S)>}
برای غشا تعریف شد، برای سطوح گسسته سازی شده بر روی تمامی درجات آزادی مِش (شبکه) تعریف می‌شود،
\begin{equation}\label{eq:<A(M)>}
\langle X \rangle = \frac{ \sum_{\cal G}\int d {\bm q}\, X({\cal G}, {\bm q}) \exp\left[-\beta U({\cal G}, {\bm q})\right] } 
                                     { \sum_{\cal G}\int d {\bm q}\,                                \exp\left[-\beta U({\cal G}, {\bm q})\right] }\ ,
\end{equation}
در این معادله انرژی
\begin{equation}
U({\cal M}) = E({\cal M}) + E_{aux}({\cal M})
\end{equation}
یک چیدمان از مش
 ${\cal M} = ({\cal G},\bm q)$
مجموع انرژی‌های گسسته سازی شده‌ی مساحت، حجم، و انحنای سطحی است که مش بیان می‌کند
$\cal S(\cal M)$,  $E({\cal M}) \approx E(\cal S(\cal M))$
و تمام پتانسیل‌های تکمیلی است که شکل، مساحت، اندازه اضلاع، ... مثلث‌ها را کنترل می‌کند است.

وزن آماری مشی
${\cal M} = ({\cal G},\bm q)$
که نماینده‌ی سطح
 $\cal S$
است
\begin{equation}\label{eq:w(M) for given S}
w({\cal G}, {\bm q}|{\cal S}) = \exp\left[-\beta U({\cal G}, {\bm q})\right] \prod_i \delta({\bm q}_i\in {\cal S}).
\end{equation}
تعریف می‌کنیم. در معادله‌ی فوق، جمله‌ی
 $\delta({\bm q}_i\in {\cal S})$
مانند یک دلتای دیراک است که مش‌هایی که نقاط آن در فاصله‌ی مناسبی نسبت به سطح 
 $\cal S$
قرار دارد را قبول می‌کند. در طول این رساله مش‌ها را به دو دسته‌ی کلی سخت و نرم دسته بندی می‌کنیم. مش‌های سخت مش‌هایی هستند که در آن پتانسیل‌های تکمیلی شکل مثلث‌های را به شکل تقریبا یکسانی بر سطح مش حفظ می‌کنند. به شکل مشابه مش‌های نرم مش‌هایی هستند که در آن پتانسیل‌های تکمیلی اجازه‌ می‌دهند مثلث‌های سطح اشکال بسیار متنوعی داشته باشند. 

در اکثر شبیه‌سازی‌های غشا از مش‌های سخت استفاده می‌شود. مهم‌ترین دلیل این امر استفاده از روش گسسته‌سازی انحنایی است که گامپر و کرول است
\cite{gompperkroll1996}
. این روش با این فرض طراحی شده است که تمامی مثلث‌هایی که بر مش قرار دارند (کم و بیش) شکل یکسان دارند. در نتیجه برای یک مش سخت احتمال
 $\exp\left[-\beta E_{aux}({\cal G}, {\bm q})\right]$
برای مش‌هایی که مثلث‌های یکسان ندارند بسیار کوچک می‌شود. مثلا برای تغییر شکل مشی مربعی مانند شکل 
\ref{fig:meshDT}a
به یک مش مستطیلی (شکل
\ref{fig:meshDT}b
) اتصالات میان نقاط باید کاملا تغییر کند. در نتیجه اگر برای شبیه‌سازی سطوحی که دائم در حال تغییر هستند از مش‌های سخت استفاده شود اتصالات مش
$\cal G$
دائما نیاز به تغییر دارد. از طرفی، با استفاده از مش‌های نرم می‌توان شکل‌های مختلفی را با اتصال یکسان
$\cal G$
نمایش داد. برای مثال در شکل
\ref{fig:meshDAR}
تغییر شکل یک مش مربعی (شکل
\ref{fig:meshDAR}a
) به یک مش مستطیلی (شکل
\ref{fig:meshDAR}b
) را با مش نرم نشان دادیم. به طور کمی، به شرطی که 
${\cal G}$
برای نمایش سطح
${\cal S}$
تطابق لازم داشته باشد، بیشترین وزن آماری مربوط به 
$\exp\left[-\beta E({\cal S})\right]$
خواهد بود و مشاهده‌ پذیر‌ها را می‌توان به این شکل اندازه‌گیری کرد
\begin{equation}\label{eq:<A(M)> soft meshes}
\langle X \rangle \approx
\langle X \rangle_{\cal G} = \frac{\int d {\bm q}\, X({\cal G}, {\bm q}) \exp\left[-\beta U({\cal G}, {\bm q})\right] } 
                                                  { \int d {\bm q}\,                                \exp\left[-\beta U({\cal G}, {\bm q})\right] }\ ,
\end{equation}

















\section{\label{sec:dynamicTriangulation}
الگوریتم مثلث‌بندی دینامیک
}
\begin{figure}[h]
\begin{center}
\includegraphics[width=13cm]{\MemDiscr/Pics/dynamicTri}
\caption{
تغییر مثلث بندی مِش با تغییر موضعی جفت مثلث‌ها میان چهار نقطه. در حالت اولیه (سمت چپ) دو مثلث با رئوس
$ABC$
و
$DBC$
تعریف شده. با تغییر ضلع مشترک بین دو مثلث از 
$BC$
به
$AD$
مثلث بندی جدید با رئوس
$BAD$
و 
$CAD$
تشکیل خواهد شد (سمت راست).
}
\label{fig:dynamicTri}
\end{center}
\end{figure}



روش  مثلث بندی دینامیک
\LTRfootnote{dynamic triangulation}
ابتدا توسط دیوید بول
\LTRfootnote{David Boal}
و همکارش 
\cite{Boal1992PRA}
در سال ۱۹۹۲برای مِش‌های مثلثی طراحی شد. هدف اصلی این روش شبیه‌سازی خاصیت سیال گون غشا با استفاده از شبکه‌های مثلثی بود. کمی‌ بعد در همان سال از بداع این روش گامپر و کرول با چاپ شبیه‌سازی موفق غشاهای سیال گون در مجله 
\cite{Gompper1992Science}
این روش را بسیار محبوب کرد. گامپر و کرول در این مطالعه از روش خمش دو سطحی برای محاسبه‌ی انرژی خمش استفاده کردند. همانطور که در این فصل اشاره شد این روش برای محاسبه‌ی خمش غلط است. در سال ۱۹۹۶ گامپر و کرول روش محاسبه‌ی ایتزیکسون را با روش مثلث بندی دینامیکی ترکیب کردند و با موفقیت رفتار سیال‌گون غشا را شبیه‌سازی کردند. 

در روش مثلث بندی دینامیک دو مثلث مجاور در نظر گرفته می‌شود. رئوس این دو مثلث مانند شکل 
\ref{fig:dynamicTri}
سمت چپ، چهار وجهی 
$ABCD$
را تشکیل خواهد داد. مثلث بندی در ابتدا دو مثلث 
$ABC$
و
$DBC$
را تعریف می‌کند. با حذف ضلع
$BC$
و ایجاد ضلع
$AD$
مثلث بندی جدید با مثلث‌های
$CAD$
و
$BAD$
تشکیل خواهد شد. تغییر مثلث بندی انرژی خمش و کششی مِش را تغییر خواهد داد. همانطور که در شکل با رنگ‌های قرمز و سبز نمایش داده شده، همسایه‌های مثلث‌های آبی در این فرآیند تغییر خواهد کرد. در نتیجه علاوه بر تغییر زاویه‌ی میان مثلث‌های آبی در دو مثلث بندی، زاویه میان همسایه‌ها نیز تغییر خواهد کرد. به لحاظ انرژی کششی، در صورتی که طول ضلع 
$BC$
و
$AD$
متفاوت باشد، انرژی کششی نیز تغییر خواهد کرد. انتخاب مثلث بندی با یک وزن متروپلیس انجام می‌شود. در این روش، ابتدا انرژی مثلث بندی در حالت اولیه محاسبه می‌شود
$E_i$
سپس مثلث بندی تغییر داده می‌شود و انرژی مش در چیدمان جدید محاسبه ‌می‌شود
$E_f$
. در صورتی که انرژی با مثلث بندی جدید کاهش پیدا کند مثلث بندی جدید حتما پذیرفته می‌شود. در صورتی که انرژی مثلث بندی جدید بیشتر باشد این چیدمان با یک وزن بولتزمن
\LTRfootnote{Boltzman}
انتخاب یا رَد خواهد شد. 

به علت ماهیت این الگوریتم بهترین روش برای شبیه‌سازی آن استفاده از روش مانتی کارلو
\LTRfootnote{Monte Carlo}
است. گامپر و کرول هنگام ترکیب الگوریتم مثلث بندی دینامیک با روش اندازه‌گیری خمش ایتزیکسون محدودیت‌های زیادی در انتخاب طول اضلاع قرار داد. نتیجه‌ی این محدودیت‌ها ایجاد توزیع یکنواخت نقاط بر سطح شبکه‌ی مثلثی و کنترل شکل و اندازه مثلث‌ها جهت پایدار کردن روش ایتزیکسون بود.

















\section{\label{sec:meshDynamics}
دینامیک مِش، روشی منطبق با دینامیک مولکولی
}
ما پیشنهاد می‌کنیم که به جای استفاده از روش دینامیک مثلثی (که هر بار بر اساس یک قدم مانتی کارلو مش را تغییر می‌دهد)، پیرو کارهایی که قبلا در این شاخه از فیزیک انجام شده بود

از مش‌های نرم دو بعدی با اتصالات ثابت برای شبیه سازی غشا استفاده شود. در این شبیه‌سازی‌ها حرکت نقاط بر روی سطح 
$\cal S$
با محدودیت بسیار کمی مواج است. در واقع ایده‌ی اصلی این است که نقاط مش تا زمانی باعث همپوشانی مثلث‌ها نشوند (شکل
\ref{fig:DARs}c
) می‌توانند آزادانه بر سطح دو بعدی غشا حرکت کنند. این شرط توسط پتانیسیل‌های تکمیلی
$E_{aux}({\cal M})$
کنترل می‌شود. یه طور عمومی مثلث‌ها یا پلاکت‌هایی که به هر نقطه اختصاص داده شده، نماینده‌ی یک مقدار ثابتی از غشا نیستند ولی نماینده‌ی کسری از جرم کل غشا هستند که با مساحت لحظه‌ای هر نقطه رابطه دارد
 $m_i = M a_i/A$
. برای غشایی که سطح آن تغییر نکند این رابطه را به شکل ساده‌تر
$m_i  \approx \rho a_i$
می‌توان نوشت. حرکت دسته‌ جمعی نقاط باعث می‌شود که سطح

در پاسخ به نیروهای خارجی سریع تغییر کند. در ۳ بُعد، می‌توان دینامیک مش را شبیه به یک توصیف شبه لاگرانژی از یک شارع دو بعدی درنظر گرفت. حرکت این رویه‌ی دوبعدی در جهات خراج از سطح لاگرانژی است ولی حرکت درون سطحی نقاط، که نماینده‌ی لیپید‌ها هستند، وابسته به پانسیل‌های تکمیلی مش است.

در این شرایط هامیلتونین سیستم را می‌توان به شکل زیر نوشت
\begin{eqnarray}
\mathcal H&=& \frac12 \sum_i^N \frac{\bm p_i^2}{m_i(\bm q)} + U(\bm q) 
\label{eq:HamiltonianGeneral}\\
 &=& \frac {A(\bm q)}{2M}  \sum_i^N \frac{\bm p_i^2}{a_i(\bm q)} + U(\bm q) 
 \label{eq:HamiltonianGeneral with explicit areas}\\
&\approx& \frac {1}{2\rho}  \sum_i^N \frac{\bm p_i^2}{a_i(\bm q)} + U(\bm q) \ ,
\label{eq:HamiltonianVariableMass}
\end{eqnarray}
که در اینجا
$\bm p=\{\bm p_1,\bm p_2, \ldots, \bm p_N \}$
و
$\bm q=\{\bm q_1,\bm q_2, \ldots, \bm q_N \}$ 
تکانه و مختصات است،
$m_i(\bm q) = Ma_i(\bm q)/A(\bm q)$
جرم نقاط،
$A(\bm q)$
مساحت لحظه‌ای مش، و 
$U(\bm q)$
مجموع انرژی‌های پتانسیل مش است که تابع مساحت، حجم، و انحنای مش است.

در این هامیلتونین، جرم نقاط تابع توزیع مختصات مش است
$m_i(\bm q)$
. اگر از تکانه‌ها انتگرال بگیریم، وزن آماری یک چیدمان مشخص از مش
$\bm q$
به شکل زیر تعریف می‌شود
\begin{equation}
\begin{aligned}
w(\bm q)&=\exp\left[-\beta U(\bm q)\right]\int d\bm p\exp\left(-\beta\frac {A(\bm q)}{2M}  \sum_i^N \frac{\bm p_i^2}{a_i(\bm q)}\right)\\
&=\prod_{i=1}^N\left[\frac{2\pi M}{\beta} \frac{a_i(\bm q)}{A(\bm q)}\right]^{\frac{3}{2}}\exp\left[-\beta U(\bm q)\right]\ .
\end{aligned}
\label{eq:microStateProbability1}
\end{equation}

همانطور که انتظار داشتیم،
$w(\bm q)$ 
به وزن بولتزمنی 
$\exp\left[-\beta U(\bm q)\right]$
وابسته است که به شکل سطح مربوط است. ضریب این جمله زمانی بیشینه است که تمامی پلاکت‌ها مساحت یکسانی داشته باشند
$a_i = A/N$
. در نتیجه می‌توانیم این وزن را به شکل وزن بولتزمن موثر بنویسیم
\begin{equation}
\begin{aligned}
w(\bm q)
&=\exp\left[\frac{3}{2}\sum_{i=1}^N\log\left\{\frac{2\pi M}{\beta} \frac{a_i(\bm q)}{A(\bm q)}\right\}-\beta U(\bm q)\right]\ .
\end{aligned}
\label{eq:microStateProbability}
\end{equation}
برای فهیمدن این وزن موثر، جمله‌ی بالا را بر اساس تغییرات مساحت پلاکت‌ها از مساحت میانگین پلاکت‌ها 
$a_i(\bm q)=A(\bm q)/N+\delta_i(\bm q)$
بازنویسی می‌کنیم
\begin{equation*}
\begin{aligned}
\log\left(\frac{2\pi M}{\beta} \frac{a_i(\bm q)}{A(\bm q)}\right)
&=\log\left(\frac{2\pi m}{\beta}\right)+\log\left(1+\frac{N\delta_i(\bm q)}{A(\bm q)}\right)\\
&\approx\log\left(\frac{2\pi m}{\beta}\right)+\frac{N\delta_i(\bm q)}{A(\bm q)}-\frac{1}{2}\left(\frac{N\delta_i\bm q)}{A(\bm q)}\right)^2.
\end{aligned}
\label{eq:logExpantion}
\end{equation*}
پس از جایگذاری این جمله در معادله‌ی 
\ref{eq:microStateProbability}
جمع جملات مرتبه‌ی صفرم یک ثابت عددی است، جمع جملات مرتبه‌ی اول بنان به تعریف صفر خواهد بود. در نتیجه جملات باقیمانده تنها از مربته‌ی دوم خواهد بود
\begin{equation}
\begin{aligned}
w(\bm q)\propto\exp\left[-\beta\left( \frac{3}{2}k_BT\sum_{i=1}^N \frac{1}{2}(\frac{N}{A_0}\delta_i(\bm q))^2+U(\bm q)\right)\right] \ ,
\end{aligned}
\label{eq:microStateProbabilityExpansion}
\end{equation}
این جمله شبیه به یک پتانسیل هارمونیک است که اندازه‌ی پلاکت‌ها را نزدیک به مساحت میانگین حفظ می‌کند.



اگر از تغییرات کوچک مساحت کل غشا صرفه نظر کنیم، معادلات حرکت نقاط مش را می‌توانیم با استفاده از معادلات حرکت هامیلتون محاسبه کنیم. از معادله‌ی اول هامیلتون 
$\dot {\bm q_i}=\frac{\partial}{\partial \bm p_i}\mathcal H$
می‌توانیم تکانه‌ی نقاط را حساب کنیم
$\bm p_i = \rho a_i(\bm q)\dot {\bm q_i}$
. و معادله‌ی دوم 
$\frac{d}{dt} \bm p_i = -\frac{\partial}{\partial \bm q_i}\mathcal H$
به شکل زیر ساده می‌شود
\begin{equation}
\frac{d}{dt}(\rho a_k(\bm q) \dot {\bm q}_k) = -\frac{\partial U(\bm q)}{\partial \bm q_k} - \frac{1}{2} \frac{\partial}{\partial \bm q_k}\sum_{i=1}^N\frac{\bm p_i^2}{\rho a_i(\bm q)} \end{equation}
یا
\begin{eqnarray}
\lefteqn{  \rho a_k(\bm q) \ddot{\bm q_k}  =}
\label{eq:HamiltonianGeneralEOMMomentum}\\
 &=& -\frac{\partial U(\bm q)}{\partial \bm q_k} + \frac{1}{2}\rho \sum_{i=1}^N\dot{\bm q_i}^2\frac{\partial a_i(\bm q)}{\partial \bm q_k}
 - \rho  \dot{\bm q_k}\sum_{i=1}^N\frac{\partial a_k(\bm q)}{\partial \bm q_i}\dot{\bm q_i}\nonumber
\end{eqnarray}
حل این معادلات دشوار است و برای بررسی دینامیک هر سیستمی که در آن تاثیر جرم مهم باشد (مانند حباب صابون) مورد نیاز است. در این رساله ما علاقمند به شبیه‌سازی رفتار غشایی هستیم که دینامیک سطح توسط جریان شارع درون و شارعی که غشا در آن قوطه‌ور شده حکم می‌شود
\cite{milnersafranPRA1987, schneider1984}
. از آنجایی که در این سیستم اینرسی نقش نخواهد داشت، برای ساده سازی معادلات فوق، جرم تمام نقاط مش را یکسان تقریب می‌زنیم
\begin{eqnarray}
\mathcal H&=& \frac12 \sum_i^N \frac{\bm p_i^2}{m_i} + U(\bm q) 
\label{eq:HamiltonianFixedMass} \ ,
\end{eqnarray}
در این حالت وزن بولتزمنی که اندازه‌گیری‌های آماری را تایین می‌کند
\begin{equation}
\begin{aligned}
w(\bm q)\propto\exp\left\{-\beta U(\bm q)\right\} \ ,
\end{aligned}
\label{eq:microStateProbability HamiltonianFixedMass}
\end{equation}
و معادلات حرکت به معادلات حرکت نیوتن ساده می‌شود
\begin{equation}
m_i \ddot{\bm q_k} = -\frac{\partial U(\bm q)}{\partial \bm q_k} \ .
\label{eq:EoM for HamiltonianFixedMass}
\end{equation}
برای حل دقیق‌تر این معادلات برای غشایی که در خلا حرکت می‌کند، می‌توان اندازه‌ی جرم تمام نقاط را با اندازه‌ی جرم متوسط هر نقطه جایگزین کرد (کاری که در این رساله انجام شده) 
$m_i = \rho \langle a_i \rangle$
یا با بازسازی مش، در هر مرحله از شبیه‌سازی از مشی‌ استفاده کرد که توزیع نقاط بر روی سطح همیشه یکنواخت باشد.





 














\section{\label{sec:DAR}
روش توزیع دینامیکِ مساحت در شبکه‌ی مثلثی
}
روش توزیع دینامیک مساحت 
\LTRfootnote{Dynamic Area Redistribution}
با هدف  شبیه‌سازی خاصیت سیال‌گون غشا در محیط دینامیک ملکولی در گروه ماده چگال نرم دانشکده فیزیک دانشگاه صنعتی شریف طراحی شده‌است.  غشا سیال است و ملکول‌های لیپید در آن رفتار پخشی دارند. در صورتی که به نقطه‌ای از غشا تنش بُرشی توسط نیروی خارجی وارد شود، ملکول‌های لیپیدی بر سطح غشا حرکت می‌کنند. هنگامی که نیروی‌ خارجی متوقف شود، ملکول‌های لیپید دلیلی برای بازگشت به محل اولیه خود ندارند و مانند یک سیال به حرکت پخشی خود ادامه می‌دهند. 

\begin{figure}[h]
\begin{center}
\includegraphics[width=12cm]{\MemMethod/Pics/meshDiffusion.pdf}
\caption{
شبکه بندی سطح غشا تشکیل شده از 
$N$
ملکول لیپیدی را نمایش می‌دهد. در هنگام تعادل ترمودینامیکی چگالی ملکول‌ها در هر تکه از شبکه به طور متوسط
$\rho=N/A_0$
است.
}
\label{fig:cylindermesh}
\end{center}
\end{figure}

غشایی با مساحت 
$A_0$
را فرض می‌کنیم که از 
$N$
ملکلول لیپیدی تشکیل شده‌است. در صورتی که سطح غشا با 
$M$
قسمت با مساحت 
$A_0/M$
شبکه بندی شود، هنگام  تعادل ترمودینامیکی، به طور متوسط چگالی ملکول‌های لیپیدی در هر قسمت از شبکه برابر با 
$\rho=N/A_0$
است. اگر تنش  به قسمت
$i$
از شبکه وارد شود که باعث انبساط آن تکه شود، چگالی ملکول‌های آن کمی کمتر از متوسط خواهد شد،
$\rho_i=\rho^-$
. از آنجایی که تعداد ملکلو‌ل‌ها در سطح غشا یکسان است و غشای لیپیدی ضریب فشردگی بسیار بزرگی دارد، چگالی تمام قسمت‌های دیگر شبکه کمی افزایش می‌یابد (
$\rho_j=\rho^+$
برای هر قسمت که
$i\neq j$
). از آنجایی که غشا در تعادل ترمودینامیکی به سر می‌برد با گذشت زمان کم ملکلول‌های غشا (با دینامیک پخشی) در قسمت‌های مختلف شبکه جابجا شده و چگالی ملکولی در تمام قسمت‌ها را دوباره یکسان می‌کنند. در نتیجه تا زمانی که مساحت کل غشا تغییر نکند اندازه‌ی قسمت‌های مختلف شبکه می‌تواند با دمای محیط اُفت  و خیز کند. در این توصیف هر قسمت از شبکه نماینده‌ی تعداد ثابتی از ملکول‌های لیپیدی نیست ولی چگالی عددی ملکلول‌ها در سراسر غشا یکسان است.

روش توزیع دینامیک مساحت روی هر شبکه‌ بندی قابل پیاده‌سازی است. در این رساله، این روش بر روی شبکه‌های مثلثی پیاده‌سازی شده‌است. در دهه‌های گذشته مطالعات زیادی بر نحوه‌ی شبیه‌سازی با استفاده از شبکه‌های مثلثی شده‌است. در نتیجه نحوه‌ی گسسته سازی انرژی‌های مورد نیاز برای شبیه‌سازی یک غشا (انرژی خمش، مساحت، و حجم) بر بستر این شبکه‌ها از مطالعات گذشتگان در دسترس است. برای شبیه‌سازی رفتار سیال‌گون غشا کافی ‌است که انحنای غشا بر روی سطح به درستی تعریف شود و حجم و مساحت کل غشا در طول شبیه‌سازی قابل کنترل باشد. 

\begin{figure}[h]
\begin{center}
\includegraphics[width=12cm]{\MemMethod/Pics/DAR.pdf}
\caption{
شکل یک مش منظم
$a)$
درهم
$b)$
و مشی با حرکت غیر مجاز 
$c)$
برای یک مربع تخت را نشان می‌دهد
}
\label{fig:DARs}
\end{center}
\end{figure}


برای آشنایی با روش توزیع دینامیک مساحت یک مثال ساده را بررسی می‌کنیم. این روش را برای یک مربع تخت به مساحت 
$S$
و خمش صفر
$C_1=C_2=C_s=0$
تعریف می‌کنیم. مانند شکل
\ref{fig:DARs}a
می‌توان مربع را با یک مش منظم نمایش داد. از آنجایی که مثلث‌ها همپوشانی نداند، جمع مساحت آن‌ها برابر با 
$S$
است. زاویه‌ی دو سطحی میان تمام جفت مثلث‌ها صفر است در نتیجه خمش نیز همه‌جای مش برابر با صفر است. مانند شکل
\ref{fig:DARs}b
نقاط مش را روی سطح جابجا می‌کنیم. تا زمانی که نقاط مش هنگام جابجا شدن از روی ضلعی عبور نکنند، همچنان جمع مساحت آن‌ها برابر با 
$S$
خواهد بود و خمش همه‌ جا صفر. اما اگر مانند شکل
\ref{fig:DARs}c
نقطه‌ی 
p
از روی ضلعی عبور کند، حداقل یک مثلث پدید می‌آید که با مثلث‌های دیگر همپوشانی دارد (مساحت نارنجی رنگ). در این حالت جمع مساحت مثلث‌ها دیگر
$S$
نیست. همچنین زاویه‌ی میان جفت مثلثی که ضلع عبور شده میان آن‌ها مشترک است دیگر صفر نیست بکله
$\phi=\pi$
است. در صورتی که حرکت نقاط به شکلی محدود شود که هیچ نقطه‌ای نتواند از ضلعی عبور کند، تمام نقاط چیدمان‌های نقاط می‌توانند به درستی خمش و مساحت مربع را توصیف کنند.

این مفهوم را می‌توان برای سطوح ۳ بعدی نیز در نظر گرفت. در روش توزیع دینامیک مساحت، مساحت سلول‌های مش می‌توانند آزادانه حرکت کنند تا زمانی که،

۱- اتصالات یا توپولوژی مش تغییر نکند

۲- سلول‌های مش همپوشانی نداشته باشند

۳- تنها تغییرات مساحت کل مش هزینه‌ی انرژی داشته باشد.






\section{\label{sec:meshRecipe}
تهیه‌ی مِش
}
\begin{figure}[h]
\begin{center}
\includegraphics[width=12cm]{\MemMethod/Pics/MeshTypes.pdf}
\caption{
عکس چهار نوع مِش استفاده شده در این رساله. مِش منظم (بالا سمت چپ)، تصادفی (بالا سمت راست)، دَرهَم منظم (پایین سمت چپ)، و مِش دَرهَم تصادفی (پایین سمت راست).
}
\label{fig:meshTypesMesthod}
\end{center}
\end{figure}
در شکل
\ref{fig:meshTypesMesthod}
نمونه‌ای از مِش‌هایی که در این رساله استفاده شده را می‌توان یافت. مِش‌ها به ترتیب زیر قابل تهیه هستند:
\subsection{
مِش منظم
}
مش منظم با شبکه‌‌بندی کردن یک بیست وجهی منتظم ساخته می‌شود. در این شبکه‌ بندی ۱۲ نقطه‌ی نقص با درجه‌ی ۵ در ۱۲ گوشه‌ی بیست وجهی ایجاد خواهد شد و نقاط دیگر همگی از درجه‌ی ۶ خواهند بود. در نتیجه به ازای هندسه بسیار متقارن کُره تنها یک نمونه از چنین شبکه‌ای به ازای تعداد نقاط وجود دارد. همچنین  برای ساخت چنین مِشی تنها می‌توان  تعداد مشخصی نقطه که با رابطه‌ی 
\begin{equation}
N_{mesh}=10\times i^2+2,
\end{equation}
محاسبه می‌شود، داشت،  که در اینجا 
$i$
مقادیر صحیح غیر صفر دارد. ‌شبکه‌ بندی با هر نرم‌افزار مش بندی قابل انجام است. در این مطالعه ما از نرم افزار 
Blender
برای تولید این مِش به خصوص استفاده کردیم. 

\subsection{
مِش تصادفی
}
مقدمه‌ی تهیه‌ی این نوع مش  قرار دادن 
$N$
نقطه به شکل تصادفی بر سطح یک کُره‌ به شعاع ۱ است. سپس به کمک 
$N$
پتانسیل فنر هارمونیک با ضریب سختی زیاد، این نقاط را به مرکز کُره متصل کرده و بر سطح کُره مقید می‌کنیم. سپس میان این نقاط برهمکنش بلند بُرد دافعه تعریف می‌کنیم. هر برهمکنش دافعه‌ی بلند بُردی که  توزیع تقریبا یکنواخت نقاط بر سطح کُره ایجاد کند مورد قبول است. در این رساله از پتانیسل 
\begin{equation}
U_{EV}=10^{-3}\left(\frac{2\rho_0}{r}\right)^6,
\end{equation}
استفاده شد. در اینجا 
$r$
فاصله‌ی فضایی نقاط از یکدیگر بوده و 
$\rho_0$
شعاع متوسط هر نقطه روی کُره است. شعاع متوسط کُره با شعاع 
$N$
دایره تخمین زده شده‌است که  سطح کُره را پوشش می‌دهند
\begin{equation}
N\pi\rho_0^2=4\pi\rightarrow \rho_0=\frac{2}{\sqrt{N}}.
\end{equation} 
سپس ذرات تحت دینامیک لانژون شروع به حرکت کرده و با انتخاب دمای پایین برای ترموستات لانژون می‌توان سرعت ذرات را پله پله کم کرد تا جایی که ذرات در فاصله‌ی تعادلی نسبت به یکدیگر قرار گیرند. البته از هر روش کمینه کردن انرژی\LTRfootnote{energy minimisation}
برای یافتن مختصات تعادلی نقاط می‌توان استفاده کرد. پس از یافت مختصات تعادلی نقاط با استفاده از الگوریتم مثلث‌ بندی دِلونی\LTRfootnote{Delaunay triangulation algorithm}
\cite{DelaunayTriangulation1997}
یک شبکه‌ی مثلثی تصادفی ایجاد خواهد شد. با تکرار این دستورنامه با استفاده از بذر‌های\LTRfootnote{seed}
 مختلف برای تولید اعداد تصادفی غیر یکسان می‌توان مِش‌های تصادفی متفاوتی تولید کرد.

\subsection{
مِش‌های درهم
}
برای تغییر توزیع مساحت مثلث‌های مِش‌های منظم و تصادفی‌  نقاط آن را با فنر‌های هارمونیک با ضریب سختی بالا مقید به حرکت بر سطح کُره کرده و سپس به تمام نقاط، سرعت و جهت تصادفی اختصاص داده می‌شود. با محدود کردن طول ارتفاع مثلث‌های مش با پتانسیل 
$U_{h}$
(جزئیات این پتانسیل در بخش 
\ref{sec:auxPotentials}
قرار دارد) حد پایین برای اندازه‌ی مثلث‌ها تعیین شد. این پتانسیل تضمین می‌کند که اضلاع مثلث‌ها بر اثر حرکت کاتوره‌ای نقاط همدیگر را قطع نکنند. ولی این پتانسیل برای جلوگیری از تا شدن مثلث‌ها بر روی سطح کافی نیست. برای پیشگیری از این اتفاق، پتانسیل غیر خطی خمشی دوسطحی
$U_{\phi}$
(جزئیات این پتانسیل در بخش 
\ref{sec:auxPotentials}
قرار دارد) میان تمام جفت مثلث‌های مش تعریف شد. در نتیجه دینامیک چنین نقاطی تحت معادله‌ی حرکت نیوتن (انتگرال گیری وِرلِه سرعتی\LTRfootnote{velocity Verlet})
 مساحت‌ مثلث‌ها افت و خیز می‌کند و توپولوژی اولیه مِش استفاده شده کاملا محفوظ می‌ماند.
 






\section{\label{sec:areaVolumeD}
پتانسیل کنترل مساحت و حجم
}
معادله‌ی انرژی هارمونیک مربعی که برای بیان رفتار تغییر مساحت غشا محاسبه شد (معادله‌ی
\ref{eq:stretchEnergySigma}
) را می‌توان بر اساس جمع بر روی لیست مثلث‌های یک مش به شکل زیر بازنویسی کرد
\begin{equation}
\begin{aligned}
U_{A}&=\frac{1}{2}\frac{k_A}{A_0}\left[\sum_{i=1}^{N_{tri}}a_i-A_0\right]^2\\
&=\frac{1}{2}\frac{k_A}{A_0}\left[\sum_{i=1}^{N_{tri}}a_i^2+A_0^2-2A_0\sum_{i=1}^{N_{tri}}a_i+2\sum_{i\neq j}a_ia_j\right]\\
&=\frac{1}{2}\frac{k_A}{A_0}\left[\sum_{i=1}^{N_{tri}}(a_i-A_0)^2-(N_{tri}-1)A_0^2+2\sum_{i\neq j}a_ia_j\right]\\
&=\sum_{i=1}^{N_{tri}}\frac{1}{2}\frac{k_A}{A_0}\left((a_i-A_0)^2-\frac{(N_{tri}-1)}{N_{tri}}A_0^2\right)\\
&~~+\sum_{i\neq j}\frac{k_A}{A_0}a_ia_j.
\label{eq:GlobalAreaPotentialExpansion}
\end{aligned}
\end{equation}
. در جمع بالا 
$A_0$
مساحت تعادلی مش
$a_i$
مساحت مثلث
$i$
ام در لیست مثلث‌های مش است. در صورتی که فرض کنیم مثلث
$i$
ام از سه راس 
$l,k$
و
$f$
تشکیل شده مساحت مثلث را می‌توان بر اساس طول دو ضلع
$\ell_{lk}$
و
$\ell_{lf}$
 ا و زاویه‌ میان این دو ضلع
 $\theta_{klf}$
  به شکل 
 \begin{equation}
a_i=\frac{1}{2}\ell_{lk}\ell_{lf}|\sin(\theta_{klf})|.
\label{eq:singleTriangleArea}
\end{equation}
 محاسبه شود. پارامتر 
 $N_{tri}$
تعداد مثلث‌های مش را مشخص می‌کند و 
$k_A$
ضریب فشردگی سطح غشا است که مقدار آن توسط شرایط فیزیکی مسئله مشخص می‌شود. تمام این پارامتر‌ها در ابتدای شبیه‌سازی تنظیم شده و تا انتهای شبیه‌سازی ثابت می‌مانند. جمع اول در معادله‌ی 
 \ref{eq:singleTriangleArea}
 را می‌توان به عنوان یک پتانسیل ۳ ذره‌ای برای تمام مثلث‌های مش تعریف کرد. جمع دوم جمله‌ای است فرآیند تبادل مساحت میان مثلث‌ها را ممکن می‌کند. این جمع به شکل یک پتانیسیل ۶ ذره‌ای میان انتخاب ۲ از تعداد تمام مثلث‌های مش
 ${N_{tri} \choose 2}$
 تعریف می‌شود. محاسبات مربوط به این بخش از پتانسیل به لحاظ منابع کامپیوتری مورد نیاز بسیار هزینه بر است.


به شکل مشابهی می‌توان انرژی حجم غشا (معادله‌ی 
\ref{eq:VolumeEnergy}
) را به شکل جمع روی لیستی از مثلث‌ها بازنویسی کرد،
\begin{equation}
\begin{aligned}
U_{V}&=\frac{1}{2}\frac{k_V}{V_0}\left[\sum_{i=1}^{N_{tri}}v_i-V_0\right]^2\\
&...\\
&=\sum_{i=1}^{N_{tri}}\frac{1}{2}\frac{k_V}{V_0}\left((v_i-V_0)^2-\frac{(N_{tri}-1)}{N_{tri}}V_0^2\right)\\
&~~+\sum_{i\neq j}\frac{k_V}{V_0}v_iv_j.
\label{eq:VolumePotentialExpansion}
\end{aligned}
\end{equation}
. در اینجا
$v_i$
حجم هرمی است که پایه‌ی آن مثلث‌
$i$
ام مش است. همانطور که در بخش
\ref{sec:areaVolumeDiscr}
توضیح داده شد، حجم غشا با یک ضرب سه گانه قابل محاسبه‌است و حجم حاصل مستقل از مختصات راس هرم است. جهت ساده شدن محاسبات راس مثلث ها  مرکز مختصات انتخاب شده‌است. در نتیجه با فرض اینکه مثلث 
$i$
ام از رئوس
$l,k$
و
$f$
تشکیل شده، حجم هر به شکل زیر در پتانیسیل فوق جایگذاری می‌شود،
\begin{equation}
\begin{aligned}
v_i=-\frac{1}{6}(&x_l(y_fz_k-z_fy_k)+\\
&x_k(y_lzf-z_ly_f)+\\
&x_f(y_kz_l-z_ky_l)).
\end{aligned}
\label{eq:VolumeTripleProductDef}
\end{equation}
. علامت منفی در معادله‌ی فوق به این دلیل است که ترتیب رئوس طوری در نظر گرفته شده که جهت ضرب برداری آن رو به خارج از غشا باشد. پارامتر 
 $N_{tri}$
تعداد مثلث‌های مش را مشخص می‌کند و 
$k_V$
مدول فشوردگی سیال درون غشا است که مقدار آن توسط شرایط فیزیکی مسئله مشخص می‌شود. تمام این پارامتر‌ها در ابتدای شبیه‌سازی تنظیم شده و تا انتهای شبیه‌سازی ثابت می‌مانند. همچنین مشابه به پتانسیل سطح، پتانسیل حجم را می‌توان به دو برهمکنش ۳ ذره‌ای و ۶ ذره‌ای تقسیم کرد. پتانسیل ۳ ذره‌ای برای تمامی مثلث‌های مش تعریف شده و پتانسیل ۶ ذره‌ای میان انتخاب ۲ از 
$N_{tri}$
برای تمام مثلث‌های متمایز روی مش است
 ${N_{tri} \choose 2}$
.

از آنجایی که هر دو پتانسیل بر لیست یکسانی از مثلث‌ها و انتخاب‌های میان آنها تعریف می‌شود، در صورتی در شبیه‌سازی هم حجم هم مساحت کنترل می‌شود، می‌توان این جمع دو پتانسیل را به شکل یک پتانسیل ۳ ذره‌ای و یک پتانسیل ۶ ذره‌ای سطح و حجم پیاده سازی کرد.












\section{\label{sec:curvatureD}
پتانسیل انحنای متوسط
}
%\begin{figure}[htbp]
%\begin{center}
%\includegraphics[width=10cm]{\MemRes/Pics/UnitSphereCurvatureORW}
%\caption{
%انرژی انحنا چهار مدل گامپر (قرمز)، یولیشر (سبز)، گامپر-بریسنتریک (آبی)، و یولیشر-ورنوی (نارنجی) برای مش‌های مختلف محاسبه شده‌است. به غیر از مش‌های درهم تصادفی، انرژی انحنا برای تمامی مش‌ها با افزایش تعداد نقاط شبکه بهتر شده. به غیر از مش منظم (که تنها یک نمونه از آن برای هر تعداد نقطه وجود دارد) مقادیر محاسبه شده حاصل از میان‌گین گیری بر روی ۵۰ نمونه‌ی مستقل انجام شده‌است.
%}
%\label{fig:unitsphereBending}
%\end{center}
%\end{figure}

انرژی یک کره‌ به شعاع یک و ضریب سختی 
$\kappa=1[\varepsilon]$
برابر 
$8\pi\kappa\approx 25.1[\varepsilon]$
است. این مقدار با جایگذاری 
$C_1=C_2=\frac{1}{R}$
و
$C_0=\frac{2}{R_\infty}=0$
در معادله‌ی 
\ref{eq:HelfrichBendingEnergy}
قابل محاسبه‌ است. ستون 
$(c)$
 در شکل 
\ref{fig:unitsphereAll}
انرژی حاصل از محاسبه‌ی انحنا با چهار مدل گامپر (قرمز)، یولیشر (سبز)، گامپر-بریسنتریک (آبی)، و یولیشر-ورنوی (نارنجی) را نمایش می‌دهد. با نگاه کلی به محور عمودی می‌توان نتیجه گرفت که هر چهار مدل برای تخمین انرژی انحنای غشا مناسب هستند. به این نکته باید توجه کرد که در بدترین حالت ممکن خطای محاسبه‌ی انحنا در شکل 
\ref{fig:unitsphereAll}
مربوط به مش‌های تصادفی درهم است که حدود
$\sim4\%$
است. این خطا در برابر خطای اندازه‌گیری ضریب سختی خمشی غشا‌ها (
$\sim50\%$
) ناچیز است.

برای تمامی مش‌ها دقت اندازه‌گیری انرژی انحنا برای مدل‌های بر پایه‌ی مساحت ورنوی (گامپر و یولیشر-ورنوی) دقت اندازه‌گیری با افزایش نقاط مش بهتر شد. این رفتار برای محاسبات انجام شده با مدل‌های بر پایه‌ی مساحت بریسنتریک (یولیشر و گامپر-بریسنتریک) روی مش‌های منظم و تصادفی نیز به همین ترتیب است. درهم شدن مش خطای اندازه‌‌گیری را برای کمترین نقاط (۱۰۰۲ نقطه) حدود 
$\sim2\%$
کرده. 

با افزایش تعداد نقاط در مش دو سناریو وجود دارد. در صورتی که مش‌ منظم باشد و نقاط نقص بر روی آن تعداد بسیار کمی داشته باشد (مش منظم ۱۲ نقطه با درجه‌ی ۵ دارد) با افزایش تعداد نقاط روی سطح دقت اندازه‌گیری بهتر می‌شود. اما در صورتی که مش تعداد نقطه‌ی نقص زیادی داشته باشد، با افزایش تعداد نقاط شبکه (و به تبع افزایش تعداد نقاط نقص) خطای اندازه‌گیری افزایش می‌یابد. در واقع اثر درهم کردن با افزایش تعداد نقاط مش قابل جبران است ولی اثر توپولوژی قابل حذف نیست.



\begin{figure}[htbp]
\begin{center}
\includegraphics[width=\columnwidth]{\MemRes/Pics/IJ_numerator.pdf}
\caption{
مقادیر صورت (ستون چپ) و مخرج (ستون راست) مدل گسسته اندازه‌گیری انحنای گامپر و یولیشر برای نقاط مختلف بر روی مش‌های منظم، منظم درهم، تصادفی، و تصادفی درهم با ۱۰۰۲ نقطه رسم شده‌است. رنگ بنفش، آبی، و قرمز به ترتیب  نقاط با درجه‌ی ۵، ۶، و ۷ را مشخص می‌کنند. برای مش‌های منظم و تصادفی صورت این دو مدل پاسخ تقریبا یکسانی داردند و اختلافشان ناشی از وزنی است که به نقاط با درجه‌ی ۵ و ۷ اختصاص داده شده‌است. داده‌ها در مش‌های درهم پخش شده اما همچنان همین روند نیز در آن‌ها دیده می‌شود.
}
\label{fig:unitsphereBendingScatter}
\end{center}
\end{figure}

در ستون سمت چپ شکل 
\ref{fig:unitsphereBendingScatter}
صورت کسر انرژی انحنای گامپر در مقایسه با صورت یولیشر (سمت چپ و راست معادله‌ی 
\ref{eq:JulicherItzyksonNumerator}
) برای نقاط روی مش‌های منظم، منظم درهم، تصادفی، و تصادفی درهم رسم شده‌است. در ستون سمت راست مساحت بریسنتریک (معادله‌ی 
\ref{eq:BarycentricArea}
) بر حسب مساحت ورنوی (معادله‌ی
\ref{eq:voronoiArea}
) برای همان نقاط محاسبه شده‌است. رنگ‌های بنفش، آبی،‌ و قرمز به ترتیب نقاط با درجه‌ی ۵، ۶، و ۷  را مشخص می‌کنند. شیب ۱ با خط مشکی مشخص شده. نقاط روی این خط مقادیر یکسان در هر دو نوع محاسبه دارند. به غیر از مش منظم،‌ نقاط رسم شده برای هر مش از ۱۰ نمونه مش با تعداد نقاط ۱۰۰۲ انتخاب شده‌است. یعنی در هر کدام از نمودارها ۱۰۰۲۰ نقطه نمایش داده شده‌است.

داده‌های ستون سمت راست نشان می‌دهد که اختلاف دو مدل گامپر و یولیشر  در  وزنی است که به نقاط اختصاص می‌دهند. همچنین به طور عمومی یولیشر وزن بیشتری (مساحت کمتری) به نقاط با درجه‌ی ۵ و وزن کمتری (مساحت بیشتری) به نقاط با درجه‌ی ۷ نسبت به مدل گامپر در نظر می‌گیرد. این روند در مش‌های درهم نیز دیده می‌شود.

\begin{figure}[htbp]
\begin{center}
\includegraphics[width=\columnwidth]{\MemRes/Pics/n5byn7}
\caption{
نسبت متوسط تعداد نقاط با درجه‌ی ۵ (
$N_5$
) به تعداد نقاط با درجه‌ی ۷ (
$N_7$
) برای مش‌های مثلثی تصادفی بر حسب تعداد نقاط روی مش رسم شده‌است. نقاط حاصل متوسط‌گیری روی ۵۰ نمونه است.
}
\label{fig:n5n7}
\end{center}
\end{figure}


تعداد نقاط با درجات مختلف در یک شبکه مثلثی به شکل مش و تعداد نقاط  و خطوط لغزش روی آن بستگی دارد 
\cite{Nelson2000PRB}
. در شکل
\ref{fig:n5n7}
نسبت متوسط تعداد نقاط با درجه‌ی ۵ (
$N_5$
) به تعداد نقاط با درجه‌ی ۷ (
$N_7$
) برای مش‌های تصادفی رسم شده‌است. مش‌های تصادفی بیشتر از نقاط با درجه‌ی ۶ ساخته شده‌اند که تقریبا در هر دو مدل محاسبه‌ی خمش به طور متوسط اندازه‌ی یکسانی دارند. در مورد مش‌های درهم تصادفی انرژی انحنا با تعداد نقاط با درجات غیر ۶ تغییر می‌کند. در صورتی که نسبت تعداد نقاط با درجه‌ی ۵ به ۷ در شبکه‌های ریز و درشت یکسان می‌بود، شاهد رشد خطی خطا با افزایش اندزاه‌ی مش‌بندی شبکه‌ در مش‌های درهم تصادفی در شکل 
\ref{fig:unitsphereAll}
می‌بودیم. از آنجایی که این نسب با افزایش اندازه‌ی مش‌بندی تغییر می‌کند و به ۱ میل می‌کند شاهد میل‌ انرژی انحا برای مدل‌های بریسنتریکی در مش‌های درهم تصادفی هستیم.



%\begin{figure}[htbp]
%\begin{center}
%\includegraphics[width=12cm]{\MemRes/Pics/UnitSphereDeformation_mesh_30}
%\caption{
%انرژی انحنا (ستون چپ) و نیروی برگرداننده (ستون راست) محاسبه‌ شده حاصل از تغییر شکل مش کروی با اضافه شدن مد
%$Y_{2,0}(\theta,\phi)$
%با شدت‌های مختلف برای چهار مدل گامپر (قرمز)، یولیشر (سبز)، گامپر-بریسنتریک (آبی)، و یولیشر-ورنوی (نارنجی) روی مش‌های مختلف رسم شده‌است. خطوط مشکی برای ستون سمت چپ و راست به ترتیب پیش بینی
% مرتبه‌ی دوم انرژی انحنا و نیروی بازگرداننده برای تغییر انحنا، یعنی معادلات
%\ref{eq:curvatureY20}
%و
%\ref{eq:curvatureForceY20}
%را نشان می‌دهند.  شدت مد 
%$u_{2,0}=0$
%مربوط به شکل کاملا کروی (عکس سمت چپ) و شدت مد 
%$u_{2,0}=1$
%مربوط به شکل دمبلی (عکس سمت راست) است. یک نمونه از هر مش برای حالت کروی و دمبلی در ردیف مربوته رسم شده‌است. به غیر از مش منظم (که تنها یک نمونه از آن برای هر تعداد نقطه وجود دارد) مقادیر محاسبه شده حاصل از میان‌گین گیری بر روی ۵۰ نمونه‌ی مستقل انجام شده‌است.
%}
%\label{fig:unitsphereBendingULM20}
%\end{center}
%\end{figure}

\begin{figure}[htbp]
\begin{center}
\includegraphics[width=\columnwidth]{\MemRes/Pics/IJ_numerator_ULM20.pdf}
\caption{
مقادیر صورت (ستون چپ) و مخرج (ستون راست) مدل گسسته اندازه‌گیری انحنای گامپر و یولیشر برای نقاط مختلف بر روی مش‌های منظم، منظم درهم، تصادفی، و تصادفی درهم که با شدت مد 
$u_{2,0}=1$
به شکلی دمبلی درآمده‌اند رسم شده‌است. داده‌ها مربوط به مش‌های ۱۰۰۲ نقطه‌ای است و  رنگ بنفش، آبی، و قرمز به ترتیب  نقاط با درجه‌ی ۵، ۶، و ۷ را مشخص می‌کنند. برای مش‌های منظم و تصادفی صورت این دو مدل پاسخ تقریبا یکسانی داردند و اختلافشان ناشی از وزنی است که به نقاط با درجه‌ی ۵ و ۷ اختصاص داده شده‌است. داده‌ها در مش‌های درهم پخش شده اما همچنان همین روند نیز در آن‌ها دیده می‌شود.
}
\label{fig:ULM20BendingScatter}
\end{center}
\end{figure}

مشابه با بخش قبلی، انرژی انحنای مش‌هایی که با مد هارمونیک کروی تغییر شکل داده شده‌اند را بررسی می‌کنیم (ستون
$(f)$
شکل
\ref{fig:AllULM20}
). در این شکل تغییر انرژی و نیرو بر اساس تغییر شدت مد از
$u_{2,0}=0$
(کاملا کروی) به
$u_{2,0}=1$
(دمبلی شکل) رسم شده‌است. انرژی انحنا برای تغییر شکل‌های کم به کمک معادله‌ی 
\ref{eq:curvatureYLM}
 و با جایگذاری 
 $\kappa=1[\varepsilon]$
 به صورت
\begin{eqnarray}
E_{b}=8\pi\kappa + 12\kappa|u_{2,0}|^2
\label{eq:curvatureY20}
\end{eqnarray}
و نیروی برگرداننده با مشتق گیری نسبت با شدت مد به صورت
\begin{eqnarray}
-\frac{\partial E_{b}}{\partial u_{2,0}}= -24\kappa |u_{2,0}|
\label{eq:curvatureForceY20}
\end{eqnarray}
محاسبه می‌شود.

در مورد انرژی انحنا نیز در شد‌ت مدهای کم همخوانی خوبی میان مدل‌ها و محاسبات نتیجه شده از بررسی افت و خیز دیده می‌شود. خطای کمی که در مش‌های درهم تصادفی دیده‌ شده بود در اینجا نیز دیده می‌شود. ولی نتیجه‌ی مهم این شکل این است که مستقل از مدل انحنا، نیروهای تولید شده میان تمام مدل‌ها یکسان رفتار می‌کند. در نتیجه انتخاب مدل محاسبه‌ی انرژی انحنا در دینامیک نقاط روی صفحه تاثیر ندارد.



جهت بررسی انحنای نقاط زین اسبی، در شکل 
\ref{fig:ULM20BendingScatter}
مشابه به شکل
\ref{fig:unitsphereBendingScatter}
صورت و مخرج مدل‌های گامپر و یولیشر را برای نقاط روی مش‌های شکل 
\ref{fig:AllULM20}
رسم شده‌است. به علت کشیده شدن مش‌ بر اثر تغییر شکل، توزیع مساحت وسیع‌تری نسبت به شکل 
\ref{fig:unitsphereBendingScatter}
مشاهده می‌شود. بررسی رفتار نقاط نشان می‌دهد که روند اختلاف انرژی نقاط با درجات مختلف که قبل‌تر مشاهده شد در شکل‌های پیچیده‌تر نیز حضور دارد.







\section{
پتانسیل‌های تکمیلی
\label{sec:auxPotentials}
}
\subsection{
پتانیسل ارتفاع
}

فاصله‌ی میان نقاط و اضلاع مثلث‌ها در شبکه را می‌توان با پتانسیل 
Weeks-Chandler-Andersen
یا به اختصار
WCA
کنترل کرد. این پتانسیل منشا فیزیکی ندارد و برای ایجاد پایداری در شبیه‌سازی به کار برده می‌شود. با کنترل فاصله‌ی نقاط با اضلاع مثلث‌ می‌توان حد پایین برای اندازه‌ی پلاکت‌ها تعیین کرد و در نتیجه اندازه قدم شبیه‌ سازی دینامیک ملکولی را تعیین کرد. از طرفی محاسبات هندسی مختلفی که برای محاسبه‌ی سطح، حجم، و انحنای مش مورد نیاز است در صورتی دو نقطه از شبکه یا یک نقطه و یک ضلع مثلث دقیقا روی هم قرار بگیرند به پاسخ گنگی منجر خواهد شد. با استفاده از این پتانسیل احتمال ناپایداری در محاسبات را می‌توان از میان برد. از آنجایی که یک مثلث ۳ اتفاع دارد، برای هر مثلث در مش ۳ پتانسیل نیاز خواهیم داشت. برای نمونه، این پتانسیل برای یکی از ارتفاتعات مثلث به شکل زیر تعریف می‌شود،
\begin{equation}
U_{h}=\epsilon\left[\left(\frac{d_h}{h}\right)^8-\left(\frac{d_h}{h}\right)^4+\frac{1}{4}\right].
\label{eq:wcah}
\end{equation} 
پارامتر
$d_h$
فاصله‌ی کمینه ممکن برای ارتفاع را مشخص می‌کند،  عمق چاه
$\epsilon=4k_BT$,
 و فاصله‌ی قطع\LTRfootnote{cut off}
پتانیسل 
$h_{cutoff}=\sqrt[6]{2}d_h$
است. ارتفاع راس
$i$
از ضلع تعریف شده میان دو راس
$j$
و
$j'$
در مثلث
$ijj'$
به صورت 
\begin{equation}
h_i=\ell_{ij}\sin\theta_{ijj'}.
\end{equation} 
محاسبه می‌شود.

\subsection{
پتانیسل غیر خطی دوسطحی
}
به طول عمومی، تمام پتانسیل‌هایی که برای محاسبه‌ی انحنای رویه در این رساله معرفی شده برای حد خمش‌های کم صادق است. جهت محاسبه‌ی صحیح انرژی انحنا لازم است که در طول شبیه‌سازی از پدید آمدن لبه‌های نوک تیز روی مش جلوگیری کرد. از آنجایی که انرژی انحنا یک پتانسیل هارمونیک است، پتانسیلی که زوایای میان مثلث‌ها را کنترل کند باید در حد انحناهای کم، هزینه‌ی انرژی تقریبا صفر داشته باشد و تنها در  انحنای زیاد ظاهر شده و در برابر خم شدن مثلث‌ها مقاومت نشان دهد. با تعریف پتانسیل به صورت
\begin{equation}
U_{\phi}=\frac{1}{2}k_{\phi^4}\left[e^{2(1-\cos\phi_{ij})}-1-\phi_{ij}^2 \right],
\label{eq:theta4}
\end{equation}
میان تمام جفت مثلث‌ها می‌توان از ایجاد زوایای دوسطحی بزرگ جلوگیری کرد. قدرت پتانسیل با پارامتر 
$k_{\phi}$
قابل تنظیم است.














\section{\label{sec:OpenMM}
موتور محاسباتی
OpenMM
}
در تحقیقات این رساله از موتور محاسباتی دینامیک ملکولی
OpenMM\cite{OpenMM2017}
جهت حل معادلات حرکت استفاده شده‌است. تمامی پتانسیل‌ها به شکلی بازنویسی شده‌اند که در این بسته‌ی نرم‌افزاری قابل پیاده‌سازی باشد. در این فصل نحوه‌ی محاسبه‌ی نیروی حاصل از پتانسیل‌ها ارائه نشده زیراکه 
OpenMM
با دانستن پتانسیل حاکم بر ذرات قادر به محاسبه‌ی نیرو است. 


شبیه‌سازی ذراتی که تحت معادله‌ی نیوتن حرکت می‌کنند، 
\begin{equation}
m_i\frac{d\vec v_i}{d_t}=\vec f_i.
\label{eq:newton}
\end{equation}
از روش انتگرال گیری پرش قورباغه‌‌ای ورله موجود در بسته‌ی نرم افزاری 
OpenMM
استفاده شد. جهت بررسی افت و خیز سطح، معادله‌ی حرکت لانژون برای محاسبه‌ی نیرو استفاده شد
 \begin{equation}
m_i\frac{dv_i}{d_t}=\vec f_i -\gamma m_i\vec v_i+R_i,
\label{eq:newton}
\end{equation}
در این بسته‌ی نرم افزاری دینامیک ملکولی این معادله با الگوریتم پرش قورباغه‌ای 
\cite{IZAGUIRRE2009}
گسسته سازی شده. 
$\gamma$
ضریب اصطکاک و 
$R_i$
نیروی غیر همبسته‌ی تصادفی با مقدار میانگین صفر و واریانس
$2m_i\gamma k_BT$
است. 

\section{\label{sec:dataAcu}
آماده سازی شبیه‌سازی
}
نتایج ارائه شده در این رساله  به طور پیش‌فرض با مِش‌های تصادفی دارای 
$N=1002$
نقطه و جرم کل
$M=N\times m$
ایجاد شده‌اند. جرم هر ذره 
$m=50 [m_0]$
مقدار دهی شده و در صورتی که مِش کُروی باشد اندازه‌ی شعاع آن 
$r_0=1000 [l]$
تنظیم شده‌است. سرعت اولیه ذرات از یک توزیع بولتزمن با دمای 
$k_BT=2.49 [\varepsilon]$
انتخاب شده.

پارامترهای مربوط به خواص ماده‌ی غشا به این ترتیب مقدار دهی شده، ضریب سختی خمش 
$\kappa =20k_BT$
ضریب فشردگی سطحی
$k_A=5.22\times10^{5}k_BT/r_0^2 [\varepsilon/l^2]$
مدول فشردگی حجمی
$k_V=1.6\times10^7k_BT/r_0^3 [\varepsilon/l^3]$
و مقادیر تعادلی مساحت
$A_0$
و حجم
$V_0$
مستقیم از هندسه‌ی اولیه مِش محاسبه‌ شده‌است. 

در مورد پتانیسل‌های تکمیلی برای تمامی مش‌ها پتانسیل 
WCAh
با اندازه‌ی کمینه ارتفاع 
$d_h=0.02r_0 [l]$
و عمق چاه
$\epsilon=4k_BT$
استفاده شد. در صورتی که نیاز به پتانسیل غیر خطی دو وجهی، ضریب سختی آن 
$k_{\phi^4}=20k_BT$
قرار داده شده‌است.

در شبیه‌سازی واحد زمان
$[\tau]$
بر حسب واحد طول 
$[l]$
واحد جرم
$[m_0]$
و واحد انرژی
$[\varepsilon]$
به صورت
\begin{equation}
\tau =\sqrt{\frac{m_0l^2}{\varepsilon}}
\end{equation}
تعیین می‌شود.











\section{\label{sec:VCM}
نرم افزار مدل مجازی سلول
}
نرم افزار مدل سلول مجازی\LTRfootnote{Virtual Cell Model}
در گروه ماده چگال نرم دانشگاه صنعتی شریف جهت مطالعه‌ی خواص مکانیکی سلول هنگام مهاجرت\LTRfootnote{cell migration}
و چسبیدن به سطوح\LTRfootnote{cell adhesion}
ایجاد شد
\cite{Tiam2017ACS, Tiam2018AFM}.
 بخش غشای‌ این نرم افزار جهت انجام شبیه‌سازی‌های مورد نیاز این رساله توسعه یافت. در حال حاضر بخش غشا توانایی شبیه‌سازی غشا‌های جامد (دارای مدول یانگ) و غشاهای سیال را دارد. توسعه‌‌ی نرم افزار حدود ۳ سال طول کشید. در این زمان بخش‌هایی مانند مدل‌سازی شبکه‌ی اسکلت سلولی و کروماتین نیز توسعه داده شده‌است. همچنین تمامی مقیاس‌های انرژی و تعاریف پتانسیل با واحد‌های قابل مقایسه با نتایج آزمایشگاهی کالیبره شدند
\cite{VCMgit}.

به لحاظ عملکرد، موتور محاسباتی دینامیک ملکولی 
OpenMM
به عنوان مرکز محاسباتی اصلی این نرم افزار قرار داده شد که تمام قابلیت‌های این موتور محاسباتی را در دسترس کاربر قرار می‌دهد. از جمله قابلیتی که مورد توجه افراد مشغول در زمینه‌ی محاسباتی است، توانایی انجام محاسبات به شکل موازی بر پردازنده‌های مرکزی چند هسته‌ای\LTRfootnote{multi-core CPU}
و پردازنده‌های گرافیکی\LTRfootnote{GPU}
است. جهت کسب اطلاعات بیشتر به صفحه‌ی 
YouTube \cite{VCMYoutube}
و صفحه‌ی راهنمای این نرم افزار 
\cite{VCMhomepage}
مراجعه فرمایید.



