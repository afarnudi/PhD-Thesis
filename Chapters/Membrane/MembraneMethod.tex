\setRL
\clearpage
\def \MemMethod {\Mempath /MembraneMethod}

\section{
چکیده
}
در این بخش روش  دینامیکِ مِش  معرفی شده‌است. در مقدمه، انگیزه‌ی طراحی این روش را بیان شده‌است. سپس بحث کوتاهی در مورد مکانیک آماری سطوح مثلث بندی شده و نحوه‌ی اندازه‌گیری مشاهده پذیر‌ها در بخش 
\ref{sec:statMech}
بیان شده‌است. پس از یادآوری الگوریتم مثلث بندی دینامیک در بخش
\ref{sec:dynamicTriangulation}
 روش تکمیلی، دینامیک مش در بخش
 \ref{sec:meshDynamics}
و معادلات حرکت معرفی شده‌است. سپس در بخش
 \ref{sec:DAR}
به توزیع دینامیک مساحت پرداخته‌ایم که حاصل استفاده از روش دینامیک مش است. ‌تعریف مش‌های استفاده شده در این روش در بخش
\ref{sec:meshRecipe}
ارائه شده و روش‌پیاده سازی پتانسیل‌های مورد نیاز برای انجام شبیه‌سازی دینامیک مولکولی با جزئیات در بخش‌های
\ref{sec:areaVolumeD}
و
\ref{sec:curvatureD}
توضیح داده شده‌است. پس از تعریف پتانسیل‌های فیزیکی، در بخش
\ref{sec:auxPotentials}
پتانسیل‌های تکمیلی مورد نیاز دینامیک مش معرفی شده‌است. در نهایت نرم‌افزار‌ها و ابزارهای استفاده شده برای انجام شبیه‌سازی‌های مورد نیاز این مطالعه (بخش‌های
\ref{sec:OpenMM}
و
\ref{sec:VCM})
و شیوه‌ی آماده‌سازی شبیه‌سازی‌ها در بخش
\ref{sec:dataAcu}
توضیح داده شده‌است.

\section{
مقدمه
}
در روش شبیه‌سازی توزیع دینامیک مساحت
\LTRfootnote{Dynamic Area Redistribution}
اندازه و شکل قسمت‌های مِش به طور پیوسته در حال تغییر است. در این مدل تنش بُرشی هزینه‌ی انرژی نداشته و چگالی دولایه‌ی لیپیدی غشا در سراسر مش ثابت فرض شده. در این رساله از شبکه‌ی مثلثی برای پیاده سازی 
Dynamic Area Redistribution (DAR)
استفاده شده‌است. جهت شبیه‌سازی موفق غشا لازم است انرژی مساحت، حجم، و انحنا به درستی در همه نقاط محاسبه شود. صحت معادلات گسسته بر شبکه‌های منظم و تصادفی در مطالعات گذشتگان اثبات شده است (بخش 
\ref{sec:simRevMesh}
). در درجه‌ی اول صحت این معادلات بر شبکه‌های درهم نیز باید اثبات شود. سپس با استفاده از این شبکه‌ها می‌توان رفتار غشا را مطالعه کرد. ابتدا افت و خیز سطح غشای شبیه‌ سازی شده با این روش مطالعه، سپس شکل غشا برای مقادیر مختلف حجم کاهیده بررسی شد. در نهایت در جهت ارائه‌ی کاربرد این مدل، نحوه‌ی تولید گلبول قرمز نمایش داده شده‌است.


\section{\label{sec:statMech}
مکانیک آماری سطوح مثلث بندی شده
}
توجه ما در این بخش به مکانیک آماری سطوح 
$\cal S$
است که شکل یک غشا را بیان می‌کنند. مقادیر قابل اندازه‌گیری که برای سطح غشا تعریف می‌شود با میانگین‌گیری آنسامبلی بر تمامی چیدمان‌های ممکن سطوح غشا به این فرم تعریف می‌شود
\begin{equation}\label{eq:<A(S)>}
\langle X \rangle = \frac{ \int {\cal D}{\cal S}\, X({\cal S}) \exp\left[-\beta E({\cal S})\right] } 
                                     { \int {\cal D}{\cal S}\,                  \exp\left[-\beta E({\cal S})\right] }.
\end{equation}
در معادله‌ی فوق 
$E({\cal S})$
مجموع انرژی ناشی از سطح، حجم، و انحنای غشاست. محاسبات عددی مربوط به اشکال غشا وابسته به شبکه
$\cal M$
است که سطح توسط آن گسسته سازی شده‌است. یک شبکه توسط مختصات نقاط
$\bm q=\{\bm q_1,\bm q_2, \ldots, \bm q_N \} \in \cal S$
و اتصالات
$\cal G$
که میان نقاط است تعریف می‌شود. به طور مشخص می‌توان از شبکه‌های‌ مثلثی برای گسسته سازی استفاده کرد. در شبکه‌های مثلثی، زمانی که اتصالات نقاط میان همسایگان نزدیک تعریف شود، مثلث‌هایی که سطح را تشکیل می‌دهند همپوشانی نخواهند داشت.

میان‌گین‌گیری آنسامبلی که در معادله‌ی 
\ref{eq:<A(S)>}
برای غشا تعریف شد، برای سطوح گسسته سازی شده بر روی تمامی درجات آزادی مِش (شبکه) تعریف می‌شود،
\begin{equation}\label{eq:<A(M)>}
\langle X \rangle = \frac{ \sum_{\cal G}\int d {\bm q}\, X({\cal G}, {\bm q}) \exp\left[-\beta U({\cal G}, {\bm q})\right] } 
                                     { \sum_{\cal G}\int d {\bm q}\,                                \exp\left[-\beta U({\cal G}, {\bm q})\right] }\ ,
\end{equation}
در این معادله انرژی
\begin{equation}
U({\cal M}) = E({\cal M}) + E_{aux}({\cal M})
\end{equation}
یک چیدمان از مش
 (${\cal M} = ({\cal G},\bm q)$),
مجموع انرژی‌های گسسته سازی شده‌ی مساحت، حجم، و انحنای سطحی است که مش بیان می‌کند
$\cal S(\cal M)$,  $E({\cal M}) \approx E(\cal S(\cal M))$
و تمام پتانسیل‌های تکمیلی است که شکل، مساحت، اندازه اضلاع، ... مثلث‌ها را کنترل می‌کند.

وزن آماری مشی
${\cal M} = ({\cal G},\bm q)$
که نماینده‌ی سطح
 $\cal S$
است، به فرم زیر تعریف می‌شود،
\begin{equation}\label{eq:w(M) for given S}
w({\cal G}, {\bm q}|{\cal S}) = \exp\left[-\beta U({\cal G}, {\bm q})\right] \prod_i \delta({\bm q}_i\in {\cal S}).
\end{equation}
 در معادله‌ی فوق، جمله‌ی
 $\delta({\bm q}_i\in {\cal S})$
مانند یک دلتای دیراک است که مش‌هایی که نقاط آن در فاصله‌ی مناسبی نسبت به سطح 
 $\cal S$
قرار دارد را قبول می‌کند. در طول این رساله مش‌ها را به دو دسته‌ی کلی سخت و نرم دسته بندی می‌کنیم. مش‌های سخت مش‌هایی هستند که در آن پتانسیل‌های تکمیلی، شکل مثلث‌ها را به شکل تقریبا یکسانی بر سطح مش حفظ می‌کنند. به شکل مشابه مش‌های نرم مش‌هایی هستند که در آن پتانسیل‌های تکمیلی اجازه‌ می‌دهند مثلث‌های سطح، اشکال بسیار متنوعی داشته باشند. 

\begin{figure}[h]
\begin{center}
\includegraphics[width=10cm]{\MemMethod/Pics/DAR2}
\caption{
نمایش یک مربع (a) و یک مستطیل (b) با مش‌هایی که دارای اتصلات و تعداد نقاط یکسان اند و مجموع مساحت‌های مثلث‌های آن‌ها برابر است.
}
\label{fig:meshDAR}
\end{center}
\end{figure} 




در اکثر شبیه‌سازی‌های غشا از مش‌های سخت استفاده می‌شود. مهم‌ترین دلیل این امر استفاده از روش گسسته‌سازی انحنایی است که گامپر و کرول معرفی کردند
\cite{gompperkroll1996}.
 این روش با این فرض طراحی شده است که تمامی مثلث‌هایی که بر مش قرار دارند (کم و بیش) شکل یکسان دارند. در نتیجه برای یک مش سخت احتمال
 $\exp\left[-\beta E_{aux}({\cal G}, {\bm q})\right]$
برای مش‌هایی که مثلث‌های یکسان ندارند بسیار کوچک می‌شود. مثلا برای تغییر شکل مشی مربعی مانند شکل 
\ref{fig:meshDT}a
به یک مش مستطیلی (شکل
\ref{fig:meshDT}b)
 اتصالات میان نقاط باید کاملا تغییر کند. در نتیجه اگر برای شبیه‌سازی سطوحی که دائم در حال تغییر هستند از مش‌های سخت استفاده شود اتصالات مش
$\cal G$
دائما نیاز به تغییر دارد. از طرفی، با استفاده از مش‌های نرم می‌توان شکل‌های مختلفی را با اتصال یکسان
$\cal G$
نمایش داد. برای مثال در شکل
\ref{fig:meshDAR}
تغییر شکل یک مش مربعی (شکل
\ref{fig:meshDAR}a
) به یک مش مستطیلی (شکل
\ref{fig:meshDAR}b
) را با مش نرم نشان دادیم. 

به طور کمی، به شرطی که 
${\cal G}$
برای نمایش سطح
${\cal S}$
تطابق لازم داشته باشد، بیشترین وزن آماری مربوط به 
$\exp\left[-\beta E({\cal S})\right]$
خواهد بود و مشاهده‌ پذیر‌ها را می‌توان به این شکل اندازه‌گیری کرد
\begin{equation}\label{eq:<A(M)> soft meshes}
\langle X \rangle \approx
\langle X \rangle_{\cal G} = \frac{\int d {\bm q}\, X({\cal G}, {\bm q}) \exp\left[-\beta U({\cal G}, {\bm q})\right] } 
                                                  { \int d {\bm q}\,                                \exp\left[-\beta U({\cal G}, {\bm q})\right] }.
\end{equation}

















\section{\label{sec:dynamicTriangulation}
الگوریتم مثلث‌بندی دینامیک
}
\begin{figure}[h]
\begin{center}
\includegraphics[width=\columnwidth]{\MemMethod/Pics/dynamicTri}
\caption{
تغییر مثلث بندی مِش با تغییر موضعی جفت مثلث‌ها میان چهار نقطه. در حالت اولیه (سمت چپ) دو مثلث با رئوس
$ABC$
و
$DBC$
تعریف شده‌است. با تغییر ضلع مشترک بین دو مثلث از 
$BC$
به
$AD$
مثلث بندی جدید با رئوس
$BAD$
و 
$CAD$
تشکیل خواهد شد (سمت راست).
}
\label{fig:dynamicTri}
\end{center}
\end{figure}



روش  مثلث بندی دینامیک\LTRfootnote{dynamic triangulation}
ابتدا توسط دیوید بول\LTRfootnote{David Boal}
و همکارش 
\cite{Boal1992PRA}
در سال ۱۹۹۲برای مِش‌های مثلثی طراحی شد. هدف اصلی این روش، شبیه‌سازی رفتار سیال گون غشا با استفاده از شبکه‌های مثلثی بود. کمی‌ بعد در همان سال، گامپر و کرول با  شبیه‌سازی موفق غشاهای سیال گون، سبب محبوبیت این روش شدند 
\cite{Gompper1992Science}.
این روش را بسیار محبوب کرد. گامپر و کرول در این مطالعه از روش انحنای دو سطحی برای محاسبه‌ی انرژی انحنا استفاده کردند. همانطور که در بخش
\ref{sec:curvatureDiscDef}
توضیح داده شد، این روش برای محاسبه‌ی انحنای غلط است. در سال ۱۹۹۶ گامپر و کرول روش محاسبه‌ی ایتزیکسون را با روش مثلث بندی دینامیکی ترکیب کردند و با موفقیت رفتار سیال‌گون غشا را شبیه‌سازی کردند
\cite{gompper1996}.

در روش مثلث بندی دینامیک دو مثلث مجاور در نظر گرفته می‌شود. رئوس این دو مثلث مانند شکل 
\ref{fig:dynamicTri}
سمت چپ، چهار وجهی 
$ABCD$
را تشکیل خواهد داد. مثلث بندی در ابتدا دو مثلث 
$ABC$
و
$DBC$
را تعریف می‌کند. با حذف ضلع
$BC$
و ایجاد ضلع
$AD$
مثلث بندی جدید با مثلث‌های
$CAD$
و
$BAD$
تشکیل خواهد شد (شکل
\ref{fig:dynamicTri}
سمت راست). تغییر مثلث بندی، انرژی انحنا و کششی مِش را تغییر خواهد داد. همانطور که در شکل با رنگ‌های قرمز و سبز نمایش داده شده، همسایه‌های مثلث‌های آبی در این فرآیند تغییر خواهد کرد. در نتیجه علاوه بر تغییر زاویه‌ی میان مثلث‌های آبی در دو مثلث بندی، زاویه میان همسایه‌ها نیز تغییر خواهد کرد. به لحاظ انرژی کششی، در صورتی که طول ضلع 
$BC$
و
$AD$
متفاوت باشد، انرژی کششی نیز تغییر خواهد کرد. انتخاب مثلث بندی با یک وزن متروپلیس\LTRfootnote{Metropolis}
 انجام می‌شود. در این روش، ابتدا انرژی مثلث بندی در حالت اولیه محاسبه می‌شود
($E_i$),
سپس مثلث بندی تغییر داده می‌شود و انرژی مش در چیدمان جدید محاسبه ‌می‌شود
($E_f$).
 در صورتی که انرژی با مثلث بندی جدید کاهش پیدا کند مثلث بندی جدید حتما پذیرفته می‌شود. در صورتی که انرژی مثلث بندی جدید بیشتر باشد این چیدمان با یک وزن بولتزمن\LTRfootnote{Boltzman}
انتخاب یا رَد خواهد شد. 

به علت ماهیت این الگوریتم بهترین روش برای شبیه‌سازی آن استفاده از روش مانتی کارلو\LTRfootnote{Monte Carlo}
است. گامپر و کرول هنگام ترکیب الگوریتم مثلث بندی دینامیک با روش اندازه‌گیری انحنای ایتزیکسون، محدودیت‌های زیادی در انتخاب طول اضلاع قرار داد. نتیجه‌ی این محدودیت‌ها ایجاد توزیع یکنواخت نقاط بر سطح شبکه‌ی مثلثی و کنترل شکل و اندازه مثلث‌ها در جهت پایدار کردن روش ایتزیکسون بود.

\begin{figure}[h]
\begin{center}
\includegraphics[width=13cm]{\MemMethod/Pics/DT.pdf}
\caption{
نمایش یک مربع (a)، یک مستطیل (b)، و دو شکل غیر مرتبط (c) که همگلی از ۳۴۰ مثلث متساوی الاضلاع تشکیل شده‌اند.
}  
\label{fig:meshDT}
\end{center}
\end{figure} 


همچنین با استفاده از الگوریتم مثلث‌بندی دینامیک 
\cite{Boal1992PRA, Gompper1992Science},
می‌توان اتصالات مختلف
 $\cal G$
را نمونه‌گیری کرد. این الگوریتم مهم‌ترین و پر کاربرد‌ترین روش برای بازسازی مش است که برای شبیه‌سازی اشکال غشا‌ها استفاده می‌شود. با تعمیم این روش
\cite{Kohyama2003PRE},
می‌توان رفتارهای پیچیده‌ی سطوح مایع‌گون (شکل 
\ref{fig:meshDT}c)
 را نیز شبیه‌سازی کرد. با وجود آزادی زیادی که این روش برای مدل سازی غشا‌ها در اختیار ما قرار می‌دهد، این روش محدودیت‌هایی نیز دارد. قدرت اصلی این الگوریتم در تغییر اتصالات مش‌های سخت است. یعنی برای یک تغییر شکل ساده مربعی به مستطیلی (شکل 
\ref{fig:meshDT})
 باید چیدمان نقاط (و اتصالات میانشان) را با دینامیک موضعیِ پخشی تغییر داد. این کار بسیار زمانبر است.















\section{\label{sec:meshDynamics}
دینامیک مِش، روشی منطبق با دینامیک مولکولی
}
در این رساله، پیشنهاد می‌کنیم که به جای استفاده از روش مثلث بندی دینامیک (که هر بار بر اساس یک قدم مانتی کارلو مش را تغییر می‌دهد)، پیرو کارهایی که قبلا در این شاخه از فیزیک انجام شده است، از مش‌های نرم دو بعدی با اتصالات ثابت برای شبیه سازی غشا استفاده شود. در این شبیه‌سازی‌ها حرکت نقاط بر روی سطح 
$\cal S$
با محدودیت بسیار کمی مواجه است. در واقع ایده‌ی اصلی این است که نقاط مش تا زمانی باعث همپوشانی مثلث‌ها نشوند (شکل
\ref{fig:DARs}c)
 می‌توانند آزادانه بر سطح دو بعدی غشا حرکت کنند. این شرط توسط پتانیسیل‌های تکمیلی
$E_{aux}({\cal M})$
کنترل می‌شود. به طور عمومی، مثلث‌ها یا پلاکت‌هایی که به هر نقطه اختصاص داده شده‌است، نماینده‌ی یک مقدار ثابتی از غشا نیستند ولی نماینده‌ی کسری از جرم کل غشا هستند که با مساحت لحظه‌ای هر نقطه رابطه دارد
 $m_i = M a_i/A$.
 برای غشایی که سطح آن تغییر نکند این رابطه را به شکل ساده‌تر
$m_i  \approx \rho a_i$
می‌توان نوشت. حرکت دسته‌ جمعی نقاط باعث می‌شود که سطح در پاسخ به نیروهای خارجی سریع تغییر کند. در ۳ بُعد، می‌توان دینامیک مش را شبیه به یک توصیف شبه لاگرانژی از یک سیال دو بعدی درنظر گرفت. حرکت این رویه‌ی دوبعدی در جهات خارج از سطح لاگرانژی است ولی حرکت درون سطحی نقاط، که نماینده‌ی لیپید‌ها هستند، وابسته به پانسیل‌های تکمیلی مش است.

در این شرایط هامیلتونین سیستم را می‌توان به شکل زیر نوشت
\begin{eqnarray}
\mathcal H&=& \frac12 \sum_i^N \frac{\bm p_i^2}{m_i(\bm q)} + U(\bm q) 
\label{eq:HamiltonianGeneral}\\
 &=& \frac {A(\bm q)}{2M}  \sum_i^N \frac{\bm p_i^2}{a_i(\bm q)} + U(\bm q) 
 \label{eq:HamiltonianGeneral with explicit areas}\\
&\approx& \frac {1}{2\rho}  \sum_i^N \frac{\bm p_i^2}{a_i(\bm q)} + U(\bm q) \ ,
\label{eq:HamiltonianVariableMass}
\end{eqnarray}
که در اینجا
$\bm p=\{\bm p_1,\bm p_2, \ldots, \bm p_N \}$
و
$\bm q=\{\bm q_1,\bm q_2, \ldots, \bm q_N \}$ 
تکانه و مختصات است،
$m_i(\bm q) = Ma_i(\bm q)/A(\bm q)$
جرم نقاط،
$A(\bm q)$
مساحت لحظه‌ای مش، و 
$U(\bm q)$
مجموع انرژی‌های پتانسیل مش است که تابع مساحت، حجم، و انحنای مش است.

در این هامیلتونین، جرم نقاط تابع توزیع مختصات مش است
($m_i(\bm q)$).
 اگر از تکانه‌ها انتگرال بگیریم، وزن آماری یک چیدمان مشخص از مش
$\bm q$
به شکل زیر تعریف می‌شود
\begin{equation}
\begin{aligned}
w(\bm q)&=\exp\left[-\beta U(\bm q)\right]\int d\bm p\exp\left(-\beta\frac {A(\bm q)}{2M}  \sum_i^N \frac{\bm p_i^2}{a_i(\bm q)}\right)\\
&=\prod_{i=1}^N\left[\frac{2\pi M}{\beta} \frac{a_i(\bm q)}{A(\bm q)}\right]^{\frac{3}{2}}\exp\left[-\beta U(\bm q)\right]\ .
\end{aligned}
\label{eq:microStateProbability1}
\end{equation}
همانطور که انتظار داشتیم،
$w(\bm q)$ 
به وزن بولتزمنی 
$\exp\left[-\beta U(\bm q)\right]$
وابسته است که به شکل سطح مربوط است. ضریب این جمله زمانی بیشینه است که تمامی پلاکت‌ها مساحت یکسانی داشته باشند
$a_i = A/N$
. در نتیجه می‌توانیم این وزن را به شکل وزن بولتزمن موثر بنویسیم
\begin{equation}
\begin{aligned}
w(\bm q)
&=\exp\left[\frac{3}{2}\sum_{i=1}^N\log\left\{\frac{2\pi M}{\beta} \frac{a_i(\bm q)}{A(\bm q)}\right\}-\beta U(\bm q)\right]\ .
\end{aligned}
\label{eq:microStateProbability}
\end{equation}
برای فهیمدن این وزن موثر، جمله‌ی بالا را بر اساس تغییرات مساحت پلاکت‌ها از مساحت میانگین پلاکت‌ها 
$a_i(\bm q)=A(\bm q)/N+\delta_i(\bm q)$
بازنویسی می‌کنیم
\begin{equation*}
\begin{aligned}
\log\left(\frac{2\pi M}{\beta} \frac{a_i(\bm q)}{A(\bm q)}\right)
&=\log\left(\frac{2\pi m}{\beta}\right)+\log\left(1+\frac{N\delta_i(\bm q)}{A(\bm q)}\right)\\
&\approx\log\left(\frac{2\pi m}{\beta}\right)+\frac{N\delta_i(\bm q)}{A(\bm q)}-\frac{1}{2}\left(\frac{N\delta_i\bm q)}{A(\bm q)}\right)^2.
\end{aligned}
\label{eq:logExpantion}
\end{equation*}
پس از جایگذاری این جمله در معادله‌ی 
\ref{eq:microStateProbability}
جمع جملات مرتبه‌ی صفرم یک ثابت عددی است، جمع جملات مرتبه‌ی اول بنا به تعریف صفر خواهد بود. در نتیجه جملات باقیمانده تنها از مرتبه‌ی دوم خواهند بود
\begin{equation}
\begin{aligned}
w(\bm q)\propto\exp\left[-\beta\left( \frac{3}{2}k_BT\sum_{i=1}^N \frac{1}{2}(\frac{N}{A_0}\delta_i(\bm q))^2+U(\bm q)\right)\right] \ .
\end{aligned}
\label{eq:microStateProbabilityExpansion}
\end{equation}
این جمله شبیه به یک پتانسیل هارمونیک است که اندازه‌ی پلاکت‌ها را نزدیک به مساحت میانگین حفظ می‌کند.



اگر از تغییرات کوچک مساحت کل غشا صرف نظر کنیم، معادلات حرکت نقاط مش را می‌توانیم با استفاده از معادلات حرکت هامیلتون محاسبه کنیم. از معادله‌ی اول هامیلتون 
$\dot {\bm q_i}=\frac{\partial}{\partial \bm p_i}\mathcal H$
می‌توانیم تکانه‌ی نقاط را حساب کنیم
$\bm p_i = \rho a_i(\bm q)\dot {\bm q_i}$
 و معادله‌ی دوم 
$\frac{d}{dt} \bm p_i = -\frac{\partial}{\partial \bm q_i}\mathcal H$
به شکل زیر ساده می‌شود
\begin{equation}
\frac{d}{dt}(\rho a_k(\bm q) \dot {\bm q}_k) = -\frac{\partial U(\bm q)}{\partial \bm q_k} - \frac{1}{2} \frac{\partial}{\partial \bm q_k}\sum_{i=1}^N\frac{\bm p_i^2}{\rho a_i(\bm q)},
\end{equation}
و یا
\begin{eqnarray}
\lefteqn{  \rho a_k(\bm q) \ddot{\bm q_k}  =}
\label{eq:HamiltonianGeneralEOMMomentum}\\
 &=& -\frac{\partial U(\bm q)}{\partial \bm q_k} + \frac{1}{2}\rho \sum_{i=1}^N\dot{\bm q_i}^2\frac{\partial a_i(\bm q)}{\partial \bm q_k}
 - \rho  \dot{\bm q_k}\sum_{i=1}^N\frac{\partial a_k(\bm q)}{\partial \bm q_i}\dot{\bm q_i}\nonumber
\end{eqnarray}
حل این معادلات دشوار است و برای بررسی دینامیک هر سیستمی که در آن تاثیر جرم مهم باشد (مانند حباب صابون) مورد نیاز است. در این رساله ما علاقمند به شبیه‌سازی رفتار غشایی هستیم که دینامیک سطح توسط جریان شارع درون و سیالی که غشا در آن غوطه‌ور است، کنترل می‌شود
\cite{milnersafranPRA1987, schneider1984}.
 از آنجایی که در این سیستم اینرسی نقش نخواهد داشت، برای ساده سازی معادلات فوق، جرم تمام نقاط مش را یکسان تقریب می‌زنیم
\begin{eqnarray}
\mathcal H&=& \frac12 \sum_i^N \frac{\bm p_i^2}{m_i} + U(\bm q) 
\label{eq:HamiltonianFixedMass} .
\end{eqnarray}
در این حالت وزن بولتزمنی که اندازه‌گیری‌های آماری را تایین می‌کند
\begin{equation}
\begin{aligned}
w(\bm q)\propto\exp\left\{-\beta U(\bm q)\right\} \ ,
\end{aligned}
\label{eq:microStateProbability HamiltonianFixedMass}
\end{equation}
و معادلات حرکت به معادلات حرکت نیوتن ساده می‌شود
\begin{equation}
m_i \ddot{\bm q_k} = -\frac{\partial U(\bm q)}{\partial \bm q_k} \ .
\label{eq:EoM for HamiltonianFixedMass}
\end{equation}
برای حل دقیق‌تر این معادلات برای غشایی که در خلا حرکت می‌کند، می‌توان اندازه‌ی جرم تمام نقاط را با اندازه‌ی جرم متوسط هر نقطه جایگزین کرد (کاری که در این رساله انجام شده‌است، 
$m_i = \rho \langle a_i \rangle$)
یا با بازسازی مش، در هر مرحله از شبیه‌سازی از مشی‌ استفاده کرد که توزیع نقاط بر روی سطح همیشه یکنواخت باشد.





 














\section{\label{sec:DAR}
روش توزیع دینامیکِ مساحت در شبکه‌ی مثلثی
}
روش توزیع دینامیک مساحت 
\LTRfootnote{Dynamic Area Redistribution}
با هدف  شبیه‌سازی خاصیت سیال‌گون غشا در محیط دینامیک ملکولی در گروه ماده چگال نرم دانشکده فیزیک دانشگاه صنعتی شریف طراحی شده‌است.  غشا سیال است و ملکول‌های لیپید در آن رفتار پخشی دارند. در صورتی که به نقطه‌ای از غشا تنش بُرشی توسط نیروی خارجی وارد شود، ملکول‌های لیپیدی بر سطح غشا حرکت می‌کنند. هنگامی که نیروی‌ خارجی متوقف شود، ملکول‌های لیپید دلیلی برای بازگشت به محل اولیه خود ندارند و مانند یک سیال به حرکت پخشی خود ادامه می‌دهند. 

\begin{figure}[h]
\begin{center}
\includegraphics[width=12cm]{\MemMethod/Pics/meshDiffusion.pdf}
\caption{
شبکه بندی سطح غشا تشکیل شده از 
$N$
ملکول لیپیدی را نمایش می‌دهد. در هنگام تعادل ترمودینامیکی چگالی ملکول‌ها در هر تکه از شبکه به طور متوسط
$\rho=N/A_0$
است.
}
\label{fig:cylindermesh}
\end{center}
\end{figure}

غشایی با مساحت 
$A_0$
را فرض می‌کنیم که از 
$N$
ملکلول لیپیدی تشکیل شده‌است. در صورتی که سطح غشا با 
$M$
قسمت با مساحت 
$A_0/M$
شبکه بندی شود، هنگام  تعادل ترمودینامیکی، به طور متوسط چگالی ملکول‌های لیپیدی در هر قسمت از شبکه برابر با 
$\rho=N/A_0$
است. اگر تنش  به قسمت
$i$
از شبکه وارد شود که باعث انبساط آن تکه شود، چگالی ملکول‌های آن کمی کمتر از متوسط خواهد شد،
$\rho_i=\rho^-$
. از آنجایی که تعداد ملکلو‌ل‌ها در سطح غشا یکسان است و غشای لیپیدی ضریب فشردگی بسیار بزرگی دارد، چگالی تمام قسمت‌های دیگر شبکه کمی افزایش می‌یابد (
$\rho_j=\rho^+$
برای هر قسمت که
$i\neq j$
). از آنجایی که غشا در تعادل ترمودینامیکی به سر می‌برد با گذشت زمان کم ملکلول‌های غشا (با دینامیک پخشی) در قسمت‌های مختلف شبکه جابجا شده و چگالی ملکولی در تمام قسمت‌ها را دوباره یکسان می‌کنند. در نتیجه تا زمانی که مساحت کل غشا تغییر نکند اندازه‌ی قسمت‌های مختلف شبکه می‌تواند با دمای محیط اُفت  و خیز کند. در این توصیف هر قسمت از شبکه نماینده‌ی تعداد ثابتی از ملکول‌های لیپیدی نیست ولی چگالی عددی ملکلول‌ها در سراسر غشا یکسان است.

روش توزیع دینامیک مساحت روی هر شبکه‌ بندی قابل پیاده‌سازی است. در این رساله، این روش بر روی شبکه‌های مثلثی پیاده‌سازی شده‌است. در دهه‌های گذشته مطالعات زیادی بر نحوه‌ی شبیه‌سازی با استفاده از شبکه‌های مثلثی شده‌است. در نتیجه نحوه‌ی گسسته سازی انرژی‌های مورد نیاز برای شبیه‌سازی یک غشا (انرژی خمش، مساحت، و حجم) بر بستر این شبکه‌ها از مطالعات گذشتگان در دسترس است. برای شبیه‌سازی رفتار سیال‌گون غشا کافی ‌است که انحنای غشا بر روی سطح به درستی تعریف شود و حجم و مساحت کل غشا رد طول شبیه‌سازی قابل کنترل باشد. 

در آخر لازم است به این نکته اشاره شود که از آنجایی که غشا‌ مدول یانگ ندارد، تغییر شکل‌های کُره-بیست وجهی که حاصل از رقابت انرژی کشسانی و انرژی خمشی است (جزئیات بیشتر در بخش 
\ref{sec:gammaTransition}
) در آن‌ها دیده نخواهد شد.






\section{\label{sec:meshRecipe}
تهیه‌ی مِش
}
\begin{figure}[h]
\begin{center}
\includegraphics[width=12cm]{\MemMethod/Pics/MeshTypes.pdf}
\caption{
عکس چهار نوع مِش استفاده شده در این رساله. مِش منظم (بالا سمت چپ)، تصادفی (بالا سمت راست)، دَرهَم منظم (پایین سمت چپ)، و مِش دَرهَم تصادفی (پایین سمت راست).
}
\label{fig:meshTypesMesthod}
\end{center}
\end{figure}
در شکل
\ref{fig:meshTypesMesthod}
نمونه‌ای از مِش‌هایی که در این رساله استفاده شده را می‌توان یافت. مِش‌ها به ترتیب زیر قابل تهیه هستند:
\subsection{
مِش منظم
}
مش منظم با شبکه‌‌بندی کردن یک بیست وجهی منتظم ساخته می‌شود. در این شبکه‌ بندی ۱۲ نقطه‌ی نقص با درجه‌ی ۵ در ۱۲ گوشه‌ی بیست وجهی ایجاد خواهد شد و نقاط دیگر همگی از درجه‌ی ۶ خواهند بود. در نتیجه به ازای هندسه بسیار متقارن کُره تنها یک نمونه از چنین شبکه‌ای به ازای تعداد نقاط وجود دارد. همچنین  برای ساخت چنین مِشی تنها می‌توان  تعداد مشخصی نقطه که با رابطه‌ی 
\begin{equation}
N_{mesh}=10\times i^2+2,
\end{equation}
محاسبه می‌شود، داشت،  که در اینجا 
$i$
مقادیر صحیح غیر صفر دارد. ‌شبکه‌ بندی با هر نرم‌افزار مش بندی قابل انجام است. در این مطالعه ما از نرم افزار 
Blender
برای تولید این مِش به خصوص استفاده کردیم. 

\subsection{
مِش تصادفی
}
مقدمه‌ی تهیه‌ی این نوع مش  قرار دادن 
$N$
نقطه به شکل تصادفی بر سطح یک کُره‌ به شعاع ۱ است. سپس به کمک 
$N$
پتانسیل فنر هارمونیک با ضریب سختی زیاد، این نقاط را به مرکز کُره متصل کرده و بر سطح کُره مقید می‌کنیم. سپس میان این نقاط برهمکنش بلند بُرد دافعه تعریف می‌کنیم. هر برهمکنش دافعه‌ی بلند بُردی که  توزیع تقریبا یکنواخت نقاط بر سطح کُره ایجاد کند مورد قبول است. در این رساله از پتانیسل 
\begin{equation}
U_{EV}=10^{-3}\left(\frac{2\rho_0}{r}\right)^6,
\end{equation}
استفاده شد. در اینجا 
$r$
فاصله‌ی فضایی نقاط از یکدیگر بوده و 
$\rho_0$
شعاع متوسط هر نقطه روی کُره است. شعاع متوسط کُره با شعاع 
$N$
دایره تخمین زده شده‌است که  سطح کُره را پوشش می‌دهند
\begin{equation}
N\pi\rho_0^2=4\pi\rightarrow \rho_0=\frac{2}{\sqrt{N}}.
\end{equation} 
سپس ذرات تحت دینامیک لانژون شروع به حرکت کرده و با انتخاب دمای پایین برای ترموستات لانژون می‌توان سرعت ذرات را پله پله کم کرد تا جایی که ذرات در فاصله‌ی تعادلی نسبت به یکدیگر قرار گیرند. البته از هر روش کمینه کردن انرژی\LTRfootnote{energy minimisation}
برای یافتن مختصات تعادلی نقاط می‌توان استفاده کرد. پس از یافت مختصات تعادلی نقاط با استفاده از الگوریتم مثلث‌ بندی دِلونی\LTRfootnote{Delaunay triangulation algorithm}
\cite{DelaunayTriangulation1997}
یک شبکه‌ی مثلثی تصادفی ایجاد خواهد شد. با تکرار این دستورنامه با استفاده از بذر‌های\LTRfootnote{seed}
 مختلف برای تولید اعداد تصادفی غیر یکسان می‌توان مِش‌های تصادفی متفاوتی تولید کرد.

\subsection{
مِش‌های درهم
}
برای تغییر توزیع مساحت مثلث‌های مِش‌های منظم و تصادفی‌  نقاط آن را با فنر‌های هارمونیک با ضریب سختی بالا مقید به حرکت بر سطح کُره کرده و سپس به تمام نقاط، سرعت و جهت تصادفی اختصاص داده می‌شود. با محدود کردن طول ارتفاع مثلث‌های مش با پتانسیل 
$U_{h}$
(جزئیات این پتانسیل در بخش 
\ref{sec:auxPotentials}
قرار دارد) حد پایین برای اندازه‌ی مثلث‌ها تعیین شد. این پتانسیل تضمین می‌کند که اضلاع مثلث‌ها بر اثر حرکت کاتوره‌ای نقاط همدیگر را قطع نکنند. ولی این پتانسیل برای جلوگیری از تا شدن مثلث‌ها بر روی سطح کافی نیست. برای پیشگیری از این اتفاق، پتانسیل غیر خطی خمشی دوسطحی
$U_{\phi}$
(جزئیات این پتانسیل در بخش 
\ref{sec:auxPotentials}
قرار دارد) میان تمام جفت مثلث‌های مش تعریف شد. در نتیجه دینامیک چنین نقاطی تحت معادله‌ی حرکت نیوتن (انتگرال گیری وِرلِه سرعتی\LTRfootnote{velocity Verlet})
 مساحت‌ مثلث‌ها افت و خیز می‌کند و توپولوژی اولیه مِش استفاده شده کاملا محفوظ می‌ماند.
 






\section{\label{sec:areaVolumeD}
پتانسیل کنترل مساحت و حجم
}
\begin{figure}[tbp]
\begin{center}
\includegraphics[width=14cm]{\MemRes/Pics/UnitSphereAreaVolume}
\caption{
مساحت (ستون چپ) و حجم (ستون راست) محاسبه‌ شده برای چهار نوع مش‌های کُروی بر اساس تعداد نقاط روی مش رسم شده‌است. مساحت کل، به دو روش، با جمع مساحت‌های ورنوی (نقاط بنفش) و جمع مساحت‌های بریسنتریک (نقاط خاکستری) محاسبه شده‌است. دقت در اندازه‌گیری مساحت و حجم (نقاط آبی) برای تمام مش‌ها با افزایش تعداد نقاط روی مش، افزایش می‌یابد. تصویر مش نمونه از هر نوع مش استفاده شده در محاسبات کنار هر ردیف رسم شده‌است. به غیر از مش منظم (که تنها یک نمونه از آن برای هر تعداد نقطه وجود دارد) مقادیر محاسبه شده حاصل از میان‌گین گیری بر روی ۵۰ نمونه‌ی مستقل انجام شده‌است.
}
\label{fig:unitsphereAreaVolume}
\end{center}
\end{figure}

جهت بررسی دقت اندازه‌گیری مساحت و حجم برای مش‌های مختلف، مش‌های کروی با تعداد نقاط مختلف انتخاب شد.  مساحت مش‌ها به دو روش محاسبه شد، با جمع  مساحت‌ وُرُنُی رئوس (نقاط بنفش) و جمع مساحت بریسنتریک (نقاط خاکستری). جهت بررسی میزان دقت در اندازه‌گیری،  نتیجه‌ی محاسبات بر مساحت کُره (
$4\pi r_0^2$
)  تقسیم شده‌است. نتایج محاسبات در شکل
\ref{fig:unitsphereAreaVolume}
رسم شده‌است. همانطور که می‌بینید دقت اندازه‌گیری برای تمامی مش‌ها با افزایش تعداد نقاط بهتر می‌شود. لازم به ذکر است  از آنجایی که تمام نقاط مش‌ها روی سطح پوسته‌ی کُروی قرار دارد، یا به عبارت دیگر مش بر داخل یک کره‌ مماس است، مساحت و حجم آن همیشه از کره کمتر خواهد بود.




مشابه به مساحت، حجم نیز برای تمامی مش‌ها با دقت بسیار خوبی قابل اندازه‌گیری‌است. در نتیجه مش‌های درهم برای اندازه‌گیری مساحت و حجم کره مناسب هستند. از آنجایی که غشا‌ها اشکال پیچیده‌تری نسبت به کره دارند، صحت محاسبات برای اشکالی به غیر از کره نیز باید بررسی شود. با اضافه کردن جمله‌ی هارمونیک کروی به مکان شعاعی تمام نقاط مش، می‌توان شکل مش را تغییر داد،
\begin{eqnarray}
r(\theta,\phi)=r_0+r_0|u_{\ell,m}|Y_{\ell,m}(\theta,\phi).
\label{eq:rDeformed}
\end{eqnarray}
با انتخاب
$\ell=2$
و
$m=0$
برای هماهنگ‌ کروی، می‌توان مش‌های دمبلی شکل  تولید کرد. میزان تغییر شکل به شدت مُد
$u_{2,0}$
بستگی دارد. هزینه‌ی انرژی تغییر مساحت یک کره به یک شکل دمبلی طبق معادله‌ی
\ref{eq:AreaGLFluctuationAmplitude}
قابل محاسبه ‌است،
\begin{equation}
E_A=\frac{2}{\pi}|u_{2,0}|^4,
\label{eq:AreaEnergyULM20}
\end{equation}
در معادله‌ی فوق 
$k_A=1[\varepsilon/l^2]$
و
$r_0=1[l]$
فرض شده‌است. نیروی بازگرداننده که با این تغییر شکل مقاومت می‌کند با مشتق گیری از انرژی در فضای مُد قابل محاسبه‌ می‌باشد، 
\begin{equation}
-\frac{\partial E_A}{\partial u_{2,0}}=-\frac{8}{\pi}|u_{2,0}|^3,
\label{eq:AreaForceULM20}
\end{equation}


به همین ترتیب با جایگذاری 
$k_V=1[\varepsilon/l^3]$
در معادله‌ی
\ref{eq:VolumeGLFluctuationAmplitude}
می‌توان انرژی تغییر حجم مش زمانی که از  شکل کره به شکل دمبلی تغییر می‌کند را بر حسب شدت مُد محاسبه کرد،
\begin{equation}
E_V= \frac{3}{8\pi}|u_{2,0}|^4.
\label{eq:VolumeEnergyULM20}
\end{equation}

با مشتق ‌گیری نسبت به شدت مُد، مقدار نیرویی که با تغییر شکل مش مقاومت می‌کند را می‌توان بر حسب شدت مد محاسبه کرد،
\begin{equation}
-\frac{\partial E_V}{\partial u_{2,0}}= -\frac{3}{2\pi}|u_{2,0}|^3,
\label{eq:VolumeForceULM20}
\end{equation}

\begin{figure}[tbp]
\begin{center}
\includegraphics[width=14cm]{\MemRes/Pics/UnitSphereDeformation_mesh_10_Areavolume}
\caption{
انرژی مساحت (ستون 
$(a)$
) و حجم (ستون 
$(b)$
) محاسبه‌ شده حاصل از تغییر شکل مش کروی با اضافه شدن مد
$Y_{2,0}(\theta,\phi)$
با شدت‌های مختلف برای چهار نوع مش رسم شده‌است. انرژی مساحت برای دو روش محاسبه‌ی مساحت کل، جمع مساحت‌های ورنوی (نقاط بنفش) و جمع مساحت‌های بریسنتریک (نقاط خاکستری)، و انرژی حجم (نقاط آبی) با جمع روی حجم هرم‌ها محاسبه شده‌است. خطوط مشکی برای ستون‌های 
$(a)$
و
$(b)$
به ترتیب از رسم معادلات
\ref{eq:AreaEnergyULM20}
و
\ref{eq:VolumeEnergyULM20}
حاصل شده‌است. نیروی بازگرداننده برای تغییر مساحت و حجم در ستون‌های 
$(c)$
و
$(d)$
رسم شده.  پیش‌بینی‌ مرتبه‌ی دوم نیروی بازگرداننده برای تغییر مساحت وحجم با توجه به معادلات
\ref{eq:AreaForceULM20}
و
\ref{eq:VolumeForceULM20}
با خط مشکی رسم شده است. شدت مد 
$u_{2,0}=0$
مربوط به شکل کاملا کروی (عکس سمت چپ) و شدت مد 
$u_{2,0}=1$
مربوط به شکل دمبلی (عکس سمت راست) است. یک نمونه از هر مش برای حالت کروی و دمبلی در ردیف مربوته رسم شده‌است. به غیر از مش منظم (که تنها یک نمونه از آن برای هر تعداد نقطه وجود دارد) مقادیر محاسبه شده حاصل از میان‌گین گیری بر روی ۵۰ نمونه‌ی مستقل انجام شده‌است.
}
\label{fig:unitsphereAreaVolumeULM20}
\end{center}
\end{figure}


در شکل
\ref{fig:unitsphereAreaVolumeULM20}
نتیجه‌ی محاسبات عددی در کنار پیش‌بینی نظری برای انرژی و نیروی حاصل از تغییر شکل رسم شده است. شدت مد
$u_{2,0}=0$
مربوط به مش کاملا کروی است (عکس‌های سمت چپ) و شدت مد 
$u_{2,0}=1$
مربوط به مش‌های دمبلی شکل است (عکس‌های سمت راست). هر ردیف محاسبات را برای یک نوع مش نشان می‌دهد که با  عکسی از آن مش مشخص شده‌است. به ترتیب در ستون‌های 
$(a)$
،
$(b)$
،
$(c)$
، و
$(d)$
تغییر انرژی مساحت، انرژی حجم، نیروی مساحت، و نیروی حجم به صورت نقاط بنفش (ورنوی) و خاکستری (بریسنتریک) برای مساحت و نقاط آبی برای حجم به همراه پیش‌بینی استخراج شده از محاسبات افت و خیز (خط مشکی) رسم شده‌است. 

داده‌های شکل 
\ref{fig:unitsphereAreaVolumeULM20}
نشان می‌دهد که رفتار انرژی و نیرو برای تغییر شکل‌های کوچک با پیش‌بینی مستخرج از محاسبات افت و خیز همخوانی دارد. از آنجایی که محاسبات افت و خیز برای شدت مد‌های کوچک و تا مرتبه‌ی دوم در 
$u_{\ell,m}$
درنظر گرفته شده‌است برای شدت‌ مد‌های بزرگ فاقد اعتبار است. ولی رفتار کلی محاسبات عددی و معادلات
\ref{eq:AreaEnergyULM20}
،
\ref{eq:VolumeEnergyULM20}
،
\ref{eq:AreaForceULM20}
، و
\ref{eq:VolumeForceULM20}
برای شد‌ت مد 
$|u_{\ell,m}|>0.5$
نیز همخوانی دارد. می‌توان نتیجه گرفت که محاسبات لازم برای اندازگیری مساحت و حجم غشا برای شکل‌های کروی و همچنین شکل‌های غیر کروی حاوی انحنای زین اسبی بر روی مش‌های معمولی  و  درهم به خوبی قابل پیاده‌سازی است.












\section{\label{sec:curvatureD}
پتانسیل انحنای متوسط
}
یک مِش ممکن است از نقاط با درجه‌های مختلفی تشکیل شده باشد. مثلا مش منظم از نقاط با درجات ۵ و ۶ ساخته شده و مش‌های تصادفی قالبا درجات ۵، ۶، و ۷ دارند. پتانسیل انحنا بر روی لیس رئوسی تعریف می‌شود که درجه‌ی یکسان دارند. در نتیجه به ازای هر درجه‌ی موجود در مش  لیست مجزایی از رئوس با آن درجه تشکیل داده می‌شود. سپس پتانسیل انحنا برای هر لیست درجه به طور مجزا تعریف می‌شود. نحوه‌ی محاسبه‌ی انحنا میان این پتانسیل‌ها یکسان بوده و تنها تعداد جملاتی که بر روی آن جمع زده می‌شود متفاوت است. در نتیجه برای هر لیست از رئوس درجه‌ی 
$n$
یک پتانسیل 
$n+1$
ذره‌ای نیاز است.


پتانیسیل  انحنا برای مدل یولیشر به شکل زیر پیاده‌سازی شده،
\begin{equation}
U_b^J=\frac{3}{4}\kappa\sum_i\frac{\left[\sum_{j(i)}\ell_{ij}\phi_{ij}\right]^2}{\sum_{j,j'(i)}\ell_{ij}\ell_{ij'}\sin(\theta_{jij'})}
\label{eq:UbJDiscrete}
\end{equation}
در اینجا تعریف پارامتر‌های 
$\ell_{ij}$
و
$\phi_{ij}$
مطابق شکل 
\ref{fig:trianglePairAngle}
است. در صورت معادله‌ی فوق جمع روی تمام رئوس همسایه نقطه‌ی 
$i$
است و در مخرج جمع روی تمام مثلث‌هایی است که نقطه‌ی 
$i$
میان رئوس آن‌هاست. باید توجه ویژه با انتخاب جهت زاویه‌ی دوسطحی کرد. زاویه‌ی دوسطح میان ۴ نقطه تعریف می‌شود و ترتیب دو نقطه‌ی میانی علامت زاویه‌ی دوسطحی را تغییر می‌دهد. برای کُره لیست نقاط دوسطحی باید طوری تنظیم شود که تمام زوایای دوسطحی روی کُره مثبت باشد.

به شکل مشابه می‌توان پتانسیل انحنا را برای مدل گامپر، گامپر-برسنتریک، و یولیشر-ورنوی تعریف کرد،
\begin{equation}
U_b^\text{GK}=\kappa\sum_i\frac{\left[\sum_{j(i)}(\cot\theta_1^{ij}+\cot\theta_2^{ij})(\vec r_i-\vec r_j)\right]^2}{\sum_{j(i)}\ell_{ij}^2(\cot\theta_1^{ij}+\cot\theta_2^{ij})},
\label{eq:UbGKDiscrete}
\end{equation}

\begin{equation}
U_b^{GKB}=\frac{3}{4}\kappa\sum_i\frac{\left[\sum_{j(i)}(\cot\theta_1^{ij}+\cot\theta_2^{ij})(\vec r_i-\vec r_j)\right]^2}{\sum_{j,j'(i)}\ell_{ij}\ell_{ij'}\sin(\theta_{jij'})},
\end{equation}

\begin{equation}
U_b^{JV}=\kappa\sum_i\frac{\left[\sum_{j(i)}\ell_{ij}\phi_{ij}\right]^2}{\sum_{j(i)}\ell_{ij}^2(\cot\theta_1^{ij}+\cot\theta_2^{ij})}.
\end{equation}

در معادلات فوق فرض شده که سطح انحنای ذاتی ندارد. با استفاده از معادله‌ی 
\ref{eq:bendingDiscretisationSpontaneous}
می‌توان معادلات فوق را برای حالتی که خمش ذاتی 
$C_0$
وجود داشته باشد، بازنویسی کرد. برای یولیشر معادلات به شکل

\begin{equation}
\begin{aligned}
U_b^J=\frac{1}{2}\kappa\sum_i&\frac{3}{2}\frac{\left[\sum_{j(i)}\ell_{ij}\phi_{ij}\right]^2}{\sum_{j,j'(i)}\ell_{ij}\ell_{ij'}\sin(\theta_{jij'})}\\
&-C_0\sum_{j(i)}\ell_{ij}\phi_{ij}\\
&+\frac{1}{6}C_0^2\sum_{j,j'(i)}\ell_{ij}\ell_{ij'}\sin(\theta_{jij'}),
\end{aligned}
\end{equation}
برای گامپر،
\begin{equation}
\begin{aligned}
U_b^\text{GK}=\frac{1}{2}\kappa\sum_i&2\frac{\left[\sum_{j(i)}(\cot\theta_1^{ij}+\cot\theta_2^{ij})(\vec r_i-\vec r_j)\right]^2}{\sum_{j(i)}\ell_{ij}^2(\cot\theta_1^{ij}+\cot\theta_2^{ij})}\\
&-C_0\sqrt{\left[\sum_{j(i)}(\cot\theta_1^{ij}+\cot\theta_2^{ij})(\vec r_i-\vec r_j)\right]^2}\\
&+\frac{1}{8}C_0^2\sum_{j(i)}\ell_{ij}^2(\cot\theta_1^{ij}+\cot\theta_2^{ij}),
\end{aligned}
\end{equation}
برای گامپر-بریسنتریک،
\begin{equation}
\begin{aligned}
U_b^{GKB}=\frac{1}{2}\kappa\sum_i&\frac{3}{2}\frac{\left[\sum_{j(i)}(\cot\theta_1^{ij}+\cot\theta_2^{ij})(\vec r_i-\vec r_j)\right]^2}{\sum_{j,j'(i)}\ell_{ij}\ell_{ij'}\sin(\theta_{jij'})}\\
&-C_0\sqrt{\left[\sum_{j(i)}(\cot\theta_1^{ij}+\cot\theta_2^{ij})(\vec r_i-\vec r_j)\right]^2}\\
&+\frac{1}{6}C_0^2\sum_{j,j'(i)}\ell_{ij}\ell_{ij'}\sin(\theta_{jij'}),
\end{aligned}
\end{equation}
و در نهایت برای یولیشر-ورنوی به شکل
\begin{equation}
\begin{aligned}
U_b^{JV}=\frac{1}{2}\kappa\sum_i&2\frac{\left[\sum_{j(i)}\ell_{ij}\phi_{ij}\right]^2}{\sum_{j(i)}\ell_{ij}^2(\cot\theta_1^{ij}+\cot\theta_2^{ij})}\\
&-C_0\sum_{j(i)}\ell_{ij}\phi_{ij}\\
&+\frac{1}{8}C_0^2\sum_{j(i)}\ell_{ij}^2(\cot\theta_1^{ij}+\cot\theta_2^{ij}),
\end{aligned}
\end{equation}

قابل محاسبه‌ است.





\section{
پتانسیل‌های تکمیلی
\label{sec:auxPotentials}
}
\subsection{
پتانیسل
WCAh
}

فاصله‌ی میان نقاط و اضلاع مثلث‌ها در شبکه را می‌توان با پتانسیل 
Weeks-Chandler-Andersen
یا به اختصار
WCA
کنترل کرد. این پتانسیل منشا فیزیکی ندارد و برای ایجاد پایداری در شبیه‌سازی به کار برده می‌شود. با کنترل فاصله‌ی نقاط با اضلاع مثلث‌ می‌توان حد پایین برای اندازه‌ی پلاکت‌ها تعیین کرد و در نتیجه اندازه قدم شبیه‌ سازی دینامیک ملکولی را تعیین کرد. از طرفی محاسبات هندسی مختلفی که برای محاسبه‌ی سطح، حجم، و انحنای مش مورد نیاز است در صورتی دو نقطه از شبکه یا یک نقطه و یک ضلع مثلث دقیقا روی هم قرار بگیرند به پاسخ گنگی منجر خواهد شد. با استفاده از این پتانسیل احتمال ناپایداری در محاسبات را می‌توان از میان برد. از آنجایی که یک مثلث ۳ اتفاع دارد، برای هر مثلث در مش ۳ پتانسیل نیاز خواهیم داشت. برای یکی از ارتفاعات یک مثلث نمونه این پتانسیل به شکل زیر تعریف می‌شود،
\begin{equation}
U_{WCAh}=\epsilon\left[\left(\frac{d_h}{h}\right)^8-\left(\frac{d_h}{h}\right)^4+\frac{1}{4}\right].
\label{eq:wcah}
\end{equation} 
پارامتر

فاصله‌ی کمینه ممکن برای ارتفاع را مشخص می‌کند،  عمق چاه
$\epsilon=4k_BT$
، و فاصله‌ی قطع 
\LTRfootnote{cut off}
پتانیسل 
$h_{cutoff}=\sqrt[6]{2}d_h$
است. ارتفاع راس
$i$
از ضلع تعریف شده میان دو راس
$j$
و
$j'$
در مثلث
$ijj'$
به صورت 
\begin{equation}
h_i=\ell_{ij}\sin\theta_{ijj'}.
\end{equation} 
محاسبه می‌شود.

\subsection{
پتانیسل غیر خطی دوسطحی
}
به طول عمومی تمامی پتانسیل‌هایی که برای محاسبه‌ی انحنای رویه در این رساله معرفی شده برای حد خمش‌های کم صادق است. جهت محاسبه‌ی صحیح انرژی انحنا لازم است که در طول شبیه‌سازی از پدید آمدن لبه‌های نوک تیز روی مش جلوگیری کرد. از آنجایی که انرژی انحنای خمش یک پتانسیل هارمونیک است، پتانسیلی که زوایای میان مثلث‌ها را کنترل کند باید در حد خمش‌های کم هزینه‌ی انرژی تقریبا صفر داشته باشد و تنها در زوایای خمش زیاد ظاهر شده و در برابر خم شدن مثلث‌ها مقاومت نشان دهد. با تعریف پتانسیل به صورت
\begin{equation}
U_{\phi^4}=\frac{1}{2}k_{\phi^4}\left[e^{2(1-\cos\phi_{ij})}-1-\phi_{ij}^2 \right].
\label{eq:theta4}
\end{equation}
میان تمام جفت مثلث‌ها از ایجاد زوایای دوسطحی بزرگ جلوگیری کرد. قدرت پتانسیل با پارامتر 
$k_{\phi^4}$
قابل تنظیم است.














\section{\label{sec:OpenMM}
موتور محاسباتی
OpenMM
}
در تحقیقات این رساله از موتور محاسباتی دینامیک مولکولی
OpenMM \cite{OpenMM2017}
جهت حل معادلات حرکت استفاده شده‌است. تمامی پتانسیل‌ها به شکلی بازنویسی شده‌اند که در این بسته‌ی نرم‌افزاری قابل پیاده‌سازی باشد. در این فصل نحوه‌ی محاسبه‌ی نیروی حاصل از پتانسیل‌ها ارائه نشده زیراکه 
OpenMM
با دانستن پتانسیل حاکم بر ذرات قادر به محاسبه‌ی نیرو است. 


شبیه‌سازی ذراتی که تحت معادله‌ی نیوتن حرکت می‌کنند، 
\begin{equation}
m_i\frac{d\vec v_i}{d_t}=\vec f_i.
\label{eq:newton}
\end{equation}
از روش انتگرال گیری پرشِ قورباغه‌‌ایِ میانه‌ای\LTRfootnote{Leap Frog Middle discretization}
ورله موجود در بسته‌ی نرم افزاری 
OpenMM
انجام شده‌است. جهت بررسی افت و خیز سطح، معادله‌ی حرکت لانژون برای محاسبه‌ی نیرو استفاده شد
 \begin{equation}
m_i\frac{dv_i}{d_t}=\vec f_i -\gamma m_i\vec v_i+R_i.
\label{eq:newton}
\end{equation}
در این بسته‌ی نرم افزاری دینامیک مولکولی این معادله با الگوریتم پرش قورباغه‌ای 
\cite{IZAGUIRRE2009}
گسسته سازی شده‌است. 
$\gamma$
ضریب اصطکاک و 
$R_i$
نیروی غیر همبسته‌ی تصادفی\LTRfootnote{Non-correlated random force}
 با مقدار میانگین صفر و واریانس
$2m_i\gamma k_BT$
است. 

\section{\label{sec:dataAcu}
آماده سازی شبیه‌سازی
}
نتایج ارائه شده در این رساله  به طور پیش‌فرض با مِش‌های تصادفی دارای 
$N=1002$
نقطه و جرم کل
$M=N\times m$
ایجاد شده‌اند. جرم هر ذره 
$m=50 [m_0]$
مقداردهی شده و در صورتی که مِش کُروی باشد اندازه‌ی شعاع آن 
$r_0=1000 [l]$
تنظیم شده‌است. سرعت اولیه ذرات از یک توزیع بولتزمن با دمای 
$k_BT=2.49 [\varepsilon]$
انتخاب شده‌است.

پارامترهای مربوط به خواص ماده‌ی غشا به این ترتیب مقداردهی شده‌اند، ضریب سختی خمش 
$\kappa=20k_BT$,
ضریب فشردگی سطحی
$k_A=5.22\times10^{5}k_BT/r_0^2 [\varepsilon/l^2]$,
مدول فشردگی حجمی
$k_V=1.6\times10^7k_BT/r_0^3 [\varepsilon/l^3]$
و مقادیر تعادلی مساحت
$A_0$,
و حجم
$V_0$
مستقیم از هندسه‌ی اولیه مِش محاسبه‌ شده‌است. 

در مورد پتانیسل‌های تکمیلی برای تمامی مش‌ها پتانسیل 
$U_h$
با اندازه‌ی کمینه ارتفاع 
$d_h=0.02r_0 [l]$
و عمق چاه
$\epsilon=4k_BT$
استفاده شد. در صورت نیاز به پتانسیل غیر خطی دو وجهی، ضریب سختی آن 
$k_{\phi}=20k_BT$
قرار داده شده‌است.

در شبیه‌سازی واحد زمان
$[\tau]$
بر حسب واحد طول 
$[l]$
واحد جرم
$[m_0]$
و واحد انرژی
$[\varepsilon]$
به صورت
\begin{equation}
\tau =\sqrt{\frac{m_0l^2}{\varepsilon}}
\end{equation}
تعیین می‌شود.











\section{\label{sec:VCM}
نرم افزار مدل مجازی سلول
}
نرم افزار مدل سلول مجازی\LTRfootnote{Virtual Cell Model}
در گروه ماده چگال نرم دانشگاه صنعتی شریف جهت مطالعه‌ی خواص مکانیکی سلول هنگام مهاجرت\LTRfootnote{cell migration}
و چسبیدن به سطوح\LTRfootnote{cell adhesion}
ایجاد شد
\cite{Tiam2017ACS, Tiam2018AFM}.
 بخش غشای‌ این نرم افزار جهت انجام شبیه‌سازی‌های مورد نیاز این رساله توسعه یافت. در حال حاضر بخش غشا توانایی شبیه‌سازی غشا‌های جامد (دارای مدول یانگ) و غشاهای سیال را دارد. توسعه‌‌ی نرم افزار حدود ۳ سال طول کشید. در این زمان بخش‌هایی مانند مدل‌سازی شبکه‌ی اسکلت سلولی و کروماتین نیز توسعه داده شده‌است. همچنین تمامی مقیاس‌های انرژی و تعاریف پتانسیل با واحد‌های قابل مقایسه با نتایج آزمایشگاهی کالیبره شدند
\cite{VCMgit}.

به لحاظ عملکرد، موتور محاسباتی دینامیک ملکولی 
OpenMM
به عنوان مرکز محاسباتی اصلی این نرم افزار قرار داده شد که تمام قابلیت‌های این موتور محاسباتی را در دسترس کاربر قرار می‌دهد. از جمله قابلیتی که مورد توجه افراد مشغول در زمینه‌ی محاسباتی است، توانایی انجام محاسبات به شکل موازی بر پردازنده‌های مرکزی چند هسته‌ای\LTRfootnote{multi-core CPU}
و پردازنده‌های گرافیکی\LTRfootnote{GPU}
است. جهت کسب اطلاعات بیشتر به صفحه‌ی 
YouTube \cite{VCMYoutube}
و صفحه‌ی راهنمای این نرم افزار 
\cite{VCMhomepage}
مراجعه فرمایید.



