\begin{figure}[htbp]
\begin{center}
\includegraphics[width=\columnwidth]{\MemRes/Pics/area_N_kappa_r.pdf}
\caption{
افت و خیز مساحت غشای شبیه‌سازی شده با 
(a)
مش‌های با تعداد نقاط مختلف، 
(b)
مش‌های با سختی خمش مختلف،
(c)
مش‌های با شعاع‌های مختلف، و
(d)
مش‌های با نرمی مختلف
$\Phi$
. دایره‌ی قرمز رنگ در هر چهار شکل شبیه‌سازی‌هایی که مشخصات 
$r_0=1\mu m$, $\kappa=20k_BT$, $\Phi=0.037$, 
 و
$N=1002$
دارند را نشان می‌دهد. در تمامی حالت‌های شبیه‌سازی افت و خیز نسبی مساحت کل مِش ناچیز است. در نتیجه‌ می‌توان مقدار آن را ثابت در نظر نگرفت. 
}
\label{fig:areaNKR}
\end{center}
\end{figure}

پتانسیل مساحت، مساحت در حال افت و خیز مِش را کنترل می‌کند.  شکل
\ref{fig:areaNKR}
نسبت مساحت متوسط مِش‌های شبیه‌سازی شده به مساحت شعاع مش را نشان می‌دهد. داده‌هایی که با دایره‌ی قرمز مشخص شده‌اند در تمام نمودارها غشایی را نشان می‌دهد که 
 $r_0=1\mu m$, $\kappa=20k_BT$, $\Phi=0.037$, 
 و
$N=1002$
دارد. نتایج نشان می‌دهد که تغییر سختی خمش، مساحت متوسط مش‌‌های شبیه‌سازی شده را تغییر نمی‌دهد ولی تغییر
$\Phi$
مساحت مش‌ها را کمی افزایش می‌دهد. در نتیجه می‌توان مقدار مساحت کلِ مِش را در طول شبیه‌سازی‌ها، مقدار ثابتی در نظر گرفت.
















