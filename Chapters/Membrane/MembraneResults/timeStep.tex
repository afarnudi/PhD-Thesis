از آنجایی که انرژی انحنا قوی‌ترین انرژی در سیستم است می‌توان قدم زمانی شبیه‌سازی را با توجه به مقیاس زمانی نوسانات مربوط به این انرژی تنظیم کرد. با استفاده از معادله‌ی 
\ref{eq:modeFrequencyBending}
دوره‌ی تناوب بر حسب مد سیستم
\begin{equation}
\tau_{\ell,m}=\frac{2\pi}{\omega_{\ell,m}}=\sqrt{\pi \frac{Mr_0^2}{\kappa}\frac{1}{(\ell+2).(\ell+1)\ell(\ell-1)}}
\label{eq:Lperiod}
\end{equation}
است. تعداد مثلث‌های روی یک مش تقریبا دو برابر تعداد نقاط مش است،
$N_{tri}\approx 2N$.
بیشترین عدد مد ایجاد شده در سیستم مربوط به تعداد درجات آزادی حرکت ذرات در راستای عمود بر سطح بوده
$\ell_{max}=\sqrt{N}-1$ 
\cite{Gompper1996}
که در اینجا 
$N$
تعداد نقاط مش است. در نتیجه کوتاه‌ترین دوره‌ی نوسان یا دوره‌ی نوسان مربوط به بزرگترین عدد مد سیستم برابر با 
\begin{equation}
\tau_{\ell_{max},m}\approx \frac{1}{N}\sqrt{\pi \frac{Mr_0^2}{\kappa}}.
\label{eq:LmaxPeriod}
\end{equation}
است. 

به طور متوسط شکل مثلث میانگین بر روی مش یک مثلث متساوی الاضلاع به ضلع
$b$
و مساحت 
 $\bar a_{iso}$
 است. مساحت این مثلث برابر است با
 \begin{equation}
\bar a_{iso}=\frac{1}{2}b^2\sin(\pi/3)=\frac{4\pi r_0^2}{2N}.
\label{eq:averageTriArea}
\end{equation}
 با بازنویسی معادله‌ی فوق می‌توان اندازه‌ی طول ضلع مثلث را محاسبه کرد،
 \begin{equation}
b=r_0\sqrt{\frac{4\pi}{N\sin(\pi/3)}}.
\label{eq:averageTriArea}
\end{equation}
 
هنگام استفاده از الگوریتم توزیع مساحت، نقاط می‌توانند بر روی صفحه حرکت کنند. در نتیجه ممکن است مناطقی تشکیل شود که از مثلث‌های بسیار کوچکی ساخته شده باشند. مساحت مثلث‌های کوچک
$\bar a_{min}$
تابع کوچکترین ارتفاع 
$h$
است که اندازه‌ی آن توسط پتانسیل
WCAh
تعیین می‌شود. مساحت این مثلث‌ها 
\begin{equation}
\bar a_{min}=\frac{1}{2}\frac{d_h^2}{\sin(\pi/3)}.
\label{eq:aMin}
\end{equation}
خواهد بود.


$\phi$
را نسبت مساحت کوچکترین مثلث درون مش به مساحت مثلث متوسط مش تعریف می‌کنیم،
\begin{equation}
\phi=\frac{\bar a_{min}}{\bar a_{iso}}=\frac{d_h^2}{r_0^2}\frac{N}{4\pi\sin(\pi/3)}
\label{eq:phiDef}
\end{equation}

از طرفی پتانسیل 

نیز پتانسیل قوی در سیستم است. در نتیجه قدم زمانی شبیه‌سازی باید به میزانی کوچک باشد که این پتانسیل نیروهای قابل قبولی تولید کند. مقیاس زمانی این پتانسیل مانند مقیاس زمانی پتانسیل لنارد-جونز محاسبه می‌شود،
\begin{equation}
\tau_{WCAh}\approx d_h\sqrt{\frac{M/N}{\epsilon}}=\frac{r_0}{N}\sqrt{\phi}\sqrt{\frac{\pi\sin(\pi/3)}{k_BT}}.
\label{eq:tauWCAh}
\end{equation}
در اینجا 
$\epsilon=4k_BT$.
حال سعی می‌کنیم 
$\tau_{WCAh}$
را برای طول قدم زمانی محاسبه کنیم که در آن هم نیرو‌های انحنا هم نیروی 
WCAh
به شکل معقولی محاسبه شود. با برابر قرار دادن دو مقیاس زمانی 
$\phi$
به شکل زیر تخمین زده می‌شود،
\begin{equation}
\phi=\frac{k_BT}{\kappa\sin(\pi/3)}.
\label{eq:phiTimeScale}
\end{equation}

برای کندترین دوره‌ی تناوب در سیستم 
(کوچکترین عدد موج)
زمان تنواب
\begin{equation}
\tau_{2,m}=\sqrt{\frac{\pi}{24} \frac{Mr_0^2}{\kappa}}
\label{eq:Lperiod}
\end{equation}
است. 

\begin{figure}[htbp]
\begin{center}
\includegraphics[width=10cm]{\MemRes/Pics/timeStep}
\caption{
نقاط زمان متوسط شکست 
 $T_f$
اجرای شبیه‌سازی یک غشای تقریبا کروی را نمایش می‌دهد. معادله‌ی حرکت ذرات توسط معادله‌ی نیوتن تعریف شده انرژی جنبشی اولیه از توزیع بولتزمن با دمای
 $k_BT=2.49[\varepsilon]$
انتخاب شده‌است. زمان شکست زمانی است که یا شبیه‌سازی به علت ناپیداری‌های عددی شکست بخورد یا زمانی که انرژی جنبشی سیستم دو برابر شود. انتخاب رنگ نقاط مطابق مدل محاسبه‌ی انحاست که در طول رساله استفاده شده‌است. در این شبیه‌سازی مش‌های با 
$N=1002$
نقطه استفاده شده و 
$\phi\approx0.037$
تنظیم شده‌است.
}
\label{fig:timeSteps}
\end{center}
\end{figure}
در شکل 
\ref{fig:timeSteps}
زمان متوسط شکست

اجرای شبیه‌سازی یک غشای تقریبا کروی با انرژی مساحت، حجم، و انحنا را نمایش داده‌ایم. در این شبیه‌سازی با استفاده از پتانسیل 

نسبت مساحتی مثلث‌ها را 

تنظیم کرده‌ایم. سرعت اولیه‌ی ذرات از توزیع بولتزمن با دمای 

انتخاب شده‌ و ذرات تحت دینامیک نیوتنی حرکت کرده‌اند. زمان شکست اجرا

 زمانی است که شبیه‌سازی به علت ناپایداری‌های عددی شکست بخوردی یا اولین زمانی که انرژی جنبشی غشا بر اثر انباشته شدن خطای انتگرال‌گیری دو برابر شود. 
 
 با توجه به بحثی که در بخش قبل شد، طبق انتظارمان، مدل‌های انحنای ایتزیکسون (قرمز) و یولیشر-ورنوی (نارنجی) بسیار ناپایدار هستند. حتی هنگامی که قدم زمانی شبیه‌سازی یک ده‌هزارم بزرگترین مد سیستم انتخاب شد
 
شبیه‌سازی ایتزیکسون فقط قادر به تکمیل دو تناوب مد دمبلی بود و یولیشر-ورنوی حتی قادر به تکمیل یک دوره‌ی تناوب نبود. 
مد‌ل‌های یولیشر و ایتزیکسون-بریسنتریک بسیار پایدار و زمان شکستشان به طور متوسط دو مرتبه بزرگی بیشتر از مدل‌های بر پایه‌ی ورنوی بود. 

در نتیجه برای انجام شبیه‌سازی دینامیک ملکول مدل‌های بر پایه‌ی بریسنتریک پیشنهاد می‌شود. 








