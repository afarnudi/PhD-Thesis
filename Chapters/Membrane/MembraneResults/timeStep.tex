از آنجایی که انرژی انحنا قوی‌ترین انرژی در سیستم است می‌توان قدم زمانی شبیه‌سازی را با توجه به مقیاس زمانی نوسانات مربوط به این انرژی تنظیم کرد. با استفاده از معادله‌ی 
\ref{eq:bendingFluctuations}
دوره‌ی تناوب بر حسب مد سیستم
\begin{equation}
\tau_{\ell,m}=\frac{2\pi}{\omega_{\ell,m}}=\sqrt{\pi \frac{Mr_0^2}{\kappa}\frac{1}{(\ell+2).(\ell+1)\ell(\ell-1)}}
\label{eq:Lperiod}
\end{equation}
است. تعداد مثلث‌های روی یک مش تقریبا دو برابر تعداد نقاط مش است،
$N_{tri}\approx 2N$.
بیشترین عدد مد ایجاد شده در سیستم مربوط به تعداد درجات آزادی حرکت ذرات در راستای عمود بر سطح بوده
$\ell_{max}=\sqrt{N}-1$ 
\cite{Gompper1996}
که در اینجا 
$N$
تعداد نقاط مش است. در نتیجه کوتاه‌ترین دوره‌ی نوسان یا دوره‌ی نوسان مربوط به بزرگترین عدد مد سیستم برابر با 
\begin{equation}
\tau_{\ell_{max},m}\approx \frac{1}{N}\sqrt{\pi \frac{Mr_0^2}{\kappa}}.
\label{eq:LmaxPeriod}
\end{equation}
است. 

از طرفی پتانسیل 
$U_h$
نیز پتانسیل قوی در سیستم است. در نتیجه قدم زمانی شبیه‌سازی باید به میزانی کوچک باشد که این پتانسیل نیروهای قابل قبولی تولید کند. مقیاس زمانی این پتانسیل مانند مقیاس زمانی پتانسیل لنارد-جونز محاسبه می‌شود،
\begin{equation}
\tau_{h}\approx d_h\sqrt{\frac{M/N}{\epsilon}}=\frac{r_0}{N}\sqrt{\Phi}\sqrt{\frac{\pi\sin(\pi/3)}{k_BT}}.
\label{eq:tauWCAh}
\end{equation}
در اینجا 
$\epsilon=4k_BT$.
حال سعی می‌کنیم 
$\tau_{h}$
را برای طول قدم زمانی محاسبه کنیم که در آن هم نیرو‌های انحنا هم نیروی 
$U_h$
به شکل معقولی محاسبه شود. با برابر قرار دادن دو مقیاس زمانی 
$\Phi$
به شکل زیر تخمین زده می‌شود،
\begin{equation}
\Phi=\frac{k_BT}{\kappa\sin(\pi/3)}.
\label{eq:phiTimeScale}
\end{equation}

برای کندترین دوره‌ی تناوب در سیستم 
(کوچکترین عدد موج)
زمان تنواب
\begin{equation}
\tau_{2,m}=\sqrt{\frac{\pi}{24} \frac{Mr_0^2}{\kappa}}
\label{eq:Lperiod}
\end{equation}
است. نسبت زمان تناوب کوچکترین به بزرگترین مد سیستم

\begin{equation}
\frac{\tau_{2,m}}{\tau_{\ell_{max},m}}=\frac{N}{\sqrt{24}},
\end{equation}
و نسبت زمان بزرگترین مد (سریع‌ترین مد) به زمان مشخصه‌ی پتانسیل لنارد جونز

\begin{equation}
\frac{\tau_{\ell_\text{max},m}}{\tau_{h}}=\frac{r_0}{d_h}\sqrt{\frac{4\pi }{N}}\sqrt{\frac{k_BT}{\kappa}}.
\end{equation}


\begin{figure}[htbp]
\begin{center}
\includegraphics[width=10cm]{\MemRes/Pics/timeStep}
\caption{
نقاط زمان متوسط شکست 
 $T_f$
اجرای شبیه‌سازی یک غشای تقریبا کروی را نمایش می‌دهد. معادله‌ی حرکت ذرات توسط معادله‌ی نیوتن تعریف شده انرژی جنبشی اولیه از توزیع بولتزمن با دمای
 $k_BT=2.49[\varepsilon]$
انتخاب شده‌است. زمان شکست زمانی است که یا شبیه‌سازی به علت ناپیداری‌های عددی شکست بخورد یا زمانی که انرژی جنبشی سیستم دو برابر شود. انتخاب رنگ نقاط مطابق مدل محاسبه‌ی انحاست که در طول رساله استفاده شده‌است. در این شبیه‌سازی مش‌های با 
$N=1002$
نقطه استفاده شده و 
$\Phi\approx0.037$
تنظیم شده‌است.
}
\label{fig:timeSteps}
\end{center}
\end{figure}
در شکل 
\ref{fig:timeSteps}
زمان متوسط شکست
$T_f$
اجرای شبیه‌سازی یک غشای تقریبا کروی با انرژی مساحت، حجم، و انحنا را نمایش داده‌ایم. در این شبیه‌سازی با استفاده از پتانسیل 
$U_h$
نسبت مساحتی مثلث‌ها را 
$\Phi=0.037$
تنظیم کرده‌ایم. سرعت اولیه‌ی ذرات از توزیع بولتزمن با دمای 
$k_BT=2.49[\varepsilon]$
انتخاب شده‌ و ذرات تحت دینامیک نیوتنی حرکت کرده‌اند. زمان شکست اجرا
$T_f$
 زمانی است که شبیه‌سازی به علت ناپایداری‌های عددی شکست بخوردی یا اولین زمانی که انرژی جنبشی غشا بر اثر انباشته شدن خطای انتگرال‌گیری دو برابر شود. 
 
 با توجه به بحثی که در بخش قبل شد، طبق انتظارمان، مدل‌های انحنای گامپر (قرمز) و یولیشر-ورنوی (نارنجی) بسیار ناپایدار هستند. حتی هنگامی که قدم زمانی شبیه‌سازی یک ده‌هزارم بزرگترین مد سیستم انتخاب شد
 $\Delta t \approx 2\times10^{-4}\tau_{h} $
شبیه‌سازی گامپر فقط قادر به تکمیل دو تناوب مد دمبلی بود و یولیشر-ورنوی حتی قادر به تکمیل یک دوره‌ی تناوب نبود. 
مد‌ل‌های یولیشر و گامپر-بریسنتریک بسیار پایدار و زمان شکستشان به طور متوسط دو مرتبه بزرگی بیشتر از مدل‌های بر پایه‌ی ورنوی بود. 

\begin{figure}[htbp]
\begin{center}
\includegraphics[width=10cm]{\MemRes/Pics/total_curvature_152_plus_aux_zoomed.pdf}
\caption{
بزرگنمایی چشم انداز انرژی نمایش داده شده در شکل 
\ref{fig:PlacketEnergyAll}b2
و
d2
که پتانیسیل‌های گامپر و یولیشر-ورنوی را به همراه پتانسیل‌های دیگر را رسم می‌کند.
}
\label{fig:PlacketEnergyAllZoomed}
\end{center}
\end{figure}

در شکل 
\ref{fig:PlacketEnergyAllZoomed}
بزرگنمایی چشم انداز انرژی نمایش داده شده در شکل
\label{fig:PlacketEnergyAll}b2
و
d2
را مشاهده می‌کنید. پتانسیل‌های تکمیلی قادر به پوشش کامل چاه‌های عمیق پتانسیل حاصل از محاسبات ورنوی نیستند. از طرفی سد انرژی نیز برای مقید محاصره‌ی این چاه‌ها کافی نیست. در نهایت هنگام شبیه‌سازی حتما نقاط مش مسیری را برای به دام افتادن پیدا خواهند کرد.  افزایش برد پتانسیل
$U_h$
سد پتانسیل بین چاه‌های عمیق و سطح انرژی معمولی ایجاد خواهد کرد، ولی این کار همچنین باعث کریستال شدن مش می‌شود. می‌توان پتانسیل‌های تکمیلی دیگری نیز طارحی کرد که ناپایداری‌های محاسبات ورنوی را پوشش دهد. این پتانسیل‌ها باید واگرایی قوی‌تری نسبت به ناپایداری‌ها داشته باشند، در نتیجه باید از قدم‌های کوچک‌تری برای شبیه‌سازیشان استفاده شود. از آنجایی که دو مدل یولیشر و گامپر-بریسنتریک پایداری بالایی زیادی دارند، انگیزه‌ای برای طراحی چنین پتانسیل‌هایی وجود ندارد. در نتیجه برای انجام شبیه‌سازی دینامیک ملکول گامپر مدل‌های بر پایه‌ی بریسنتریک پیشنهاد می‌شود. 

پتانسیل‌های بریسنتریکی نیازی به پتانسیل‌ غیر خطی زاویه‌ای برای شبیه‌سازی ندارند ولی در صورتی که در شبیه‌سازی تغییر شکل شدیدی وجود داشته باشد، استفاده از پتانسیل
$U_\phi$
با این مدل‌ها پیشنهاد می‌شود.








