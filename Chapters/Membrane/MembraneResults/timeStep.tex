تا به اینجا تمام عناصر لازم برای اجرای یک شبیه‌سازی دینامیک مِش را داریم. برای اینکه به طور کیفی از کار کردن این روش مطمئن شوید، تشویقتان می‌کنیم که فیلم‌ شبیه‌سازی
\textbf{SM\_AVh\_GKB\_V} 
را مشاهده کنید. این فیلم شبیه‌سازی یک مش نرم با پتانسیل‌های 
$U_A$, $U_V$, $U_A$, 
و پتانسیل خمش گامپر کرول بریسنتر 
$U_b^\text{GKB}$
را نمایش می‌دهد. این شبیه‌سازی با یک مِش تصادفی با حجم کاهیده‌ی 
 $\nu\simeq1$
آماده شده‌است. به علت وجود پتانسیل انحنا، سطح این غشا نسبت به غشای در فیلم
\textbf{SM\_AVh\_V}
که پتانسیل انحنا ندارد، صاف‌تر است.

در این بخش، قدم زمانی مناسب برای شبیه‌سازی‌های دینامیکِ مِش را می‌یابیم و سپس محاسبات را برای پتانسیل‌های انحنای مختلف ارزیابی می‌کنیم.






\subsection{\label{sec:ResultsTimeStep}
انتخاب قدم زمانی برای شبیه‌سازی
}

حرکت درون-صفحه‌ای نقاط مِش به وسیله‌ی پتانسیل تکمیلی
$U_h$
محدود می‌شود. همانطور که قبلا مطرح شد، زمان مشخصه‌ی این پتانسیل برابر است با
\begin{equation}
\tau_{h}= d_h\sqrt{\frac{m}{\epsilon_h}}.
\label{eq:tauWCAh}
\end{equation}
همچنین می‌توان از روابط
 $4\pi r_0^2\approx2N\bar a_{iso}=\frac{2}{\sqrt{3}} N {\bar h}^2 = \frac{2}{\sqrt{3}} \frac{N}{\Phi} d_h^2$
برای بیان زمان مشخصه‌
$\tau_h$
استفاده کرد. در نتیجه زمان مشخصه‌ی رفتار درون-صفحه‌ای مِش بر حسب شعاع مِش 
 $r_0$
و جرم کل مِش
 $M$
به شکل زیر تعریف می‌شود،
\begin{equation}
\tau_{h}
=  \sqrt{2\pi\sqrt{3} }  \frac{\sqrt{\Phi}}{N}\sqrt{\frac{M r_0^2}{\epsilon_h}}.
\label{eq:tauWCAh M r0}
\end{equation}

حرکت برون-صفحه‌ای مِش توسط پتانسیل انحنا مشخص می‌شود. سریع‌ترین دوره‌ی تناوب نقاط بر سطح یک مِش دارای توزیع همگن نقاط، 
\begin{equation}
\tau_{\ell_{max},m}= \frac{2\pi}{N}\sqrt{\frac{Mr_0^2}{4\pi\kappa}}=\frac{1}{\sqrt{2\sqrt{3}}}\bar h\sqrt{\frac{m}{\kappa}}
\label{eq:tauL}
\end{equation}
است. از طرفی در مِش‌هایِ نرم مناطقی وجود دارد که مثلث‌های خیلی کوچکی می‌توانند شکل بگیرند که ارتفاع آن 
$d_h$
است. در واقع کوتاه‌ترین زمان تناوب برای حرکت برون-صفحه‌ای مِش مربوط به چنین مثلث‌هایی‌است
\begin{equation}
\sqrt{\Phi}\tau_{\ell_{max},m}=\frac{1}{\sqrt{2\sqrt{3}}}d_h\sqrt{\frac{m}{\kappa}}=\sqrt{\Phi}\frac{1}{N}\sqrt{\pi\frac{Mr_0^2}{\kappa}}.
\label{eq:tauLim}
\end{equation}
 قدم زمانی شبیه‌سازی باید کسری از  زمانی باشد که میان 
$\tau_h$
و
$\sqrt{\Phi}\tau_{\ell_{max},m}$
کمینه است. از این نتیجه برداشت‌های مختلفی می‌توان داشت. در درجه‌ی اول به نظر می‌آید که زمان شبیه‌سازی برای مش‌هایی که مشخصات فیزیکی یکسان داشته باشند، در صورتی که جرم نقاط مِش یکی باشد زمان مشخصه‌ی تناوب از تعداد نقاط مش مستقل خواهد بود. در صورتی که زمان تناوب را بر حسب جرم و مساحت کل مِش بنویسیم، وابستگی این زمان به مشخصات مش ظاهر می‌شود. بنا به انتظار عمومی، نتایج نشان می‌دهد که قدم شبیه‌سازی برای مِش‌هایی که تعداد نقاط بیشتری دارند کوتاه‌تر است. همچنین انتخاب مقادیر کوچکتر برای 
$\Phi$
(معادله‌ی 
\ref{eq:Phi})
باعث می‌شود مثلث‌های کوچکتری پدید آیند و این امر خود زمان قدم شبیه‌سازی را کاهش می‌دهد. اما از این مهم‌تر ما نیاز داریم که مقیاس انرژی پتانسیل تکمیلی را به گونه‌ای انتخاب کنیم که
\begin{equation}
\epsilon_h<2\sqrt{3}~\kappa.
\label{eq:EhKappa}
\end{equation}
در این صورت مطمئن خواهیم بود که حرکت برون صفحه‌ای نقاط مش همیشه سریع‌ترین پدید‌ه‌های شبیه‌سازی دینامیک مش خواهند بود.

\subsection{\label{sec:instabilities}
چشم انداز انرژی
}


\begin{figure}[h]
\begin{center}
\includegraphics[width=\columnwidth]{\MemRes/Pics/total_curvature_152_plain_slices.pdf}
\caption{
سمت چپ، نمایش ۲ بُعدی پلاکتی (مجموعه مثلث‌هایی که یک نقطه‌ی مشترک دارند) از مِشِ درهمِ تصادفی را با رنگ سبز نشان می‌دهد. نقطه‌ی وسط پلاکتِ سبز، عضو پلاکت‌های همسایه‌ نیز هست که با رنگ نارنجی مشخص شده‌است. سمت راست، دو بُرِش در چشم انداز انرژی انتخاب شده‌است: ۱) سطح کروی که تمام نقاط مِش بر آن قرار دارند (ستون سمت چپ در شکل‌های
\ref{fig:PlacketEnergyArea}, \ref{fig:PlacketEnergy},
و
\ref{fig:PlacketEnergyAll}
و ۲)  صفحه‌ای صاف (صفحه‌ی آبی) که بر سطح مِش عمود است ( ستون سمت راست در شکل‌های 
\ref{fig:PlacketEnergyArea}, \ref{fig:PlacketEnergy},
و
\ref{fig:PlacketEnergyAll}).
}
\label{fig:PlacketRepresentaion}
\end{center}
\end{figure}
همانطور که در بخش‌های قبل نشان داده شد، هر سه انرژی لازم برای مدل کردن غشا (یعنی انرژی مساحت، حجم، و انحنا) بر روی مِش‌های درهم با دقت قابل قبولی تعریف شده و همچنین نیروی محاسبه شده از آنها با محاسبات نظری همخوانی دارد. برای بررسی اثر پتانسیل‌های فیزیکی (پتانسیل مساحت، حجم، و انحنا) و پتانسیل‌های تکمیلی 
($U_h$
و
$U_\phi$)
 در شبیه‌سازی دینامیکِ مِش، چشم انداز انرژی در فضای اطراف یک نقطه
 ($i$)
  از مِشِ درهمِ تصادفی را مطالعه می‌کنیم. چشم انداز انرژی تغییرات انرژی پتانسیل حاصل از تغییر مختصات این نقطه را بیان می‌کند،
  \begin{equation}
\begin{aligned}
&\Delta U(\bm q_{i}') = U({\cal G}, \{\bm q'\}) - U({\cal G}, \{\bm q\});\\
&\{\bm q\}=\{\bm q_1, \bm q_2, , ..., \bm q_{i},...,\bm q_N\},\\
&\{\bm q'\}=\{\bm q_1, \bm q_2, , ..., \bm q_{i}',...,\bm q_N\}.
\label{eq:energyLandscape}
\end{aligned}
\end{equation}
البته نمی‌توان با بررسی تنها یک نمونه چیدمان‌ نقاط مِش‌ رفتار پیچیده‌ی دینامیکِ مِش هنگام شبیه‌سازی را توضیح داد، ولی می‌توان تاثیر استفاده از مساحت وُرُنوی و بریسنتر در پتانسیل‌های گسسته شده‌ی انحنا را فهمید. در شکل 
\ref{fig:PlacketRepresentaion},
سمت چپ، مختصات مرجع نقاط مِش
($\{{\bm q}\}$)
که بر سطح یک کره قرار دارند رسم شده‌است. از آنجایی که نقاط مِش روی سطح کره هستند، تمام اضلاع مِش زیر این سطح قرار می‌گیرند. مختصات نقطه‌ی متغیر
(${\bm q_i}$)
مربوط به نقطه‌ی مرکزی پلاکتِ سبز است. ما چشم انداز انرژی 
($\Delta U({\bm q_i'})$)
را برای ۲ بُرِش ۲-بُعدی بررسی می‌کنیم (شکل
\ref{fig:PlacketRepresentaion}
سمت راست): ستون سمت چپ شکل‌های 
\ref{fig:PlacketEnergyArea}, \ref{fig:PlacketEnergy},
و
\ref{fig:PlacketEnergyAll})
مقادیر 
$\Delta U$
زمانی که مختصات نقطه‌ی 
$i$
بر روی صفحه‌ی کروی، و ستون سمت راست شکل‌های 
\ref{fig:PlacketEnergyArea}, \ref{fig:PlacketEnergy},
و
\ref{fig:PlacketEnergyAll})
مقادیر 
$\Delta U$
زمانی که مختصات نقطه‌ی 
$i$
بر صفحه صاف عمود بر سطح (که شامل مختصات
${\bm q_i}$
و مرکز کره است) را نشان می‌دهند.
 

\begin{figure}[htbp]
\begin{center}
\includegraphics[width=12cm]{\MemRes/Pics/area_volume_WCAh_Exp46_152}
\caption{
چشم انداز انرژی پتانسیل مساحت، حجم،
$U_h$
، و خمش غیر خطی دوسطحی
$U_\phi$
 بر حسب مختصات نقطه‌ی میانی پلاکت سبز رنگ. در ستون سمت چپ تغییر انرژی حاصل از حرکت نقطه بر سطح کره و ستون سمت راست تغییر انرژی حاصل از حرکت نقطه بر صفحه‌ی عمود بر سطح کره  نشان داده شده‌است. تمامی پتانسیل‌ها در مقابل خروج نقطه از منطقه‌ی پلاکت سبز مقاومت می‌کنند. واحد انرژی
$k_BT=2.49[\varepsilon]$
 و مقادیر 
 $k_A=5.12\times10^5k_BT/l^2$
 و
 $k_V=1.6\times10^7k_BT/l^3$
 به ترتیب برای ضریب فشردگی سطحی و مدول حجمی در نظر گرفته شده است. در پتانسیل‌های تکمیلی، پارامتر‌های پتانسیل 
 $U_h$, $\epsilon=4k_BT$
 و کمینه ارتفاع مثلث‌ها 
 $d_h/r_0=0.02$.
 برای ضریب سختی خمش پتانسیل غیر خطی دوسطحی 
 $\kappa_{\phi}=20k_BT$
 و انرژی در واحد‌های کاهیده
 $k_BT=2.49[\varepsilon]$
 است.
}
\label{fig:PlacketEnergyArea}
\end{center}
\end{figure}

در شکل 
\ref{fig:PlacketEnergyArea} $(a)$
 انرژی مساحت نمایش داده شده‌است. مقدار این انرژی تنها تابع تغییر شکل مثلث‌های درون پلاکت سبز رنگ است. در ستون سمت چپ تا زمانی که نقطه‌ی وسط داخل محدوده‌ی پلاکت حرکت می‌کند، مساحت کل غشا کم و بیش ثابت است. هنگامی‌ که نقطه‌ی وسط از محدوده‌ی پلاکت خارج شود مساحت غشا افزایش پیدا کرده و انرژی مساحت نیز افزایش می‌یابد. در ستون سمت راست نیز هنگامی که نقطه‌ی وسط از چیدمان اولیه فاصله می‌گیرد مساحت و در نتیجه انرژی مساحت افزایش پیدا می‌کند. 

در شکل 
\ref{fig:PlacketEnergyArea} $(b)$
 انرژی حجم نمایش داده شده‌است. مقدار این انرژی نیز تنها تابع تغییر شکل مثلث‌های درون پلاکت سبز رنگ است. از آنجایی که حجم هرم‌ها بسته به جهت سطح منفی یا مثبت هستند، تا زمانی که نقطه‌ی وسط بر سطح کره حرکت می‌کند (سمت چپ) حجم کل مش تغییر نکرده و در نتیجه انرژی حجم ثابت است.‌ از آنجایی که حجم هرم با ضرب داخلی سطح و ارتفاع محاسبه می‌شود در ستون سمت راست  هنگامی که نقطه‌ی وسط از چیدمان اولیه فاصله می‌گیرد، حجم تنها طبق فاصله‌ی شعاعی نسبت به جهت عمودی پلاکت تغییر می‌کند و حاصل آن صفحات موازی هم انرژی‌است.

در شکل 
\ref{fig:PlacketEnergyArea} $(c)$
تغییرات انرژی 
$U_h$
 نمایش داده شده‌است. از آنجایی که هنگام تغییر مختصات نقطه‌ی میانی تنها ارتفاع نقطه‌ی میانی با اضلاع مجاور تغییر خواهد کرد، مقدار این انرژی نیز تنها تابع تغییر شکل مثلث‌های درون پلاکت سبز رنگ است. در تصویر سمت چپ، تا زمانی که ارتفاع نقطه‌ی میانی از حد قطع بیشتر باشد، تغییری اتفاق نمی‌افتد. اما زمانی که نقطه به اضلاع یا رئوس نزدیک شود این انرژی با شیب توانی افزایش می‌یابد. در این نمایش چشم انداز انرژی زمانی که نقطه قادر به عبور از سطح بالای انرژی 
 $U_h$
 باشد نیز رسم شده‌است. در تصویر سمت راست نیز اثر پتانسل تنها زمانی مشاهده می‌شود که مختصات نقطه به اضلاع نزدیک شود.
 
 در شکل 
\ref{fig:PlacketEnergyArea} $(d)$
تغییرات انرژی پتانسیل غیر خطی دوسطحی نمایش داده شده‌است. باید توجه کرد که در تصویر سمت چپ نقطه‌ی وسط بر سطح کره حرکت می‌کند و نه بر سطح مش. در این صورت نقطه از روی اضلاع عبور کرده و آنها را قطع نخواهد کرد. به همین علت تا زمانی که نقطه در محدوده‌ی پلاکت سبز حرکت می‌کند زاویه‌ی تیزی تشکیل نخواهد شد. اما هنگامی که نقطه به اضلاع نزدیک شود، زوایای بسیار تیزی تشکیل شده و انرژی غیر خطی
$U_\phi$
 را سریع افزایش خواهد داد. در تصویر سمت راست اثر این پتانسیل برای زوایای تیز به وضوح دیده می‌شود.

%به طور کلی نقطه هنگامی که از محدوده‌ی  پلاکت سبز خارج شود، چه بر سطح کره و چه خارج از صفحه‌، این انرژی‌ها را افزایش خواهد داد.

\begin{figure}[htbp]
\begin{center}
\includegraphics[width=12cm]{\MemRes/Pics/total_curvature_152.pdf}
\caption{
تغییر انرژی انحنای یولیشر، گامپر، گامپر-بریسنتریک، و یولیشر-ورنوی هنگام حرکت یک نقطه بر سطح کره و بر روی صفحه‌ی عمود بر سطح کره
(شکل
\ref{fig:PlacketRepresentaion}
سمت راست)
 نسبت به انرژی انحنای اولیه‌ی مِش.  تا زمانی که مختصات نقطه‌ی میانی در محدوده‌ی پلاکت سبز قرار دارد، پتانسیل‌های بر اساس مساحت ورنوی (ردیف‌های 
$b$
و
$d$)
 رفتار مشابه به پتانسیل‌های بر اساس مساحت بریسنتر (ردیف‌های 
$a$
و
$c$)
دارند. در صورتی که مختصات نقطه‌ی میانی خارج از محدوده‌ی پلاکت  باشد، چشم انداز پتانسیل‌های بر اساس مساحت ورونی، مناطقی با عمق چاه بینهایت ایجاد می‌کنند. ولی پتانسیل‌های بر اساس مساحت بریسنتر، قادر به هدایت نقطه به محدوده‌ی پلاکت سبز هستند. سختی خمش در تمام مدل‌ها 
$\kappa=20k_BT$
در نظر گرفته شده‌است.
}
\label{fig:PlacketEnergy}
\end{center}
\end{figure}


در شکل
\ref{fig:PlacketEnergy}
تغییر انرژی انحنای مش به ترتیب از بالا به پایین برای مدل یولیشر، گامپر، گامپر-بریسنتریک، و یولیشر-ورنوی نمایش داده شده‌است. نکته‌ی مهم در مورد انرژی انحنا اینجاست که نقطه‌ی میانی پلاکت سبز رنگ، نقش نقطه‌ی گوشه‌ای برای پلاکت‌های که با رنگ نارنجی در شکل
\ref{fig:PlacketRepresentaion}
سمت چپ نمایش داده‌شده، دارد. در نتیجه چیدمان مختصات نقاط اطراف پلاکت بر چشم انداز انرژی تاثیر دارد. در ستون سمت چپ، انرژی انحنای هر چهار مدل تا زمانی که مختصات نقطه‌ی میانی در محدود‌ه‌ی پلاکت سبز قرار دارد کم و بیش ثابت است. در مورد انرژی‌ها بر اساس مساحت بریسنتریک، در صورتی که مختصات نقطه از محدوده‌ی پلاکت خارج شود، انرژی افزایش یافته و نقطه را به داخل پلاکت هدایت می‌کند. اما در مورد پتانسیل‌های بر پایه‌ی مساحت ورنوی، فضای اطراف پلاکت پر از چاه‌های منفی بینهایت و ناپایداری‌های عددی‌ است. رفتار مشابه نیز در ستون سمت راست هنگام خروج نقطه از صفحه دیده می‌شود. 


\begin{figure}[htbp]
\begin{center}
\includegraphics[width=12cm]{\MemRes/Pics/total_curvature_152_plus_aux.pdf}
\caption{
هر ردیف، جمع انرژی انحنای نمایش داده شده در ردیف متناظر در  شکل
\ref{fig:PlacketEnergy},
با چهار انرژی مساحت، حجم، 
$U_h$,
و خمش غیر خطی دوسطحی 
$U_\phi$
را نمایش می‌دهد. تمامی‌ پارامتر‌ها طبق پارامتر‌های عنوان شده در دو شکل 
\ref{fig:PlacketEnergy}
و
\ref{fig:PlacketEnergyArea}
است. تا زمانی که پتانسیل‌های تکمیلی بتوانند مختصات نقطه‌ی میانی را در محدوده‌ی پلاکت مقید کنند، رفتار مجموع پتانسیل‌ها قابل قبول است. 
}
\label{fig:PlacketEnergyAll}
\end{center}
\end{figure}


در شکل 
\ref{fig:PlacketEnergyAll}
مجموع انرژی انحنا، مساحت، و حجم به همراه انرژی‌های تکمیلی 
$U_h$
و خمش غیر خطی دوسطحی 
$U_\phi$
رسم شده‌است. مشاهده می‌شود که فضای افت وخیز نقطه‌ی وسط پلاکت برای انرژی‌های بر اساس بریسنتریک به همسایگی پلاکت محدود شده‌است. اما در چشم انداز انرژی مدل‌های انحنای بر پایه‌ی ورنوی چشم انداز انرژی هم بر سطح کره و هم بر سطح صفحه‌ی عمود بر آن، پُر از ناپایداری‌های عددی است. البته می‌توان پتانسیل‌های تکمیلی طراحی کرد که تمام نقاط ناپایداری مدل‌های گامپر و یولیشر-ورنوی را پوشش دهند. از آنجایی که چنین پتانسیل‌هایی نیاز به واگرایی قوی دارند، استفاده از آن‌ها در شبیه‌سازی دینامیک مولکولی با انتخاب قدم زمانی بسیار کم امکان پذیر خواهد بود. چون دو مدل دیگر انحنا بدون نیاز به چنین پتانسیل‌هایی قابل استفاده هستند، در حال حاضر هیچ انگیزه‌ای برای طراحی چنین پتانسیلی وجود ندارد.

هرچند نتایج استاتیک اندازه‌گیری انحنا (ستون راست در شکل
\ref{fig:unitsphereAll})
 برای تمامی‌ مدل‌های انحنا امکان پذیر است ولی در صورتی که مختصات نقاط مِش تغییر کند، دقیق‌ترین مدل‌های محاسبه‌ی انحنا (مدل‌های بر پایه‌ی مساحت ورنوی) پایداری عددی لازم برای انجام شبیه‌سازی دینامیک مولکولی را ندارند. از سوی دیگر مدل‌های انحنای بر پایه‌ی مساحت بریسنتر انتخاب خوبی برای شبیه‌سازی دینامیک مولکولی خواهند بود.









\subsection{\label{sec:ResultsStability}
پایداری شبیه‌سازی دینامیکِ مِش
}

\begin{figure}[tbp]
\begin{center}
\includegraphics[width=13cm]{\MemRes/Pics/timeStepStupid.pdf}
\caption{
نقاط زمان متوسط اجرای شبیه‌سازی قبل از شکست
$\langle T_f\rangle$
شبیه‌سازی‌های مش‌های تقریبا کروی. معادلات حرکت توسط الگوریتم ورله سرعتی جمع شدند تا زمانی که شبیه‌سازی به علت ناپایداری عددی متوقف شود. در هر دو نمودار، شبیه‌سازی دینامیک مش با مدل انحانی یولیشر (سبز)، گامپر (قرمز)، یولیشر-ورنوی (نارنجی) و گامپر-بریسنتریک (آبی) برای مقادیر 
$\Phi\approx0.037$ 
(نمودار بالا) و
$\Phi\approx0.59$ 
(نمودار پایین) رسم شده‌است. پایداری عددی پتانسیل‌های بر پایه مساحت بریسنتر برای مش‌های نرم (بالا) و سخت (پایین) بسیار بالاست. اما پتانسیل‌های بر پایه‌ی مساحت ورنوی تنها برای مش‌های سخت پایدار هستند.
}
\label{fig:timeSteps}
\end{center}
\end{figure}


جهت ارزیابی قابلیت شبیه‌سازی مش دینامیک و تاثیر پتانسیل‌های انحنای گسسته‌سازی شده، شبیه‌سازی‌های متعددی از مِش‌های کروی با چهار مدل انحنا انجام دادیم. در این شبیه سازی‌ها
$\epsilon_h=4k_bT$
و
$\kappa=20k_BT$
انتخاب شد که معادله‌ی
\ref{eq:EhKappa}
را ارضا می‌کند. سپس با انتخاب کسر‌های مختلفی از
 $\sqrt{\Phi}\tau_{\ell_{max},m}$
شبیه‌سازی را آغاز کردیم. مقدار
$\Phi=0.037$
برای شبیه‌سازی مِش‌های بسیار نرم انتخاب شد. در حالت عادی قدم زمانی
$\Delta t\sim0.01\sqrt{\Phi}\tau_{\ell_{max}}$
انتخاب مناسبی برای داشتن یک شبیه‌سازی دینامیک مولکولی پایدار است. مِش‌ها در آغاز شبیه‌سازی حجم کاهیده‌ی نزدیک به واحد و 
$\Phi\sim1$
داشتند. برای هر شبیه‌سازی زمان اتمام به علت شکست 
$T_f$
اندازه‌گیری شد. یک شبیه‌سازی به ۲ علت ممکن است شکست بخورد. یکی ایجاد انرژی بی‌نهایت در سیستم ناشی از معادلات پتانسیل‌های حاکم بر سیستم، و دیگری افزایش تدریجی انرژی سیستم به علت خطای ناشی از جمع‌کردن جملات حرکت با الگوریتم ورله است. ما در شبیه‌سازی‌هایمان زمانی که انرژی سیستم 
$50\%$
افزایش یافت، شبیه‌سازی را متوقف  و 
$T_f$
را اندازه‌گیری کردیم. زمان متوسط شکست شبیه‌سازی دینامیکِ مِش‌های تصادفی با درجه‌ی نرمی 
$\Phi$
مختلف در شکل
\ref{fig:timeSteps}
گزارش شده‌است. بنا به نتایج، دینامیکِ مِش‌های سخت 
$\Phi\approx 0.59$
را می‌توان با قدم‌های زمانی حدود
$\Delta t \sim10^{-1}\sqrt{\Phi}\tau_{\ell_{max},m}$
با پایداری خوبی اجرا کرد. پایداری عجیب این شبیه‌سازی‌ها برای مقادیر کم قدم شبیه‌سازی به علت تطابق بسیار خوب آلگوریتم جمع ورله سرعتی با پتانسیل‌های هارمونیک (پتانسیل‌های انحنای مورد استفاده) است. این رفتار در مورد پتانسیل‌هایی که از مساحت بریسنتر استفاده می‌کنند برای مش‌های نرم
$\Phi\approx 0.037$
همچنان صادق است. اما برای پتانسیل‌هایی که از مساحت ورنوی استفاده می‌کنند شبیه‌سازی‌های مش‌های نرم به سرعت ناپایدار می‌شود. 

\begin{figure}[h]
\begin{center}
\includegraphics[width=13cm]{\MemRes/Pics/total_curvature_152_plus_aux_zoomed}
\caption{
نسخه‌ی بزرگنمایی شده‌ای از چشم‌ انداز انرژی رسم شده در شکل‌های 
\ref{fig:PlacketEnergyAll} $(b2)$
و
$(d2)$
برای پتانسیل‌ انحنای گامپر (چپ) و یولیشر-ورنوی (راست). حتی در حضور پتانسیل‌های تکمیلی، مسیرهایی برای فرار نقاط به مناطق با چاه پتانسیل منفی بی‌نهایت وجود دارد.
}
\label{fig:energyLandscapeZoomed}
\end{center}
\end{figure}

همانطور که در قسمت قبل نشان دادیم، نقاط مِش می‌توانند آزادانه در محدوده‌ی پلاکت‌های خود حرکت کنند. در حضور پتانسیل‌های برسینتریک، زمانی که نقاط از این محدوده خارج می‌شوند، با پتانسیل  بزرگی مواجه می‌شوند که آنها را به داخل محدوده‌ی پلاکت هدایت می‌کند. ولی در حضور  پتانسیل‌های ورنوی، در خارج از این محدوده، مناطقی وجود دارد که دارای انرژی پتانسیل بی‌نهایت منفی است. نتایج ما در شکل
\ref{fig:energyLandscapeZoomed}
 نشان می‌دهد که حتی در حضور پتانسیل‌های تکمیلی، مسیر‌هایی در چشم انداز انرژی این پتانسیل‌ها وجود دارد که  همچنان نقاط مِش را به این ناحیه‌ها هدایت می‌کند. 
 
 در عمل می‌توان پتانسیل‌های تکمیلی بیشتری طراحی کرد که مسیر‌های هادی به نواحی پتانسیل بی‌نهایت منفی مدل‌های انحنای ورنوی را مسدود کند. اما از آنجایی که مدل‌های بریسنتریک به اندازه‌ی کافی دقیق و پایدار هستند، انگیزه‌ای برای این کار نمی‌بینیم.



























