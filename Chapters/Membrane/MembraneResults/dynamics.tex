\begin{figure}[htbp]
\begin{center}
\includegraphics[width=\columnwidth]{\MemRes/Pics/auto_samps.pdf}
\caption{
تابع خود همبستگی دامنه‌ی افت و خیز‌های مد‌های سطحی مشی با 
 $1002$
نقطه، شعاع
 $r_0=1\mu m$,
 سختی خمش
$\kappa=2-k_BT$,
 و نرمی
 $\Phi\approx0.037$
برای مقادیر  مختلف عدد مُد. ستون‌های مختلف مربوط به شبیه‌سازی‌های انجام شده با روش ورله سرعتی
(I)
، لانژون
(II)
و
(III),
 و براونی 
(IV)
است. هر خم رسم شده مربوط به یک مد اندازه‌گیری شده بر سطح مش است که با یک عدد تصادفی مختلف تولید شده‌است. در ستون 
I
رفتار سینوسی هر مد بسامد مختص خود را دارد. در ستون‌های 
II
و
III
رفتار زیر میرا و فوق میرا، تابع فرکانس نوسان هر مد و ضریب اتلاف
 $\gamma$
است. در ستون آخر، افت نمایی ناشی از اتلاف بالا مشاهده می‌شود.
}
\label{fig:autoSampling}
\end{center}
\end{figure}


برای کمی‌ سازی نتایج دینامیک غشا در شبیه‌سازی دینامیک ملکولی دینامیک مش، تابع خود همبستگی دامنه‌ی افت و خیز مد‌های سطحی مش را بررسی می‌نماییم. این رفتار برای سه نوع معادله‌ی حرکت مختلف نقاط مش (دینامیک نیوتن، لانژون، و براونی) با استفاده از نرم افزار VCM بررسی شده‌است. 


شکل
\ref{fig:autoSampling}
خلاصه‌ای از داده‌های خام مربوط به غشاهایی با ضریب سختی خمش 
 $\kappa = 20 k_BT$
و شعاع 
$r_0=1\mu m$
که با مش‌های نرم
($\Phi=0.037$)
 که با 
$N=1002$
نقطه شبیه‌سازی شده‌اند را نشان می‌دهد. نتایج برای شبیه‌سازی در رژیم‌های نیوتنی، لانژون با دو ضریب اتلاف 
($\gamma=0.001$
و
$\gamma=0.01$),
 و براونی را به ترتیب از سمت چپ به راست در ستون‌های مختلف نشان می‌دهد. با اینکه در معادله‌ی حرکت براونی جرم وجود ندارد (جرم صفر است) 
$\gamma$
را می‌توانیم با تقسیم ضریب اتلاف براونی
$\zeta$
بر جرم غشایی که در شبیه‌سازی استفاده شده محاسبه‌کرده و با نتایج شبیه‌سازی‌های دینامیک لانژون مقایسه نمود.

نمودار‌های هر  ردیف‌ نتایج مربوط به اعداد مد
$\ell\in[2,7]$
را نشان می‌دهد. هر خم، متوسط زمانی یک تک مد
$(\ell,m)$
که از پنج اندازه‌گیری مستقل بدست آمده را نشان می‌دهد. از آنجایی که به ازای هر عدد مد،
$2\ell+1$
مقدار مد مستقل (که با
$m$
متمایز می‌شوند) وجود دارد، در ردیف‌های پایین‌تر تعداد خم‌های بیشتری وجود دارد (مثلا، در ستون 
I
برای 
$C_2(t)$
۲۵ خم و برای 
$C_7(t)$
۷۵ خم). 

تعبیر کیفی داده‌های شکل
\ref{fig:autoSampling}
نسبتا ساده‌است. برای دینامیک براونی و لانژون که ضریب اتلاف بزرگتر از 
$\gamma=0.01$
دارند، همگی رفتار فوق میرا نشان می‌دهند. در ستون 
II
که شبیه‌سازی لانژون با ضریب اتلاف
$\gamma=0.001$
را نشان می‌دهد، تنها مد
$\ell=2$
فوق میراست، مد
$\ell=3$
رفتار نزدیک به میرایی بحرانی نشان می‌دهد، و تمامی‌ مد‌هایی با
$\ell\ge4$,
رفتار زیر میرا دارند. به علاوه، مش‌هایی که تحت دینامیک نیوتنی هستند، همگی رفتار زیر میرا دارند.

















\begin{figure}[htbp]
\begin{center}
\includegraphics[width=\columnwidth]{\MemRes/Pics/auto_analysis.pdf}
\caption{
تابع خود همبستگی دامنه‌ی افت و خیز‌های مد‌های سطحی مشی با 
 $1002$
نقطه، شعاع
 $r_0=1\mu m$,
 سختی خمش
$\kappa=2-k_BT$,
 و نرمی
 $\Phi\approx0.037$
برای اعداد مد مختلف. ستون‌های مختلف مربوط به شبیه‌سازی‌های انجام شده با روش ورله سرعتی
(I)
، لانژون
(II)
و
(III)
، و براونی 
(IV)
است. خم‌های رنگی، میانگین و خطای داده‌هایی که در شکل
\ref{fig:autoSampling}
رسم شده بود را نشان می‌دهند. خطوط مقطع مشکی معادله‌ی
$cos(\omega_{\ell,m}t)$
را برای ستون اول و معادلات
\ref{eq:autoUnderDamped}, \ref{eq:autoCriticallyDamped},
 یا 
\ref{eq:autoOverDamped}
را برای ستون 
II
و
III
 و در نهایت معادله‌ی 
 \ref {eq:BrowninaModeAuto}
 را برای ستون آخر نشان می‌دهد. خطوط نقطه‌چین در ستون اول مقدار برانداز شده برای اندازه‌گیری مقدار 
  $\gamma_\ell$
 را نشان می‌دهد.
}
\label{fig:autoAnalysis}
\end{center}
\end{figure}

جهت بررسی کمی‌ این داده‌ها، میانگین‌های آماری داده‌های خام شکل
\ref{fig:autoSampling},
 در شکل 
\ref{fig:autoAnalysis}
رسم شده‌اند و رفتار تابع خود همبستگی  با پیشبینی‌های نظری که در بخش 
\ref{sec:TimeStep}
انجام شد، مقایسه شده‌اند. بر خلاف شکل
\ref{fig:autoSampling}
داده‌ها در شکل
\ref{fig:autoAnalysis}
بر محور زمانی مشابه رسم نشده‌اند، بلکه در اینجا محور زمانی بر حسب مقیاس‌های زمانی مرتبط با نرخ اتلاف (که تابع  عدد مد یا دوره‌ی تناوب مدهاست) بیان شده‌است.

به لحاظ نظری، تابع خودهمبستگی دامنه‌ی افت وخیز‌های سطحی  مش‌هایی که با معادلات حرکت نیوتن نوسان می‌کنند، رفتار پایدار سینوسی خواهد داشت. اما در شبیه‌سازی‌های انجام شده در این مطالعه، شاهد رفتار میرایی بسیار ضعیفی هستیم که به علت جفت شدگی دینامیکی فرمی-پاستا-اولام\LTRfootnote{Fermi–Pasta–Ulam}
\cite{FPU}
است که از طریق پتانسیل غیر خطی تکمیلی، میان مد‌های مختلف ایجاد شده‌است. جالب است که تابع خود همبستگی مد‌های مختلف زمانی که بر حسب محور 
 $\omega_{\ell,m}~t$
رسم می‌شود بسیار شبیه به هم رفتار می‌کنند. این مشاهده نشان می‌دهد که ضریب اتلاف مد‌ها حدود کسری از فرکانس هر مد‌ است. جهت تایید این مشاهده، در شکل
\ref{fig:gammaEll}
مقادیر
 $\gamma_{\ell}$
را برای نرمی‌های مختلف مش
$\Phi$
مطالعه نمو‌دیم. این بررسی نشان می‌دهد که برای مش‌های بسیار نرم
($\Phi\ll1$),
برای تمامی‌ مد‌ها، ضریب اتلاف داخلی رابطه‌ی
 $\gamma_\ell\approx0.15\omega_{\ell,m}$
دارد. ضریب اتلاف برای مش‌های سخت
$\Phi = {\cal O}(1)$
 بزرگتر است ولی در هیچ شرایطی بر دینامیک داخلی مد‌ها قالب نیست.




\begin{figure}[h]
\begin{center}
\includegraphics[width=\columnwidth]{\MemRes/Pics/gamma_ell.pdf}
\caption{
مقدار اتلاف ذاتی مش‌های در حال افت و خیز در خلا (رژیم نیوتنی)‌. خطوط مقطع،‌ خط
$a\omega_{\ell,m}+b$
را برای مقدار
$\Phi=0.037$
و خط
$a\omega_{\ell,m}(\Phi)+b$
را برای مقدار 
$\Phi\geq0.15$
رسم می‌کنند. در این معادلات 
$a$
و
$b$
پارامتر‌های آزاد برازش شده هستند.
}
\label{fig:gammaEll}
\end{center}
\end{figure}

در نتیجه، به غیر از اتلاف داخلی بسیار ضعیف مشاهده  شده، رفتار دینامیک تمامی مد‌های اندازه‌گیری شده طبق انتظارات نظری رفتار می‌کند. این رفتار در حالی مشاهده شده که جرم تمامی‌ نقاط مش یکسان در نظر گرفته شده‌است. البته در صورتی معادلات صحیح حرکت، یعنی تغییر جرم لحظه‌ای نقاط مش در هر قدم دینامیک ملکولی، پیاده‌سازی شود، نتیجه‌ی شبیه‌سازی ستون
I
در شکل
\ref{fig:autoAnalysis}
بر پیش‌بینی نظری تطابق بهتری خواهد داشت. ولی همچنین باید در نظر داشت که برای بررسی غشا‌هایی که در رژیم عدد رنولدز‌های بسیار کوچک حرکت می‌کنند، این تصحیحات اهمیت نخواهد داشت. رفتار افت و خیز غشا‌ها در طبیعت بسیار میرا است و حرکت سطح آن توسط هیدرودینامیک سیال بیرون و خارج غشا حکم می‌شود
\cite{schneider1984, milnersafranPRA1987}
.

در ستون
I
شکل
 \ref{fig:autoAnalysis}
داده‌ها با مقیاس زمانی فرکانس زاویه‌ای مد‌ها
$\omega_{\ell,m}$
رسم شده است. 
$\omega_{\ell,m}^\prime$
با جایگزین کردن پارامتر‌های ورودی شبیه‌سازی 
($\gamma$, $\kappa$, $M$,
 و
$r_0$)
به شکل نظری محاسبه‌ شده‌است. بنا به نتیجه‌ی محاسبه‌ی 
$\omega_{\ell,m}^\prime$
یکی از معادلات
\ref{eq:autoUnderDamped}, \ref{eq:autoCriticallyDamped},
 یا
\ref{eq:autoOverDamped}
برای رسم خط مشکی به عنوان دوره‌ی تناوب انتخاب شده‌است. فضای خاکستری اطراف خط مشکی، خطای ناشی از برازش جمله‌ی اتلافی داخلی مش را مشخص می‌کند. با استفاده از پارامتر‌های شبیه‌سازی، برای ستون
II
شکل 
 \ref{fig:autoAnalysis},
مد بحرانی
$\ell_{crit}\approx 2.11$ 
،است
($\approx2.26$
اگر که 
$\gamma_{\ell=2}\approx0.00015$
نیز درنظر گرفته شود) که پیشبینی می‌کند که کُند‌ترین مد 
($\ell=2$)
رفتار فوق میرا باید نشان دهد و تمامی مد‌های
$\ell>2$
باید رفتار زیر میرا نشان دهند. محاسبات مشابه نشان می‌دهد که در ستون
III
شکل
 \ref{fig:autoAnalysis},
 مد بحرانی
$\ell_{crit}\approx 6.68$ ($\approx 7.05$
با در نظر گرفتن
$\gamma_{\ell=7}\approx0.0011$)
است. همانطور که مشاهده می‌کنید، مطالعات تئوری تغییر رفتار از فوق میرا به زیر میرای مُدها را با دقت بسیار خوبی پیش‌بینی می‌کنند. 

خط مقطع در ستون آخر با استفاده از معادله‌ی
\ref{eq:BrowninaModeAuto}
رسم شده‌است. در رژیم براونی تمام مد‌ها با نرخ مشابه، نمایی، میرا می‌شوند. این نکته با رسم داده‌ها بر محور زمان با ضریب
$\omega_{\ell,m}^2/\gamma$
نشان داده شده‌است. در محاسبات نظری مربوط به این بخش از ضریب اتلاف داخلی صرف نظر شد زیراکه سه مرتبه‌ی بزرگی از ضریب اتلاف محیط کوچکتر است.

در شکل 
\ref{fig:gammaEll},
از معادله‌ی
\ref{eq:NewtonModesFreq}
برای برازش ضریب اتلاف داخلی بر حسب فرکانس مد‌های مش‌های نرم
$\Phi=0.037$
استفاده شد. از آنجایی که مش‌های 
$\Phi\geq0.15$,
مدول یانگ دو بعدی دارند، از معادله‌ی 
\ref{eq:NewtonModesFreq}
برای محاسبه‌ی فرکانس مد‌ها نمی‌توان استفاده کرد و باید تاثیر مدول یانگ را نیز در نظر گرفت
\begin{equation}
\begin{aligned}
\omega_{\ell,m}(\Phi)^2=&\frac{4\pi}{Mr_0^2}\left[\kappa(\ell+2)(\ell+1)\ell(\ell-1)\right.\\
&\left.+Y_{2D}(\Phi)\left(\frac{3(\ell^2+\ell-2)}{3(\ell^2+\ell)-2}\right)\right].
\end{aligned}
\label{eq:YoungFreq}
\end{equation}
در این معادله، مقدار مدول یانگ از معادله‌ی
\ref{eq:YoungPhi}
محاسبه شده‌است.



\begin{figure}[h]
\begin{center}
\includegraphics[width=\columnwidth]{\MemRes/Pics/shape_deformation_evolution.pdf}
\caption{
تغییرات زمانی انرژی انحنای مش‌های بیضوی‌ پَخ (نشانه‌های پُر) و کشیده (نشانه‌های تو خالی) با نرمی
$\Phi\approx0.037$. 
در نتیجه‌ی انجام شبیه‌سازی به روش دینامیک مش، تغییرات بزرگ مقیاس در کسری از دوره‌ی تناوب طول موج بلند آن اتفاق می‌افتد.
}
\label{fig:deformationEvo}
\end{center}
\end{figure}

در شکل
\ref{fig:deformationEvo},
تغییرات زمانی انرژی انحنای مش‌هایی که دارای حجم کاهیده‌ی 
$\nu\le1$
می‌باشند، رسم شده‌است. تغییرات انرژی انحنا ملاک خوبی برای اندازه‌گیری زمان لازم برای تغییر شکل مش است. انرژی اولیه مش‌هایی که حجم کاهیده‌ی
$\nu=1$
دارند،
$E_b=8\pi\kappa = 502.6 k_BT$
است. پس از اینکه 
$N=1002$
درجات آزادی مد‌های روی سطح مش انرژی حرارتی
$k_BT/2$
را جذب می‌کنند، انرژی کل مش حدود دو برابر می‌شود. برای مش‌هایی که حجم کاهیده‌ی آن‌ها کمتر از واحد است، انرژی اولیه شامل هزینه‌ی بالاتر شکل بیضوی پَخ یا کشیده اولیه است و علاوه بر جذب حرارت محیط هزینه‌ی انرژی انحنای شکل نهایی (همانطور که در شکل
\ref{fig:nuShapes}
نشان داده شد) را نیز شامل می‌شود. در شکل 
\ref{fig:deformationEvo},
محور زمانی بر حسب زمان لازم برای تکمیل کُندترین دوره‌ی تناوب مد‌های سطحی 
$\tau_{2,m}$
رسم شده‌است. همانطور که در نمودار مشخص است، در تمام شبیه‌سازی‌ها، تغییر شکل بسیار سریع و در کسری از این زمان تناوب اتفاق می‌افتد. برخلاف روش‌ مانتی کارلو دینامیک مثلثی، که برای ایجاد تغییر شکل مش، هم چیدمان نقاط و هم اتصالات میانشان باید تغییر کند، هنگام شبیه سازی دینامیک مش، حرکتِ درون صفحه‌ایِ نقاط، زمان اتفاق  تغییر شکل مش را افزایش نمی‌دهند.

















