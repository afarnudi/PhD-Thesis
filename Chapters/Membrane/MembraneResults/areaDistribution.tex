به طور متوسط شکل مثلث میانگین بر روی مش یک مثلث متساوی الاضلاع به ضلع
$b$
و مساحت 
 $\bar a_{iso}$
 است. مساحت این مثلث برابر است با
 \begin{equation}
\bar a_{iso}=\frac{1}{2}b^2\sin(\pi/3)=\frac{4\pi r_0^2}{2N}.
\label{eq:averageTriArea}
\end{equation}
 با بازنویسی معادله‌ی فوق می‌توان اندازه‌ی طول ضلع مثلث را محاسبه کرد،
 \begin{equation}
b=r_0\sqrt{\frac{4\pi}{N\sin(\pi/3)}}.
\label{eq:averageTriArea}
\end{equation}
 
هنگام استفاده از الگوریتم توزیع مساحت، نقاط می‌توانند بر روی صفحه حرکت کنند. در نتیجه ممکن است مناطقی تشکیل شود که از مثلث‌های بسیار کوچکی ساخته شده باشند. مساحت مثلث‌های کوچک
$\bar a_{min}$
تابع کوچکترین ارتفاع 
$h$
است که اندازه‌ی آن توسط پتانسیل
$U_h$
تعیین می‌شود. مساحت این مثلث‌ها 
\begin{equation}
\bar a_{min}=\frac{1}{2}\frac{d_h^2}{\sin(\pi/3)}.
\label{eq:aMin}
\end{equation}
خواهد بود.


$\Phi$
را نسبت مساحت کوچکترین مثلث درون مش به مساحت مثلث متوسط مش تعریف می‌کنیم،
\begin{equation}
\Phi=\frac{\bar a_{min}}{\bar a_{iso}}=\frac{d_h^2}{r_0^2}\frac{N}{4\pi\sin(\pi/3)}
\label{eq:phiDef}
\end{equation}



\begin{figure}[htbp]
\begin{center}
\includegraphics[width=12cm]{\MemRes/Pics/ULM_Fluidity.pdf}
\caption{
رفتار توزیع نقاط و تابع توزیع احتمال مساحت مثلث‌ها با تغییر نسبت مساحتی 
$\Phi$
بررسی شده‌است. نمودار بالا دامنه‌ی مد‌های جرمی هماهنگ‌های کروی بر روی مش تقریبا کروی که با پتانسیل سطح، حجم، و 
$U_h$
با حجم کاهیده‌ی 
$\nu=1$
را نشان می‌دهد. نقاط مشکی دامنه‌ی مد توزیع تصادفی جرم بر روی سطح کره را نشان داده (گاز ایده‌آل دو بُعدی). نمودار پایین تابع توزیع احتمال مساحت مثلث‌ها نسبت به مساحت متوسط مثلث‌ها را رسم کرده‌است.
}
\label{fig:vertexULM}
\end{center}
\end{figure}

پتانسیل
$U_h$
حد کمینه‌ی  فاصله‌ی نقاط از اضلاع پلاکت را در مش‌های درهم یا نرم تعیین می‌کند. می‌دانیم که در حد 
$d_h$
بزرگ، نقاط فضای زیادی برای جابجایی نخواهند داشت و مش به حالت کریستالی در می‌آید. با بررسی تبدیل فوریه مکان نقاط بر سطح کره می‌توان وجود نظم در آرایش توزیع نقاط بر سطح مش پی برد. 

غشا با انرژی مساحت، حجم، و 
$U_h$
با حجم کاهیده‌ی 
$\nu=1$
با دینامیک لانژونی شبیه‌سازی شد. با تنظیم ارتفاع کمینه‌ی مثلث‌ها، 
$d_h$
، دامنه‌ی مد‌های هارمونیک کروی مختصات نقاط بر روی مش اندازه‌گیری شد (شکل
\ref{fig:vertexULM}
بالا).  سطح کره مشبک سازی شد و تعداد نقاطی که درون هر شبکه‌ قرار گرفت، شمرده شد. سپس تبدیل فوریه در مختصات کروی بر مختصات عناصر  و وزن آن محاسبه‌ شد. این کار مشخص خواهد کرد که آیا توزیع نقاط بر سطح کره آرایش منظمی دارد یا خیر. این محاسبه برای توزیع تصادفی از نقاط بر سطح کره (گاز ایده‌آل دو بعدی) نیز تکرار شد و حاصل آن با نقاط مشکی رنگ در شکل
\ref{fig:vertexULM}
رسم شد. 


برای نسبت مساحت 
$\Phi$
بالا کریستال شدن نقاط غشا به وضوح مشخص است. با کم کردن نسبت مساحت نظم ساختاری کم‌تر و کم‌تر می‌شود. با وضوح اندازه‌گیری استفاده شده، نسبت مساحتی 
$\Phi=0.037$
بزرگترین نسبتی است که در آن اثری از نظم دیده نمی‌شود. البته که به علت وجود شبکه‌ی مثلثی و اتصالات تعریف شده میان نقاط، در فاصله‌های بزرگ (عدد مد‌های کوچک) تاثیر قید دیده می‌شود. 
%با استفاده از  معادله‌ی
%\ref{eq:phiTimeScale}
%می‌توان برای غشایی که در آن 
%$\kappa=20k_BT$
%نسبت مساحتی را 
%$\phi\approx0.057$
%تخمین زد. 
%در نتیجه مقدار محاسبه شده برای نسبت مساحتی در این بخش برای استفاده با انرژی انحنا مناسب است. 

باید به این نکته توجه کرد که در نسبت‌های
$\Phi$
بالا نقاط غشا آزادی حرکت ندارند، اما از طرفی می‌توان تصور کرد که جابجایی نقاط در برا در برابر تنش‌ برشی خارجی رفتار قابل پیش‌بینی نخواهد بود. چون در این رساله انگیزه‌‌ای برای مطالعه فیزیک این سیستم وجود نداشت، اندازه‌گیری  در این مورد انجام نشده.


در  نمودار پایینی شکل 
\ref{fig:vertexULM}
 تابع توزیع نرمال شده‌ی مساحت مثلث‌ها رامشاهده می‌کنید. نرمال شدن با تقسیم داده‌ها بر مساحت مثلث متوسط انجام شده‌است. همانطور که انتظار می‌رود برای حالت کریستالی بیشتر مثلث‌ها مساحت متوسط دارند. ولی با کاهش نسبت مساحتی،‌ اثر توزیع دینامیک مساحت دیده می‌شود. به نظر می‌رسد که حاصل توزیع دینامیک مساحت  ایجاد تعداد زیادی مثلث با مساحت کوچک و تعداد محدودی مثلث با اندازه‌ی بزرگ است. همچنین به نظر می‌رسد که بیشترین مساحت محتمل در مش  توسط 
 $\Phi$
  تعیین می‌شود. می‌توان تصور کرد که در عدم حضور پتانسیل 
$U_h$
توزیعی با مقدار بیشینه در مساحت‌های خیلی کوچک دیده‌ خواهد شد.

\begin{figure}[htbp]
\begin{center}
\includegraphics[width=10cm]{\MemRes/Pics/area_relaxation.pdf}
\caption{
نمودار بالا: دینامیک تغییر تابع نرمال توزیع احتمال مساحت مثلث‌ها برای مش‌های  تصادفی ۱۰۰۲نقطه‌ای (۲۰۰۰ مثلث). مشی که در ابتدا معمولی است به سرعت به مش درهم تبدیل می‌شود. نمودار پایین: تابع خود همبستگی مساحت مثلث‌های مش برای مقادیر مختلف نسبت مساحتی
$\Phi$
رسم شده‌است. هر چه 
$\Phi$
بزرگتر باشد، توزیع تعادلی مساحت به توزیع اولیه‌ی مش نزدیکتر بوده، در نتیجه زمان از دست دادن همبستگی سریع‌تر خواهد بود.
}
\label{fig:areaRelaxation}
\end{center}
\end{figure}

حالا می‌خواهیم دینامیک تغییر توزیع مساحت‌ها را بررسی کنیم. در نتیجه مش‌ها تصادفی مختلف با پتانسیل مساحت، حجم، و
$U_h$
با حجم کاهیده‌ی 
$\nu=1$
در نظر گرفته شد. سرعت اولیه‌ی نقاط را از توزیع بولتزمن با دمای
$k_BT$
انتخاب کرده و سیستم با دینامیک نیوتونی متحول شد. در زمان اولیه تمام مثلث‌ها کم و بیش یک مساحت دارند (با رنگ مشکی) و با گذشت زمان توزیع مساحت‌ها به تعادل می‌رسد. زمان این دینامیک بر اساس زمان مشخصه‌ی پتانسیل لنارد جونز گزارش شده است. می‌دانیم که زمان مشخصه‌ی لنارد جونز برای واحد‌های معرفی شده در قسمت شبیه‌سازی
$\tau_h=l\sqrt{m/\epsilon}$
است. در نمودار بالا شکل
\ref{fig:areaRelaxation}
مشاهده می‌کنیم که این روش دینامیک توزیع مثلث در زمان بسیار کوتاهی، حدود
$\Delta t \approx6\tau_h$
توزیع مساحت‌ها را به توزیع تعادلی می‌رساند. همچنین زمان همبستگی نیز برای مساحت مثلث‌ها برای 
$\Phi$
های مختلف محاسبه‌ شد (نمودار پایین در شکل
\ref{fig:areaRelaxation}
). از آنجایی که هر چه
$\Phi$
بزرگتر باشد، توزیع تعادلی مثلث‌ها به توزیع اولیه مش‌های تصادفی نزدیک‌تر است، زمان واحلش مساحت‌ها نیز کوتاه‌تر خواهد بود.


