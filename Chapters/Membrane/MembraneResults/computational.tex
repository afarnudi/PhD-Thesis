طبیعی‌ است که شبیه‌سازی‌های درشت دانه‌ی غشا که با دینامیک مش انجام می‌شود بسیار سریع‌تر از مدل‌های تمام اتم است. شبیه‌سازی دینامیک مش، ماهیت خاصی دارد که اجازه‌ می‌دهد در موتور‌های دینامیک ملکولی پیاده سازی شود و در نتیجه از منابع کامپیوتری به خوبی استفاده کند. ما باور داریم که این خاصیت باعث می‌شود که در میان مدل‌های درشت دانه نیز این روش به لحاظ محاسباتی بسیار سریع باشد.

هزینه محاسباتی لازم برای نمونه برداری را می‌توان با نسبت طولانی ترین زمان واحلش و قدم زمانی شبیه‌سازی اندازه‌گیری کرد. برای دینامیک مش زیر میرا این رابطه
\begin{equation}
\frac{\tau_{2,m}}{\sqrt{\Phi} \tau_{\ell_{max},m}} \propto \frac{N}{\sqrt{\Phi}}\ ,
\end{equation}
است، در حالی که برای شبیه‌سازی فوق میرا با ضریب اتلاف ثابت
$\gamma= \left(\sqrt{\Phi} \tau_{\ell_{max},m}\right)^{-1}$
این زمان به این شکل افزایش می‌یابد
\begin{equation}
\frac{\gamma \tau_{2,m}^2}{\sqrt{\Phi} \tau_{\ell_{max},m}} = \frac{\tau_{2,m}^2}{\left(\sqrt{\Phi} \tau_{\ell_{max},m}\right)^2} \propto \frac{N^2}{\Phi}\ .
\end{equation}

زمان نسبی که در سمت راست معادله‌ی بالا نوشته شده، زمان لازم برای تکمیل تمام محاسبات جهت پیش‌روی به قدم زمانی شبیه‌سازی بعدی است. این زمان برای محاسبه‌ی نیروها و انرژی انحنا به شکل خطی با تعداد نقاط مش رشد می‌کند. ولی جهت تنظیم مساحت و حجم مش، هزینه‌ی زمانی لازم برای تکمیل هر قدم شبیه‌سازی با نرخ توان دوم تعداد ذرات افزایش می‌یابد.

از آنجایی که دینامیک مش با شبیه‌سازی‌های استاندارد دینامیک ملکولی تطابق دارد، می‌توان از کار عظیمی که در این شاخه جهت بهینه‌سازی محاسباتی بر زیرساخت‌های کامپیوتری انجام شده، بهره برد. برای مثال در این رساله تمامی داده‌های شبیه‌سازی بر 
GPU
انجام شده‌است. نکته‌ی قابل توجه این است که نرم افزار طراحی شده برای این کار، یعنی نرم افزار مدل سلول مجازی
VCM
بدون زحمت بیشتر از سمت کاربر، تمام محاسبات را از طریق موتور محاسباتی
OpenMM
به شکل موازی بر زیرساخت
GPU
انجام می‌دهد.

ما شبیه سازی‌های دینامیک مش را بر دو 
GPU
که بر دو سامانه‌ی مشابه سوار بودند، اندازه‌گیری نمودیم.
\\
\begin{table}[h!]
\centering
 \begin{tabular}{|| c | c | c | c  | c | c ||} 
 \hline
GPU &~cores~&~Freq (base)~&~Memmory~&power&~Price~\\

 \hline\hline
1 & ~$3,000$~ & ~$1.6~GHz$~& ~$8~GB$~&~250W~&~$\sim\$700$~\\ 
2 & ~$16,000$~ & ~$2.2~GHz$~& ~$24~GB$~&~450W~&~$\sim\$2000$~\\ 
 \hline
 \end{tabular}
\end{table}
\\
شبیه‌سازی‌های غشاهای ما حافظه‌ی زیادی نیاز ندارند
($<250MB$)
زیراکه تمام رفتار غشا با حدود
$\sim10^3$
نقطه قابل شبیه‌سازی‌است. زمان واقعی شبیه‌سازی غشایی با 
$N$
نقطه که به مدت 
$10\tau_{2,0}$ 
شبیه‌سازی شده و هر
$1\tau_{2,0}$
یک بار نمونه برداری انجام شده‌است (یعنی در این قدم شبیه‌سازی مختل شده و با اطلاعات لازم مانند، مختضات نقاط، سرعت‌ها، ... بر حافظه‌ی دائمی سامانه ضبط شده‌است) 
\begin{table}[h!]
\centering
 \begin{tabular}{||  c  |    c   |   c      |  c  |    c      ||} 
 \hline
 \multirow{2}{*}{N} &
      \multicolumn{2}{c|}{GPU 1} &
      \multicolumn{2}{c||}{GPU 2} \\
% N & GPU 1 & GPU 2  \\
%\hline 
&runtime&consumption&runtime&consumption\\ 
 \hline\hline
252&12 sec&$8.3\times10^{-4}$kWh&10 sec&$1.25\times10^{-3}$kWh\\ 
1002& 3.83 min~&~$1.6\times10^{-2}$kWh&3.66 min&$2.75\times10^{-2}$kWh\\ 
 4002&110 min&$0.46$~kWh&72 min&$0.54$~kWh\\
 \hline
 \end{tabular}
\end{table}
\\
برای اینکه ارزش این اعداد بیشتر مشخص شود باید یادآوری شود که این زمان مربوط به تهیه‌ی یک نمونه‌ مستقل است که به تعادل ترمودینامیکی رسیده است. به عبارت دیگر، زمان لازم برای مطالعه‌ی تغییر شکل بزرگ‌مقیاس مش است.





