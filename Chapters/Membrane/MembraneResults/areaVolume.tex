\begin{figure}[tbp]
\begin{center}
\includegraphics[width=\columnwidth]{\MemRes/Pics/UnitSphereAll}
\caption{
مساحت (ستون چپ) و حجم (ستون وسط) برای چهار نوع مِش کُروی بر حسب تعداد نقاط روی مِش. مساحت کل، به دو روش، جمع مساحت‌های ورنوی (نقاط بنفش) و جمع مساحت‌های بریسنتریک (نقاط خاکستری) محاسبه شده‌است. در اینجا نیجه‌ی تقسیم مساحت مِش‌ بر مساحت کره رسم شده‌است. دقت در اندازه‌گیری مساحت و حجم (نقاط آبی) برای تمام مِش‌ها با افزایش تعداد نقاط روی مش، افزایش می‌یابد. نمونه تصویری از مِش کنار هر ردیف رسم شده‌است. انرژی انحنا چهار مدل گامپر (قرمز)، یولیشر (سبز)، گامپر-بریسنتریک (آبی)، و یولیشر-ورنوی (نارنجی) برای مِش‌های مختلف در ستون سمت راست گزارش شده‌است. به غیر از مش‌های درهم تصادفی، انرژی انحنا برای تمامی مش‌ها با افزایش تعداد نقاط شبکه بهتر شده. به غیر از مش منظم (که تنها یک نمونه از آن برای هر تعداد نقطه وجود دارد) مقادیر محاسبه شده حاصل از میانگین گیری بر روی ۵۰ نمونه‌ی مستقل است.
}
\label{fig:unitsphereAll}
\end{center}
\end{figure}

دقت اندازه‌گیری مساحت و حجم در مِش‌های کروی مختلف، بر حسب تعداد نقاطِ مِش، در شکل
\ref{fig:unitsphereAll}
بررسی شده‌است. نمونه تصویری از مِش کنار هر ردیف رسم شده‌است. مساحت مِش‌ها (ستون سمت چپ) به دو روش جمع  مساحت‌ وُرُنُی رئوس (نقاط بنفش) و جمع مساحت بریسنتریک (نقاط خاکستری) محاسبه شده‌است. نتیجه‌ی محاسبات بر مساحت 
$4\pi r_0^2$
 تقسیم شده‌است. همانطور که مشاهده می‌شود دقت اندازه‌گیری برای تمامی مِش‌ها با افزایش تعداد نقاط بهتر می‌شود. لازم به ذکر است  از آنجایی که تمام نقاط مش‌ها روی سطح پوسته‌ی کُروی قرار دارد، یا به عبارت دیگر مِش بر داخل یک کره‌ مماس است، مساحت و حجم آن همیشه از کره کمتر خواهد بود.




مانند مساحت، حجم (ستون وسط) نیز برای تمامی مش‌ها با دقت بسیار خوبی قابل اندازه‌گیری‌است. نتایج ثبت شده در ستون‌های سمت چپ و وسط نشان‌ می‌دهد که تمامی‌ مِش‌ها برای اندازه‌گیری مساحت و حجم کره مناسب هستند. از آنجایی که غشا‌ها اشکال پیچیده‌تری نسبت به کره دارند، صحت محاسبات برای اشکالی به غیر از کره نیز باید بررسی شود. با اضافه کردن جمله‌ی هارمونیک کروی به مکان شعاعی تمام نقاط مش، می‌توان شکل مش را تغییر داد،
\begin{eqnarray}
r(\theta,\phi)=r_0+r_0|u_{\ell,m}|Y_{\ell,m}(\theta,\phi).
\label{eq:rDeformed}
\end{eqnarray}
با انتخاب
$\ell=2$
و
$m=0$
برای هماهنگ‌ کروی، می‌توان مش‌های دمبلی شکل  تولید کرد. میزان تغییر شکل به شدت مُد
$u_{2,0}$
بستگی دارد. هزینه‌ی انرژی تغییر مساحت یک کره به یک شکل دمبلی طبق معادله‌ی
\ref{eq:AreaGLFluctuationAmplitude}
قابل محاسبه ‌است،
\begin{equation}
E_A=\frac{2}{\pi}|u_{2,0}|^4,
\label{eq:AreaEnergyULM20}
\end{equation}
در معادله‌ی فوق 
$k_A=1[\varepsilon/l^2]$
و
$r_0=1[l]$
فرض شده‌است. نیروی بازگرداننده که با این تغییر شکل مقاومت می‌کند با مشتق گیری از انرژی در فضای مُد قابل محاسبه‌ می‌باشد، 
\begin{equation}
-\frac{\partial E_A}{\partial u_{2,0}}=-\frac{8}{\pi}|u_{2,0}|^3,
\label{eq:AreaForceULM20}
\end{equation}


به همین ترتیب با جایگذاری 
$k_V=1[\varepsilon/l^3]$
در معادله‌ی
\ref{eq:VolumeGLFluctuationAmplitude}
می‌توان انرژی تغییر حجم مش زمانی که از  شکل کره به شکل دمبلی تغییر می‌کند را بر حسب شدت مُد محاسبه کرد،
\begin{equation}
E_V= \frac{3}{8\pi}|u_{2,0}|^4.
\label{eq:VolumeEnergyULM20}
\end{equation}

با مشتق ‌گیری نسبت به شدت مُد، مقدار نیرویی که با تغییر شکل مش مقاومت می‌کند را می‌توان بر حسب شدت مد محاسبه کرد،
\begin{equation}
-\frac{\partial E_V}{\partial u_{2,0}}= -\frac{3}{2\pi}|u_{2,0}|^3,
\label{eq:VolumeForceULM20}
\end{equation}

\begin{figure}[tbp]
\begin{center}
\includegraphics[width=\columnwidth]{\MemRes/Pics/UnitSphereDeformation_mesh_30_All.pdf}
\caption{
انرژی مساحت (ستون 
$(a)$)
 و حجم (ستون 
$(b)$)
 محاسبه‌ شده حاصل از تغییر شکل مِش کروی با اضافه شدن مُد
$Y_{2,0}(\theta,\phi)$
با شدت‌های مختلف برای چهار نوع مِشِ مختلف. انرژی مساحت برای دو روش محاسبه‌ی مساحت کل، جمع مساحت‌های ورنوی (نقاط بنفش) و جمع مساحت‌های بریسنتریک (نقاط خاکستری)، و انرژی حجم (نقاط آبی) با جمع روی حجم هرم‌ها محاسبه شده‌است. خطوط مشکی در ستون‌های 
$(a)$
و
$(b)$
به ترتیب معادلات
\ref{eq:AreaEnergyULM20}
و
\ref{eq:VolumeEnergyULM20}
را نشان می‌دهند. انرژی انحنا محاسبه‌ شده حاصل از تغییر شکل برای چهار مدل گامپر (قرمز)، یولیشر (سبز)، گامپر-بریسنتریک (آبی)، و یولیشر-ورنوی (نارنجی) روی مش‌های مختلف در ستون
$c$
 رسم شده‌است. خطوط مشکی پیش بینی
 مرتبه‌ی دوم انرژی انحنا ، یعنی معادله
\ref{eq:curvatureY20}
را نشان می‌دهند. نیروی بازگرداننده برای تغییر مساحت، حجم، و انحنا در ستون‌های 
$(d)$, $(e)$,
و
$(f)$
رسم شده‌است.  پیش‌بینی‌ مرتبه‌ی دوم نیروی بازگرداننده برای چنین تغییراتی با توجه به معادلات
\ref{eq:AreaForceULM20}, \ref{eq:VolumeForceULM20}, 
و
\ref{eq:curvatureForceY20}
با خط مشکی رسم شده است. شدت مُد 
$u_{2,0}=0$
مربوط به شکل کاملا کروی و شدت مُد 
$u_{2,0}=1$
مربوط به شکل دمبلی (عکس‌های بالا) است. یک نمونه از هر مش برای حالت کروی و دمبلی در ردیف مربوطه رسم شده‌است. به غیر از مِش منظم (که تنها یک نمونه از آن برای هر تعداد نقطه وجود دارد) مقادیر محاسبه شده حاصل از میانگین گیری بر روی ۵۰ نمونه‌ی مستقل ‌است.
}
\label{fig:AllULM20}
\end{center}
\end{figure}


در شکل
\ref{fig:AllULM20}
نتیجه‌ی محاسبات عددی در کنار پیش‌بینی نظری برای انرژی و نیروی حاصل از تغییر شکل، رسم شده است. شدت مُد
$u_{2,0}=0$
مربوط به مِش کاملا کروی است و شدت مُد 
$u_{2,0}=1$
مربوط به مِش‌های دمبلی شکل است (عکس‌های بالا). هر ردیف، محاسبات را برای یک نوع مِش نشان می‌دهد. نمونه‌ عکسی از مِش‌ها در بالای شکل نمایش داده‌ شده‌است. به ترتیب در ستون‌های 
$(a)$, $(b)$, $(d)$, 
 و
$(e)$
تغییر انرژی مساحت، انرژی حجم، نیروی مساحت، و نیروی حجم به صورت نقاط بنفش (ورنوی) و خاکستری (بریسنتریک) برای مساحت و نقاط آبی برای حجم به همراه پیش‌بینی استخراج شده از محاسبات افت و خیز (خط مشکی) رسم شده‌است. 

داده‌های شکل 
\ref{fig:AllULM20}
نشان می‌دهد که رفتار انرژی و نیرو برای تغییر شکل‌های کوچک با پیش‌بینی محاسبات افت و خیز همخوانی دارد. از آنجایی که محاسبات افت و خیز برای شدت مد‌های کوچک و تا مرتبه‌ی دوم در 
$u_{\ell,m}$
درنظر گرفته شده‌است برای شدت‌ مد‌های بزرگ فاقد اعتبار است. ولی رفتار کلی محاسبات عددی و معادلات
\ref{eq:AreaEnergyULM20}, \ref{eq:VolumeEnergyULM20}, \ref{eq:AreaForceULM20}, 
 و
\ref{eq:VolumeForceULM20}
برای شد‌ت مُد 
$|u_{\ell,m}|>0.5$
نیز همخوانی دارد. می‌توان نتیجه گرفت که محاسبات لازم برای اندازگیری مساحت و حجم غشا برای شکل‌های کروی و همچنین شکل‌های غیر کروی حاوی انحنای زین اسبی بر روی مش‌های معمولی  و  درهم به خوبی قابل پیاده‌سازی است.










