\begin{figure}[th]
\begin{center}
\includegraphics[width=12cm]{\MemRes/Pics/UnitSphereAreaVolume}
\caption{
مساحت (ستون چپ) و حجم (ستون راست) محاسبه‌ شده برای چهار نوع مش‌های کُروی بر اساس تعداد نقاط روی مش رسم شده‌است. مساحت کل، به دو روش، با جمع مساحت‌های ورنوی (نقاط بنفش) و جمع مساحت‌های بریسنتریک (نقاط خاکستری) محاسبه شده‌است. دقت در اندازه‌گیری مساحت و حجم (نقاط آبی) برای تمام مش‌ها با افزایش تعداد نقاط روی مش، افزایش می‌یابد. تصویر مش نمونه از هر نوع مش استفاده شده در محاسبات کنار هر ردیف رسم شده‌است. به غیر از مش منظم (که تنها یک نمونه از آن برای هر تعداد نقطه وجود دارد) مقادیر محاسبه شده حاصل از میان‌گین گیری بر روی ۵۰ نمونه‌ی مستقل انجام شده‌است.
}
\label{fig:unitsphereAreaVolume}
\end{center}
\end{figure}
جهت بررسی دقت اندازه‌گیری مساحت و حجم برای مش‌های مختلف، مش‌های کروی با تعداد نقاط مختلف انتخاب شد. نتایج محاسبات در شکل
\ref{fig:unitsphereAreaVolume}
رسم شده‌است. مساحت مش‌ها به دو روش انجام شد، یکی با جمع کردن مساحت‌ وُرُنُی رئوس (نقاط بنفش) و درگری با جمع کردن مساحت بریسنتریک (نقاط خاکستری). با تقسیم نتیجه‌ی محاسبات بر مساحت کُره در حالت پیوسته،
$4\pi r_0^2$
میزان دقت در اندازه‌گیری بر حسب تعداد نقاط تشکیل دهنده‌ی مش‌های مختلف قابل مقایسه است. همانطور که می‌بینید دقت اندازه‌گیری برای تمامی مش‌ها با افزایش تعداد نقاط بهتر می‌شود. لازم به ذکر است که از آنجایی که تمام نقاط مش‌ها بر پوسته‌ی کُروی قرار دارد، یا به عبارت دیگر بر داخل یک کره‌ مماس هستند، مساحت و حجم آن همیشه از کره کمتر خواهد بود.

\begin{figure}[th]
\begin{center}
\includegraphics[width=12cm]{\MemRes/Pics/UnitSphereDeformation_mesh_10_Areavolume}
\caption{
انرژی مساحت (ستون 
$(a)$
) و حجم (ستون 
$(b)$
) محاسبه‌ شده حاصل از تغییر شکل مش کروی با اضافه شدن مد
$Y_{2,0}(\theta,\phi)$
با شدت‌های مختلف برای چهار نوع مش رسم شده‌است. انرژی مساحت برای دو روش محاسبه‌ی مساحت کل، جمع مساحت‌های ورنوی (نقاط بنفش) و جمع مساحت‌های بریسنتریک (نقاط خاکستری)، و انرژی حجم (نقاط آبی) با جمع روی حجم هرم‌ها محاسبه شده‌است. خطوط مشکی برای ستون‌های 
$(a)$
و
$(b)$
به ترتیب معادلات
\ref{eq:AreaEnergyULM20}
و
\ref{eq:VolumeEnergyULM20}
را رسم می‌کنند. نیروی بازگرداننده برای تغییر مساحت و حجم در ستون‌های 
$(c)$
و
$(d)$
رسم شده.  پیش‌بینی‌ مرتبه‌ی دوم نیروی بازگرداننده برای تغییر مساحت وحجم با توجه به معادلات
\ref{eq:AreaForceULM20}
و
\ref{eq:VolumeForceULM20}
با خط مشکی رسم شده است. شدت مد 
$u_{2,0}=0$
مربوط به شکل کاملا کروی (عکس سمت چپ) و شدت مد 
$u_{2,0}=1$
مربوط به شکل دمبلی (عکس سمت راست) است. یک نمونه از هر مش برای حالت کروی و دمبلی در ردیف مربوته رسم شده‌است. به غیر از مش منظم (که تنها یک نمونه از آن برای هر تعداد نقطه وجود دارد) مقادیر محاسبه شده حاصل از میان‌گین گیری بر روی ۵۰ نمونه‌ی مستقل انجام شده‌است.
}
\label{fig:unitsphereAreaVolumeULM20}
\end{center}
\end{figure}


مشابه به مساحت، حجم نیز برای تمامی مش‌ها با دقت بسیار خوبی قابل اندازه‌گیری‌است. در نتیجه مش‌های درهم برای اندازه‌گیری مساحت و حجم کره مناسب هستند. از آنجایی که غشا‌ها اشکال پیچیده‌تری نسبت به کره دارند، صحت محاسبات برای اشکالی به غیر از کره نیز باید بررسی شود. با اضافه کردن جمله‌ی هارمونیک کروی به مکان شعاعی تمام نقاط مش، می‌توان شکل مش را تغییر داد،
\begin{eqnarray}
r(\theta,\phi)=r_0+r_0|u_{\ell,m}|Y_{\ell,m}(\theta,\phi).
\label{eq:rDeformed}
\end{eqnarray}
با اضافه کردن مد هارمونیک کروی
$Y_{2,0}(\theta,\phi)$
به مختصات نقاط، مش‌ها به شکل دمبلی درمی‌آیند. میزان تغییر شکل به شدت مُد
$u_{2,0}$
بستگی دارد. هزینه‌ی تغییر مساحت یک کره به یک شکل دمبلی طبق معادله‌ی
\ref{eq:AreaGLFluctuationAmplitude}
قابل محاسبه ‌است،
\begin{equation}
E_A=\frac{2}{\pi}|u_{2,0}|^4,
\label{eq:AreaEnergyULM20}
\end{equation}
در معادله‌ی فوق 
$k_A=1[\varepsilon/l^2]$
و
$r_0=1[l]$
فرض شده‌است. نیروی بازگرداننده که با این تغییر شکل مقاومت می‌کند با مشتق گیری از انرژی در فضای مُد قابل محاسبه‌ می‌باشد، 
\begin{equation}
-\frac{\partial E_A}{\partial u_{2,0}}=-\frac{8}{\pi}|u_{2,0}|^3,
\label{eq:AreaForceULM20}
\end{equation}


به همین ترتیب با جایگذاری 
$k_V=1[\varepsilon/l^3]$
در معادله‌ی
\ref{eq:VolumeGLFluctuationAmplitude}
می‌توان انرژی تغییر حجم مش زمانی که از  شکل کره به شکل دمبلی تغییر می‌کند را بر حسب شدت مُد محاسبه کرد،
\begin{equation}
E_V= \frac{3}{8\pi}|u_{2,0}|^4,
\label{eq:VolumeEnergyULM20}
\end{equation}
و با مشتق ‌گیری نسبت به شدت مُد، مقدار نیرویی که با تغییر شکل مش مقاومت می‌کند را بر حسب شدت مد محاسبه کرد،
\begin{equation}
-\frac{\partial E_V}{\partial u_{2,0}}= -\frac{3}{2\pi}|u_{2,0}|^3,
\label{eq:VolumeForceULM20}
\end{equation}
در شکل
\ref{fig:unitsphereAreaVolumeULM20}
نتیجه‌ی محاسبات عددی در کنار پیش‌بینی نظری برای انرژی و نیروی حاصل از تغییر شکل رسم شده است. شدت مد
$u_{2,0}=0$
مربوط به مش کاملا کروی است (عکس‌های سمت چپ) و شدت مد 
$u_{2,0}=1$
مربوط به مش‌های دمبلی شکل است (عکس‌های سمت راست). هر ردیف محاسبات را برای یک نوع مش نشان می‌دهد که به صورت یک عکس از مش مشخص شده‌است. به ترتیب در ستون‌های 
$(a)$
،
$(b)$
،
$(c)$
، و
$(d)$
تغییر انرژی مساحت، انرژی حجم، نیروی مساحت، و نیروی حجم به صورت نقاط بنفش (ورنوی) و خاکستری (بریسنتریک) برای مساحت و نقاط آبی برای حجم به همراه پیش‌بینی استخراج شده از محاسبات افت و خیز (خط مشکی) رسم شده‌است. داده‌های شکل 
\ref{fig:unitsphereAreaVolumeULM20}
نشان می‌دهد که رفتار انرژی و نیرو برای تغییر شکل‌های کوچک با پیش‌بینی مستخرج از محاسبات افت و خیز همخوانی دارد. از آنجایی که محاسبات افت و خیز برای شدت مد‌های کوچک و تا مرتبه‌ی دوم در 
$u_{\ell,m}$
درنظر گرفته شده‌است برای شدت‌ مد‌های بزرگ فاقد اعتبار است. ولی رفتار کلی محاسبات عددی و معادلات
\ref{eq:AreaEnergyULM20}
،
\ref{eq:VolumeEnergyULM20}
،
\ref{eq:AreaForceULM20}
، و
\ref{eq:VolumeForceULM20}
برای شد‌ت مد 
$|u_{\ell,m}|>0.5$
نیز همخوانی دارد. می‌توان نتیجه گرفت که محاسبات لازم برای اندازگیری مساحت و حجم غشا برای شکل‌های کروی و همچنین شکل‌های غیر کروی حاوی انحنای زین اسبی بر روی مش‌های منظم و تصادفی و بر روی مش‌های درهم منظم و درهم تصادفی به خوبی قابل پیاده‌سازی است.










