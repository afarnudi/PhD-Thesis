\begin{figure}[htbp]
\begin{center}
\includegraphics[width=10cm]{\MemRes/Pics/RBCShapes.pdf}
\caption{
تصاویر گرفته شده حاصل از شبیه‌سازی غشا به روش توزیع دینامیک مساحت با حجم‌های کاهیده‌ی مختلف. شکل‌ها با استفاده از مدل انحنای ایتزیکسون (قرمز)، یولیشر (سبز)، و ایتزیکسون بری‌سنتریک (آبی) تولید شده‌است. برای ایجاد پایداری عددی جهت استفاده از مدل گامپر از نسبت مساحتی
$\Phi=0.15$
استفاده شده و در دو مدل دیگر 
$\Phi=0.037$
.
}
\label{fig:LiquidShapes}
\end{center}
\end{figure}

غشا با انرژی مساحت، حجم، و انرژی انحنا با حجم‌های کاهیده‌ی 
$\nu$
مختلف شبیه‌سازی شد. با تغییر اندازه‌ی مش در راستای 
$z$
بیضی‌گون با حجم‌های کاهیده به مقادیر
$0.6,0.4,0.3$
و
$0.25$
جهت شروع شبیه‌سازی استفاده شده. پتانسیل 
$U_h$
با جهت تنظیم نسبت مساحتی
$\Phi=0.037$
برای مدل‌های انحنای گامپر-بریسنتریک (آبی) و یولیشر (سبز) و جهت ایجاد پایداری عددی 
$\Phi=0.15$
برای مدل انحنای گامپر (قرمز) استفاده شد. تصویر نمونه از شبیه‌سازی در شکل
\ref{fig:LiquidShapes}
رسم شده‌است. اشکال تولید شده برای حجم‌های کاهیده (به غیر از 
$\nu=0.25$
) طبق مطالعات گذشتگان بوده
\cite{BIAN2020}
و عملکرد درست الگوریتم توزیع دینامیک مساحت را تایید کرد. در مورد حجم کاهیده‌ی 
$\nu=0.25$
الگوریتم مثلث‌بندی دینامیک دو کره‌ی تو در تو تولید می‌کند که مطابق با مشاهدات آزمایشگاهی است. الگوریتم توزیع دینامیک مساحت به دلیل اینکه حافظه‌ی مش اولیه را حفظ می‌کند توانایی قادر به ایجاد اشکالی که با توپولوژی مش اولیه تفاوت زیاد داشته باشد، نیست. جهت بررسی تغییر شکل‌های شدید پیشنهاد می‌شود نزدیک تغییر شکل‌های کلیدی از یک قدم باز مش‌بندی استفاده شود
\cite{Peskin972JourCompPhys, Thomas1979AIAA, Chimera1986,Drabik2016}
. 





