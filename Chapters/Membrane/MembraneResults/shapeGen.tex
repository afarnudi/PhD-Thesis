\begin{figure}[htbp]
\begin{center}
\includegraphics[width=\columnwidth]{\MemRes/Pics/nus_vs_phi.pdf}
\caption{
تصاویر گرفته شده حاصل از شبیه‌سازی غشا به روش دینامیک مش برای ترکیب‌های مختلف حجم‌های کاهیده‌
 $\nu$
و نرمی مش
$\Phi$.
مساحت تمامی‌ مش‌ها 
$A_0=4\pi (\mu m)^2$
و ضریب سختی خمش آن‌ها
$\kappa=20k_BT$
است. در هر ردیف تغییرات شکل مش ناشی از افزایش تدریجی نرمی مش از مقدار
$\Phi=0.037$
به
$\Phi=0.59$
نشان داده شده‌است. شکل‌هایی که با رنگ سبز مشخص شده‌اند، اشکالی هستند که با اتصالات مش مطابقت دارند. خطوط زرد و قرمز به ترتیب مش‌هایی را نشان می‌دهند که برای نمایش شکل غشا متناظر با آن سختی، نیاز به بازسازی کم و زیاد اتصالات دارند.
}
\label{fig:nuShapes}
\end{center}
\end{figure}

اولین چالش  در استفاده از مش‌های نرم، بررسی اشکال حاصل از شبیه‌سازی غشاهای با حجم‌کاهیده‌ی مختلف است. با این هدف حجم‌های کاهیده‌ میان مقادیر
$\nu \approx 1$
و
 $\nu = 0.1$
را بررسی کردیم. برای تولید نتایج برای شبیه‌سازی مِش‌هایی که کمترین مقدار نرمی را دارند
($\Phi=0.037$)
ابتدا با تبدیل هندسی خطی، مش‌هایی با شکل بیضوی با مساحت‌های یکسان ولی حجم‌های مختلف تولید نموده، سپس با استفاده از روش انتگرال‌گیری لانژون، رفتار زمانی تغییر شکل مش‌ها را بررسی نمودیم. مش‌های بیضوی برای تولید اشکال، با مش‌های سخت مناسب نیستند. در نتیجه برای تولید این اشکال شبیه‌سازی‌های حاصل از مش‌های نرم را با افزایش تدریجی سختی مش ادامه دادیم. اشکال حاصل از این شبیه‌سازی‌ها در شکل
\ref{fig:nuShapes}
نمایش داده‌ شده‌اند. مقادیر میانی حجم کاهیده، شکل‌هایی شبیه به گلبول‌های قرمز را به خود می‌گیرند. از آنجایی که در شبیه‌سازی از پتانسیل‌های دافعه‌ی میان مشی استفاده نکردیم، ذرات مش‌ قادرند مانند شبه  از یکدیگر عبور کنند. درنتیجه اشکال حاصل از حجم‌های کاهیده‌ی بسیار کوچک، غیر فیزیکی به نظر می‌رسند. اشکال تولیدی با استفاده از دینامک مش برای مش‌های نرم (که با رنگ سبز مشخص شده‌اند) با اشکال مشاهده شده با روش‌های دیگر تطابق بسیار خوبی دارد
\cite{Peskin972JourCompPhys, Thomas1979AIAA, Chimera1986, Drabik2016, BIAN2020}.
باید توجه داشت که با روش دینامیک مش علاوه بر داشتن اطلاعات زمانی مربوط به تغییر شکل، همچنان اطلاعات مربوط به افت و خیز سطح را هم خواهیم داشت. این نتایج نشان می‌دهد که بدون نیاز به تغییر اتصالات میان نقاط نیز می‌توان تغییر شکل‌های بسیار گسترده را با مش‌های نرمی که اتصالات کاملا مطابق با شکل نهایی ندارد را تولید کرد. سختی مش بر اشکال نهایی حجم‌های کاهیده‌ی بزرگ، تاثیر قابل ملاحظه‌ای ندارد. هر چقدر که سختی مش بیشتر باشد، اختلاف در اشکال با حجم کاهیده‌ی کوچکتر زودتر اتفاق می‌افتد.

\begin{figure}[h]
\begin{center}
\includegraphics[width=\columnwidth]{\MemRes/Pics/shapeNuEnergy.pdf}
\caption{
تغییرات در انرژی انحنا 
($U_b^{IB}$)
در نمودار بالا و انرژی پتانسیل تکمیلی
($U_h$)
در نمودار پایین برای تغییر مش‌ها از شکل کره
($\nu=1$)
به حجم‌های کاهیده‌ی کمتر. با افزایش سختی مش، انرژی انحنای مش کاهش و بر انرژی پتانسیل تکمیلی افزوده می‌شود.
}
\label{fig:UhvsNu}
\end{center}
\end{figure}


در شکل 
\ref{fig:UhvsNu}
نشان دادیم که اختلاف در تغییر شکل با افزایش انرژی پتانسیل تکمیلی و کاهش انرژی انحنا همراه است.


