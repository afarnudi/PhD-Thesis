\begin{figure}[htbp]
\begin{center}
\includegraphics[width=14cm]{\MemRes/Pics/HybridMeshes.pdf}
\caption{
ردیف اول به ترتیب از سمت چپ به راست مش‌ درشت (سبز)  استفاده شده برای شبیه‌سازی رفتار الاستیک، مش ریز (قرمز) برای شبیه‌سازی رفتار سیالگون، و در آخر دو غشا کنار هم رسم شده‌است. ردیف میانی نیز مش‌ها را به تبع ردیف اول برای 
مدل گلبول قرمز با خمش ذاتی صفر را نمایان کرده و در نهایت در ردیف آخر تصاویر مشابه برای گلبول قرمز با خمش ذاتی 
$C_0=10/r_0$
نشان داده است.
}
\label{fig:hybridMeshes}
\end{center}
\end{figure}

 گلبول قرمز را می‌توان به کمک ۲ مش انجام داد، که یکی رفتار سیال‌گون آن  و دیگری رفتار الاستیکی آن را مدل کند. در گذشته غشای سیال‌گون به کمک مثلث‌بندی دینامیک با روش مانتی کارلو متحول شده و قسمت الاستیک به روش دینامیک ملکولی. با اتصال نقاط شبکه‌ی الاستیک به شبکه‌ی سیالگون می‌توان این دو شبکه را در کنار یکدیگر حرکت داد. 

با استفاده از روش توزیع دینامیک مساحت می‌توان گلبول قرمز را با اتصال یک شبکه‌ی الاستیکی (دارای مدول یانگ) و شبکه‌ی سیال‌ (توزیع دینامیک مساحت) مدل کرد (شکل
\ref{fig:hybridMeshes}
).  در این صورت هر دو شبکه می‌توانند با روش شبیه‌سازی دینامیکی ملکولی حرکت کنند. از آنجایی که شبکه‌ی الاستیکی نماینده‌ی ، شبکه‌ی پلمیری است و اندازه‌ی اتصالات آن نسبت به فاصله‌ی ملکول‌های لیپیدی بسیار بزرگتر است، می‌توان شبکه‌ی درشت‌تری برای مش الاستیک نسبت به مش نماینده رفتار سیال‌گون گلبول انتخاب کرد. در این صورت با انتخاب زیرمجموعه‌ای از رئوس غشای سیال‌گون، می‌توان یک مش مثلثی ساخت. چنین مش‌هایی در ردیف اول شکل
\ref{fig:hybridMeshes}
نشان داده ‌شده‌است. با انتخاب مش با ۱۰۰۲ راس و یک شبکه‌ی ۹۲ راسی برای مش الاستیک، می‌توان مدلی از گلبول قرمز ارائه داد. مدول یانگ گلبول قرمز حدود 
$Y_{2D}=0.003 [\varepsilon/l^2]$
 و سختی خمش آن حدود
$\kappa=120 [\varepsilon]$ 
گزارش شده‌است
\cite{gomppernelson2012}
. در این صورت تمامی خواص خمشی از یک مش و خواص الاستیک از مش دیگری ناشی می‌شود. با انتخاب خمش ذاتی صفر می‌توان شکل معمول گلبول قرمز (شکل 
\ref{fig:hybridMeshes}
ردیف وسط) را باز تولید کرد. با ایجاد خمش ذاتی غیر صفر، می‌توان اشکال پیچیده‌تر همچون ایکینوسایت 
\LTRfootnote{echinocyte}
(شکل
\ref{fig:hybridMeshes}
ردیف آخر) تولید کرد. با اضافه کردن رفتار الاستیک غیر خطی نیز می‌توان تمامی تغییر شکل‌های گلبول قرمز را نیز مطالعه کرد
\cite{Lim2002PNAS, Noguchi2005PNAS}
. همچنین از آنجایی که این یک مدل کاملا دینامیک ملکولی‌ است، مطالعه‌ی افت و خیز سطح و رفتار در میان جریان‌های هیدرودنامیکی امکان پذیر است.






