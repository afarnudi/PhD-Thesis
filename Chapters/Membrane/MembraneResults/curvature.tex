\begin{figure}[htbp]
\begin{center}
\includegraphics[width=10cm]{\MemRes/Pics/UnitSphereCurvatureORW}
\caption{
انرژی انحنا چهار مدل ایتزیکسون (قرمز)، یولیشر (سبز)، ایتزیکسون-بریسنتریک (آبی)، و یولیشر-ورنوی (نارنجی) برای مش‌های مختلف محاسبه شده‌است. به غیر از مش‌های درهم تصادفی، انرژی انحنا برای تمامی مش‌ها با افزایش تعداد نقاط شبکه بهتر شده. به غیر از مش منظم (که تنها یک نمونه از آن برای هر تعداد نقطه وجود دارد) مقادیر محاسبه شده حاصل از میان‌گین گیری بر روی ۵۰ نمونه‌ی مستقل انجام شده‌است.
}
\label{fig:unitsphereBending}
\end{center}
\end{figure}

انرژی یک کره‌ به شعاع یک 
$8\pi\kappa\approx 25.1[\varepsilon],\kappa=1[\varepsilon]$
با جایگذاری 
$C_1=C_2=\frac{1}{R}$
و
$C_0=\frac{2}{R_\infty}=0$
در معادله‌ی 
\ref{eq:BendingEnergy}
قابل محاسبه‌ است. ستون سمت چپ در شکل 
\ref{fig:unitsphereBending}
انرژی حاصل از محاسبه‌ی انحنا با چهار مدل ایتزیکسون (قرمز)، یولیشر (سبز)، ایتزیکسون-بریسنتریک (آبی)، و یولیشر-ورنوی (نارنجی) را نمایش می‌دهد. این مقادیر با انرژی کره یکسان سازی شده‌اند. با نگاه کلی به محور عمودی می‌توان نتیجه گرفت که هر چهار مدل برای تخمین انرژی انحنای غشا مناسب هستند. به این نکته باید توجه کرد که در بدترین حالت ممکن خطای محاسبه‌ی انحنا در شکل 
\ref{fig:unitsphereBending}
مربوط به مش‌های تصادفی درهم است که حدود
$\sim4\%$
است. این خطا در برابر خطای اندازه‌گیری ضریب سختی خمشی غشا‌ها (
$\sim50\%$
) ناچیز است.

برای تمامی مش‌ها دقت اندازه‌گیری انرژی انحنا برای مدل‌های بر پایه‌ی مساحت ورنوی (ایتزیکسون و یولیشر-ورنوی) دقت اندازه‌گیری با افزایش نقاط مش بهتر شد. این رفتار برای محاسبات انجام شده با مدل‌های بر پایه‌ی مساحت بریسنتریک (یولیشر و ایتزیکسون-بریسنتریک) روی مش‌های منظم و تصادفی نیز به همین ترتیب است. درهم کردن مش خطای اندازه‌‌گیری را برای مش با کمترین نقاط (۱۰۰۲ نقطه) را حدود 
$\sim2\%$
کرده. با افزایش تعداد نقاط در مش دو سناریو وجود دارد. در صورتی که مش‌ منظم باشد و نقاط نقص بر روی آن تعداد بسیار کمی داشته باشد (مش منظم ۱۲ نقطه با درجه‌ی ۵ دارد) با افزایش تعداد نقاط روی سطح دقت اندازه‌گیری بهتر می‌شود. اما در صورتی که مش تعداد نقطه‌ی نقص زیادی داشته باشد، با افزایش تعداد نقاط شبکه (و به تبع افزایش تعداد نقاط نقص) خطای اندازه‌گیری افزایش می‌یابد. در واقع اثر درهم کردن با افزایش تعداد نقاط مش قابل جبران است ولی اثر توپولوژی قابل حذف نیست.



\begin{figure}[htbp]
\begin{center}
\includegraphics[width=10cm]{\MemRes/Pics/IJ_numerator.pdf}
\caption{
مقادیر صورت (ستون چپ) و مخرج (ستون راست) مدل گسسته اندازه‌گیری انحنای ایتزیکسون و یولیشر برای نقاط مختلف بر روی مش‌های منظم، منظم درهم، تصادفی، و تصادفی درهم با ۱۰۰۲ نقطه رسم شده‌است. رنگ بنفش، آبی، و قرمز به ترتیب  نقاط با درجه‌ی ۵، ۶، و ۷ را مشخص می‌کنند. برای مش‌های منظم و تصادفی صورت این دو مدل پاسخ تقریبا یکسانی داردند و اختلافشان ناشی از وزنی است که به نقاط با درجه‌ی ۵ و ۷ اختصاص داده شده‌است. داده‌ها در مش‌های درهم پخش شده اما همچنان همین روند نیز در آن‌ها دیده می‌شود.
}
\label{fig:unitsphereBendingScatter}
\end{center}
\end{figure}

در ستون سمت چپ شکل 
\ref{fig:unitsphereBendingScatter}
صورت کسر انرژی انحنای ایتزیکسون در مقایسه با صورت یولیشر (سمت چپ و راست معادله‌ی 
\ref{eq:JulicherItzyksonNumerator}
) برای نقاط روی مش‌های منظم، منظم درهم، تصادفی، و تصادفی درهم رسم شده‌است. در ستون سمت راست مساحت بریسنتریک (معادله‌ی 
\ref{eq:BarycentricArea}
) بر حسب مساحت ورنوی (معادله‌ی
\ref{eq:voronoiArea}
) برای همان نقاط محاسبه شده‌است. رنگ‌های بنفش، آبی،‌ و قرمز به ترتیب نقاط با درجه‌ی ۵، ۶، و ۷  را مشخص می‌کنند. شیب ۱ با خط مشکی مشخص شده. نقاط روی این خط مقادیر یکسان در هر دو نوع محاسبه دارند. به غیر از مش منظم،‌ نقاط رسم شده برای هر مش از ۱۰ نمونه مش با تعداد نقاط ۱۰۰۲ انتخاب شده‌است. یعنی در هر کدام از نمودارها ۱۰۰۲۰ نقطه نمایش داده شده‌است.

داده‌های ستون سمت راست نشان می‌دهد که اختلاف دو مدل ایتزیکسون و یولیشر  در  وزنی است که به نقاط اختصاص می‌دهند. همچنین به طور عمومی یولیشر وزن بیشتری (مساحت کمتری) به نقاط با درجه‌ی ۵ و وزن کمتری (مساحت بیشتری) به نقاط با درجه‌ی ۷ نسبت به مدل ایتزیکسون در نظر می‌گیرد. این روند در مش‌های درهم نیز دیده می‌شود.

\begin{figure}[htbp]
\begin{center}
\includegraphics[width=12cm]{\MemRes/Pics/n5byn7}
\caption{
نسبت متوسط تعداد نقاط با درجه‌ی ۵ (
$\vartheta_5$
) به تعداد نقاط با درجه‌ی ۷ (
$\vartheta_7$
) برای مش‌های مثلثی تصادفی بر حسب تعداد نقاط روی مش رسم شده‌است. نقاط حاصل متوسط‌گیری روی ۵۰ نمونه است.
}
\label{fig:n5n7}
\end{center}
\end{figure}


تعداد نقاط با درجات مختلف در یک شبکه مثلثی به شکل مش و تعداد نقاط نقط و خطوط لغزش روی آن بستگی دارد 
\cite{Nelson2000PRB}
. در شکل
\ref{fig:n5n7}
نسبت متوسط تعداد نقاط با درجه‌ی ۵ (
$\vartheta_5$
) به تعداد نقاط با درجه‌ی ۷ (
$\vartheta_7$
) برای مش‌های تصادفی رسم شده‌است. مش‌های تصادفی بیشتر از نقاط با درجه‌ی ۶ ساخته شده‌اند که تقریبا در هر دو مدل محاسبه‌ی خمش به طور متوسط اندازه‌ی یکسانی دارند. در مورد مش‌های درهم تصادفی انرژی انحنا با تعداد نقاط با درجات غیر ۶ تغییر می‌کند. در صورتی که نسبت تعداد نقاط با درجه‌ی ۵ به ۷ در شبکه‌های ریز و درشت یکسان می‌بود، شاهد رشد خطی خطا با افزایش اندزاه‌ی مش‌بندی شبکه‌ در مش‌های درهم تصادفی در شکل 
\ref{fig:unitsphereBending}
می‌بودیم. از آنجایی که این نسب با افزایش اندازه‌ی مش‌بندی تغییر می‌کند و به ۱ میل می‌کند شاهد میل‌ انرژی انحا برای مدل‌های بریسنتریکی در مش‌های درهم تصادفی هستیم.



\begin{figure}[htbp]
\begin{center}
\includegraphics[width=12cm]{\MemRes/Pics/UnitSphereDeformation_mesh_30}
\caption{
انرژی انحنا (ستون چپ) و نیروی برگرداننده (ستون راست) محاسبه‌ شده حاصل از تغییر شکل مش کروی با اضافه شدن مد
$Y_{2,0}(\theta,\phi)$
با شدت‌های مختلف برای چهار مدل ایتزیکسون (قرمز)، یولیشر (سبز)، ایتزیکسون-بریسنتریک (آبی)، و یولیشر-ورنوی (نارنجی) روی مش‌های مختلف رسم شده‌است. خطوط مشکی برای ستون سمت چپ و راست به ترتیب پیش بینی
 مرتبه‌ی دوم انرژی انحنا و نیروی بازگرداننده برای تغییر انحنا، یعنی معادلات
\ref{eq:curvatureY20}
و
\ref{eq:curvatureForceY20}
را نشان می‌دهند.  شدت مد 
$u_{2,0}=0$
مربوط به شکل کاملا کروی (عکس سمت چپ) و شدت مد 
$u_{2,0}=1$
مربوط به شکل دمبلی (عکس سمت راست) است. یک نمونه از هر مش برای حالت کروی و دمبلی در ردیف مربوته رسم شده‌است. به غیر از مش منظم (که تنها یک نمونه از آن برای هر تعداد نقطه وجود دارد) مقادیر محاسبه شده حاصل از میان‌گین گیری بر روی ۵۰ نمونه‌ی مستقل انجام شده‌است.
}
\label{fig:unitsphereBendingULM20}
\end{center}
\end{figure}

مشابه با بخش قبلی، انرژی انحنای مش‌هایی که با مد هارمونیک کروی تغییر شکل داده شده‌اند را بررسی می‌کنیم (شکل
\ref{fig:unitsphereBendingULM20}
). در این شکل تغییر انرژی و نیرو بر اساس تغییر شدت مد از
$u_{2,0}=0$
(کاملا کروی) به
$u_{2,0}=1$
(دمبلی شکل) رسم شده‌است. انرژی انحنا برای تغییر شکل‌های کم به کمک معادله‌ی 
\ref{eq:curvatureYLM}
 و با جایگذاری 
 $\kappa=1[\varepsilon]$
 به صورت
\begin{eqnarray}
E_{b}=8\pi\kappa + 12\kappa|u_{2,0}|^2
\label{eq:curvatureY20}
\end{eqnarray}
و نیروی برگرداننده با مشتق گیری نسبت با شدت مد به صورت
\begin{eqnarray}
-\frac{\partial E_{b}}{\partial u_{2,0}}= -24\kappa |u_{2,0}|
\label{eq:curvatureForceY20}
\end{eqnarray}
محاسبه می‌شود.

در مورد انرژی انحنا نیز در شد‌ت مدهای کم همخوانی خوبی میان مدل‌ها و محاسبات نتیجه شده از بررسی افت و خیز دیده می‌شود. خطای کمی که در مش‌های درهم تصادفی دیده‌ شده بود در اینجا نیز دیده می‌شود. ولی نتیجه‌ی مهم این شکل این است که مستقل از مدل انحنا، نیروهای تولید شده میان تمام مدل‌ها یکسان رفتار می‌کند. در نتیجه انتخاب مدل محاسبه‌ی انرژی انحنا در دینامیک نقاط روی صفحه تاثیر ندارد.

\begin{figure}[htbp]
\begin{center}
\includegraphics[width=10cm]{\MemRes/Pics/IJ_numerator_ULM20.pdf}
\caption{
مقادیر صورت (ستون چپ) و مخرج (ستون راست) مدل گسسته اندازه‌گیری انحنای ایتزیکسون و یولیشر برای نقاط مختلف بر روی مش‌های منظم، منظم درهم، تصادفی، و تصادفی درهم که با شدت مد 
$u_{2,0}=1$
به شکلی دمبلی درآمده‌اند رسم شده‌است. داده‌ها مربوط به مش‌های ۱۰۰۲ نقطه‌ای است و  رنگ بنفش، آبی، و قرمز به ترتیب  نقاط با درجه‌ی ۵، ۶، و ۷ را مشخص می‌کنند. برای مش‌های منظم و تصادفی صورت این دو مدل پاسخ تقریبا یکسانی داردند و اختلافشان ناشی از وزنی است که به نقاط با درجه‌ی ۵ و ۷ اختصاص داده شده‌است. داده‌ها در مش‌های درهم پخش شده اما همچنان همین روند نیز در آن‌ها دیده می‌شود.
}
\label{fig:ULM20BendingScatter}
\end{center}
\end{figure}

جهت بررسی انحنای نقاط زین اسبی، در شکل 
\ref{fig:ULM20BendingScatter}
مشابه به شکل
\ref{fig:unitsphereBendingScatter}
صورت و مخرج مدل‌های ایتزیکسون و یولیشر را برای نقاط روی مش‌های شکل 
\ref{fig:unitsphereBendingULM20}
رسم شده‌است. به علت کشیده شدن مش‌ بر اثر تغییر شکل، توزیع مساحت وسیع‌تری نسبت به شکل 
\ref{fig:unitsphereBendingScatter}
مشاهده می‌شود. بررسی رفتار نقاط نشان می‌دهد که روند اختلاف انرژی نقاط با درجات مختلف که قبل‌تر مشاهده شد در شکل‌های پیچیده‌تر نیز حضور دارد.




