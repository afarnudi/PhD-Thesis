در روش شبیه‌سازی توزیع دینامیک مساحت
\LTRfootnote{Dynamic Area Redistribution}
اندازه و شکل قسمت‌های مِش به طور پیوسته در حال تغییر است. در این مدل تنش بُرشی هزینه‌ی انرژی نداشته و چگالی دولایه‌ی لیپیدی غشا در سراسر مش ثابت فرض شده. در این رساله از شبکه‌ی مثلثی برای پیاده سازی 
Dynamic Area Redistribution (DAR)
استفاده شده‌است. جهت شبیه‌سازی موفق غشا لازم است انرژی مساحت، حجم، و انحنا به درستی در همه نقاط محاسبه شود. صحت معادلات گسسته بر شبکه‌های منظم و تصادفی در مطالعات گذشتگان اثبات شده است (بخش 
\ref{sec:simRevMesh}
). در درجه‌ی اول صحت این معادلات بر شبکه‌های درهم نیز بایت اثبات شود. سپس با استفاده از این شبکه‌ها می‌توان رفتار غشا را مطالعه کرد.