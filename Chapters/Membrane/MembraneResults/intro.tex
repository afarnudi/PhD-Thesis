در این فصل به طور کمی، «نرمی‌» مش‌هایمان را اندازه‌گیری کرده و رابطه‌ی آن را با پارامتر‌های پتانسیل‌ تکمیلی که ارتفاع مثلث‌ها را کنترل می‌کند، بررسی می‌کنیم (بخش
\ref{sec:Soft meshes}).
 در قدم بعد، در بخش 
\ref{sec:MeshObservables}
کیفیت اندازه‌گیری مساحت، حجم، و انحنا پس از گسسته‌سازی سطح را بررسی می‌کنیم. در بخش
\ref{sec:Results DAR MD}
قدم زمانی لازم برای شبیه‌سازی مش‌های دینامیک را محاسبه کرده و سپس پایداری مدل‌های مختلف اندازه‌گیری انحنا را در شبیه‌سازی مشخص می‌کنیم. بخش 
\ref{sec:resultsBendingFluctuations}  
نتایج اندازه‌گیری دامنه افت و خیزهای سطحی مش را نشان می‌دهد و بخش 
\ref{sec:larger shape changes}
به طور کیفی قابلیت تغییر شکل مش‌های نرم را نشان می‌دهد. در بخش
\ref{sec:dynamics}
نتایج بررسی دینامیک سطح را ارائه می‌دهیم. این فصل با صحبتی کوتاه در مورد کارایی محاسبات انجام شده در بخش
\ref{sec:computational efficiency}
به پایان می‌رسد.