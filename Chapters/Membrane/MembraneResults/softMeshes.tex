همانطور که در فصل قبل (بخش 
\ref{sec:statMech})
بحث شد، سطح
$\mathcal{S}$
(در فضای ۳ بعدی) را می‌توان با مِش
$\mathcal{M}({\bm q},\mathcal{G})$
نمایش داد. مِش
$\mathcal{M}$
با مجموعه مختصات نقاط 
${\bm q}=\{{\bm q_1},{\bm q_2},...,{\bm q_N}\} \in \mathcal{S}$
و اتصالات میان آن‌ها
$\mathcal{G}$
تعریف می‌شود. مِش سخت مِشی‌است که در آن مختصات نقاط به طور تقریبا یکنواخت بر سطح 
$\mathcal{S}$
توزیع شده‌ باشد. هنگامی که سطح 
$\mathcal{S}$
دچار تغییر شکل شود، مختصات نقاطِ مِش نیز به ناچار تغییر خواهد کرد. از آنجایی که نقاط مِشِ سخت باید  توزیع یکنواخت بر سطح
$\mathcal{S}$
داشته باشند، ضمن تغییر مختصات نقاطِ مِش، اتصالاتِ میان نقاط
($\mathcal{G}$)
نیز باید تغییر کند. برای بازسازی (یا باز تعریف) 
$\mathcal{G}$
معمولا از الگوریتم دینامیک مثلثی
\cite{Boal1992PRA, Gompper1992Science}
استفاده می‌شود. در نتیجه، هنگام شبیه‌سازی سطح
$\mathcal{S}$
با مِشِ سخت، اندازه و شکل مثلث‌هایِ مِش در طول شبیه‌سازی ‌کم و بیش یکسان است. از سوی دیگر، در تعریف مِش‌های نرم، مختصات نقاط،  توزیع مشخصی ندارند ولی اتصالات تعریف شده میان نقاط مِش در طول شبیه‌سازی ثابت است. در نتیجه بر سطح مش‌های نرم، مثلث‌هایی با شکل و اندازه‌های بسیار متفاوتی پدید می‌آید.  دامنه‌ی تغییر شکل مثلث‌ها توسط پتانیسلی شبیه به 
WCA 
یا همان پتانسیل تکمیلی 
$U_h$
(معادله‌ی 
\ref{eq:wcah}
) تعیین می‌شود که وظیفه‌ی آن کنترل کمترین ارتفاع مجاز مثلث‌هاست. 


\subsection{\label{sec:MeshEquilibration}
به تعادل رسیدن مِش
}

برای فهمیدن بهتر دینامیک مِش‌های نرم، ابتدا مِش‌های تصادفی را همراه با پتانسیل مساحت
$U_A$ (\ref{eq:GlobalAreaPotentialExpansion})
، پتانسیل حجم 
$U_V$ (\ref{eq:VolumePotentialExpansion})
، و پتانیسل تکمیلی 
$U_h$ (\ref{eq:wcah})
 شبیه‌سازی می‌کنیم. در چنین سیستمی، کوتاه‌ترین زمان مشخصه‌ی سیستم مربوط به زمان مشخصه‌ی پتانسیل 
WCA
است که مقدار آن
 $\tau_h=d_h\sqrt{m/\epsilon_h}$
وابسته به جرم نقاط مش
$m$,
ارتفاع کمینه‌ی مجاز مثلث‌ها
$d_h$,
و عمق چاه 
WCA
یا 
$\epsilon_h$
است. در چنین شبیه‌سازی، نقاط مش در فضا حرکت می‌کنند و سطح و حجم مش نسبت به مقادیر اولیه‌شان 
($A_0$
و
$V_0$)
افت و خیز می‌کند. از آنجایی که در این شبیه‌سازی از پتانسیل انحنا استفاده نشده‌است، تا زمانی که مساحت و حجم توسط پتانسیل‌های
$U_A$
و
$U_V$
تنظیم شود، هیچ محدودیتی بر زاویه‌ی میان مثلث‌های همسایه وجود ندارد.لازم به ذکر است که در این شبیه‌سازی‌ها حجم کاهیده نزدیک به واحد است، در نتیجه، حجم‌های شبیه به سیخ‌های بلند بر سطح غشا پدیدار نمی‌شود. برای آشنایی با طبیعت این شبیه‌سازی‌ها، فیلم ضمیمه‌ی 
\textbf{SM\_AVh\_V}
را مشاهده کنید.
\begin{figure}[htbp]
\begin{center}
\includegraphics[width=13cm]{\MemRes/Pics/area_relaxation.pdf}
\caption{
به تعادل رسیدن تابع توزیع احتمال مساحت مثلث‌های سطح مِشِ نزم با متوسط گیری از 
$10$
 مِش‌ تصادفی دارای
$2000$
مثلث 
($N=1002$).
خط چین عمودی سبز حد کمینه‌ی مجاز مساحتِ مثلث (که توسط پتانسیل
$U_h$
تعیین می‌شود) را نشان می‌دهد.
}
\label{fig:areaRelaxation}
\end{center}
\end{figure}


\begin{figure}[htbp]
\begin{center}
\includegraphics[width=13cm]{\MemRes/Pics/bond_length_relaxation.pdf}
\caption{
به تعادل رسیدن تابع توزیع احتمال اندازه‌ی اضلاع مثلث‌های سطح مِشِ نرم با متوسط گیری از 
$10$
 مِش‌ تصادفی دارای
$3000$
ضلع 
($N=1002$). خط چین عمودی سبز حد کمینه‌ی مجاز اندازه‌ی ضلع مثلث ( که توسط پتانسیل
$U_h$
تعیین می‌شود) را نشان می‌دهد.
}
\label{fig:bondRelaxation}
\end{center}
\end{figure}

شکل‌های 
\ref{fig:areaRelaxation}
و
\ref{fig:bondRelaxation}
به ترتیب تغییرات زمانی تابع توزیع اندازه‌ی مساحت و اضلاع مثلث‌های مِش‌ها را نشان می‌دهد. شبیه‌سازی‌های دینامیکِ مِش بسیار سریع است و توزیع چیدمان نقاط، سریع به تعادل می‌رسد. دقت کنید که در این دو شکل، هیچ تفاوت چشم‌گیری میان تابع توزیع اندازه‌گیری شده در زمان
$t=6.5 \tau_h$
و
$t=1100 \tau_h$
که به رنگ آبی در پس زمینه است، وجود ندارد.

\subsection{\label{sec:phi}
تعریف و کنترل نرمیِ مِش
}

شکل متوسط مثلث‌هایی که سطح یک مش را می‌پوشانند را می‌توان با مثلث متساوی الاضلاعی تخمین زد که ارتفاع 
$\bar h$
و مساحت 
$\bar a_{iso}$
متوسط آن،
\begin{equation}
\bar a_{iso}=\frac{1}{\sqrt{3}}\bar h^2.
\label{eq:averageTriArea}
\end{equation}
است. در طول شبیه‌سازی دینامیکِ مِش، نقاط می‌توانند در فضا حرکت کنند، در نتیجه مناطقی ممکن است ایجاد شود که در آن اندازه‌ی مثلث‌ها بسیار کوچک باشد. مساحت چنین مثلث‌هایی 
($\bar a_{min}$)
بر اساس کوتاه‌ترین ارتفاع مجاز مثلث‌ها
($d_h$)
قابل محاسبه‌است،
\begin{equation}
\bar a_{min}=\frac{1}{\sqrt{3}}d_h^2.
\label{eq:aMin}
\end{equation}
پارامتر ساده‌ای را تعریف می‌کنیم که مقدار افت و خیز اندازه‌ی مثلث‌ها را با نسبت مساحت کوچکترین مثلث ممکن به اندازه‌ی متوسط مثلث‌ها نشان می‌دهد،
\begin{equation}\label{eq:Phi}
\Phi=\frac{\bar a_{min}}{\bar a_{iso}}=\left(\frac{d_h}{\bar h}\right)^2.
\end{equation}
\subsection{\label{sec:AreaVertex}
تاثیر
$\Phi$
بر مشخصات مِش
}

\begin{figure}[ht]
\begin{center}
\includegraphics[width=10cm]{\MemRes/Pics/ULM_Fluidity.pdf}
\caption{
مشخصه‌های استاتیک و دینامیک مِش‌های نرم بر حسب نسبت
$\Phi$. (a)
آنالیز هماهنگ‌های کروی توزیع نقاط بر سطح یک مِش تقریبا کروی دارای حجم کاهیده‌ی 
$\nu=1$.
 نقاط مشکی نتیجه محاسبه برای مختصات تصادفی بر سطح کره (گاز ایده‌آل دو بعدی) را جهت مقایسه نشان می‌دهد. 
(b)
تابع توزیع احتمال مساحت مثلث‌ها و 
(c)
تابع خودهمبستگی مساحت مثلث‌ها را نشان می‌دهد.
}
\label{fig:vertexULM}
\end{center}
\end{figure}

شکل
\ref{fig:vertexULM}
چندین نتیجه از تاثیر 
 $\Phi$
بر رفتار مش را نشان می‌دهد. شکل 
\ref{fig:vertexULM}a
 مُد‌های چگالی
$\rho_{\ell,m}$
را نشان می‌دهد که با مختصات نقاط مش تقریبا کروی با حجم کاهیده‌ی 
$\nu=1$
محاسبه‌ شده است. این مُد‌ها مشابه به مُد‌های خمشی که در مراجع 
\cite{safran1983, milnersafranPRA1987}
محاسبه شده‌است تعریف می‌شوند و خواص هندسی توزیع نقاط را نشان می‌دهد. برای 
$\ell$
کوچک، دامنه‌ی مُد‌های اعداد مُد بزرگ به مقدار
$\rho_{\ell,m}=1$
میل می‌کند. تاثیر ثابت بودن اتصالات مثلث‌ها به شکل یک تاثیر ضعیف خود را در طول‌ موج‌های بلند 
($\ell$
کوچک) نشان می‌دهد. بر عکس، یک قله‌ی کاملا مشخص در عدد مُد تقریبا دو برابر بیشترین عدد مُد ممکن 
($\ell_{max}=\sqrt{N}-1$)
 زمانی که 
$\Phi = {\cal O}(1)$,
دیده می‌شود. این پدیده با کاهش شدید در افت و خیز مساحت‌های مثلث‌ها  (شکل
\ref{fig:vertexULM}b)
و تسریع افت تابع خودهمبستگی اندازه‌ی مثلث‌ها (شکل
\ref{fig:vertexULM}c)
 کاملا هماهنگ است. این پدیده خبر از تشکیل یک مِش مایع‌گون با تراکم پذیری بالا می‌دهد که در آن امواج طولی با سرعت بیشتری حرکت می‌کنند. چنین مشی برای برآورده کردن اهداف ما مناسب نیست زیراکه قادر به انجام تغییر شکل‌های بزرگ نخواهد بود.












