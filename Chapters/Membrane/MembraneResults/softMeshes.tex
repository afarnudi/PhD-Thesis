ابتدا باید مش‌هایی که برای شبیه‌سازی غشا استفاده می‌کنیم را تعریف کنیم. دو نوع اختلال مش‌های ما را از یکدیگر متمایز می‌کند. اختلالات در اتصالات میان نقاط مش در طول هر شبیه‌سازی دینامیک مش ثابت است و از هیچ روش باز سازی مش مانند دینامیک مثلثی استفاده نمی‌شود
\cite{Boal1992PRA, Gompper1992Science}
. از طرفی دیگر اندازه و شکل مثلث‌های مش تقریبا آزادانه پذیرفته می‌شود. دامنه‌ی چنین تغییراتی توسط پتانیسل شبیه به 
WCA 
یا همان پتانسیل تکمیلی 
$U_h$
(معادله‌ی 
\ref{eq:wcah}
) تعیین می‌شود که وظیفه‌ی آن تغیین کمترین ارتفاع مجاز مثلث‌هاست. 


\subsection{\label{sec:MeshEquilibration}
به تعادل رسیدن مش
}

ابتدا توجه‌مان را به دینامیک مش‌های تصادفی همراه با پتانسیل مساحت و حجم می‌دهیم. در چنین سیستمی کوتاه‌ترین زمان مشخصه‌ی سیستم مربوط به زمان مشخصه‌ی پتانسیل 
WCA
است که مقدار آن
 $\tau_h=d_h\sqrt{m/\epsilon_h}$
وابسته به جرم نقاط مش
$m$
ارتفاع کمینه‌ی مجاز مثلث‌ها
$d_h$
و عمق چاه 
WCA
یا 
$\epsilon_h$
است. در چنین شبیه‌سازی، نقاط مش در فضا حرکت می‌کنند و سطح و حجم مش نسبت به مقادیر اولیه‌شان افت و خیز می‌کند. از آنجایی که در این شبیه‌سازی از پتانسیل انحنا استفاده نشده‌است، تا زمانی که مساحت و حجم توسط پتانسیل‌های
$U_A$
و
$U_V$
تنظیم باشد، هیچ محدودیتی بر زاویه‌ی میان مثلث‌های همسایه وجود ندارد. ولی از آنجایی که در این شبیه‌سازی‌ها حجم کاهیده نزدیک به واحد است، حجم‌های شبیه به سیخ‌های بلند بر سطح غشا پدیدار نمی‌شود. خواننده را تشویق می‌کنیم که فیلم ضمیه‌ی 
\textbf{SM\_AVh\_V}
را مشاهده کند تا با طبیعت این شبیه‌سازی‌ها بیشتر آشنا شود. برای شبیه‌سازی سطحی که شبیه‌ به یک غشا رفتار کند نیاز به پتانسیل انحنا است اما این شبیه‌سازی سیستم ساده‌ایست که بتوان در آن خواص دینامیک مش را بررسی کرد.
\begin{figure}[htbp]
\begin{center}
\includegraphics[width=13cm]{\MemRes/Pics/area_relaxation.pdf}
\caption{
به تعادل رسیدن تابع توزیع احتمال مساحت مثلث‌ها که از متوسط گیری مش‌هایی با ۲۰۰۰ مثلث (
$N=1002$
) تشکیل شده‌است. خط چین عمودی سبز اندازه کمترین مساحت مثلث مجازی که پتانسیل
$U_h$
تعیین می‌کند را نشان می‌دهد.
}
\label{fig:areaRelaxation}
\end{center}
\end{figure}


\begin{figure}[htbp]
\begin{center}
\includegraphics[width=13cm]{\MemRes/Pics/bond_length_relaxation.pdf}
\caption{
به تعادل رسیدن تابع توزیع احتمال اندازه‌ی اضلاع مثلث‌ها که از متوسط گیری مش‌هایی با ۳۰۰۰ ضلع (
$N=1002$
) تشکیل شده‌است. خط چین عمودی سبز اندازه کمترین اندازه‌ی ضلع مثلث مجازی که پتانسیل
$U_h$
تعیین می‌کند را نشان می‌دهد.
}
\label{fig:bondRelaxation}
\end{center}
\end{figure}

شکل‌های 
\ref{fig:areaRelaxation}
و
\ref{fig:bondRelaxation}
به ترتیب تغییرات زمانی تابع توزیع اندازه‌ی مساحت و اضلاع مثلث‌های مش‌ها را نشان می‌دهد. شبیه‌سازی‌های دینامیک مش بسیار سریع هستند و توزیع چیدمان نقاط سریع به تعادل می‌رسد. دقت کنید که در این دو شکل، هیچ تفاوت چشم‌گیری میان تابع توزیع اندازه‌گیری شده در زمان
$6.5 \tau_h$
و
$1100 \tau_h$
که به رنگ آبی در پس زمینه است، وجود ندارد.

\subsection{\label{sec:phi}
تعریف و کنترل نرمی مش
}

شکل متوسط مثلث‌هایی که سطح یک مش را می‌پوشانند را می‌توان با مثلث متساوی الاضلاعی تخمین زد که ارتفاع 
$\bar h$
و مساحت 
$\bar a_{iso}$
متوسط آن
\begin{equation}
\bar a_{iso}=\frac{1}{\sqrt{3}}\bar h^2.
\label{eq:averageTriArea}
\end{equation}
است. در طول شبیه‌سازی دینامیک مش، نقاط می‌توانند در فضا حرکت کنند، در نتیجه مناطقی ممکن است ایجاد شود که در آن اندازه‌ی مثلث‌ها بسیار کوچک باشد. مساحت چنین مثلث‌هایی 
$\bar a_{min}$
بر اساس کوتاه‌ترین ارتفاع مجاز مثلث‌ها
$d_h$
قابل محاسبه‌است
\begin{equation}
\bar a_{min}=\frac{1}{\sqrt{3}}d_h^2.
\label{eq:aMin}
\end{equation}
پارامتر ساده‌ای را تعریف می‌کنیم که مقدار افت و خیز اندازه‌ی مثلث‌ها را با نسبت مساحت کوچکترین مثلث ممکن به اندازه‌ی متوسط مثلث‌ها نشان می‌دهد
\begin{equation}\label{eq:Phi}
\Phi=\frac{\bar a_{min}}{\bar a_{iso}}=\left(\frac{d_h}{\bar h}\right)^2
\end{equation}
.
\subsection{\label{sec:AreaVertex}
تاثیر
$\Phi$
بر مشخصات مش
}

\begin{figure}[ht]
\begin{center}
\includegraphics[width=10cm]{\MemRes/Pics/ULM_Fluidity.pdf}
\caption{
مشخصه‌های استاتیک و دینامیک مش‌های نرم بر حسب نسبت
$\Phi$
را نشان می‌‌دهد.
(a)
آنالیز هماهنگ‌های کروی توزیع نقاط بر سطح یک مش تقریبا کروی دارای حجم کاهیده‌ی 
$\nu=1$
. نقاط مشکی نمایانگر نتیجه محاسبه برای مختصات تصادفی بر سطح کره (گاز ایده‌آل دو بعدی) را جهت مقایسه نشان می‌دهد. 
قسمت
(b)
تابع توزیع احتمال مساحت مثلث‌ها و قسمت
(c)
تابع خودهمبستگی مساحت مثلث‌ها را نشان می‌دهد.
}
\label{fig:vertexULM}
\end{center}
\end{figure}

شکل
\ref{fig:vertexULM}
چندین نتیجه از تاثیر 
 $\Phi$
بر رفتار مش را نشان می‌دهد. در شکل 
\ref{fig:vertexULM}a
نشان می‌دهیم که مد‌های چگالی
$\rho_{\ell,m}$
را نشان می‌دهد که با مختصات نقاط مش تقریبا کروی با حجم کاهیده‌ی 
$\nu=1$
محاسبه‌ شده است. این مد‌ها مشابه به مد‌های خمشی که در مراجع 
\cite{safran1983, milnersafranPRA1987}
محاسبه شده‌است تعریف می‌شوند و خواص هندسی توزیع نقاط را نشان می‌دهد. برای 
$\ell$
کوچک دامنه‌ی مد‌های اعداد مد بزرگ به مقدار
$\rho_{\ell,m}=1$
میل می‌کند. تاثیرثابت بودن اتصالات مثلث‌ها به شکل یک تاثیر ضعیف خود را در طول‌ موج‌های بلند (
$\ell$
کوچک) نشان می‌دهد. بر عکس، یک قله‌ی کاملا مشخص در عدد مد تقریبا دو برابر بیشترین عدد مد ممکن (
$\ell_{max}=\sqrt{N}-1$
) وقتی که 
$\Phi = {\cal O}(1)$
دیده می‌شود. این پدیده با کاهش شدید در افت و خیز مساحت‌های مثلث‌ها  (شکل
\ref{fig:vertexULM}b
) و تسریع افت تابع خودهمبستگی اندازه‌ی مثلث‌ها (شکل
\ref{fig:vertexULM}c
) کاملا هماهنگ است. این پدیده خبر از تشکیل یک مش مایع‌گون با تراکم پذیری بالا می‌دهد که در آن امواج طولی با سرعت بیشتری حرکت می‌کنند. چنین مشی برای برآورده کردن اهداف ما مناسب نیست زیراکه قادر به انجام تغییر شکل‌های بزرگ نخواهد بود.












