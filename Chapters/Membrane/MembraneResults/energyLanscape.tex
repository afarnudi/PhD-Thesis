\begin{figure}[htbp]
\begin{center}
\includegraphics[width=10cm]{\MemRes/Pics/total_curvature_152_plain_slices.pdf}
\caption{
در سمت چپ پلاکتی از مش درهم تصادفی را با رنگ سبز نمایان شده. مکان تمام نقاط به غیر از نقطه‌ی وسط پلاکت سبز ثابت است. نقطه‌ی وسط همچنین عضو پلاکت‌های همسایه‌ است که با رنگ نارنجی مشخص شده‌اند. نمایش دو بعدی سمت چپ معادل صفحه‌ی خمیده‌ی نارنجی رنگ در نمایش ۳ بعدی تصویر سمت راست است. تغییر چشم انداز انرژی هنگامی که مختصات نقطه‌ی وسط بر روی سطح مش (صفحه‌ی نارنجی) و صفحه‌ی عمود بر سطح (صفحه‌ی آبی) بررسی شده‌است.
}
\label{fig:PlacketRepresentaion}
\end{center}
\end{figure}

هر سه انرژی لازم برای مدل کردن غشا یعنی انرژی مساحت، حجم، و انحنا بر روی مش‌ها درهم با دقت قابل قبولی تعریف شده و همچنین نیروی محاسبه شده از آنها با محاسبات نظری همخوانی دارد. حال لازم است که چشم انداز انرژی در اطراف یک پلاکت در حال حرکت را بررسی کنیم. با این هدف مش درهم تصادفی را در نظر می‌گیریم که تمامی نقاط آن بر سطح یک کُره‌ قرار دارد. مختصات تمام نقاط را در فضا ثابت کرده، سپس یک نقطه روی مش را انتخاب کرده (نقطه‌ی وسط در پلاکت سبز رنگ در تصویر سمت چپ شکل
\ref{fig:PlacketRepresentaion}
) و در حالی که دیگر نقاط در جای خود ثابتند، مختصات این نقطه را تغییر دادیم. جهت ساده‌سازی مختصات این نقطه را  بر سطح کره‌ حرکت داده (سطح نارنجی در تصویر سمت چپ شکل
\ref{fig:PlacketRepresentaion}
) و همچنین بر سطح صفحه‌ای عمود بر کره (سطح آبی در تصویر سمت چپ شکل
\ref{fig:PlacketRepresentaion}
) حرکت دادیم. هنگام تغییر مختصات نقطه‌ی وسط انرژی پتانسیل‌های حاکم بر غشا اندازه‌گیری شد. 




\begin{figure}[htbp]
\begin{center}
\includegraphics[width=11cm]{\MemRes/Pics/area_volume_WCAh_Exp46_152}
\caption{
چشم انداز انرژی پتانسیل مساحت، حجم،
WCAh
، و خمش غیر خطی دوسطحی را تابعی از مختصات نقطه‌ی میانی پلاکت سبز رنگ نشان می‌دهد. در ستون سمت چپ تغییر انرژی حاصل از حرکت نقطه بر سطح کره و ستون سمت راست تغییر انرژی حاصل از حرکت نقطه بر صفحه‌ی عمود بر سطح کره  نشان داده شده‌است. واحد انرژی
$k_BT=2.49[\varepsilon]$
 و مقادیر 
 $k_A=5.12\times10^5k_BT/l^2$
 و
 $k_V=1.6\times10^7k_BT/l^3$
 به ترتیب برای ضریب فشردگی سطحی و مدول حجمی در نظر گرفته شده است. در پتانسیل‌های تکمیلی، پارامتر‌های پتانسیل 
 WCAh
 ، 
$\epsilon=4k_BT$
 و کمینه ارتفاع مثلث‌ها 
 $d_h/r_0=0.02$ 
 . برای ضریب سختی خمش پتانسیل غیر خطی دوسطحی 
 $\kappa_{\phi^4}=20k_BT$
 . و انرژی در واحد‌های کاهیده
 $k_BT=2.49[\varepsilon]$
 است.
}
\label{fig:PlacketEnergyArea}
\end{center}
\end{figure}

در شکل 
\ref{fig:PlacketEnergyArea} $a)$
 انرژی مساحت نمایش داده شده‌است. مقدار این انرژی تنها تابع تغییر شکل مثلث‌های درون پلاکت سبز رنگ است. در ستون سمت چپ تا زمانی که نقطه‌ی وسط داخل محدوده‌ی پلاکت حرکت می‌کند، مساحت کل غشا کم و بیش ثابت است. هنگامی‌ که نقطه‌ی وسط از محدوده‌ی پلاکت خارج شود مساحت غشا افزایش پیدا کرده و کم کم انرژی مساحت افزایش می‌یابد. در ستون سمت راست نیز هنگامی که نقطه‌ی وسط از چیدمان اولیه فاصله می‌گیرد مساحت و در نتیجه انرژی مساحت افزایش پیدا می‌کند. 

در شکل 
\ref{fig:PlacketEnergyArea} $b)$
 انرژی حجم نمایش داده شده‌است. مقدار این انرژی نیز تنها تابع تغییر شکل مثلث‌های درون پلاکت سبز رنگ است. از آنجایی که حجم هرم‌ها بسته به جهت سطح منفی یا مثبت هستند، تا زمانی که نقطه‌ی وسط بر سطح کره حرکت می‌کند (سمت چپ) حجم کل مش تغییر نکرده و در نتیجه انرژی حجم ثابت است.‌ از آنجایی که حجم هرم با ضرب داخلی سطح و ارتفاع محاسبه می‌شود در ستون سمت راست  هنگامی که نقطه‌ی وسط از چیدمان اولیه فاصله می‌گیرد حجم تنها طبق فاصله‌ی شعاعی نسبت به جهت عمودی پلاکت تغییر می‌کند و حاصل آن صفحات موازی هم انرژی‌است.

در شکل 
\ref{fig:PlacketEnergyArea} $c)$
تغییرات انرژی 
WCAh
 نمایش داده شده‌است. از آنجایی که هنگام تغییر مختصات نقطه‌ی میانی تنها ارتفاع نقطه‌ی میانی با اضلاع مجاور تغییر خواهد کرد، مقدار این انرژی نیز تنها تابع تغییر شکل مثلث‌های درون پلاکت سبز رنگ است. در تصویر سمت چپ، تا زمانی که ارتفاع نقطه‌ی میانی از حد قطع بیشتر باشد، تغییری اتفاق نمی‌افتد. اما زمانی که نقطه به اضلاع یا رئوس نزدیک شود این انرژی با شیب توانی افزایش می‌یابد. در این نمایش چشم انداز انرژی زمانی که نقطه قادر به عبور از سطح بالای انرژی 
 WCAh
 باشت نیز رسم شده‌است. در تصویر سمت راست نیز اثر پتانسل تنها زمانی مشاهده می‌شود که مختصات نقطه به اضلاع نزدیک شود.
 
 در شکل 
\ref{fig:PlacketEnergyArea} $d)$
تغییرات انرژی پتانسیل غیر خطی دوسطحی نمایش داده شده‌است. باید توجه کرد که در تصویر سمت چپ نقطه‌ی وسط بر سطح کره حرکت می‌کند و نه بر سطح مش. در این صورت در این نمایش درواقع نقطه از روی اضلاع عبور کرده و آنها را قطع نخواهد کرد. به همین علت تا زمانی که نقطه در محدوده‌ی پلاکت سبز حرکت می‌کند زاویه‌ی تیزی تشکیل نخواهد شد. اما هنگامی که نقطه به اضلاع نزدیک شود، زوایای بسیار تیزی تشکیل شده و انرژی غیر خطی را سریع افزایش خواهد داد. در تصویر سمت راست اثر این پتانسیل برای زوایای تیز به وضوح دیده می‌شود.

به طور کلی نقطه هنگامی که از محدوده‌ی  پلاکت سبز خارج شود، چه بر سطح کره و چه خارج از صفحه‌، این انرژی‌ها را افزایش خواهد داد.

\begin{figure}[htbp]
\begin{center}
\includegraphics[width=11cm]{\MemRes/Pics/total_curvature_152.pdf}
\caption{
تغییر انرژی انحنای یولیشر، ایتزیکسون، ایتزیکسون-بریسنتریک، و یولیشر-ورنوی هنگام حرکت یک نقطه بر سطح کره و بر روی صفحه‌ی عمود بر سطح کره نسبت به انرژی انحنای اولیه‌ی مش را نشان داده‌است. سختی خمش در تمام مدل‌ها 
$\kappa=20k_BT$
در نظر گرفته شده‌است.
}
\label{fig:PlacketEnergy}
\end{center}
\end{figure}


در شکل
\ref{fig:PlacketEnergy}
تغییر انرژی انحنای مش به ترتیب از بالا به پایین برای مدل یولیشر، ایتزیکسون، ایتزیکسون-بریسنتریک، و یولیشر-ورنوی نمایش داده شده‌است. نکته‌ی مهم در مورد انرژی انحنا اینجاست که نقطه‌ی میانی پلاکت سبز رنگ نقش نقطه‌ی گوشه‌ای برای پلاکت‌های که با رنگ نارنجی در شکل
\ref{fig:PlacketRepresentaion}
سمت چپ نمایش داده‌شده، دارد. در نتیجه چیدمان مختصات نقاط اطراف پلاکت بر چشم انداز انرژی تاثیر دارد. در ستون سمت چپ، انرژی انحنای هر چهار مدل تا زمانی که مختصات نقطه‌ی میانی در محدود‌ه‌ی پلاکت سبز قرار دارد کم و بیش ثابت است. در مورد انرژی‌ها بر اساس مساحت بریسنتریک، در صورتی که مختصات نقطه از محدوده‌ی پلاکت خارج شود، انرژی افزایش یافته و نقطه را به داخل پلاکت حدایت می‌کند. امت در مورد پتانسیل‌های بر پایه‌ی مساحت ورنوی، فضای اطراف پلاکت پر از چاه‌های منفی بینهایت و ناپایداری‌های عددی‌ است. رفتار مشابه نیز در ستون سمت راست هنگام خروج نقطه از صفحه دیده می‌شود. 


\begin{figure}[htbp]
\begin{center}
\includegraphics[width=11cm]{\MemRes/Pics/total_curvature_152_plus_aux.pdf}
\caption{
در هر ردیف جمع انرژی انحنای نمایش داده شده در ردیف متناظر در  شکل
\ref{fig:PlacketEnergy}
با چهار انرژی مساحت، حجم، 
WCAh
و خمش غیر خطی دوسطحی را نمایش می‌دهد. تمامی‌ پارامتر‌ها طبق پارامتر‌های عنوان شده در دو شکل 
\ref{fig:PlacketEnergy}
و
\ref{fig:PlacketEnergyArea}
است. 
}
\label{fig:PlacketEnergyAll}
\end{center}
\end{figure}


در شکل 
\ref{fig:PlacketEnergyAll}
مجموع انرژی انحنا، مساحت، و حجم به همراه انرژی‌های تکمیلی 
WCAh
و خمش غیر خطی دوسطحی رسم شده‌است. مشاهده می‌شود که فضای افت وخیز نقطه‌ی وسط پلاکت برای انرژی‌های بر اساس بریسنتریک به همسایگی پلاکت محدود شده. اما در چشم انداز انرژی مدل‌های انحنای بر پایه‌ی ورنوی چشم انداز انرژی هم بر سطح کره و هم بر سطح صفحه‌ی عمود بر آن، با ناپایداری‌های عددی تزئین شده‌است. 

می‌توان نتیجه‌گیری کرد که هرچند نتایج استاتیک اندازه‌گیری انحنا برای تمامی‌ مدل‌های انحنا امکان پذیر است ولی در صورتی که مش قادر به تغییر شکل در فضا باشد، دقیق‌ترین مدل‌های محاسبه‌ی انحنا (مدل‌های بر پایه‌ی مساحت ورنوی) پایداری لازم را ندارند و مدل‌های بریسنتریک انتخاب خوبی برای شبیه‌سازی دینامیک ملکولی خواهند بود.


