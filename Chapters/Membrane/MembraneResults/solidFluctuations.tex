\begin{figure}[htbp]
\begin{center}
\includegraphics[width=\columnwidth]{\MemRes/Pics/3DAnalysis_low}
\caption{
مقایسه‌ی افت و خیز سطح اندازه‌گیری شده از شبیه‌سازی غشا‌های جامد با دمای‌ موثر مختلف و تحلیل نظری. رفتار طول موج کوتاه (مد‌های بزرگ) توسط نمای جمله‌ی انرژی انحنا 
($\sim \ell^{4}$)
تعیین می‌شود. با افزایش دمای موثر سیستم، طول موج‌های بلند (مد‌های کوچک) رفتار جامدگون
($\sim \ell^{0}$)
 نشان می‌دهند.
}
\label{fig:3DAnalysis_low}
\end{center}
\end{figure}

دامنه‌ی افت و خیز سطح غشای جامد طبق رابطه‌ی 
\ref{eq:ulmSolidTri}
تعیین می‌شود. با فرض اینکه اختلاف فشار داخل و خارج غشا صفر باشد،
$\Delta P =0$
می‌توان رفتار افت و خیز را بر اساس سه پارامتر شعاع غشا
$r_0$,
مدول یانگ 
$Y_{2D}$,
 و سختی خمش 
$\kappa$
بررسی نمود، 
\begin{equation}
\langle|u_{\ell,m}|^2\rangle=\frac{k_BT}{\kappa(\ell+2)^2(\ell-1)^2+Y_{2D}R^2\frac{3(\ell^2+\ell-2)}{3(\ell^2+\ell)-2}}.
\label{eq:ulmSolidTriP0}
\end{equation}
غشا‌های جامد به علت وجود مدول یانگ، شبکه‌هایی با توزیع بسیار منظمی از نقاط دارند (مش‌های تصادفی). عدد فاپِل فون کارمان
\LTRfootnote{F\"oppl-von K\a'arm\a'an}
$\gamma=Y_{2d}r_0^2/\kappa$
ملاک خوبی برای نسبت انرژی کششی به خمشی در سیستم است (جزئیات بیشتر در بخش 
\ref{sec:gammaTransition}
). از آنجایی که مقیاس انرژی با دمای سیستم
$k_BT$
تعیین می‌شود، می‌توان دمای موثر برای بررسی غشاهای جامد را به صورت
\cite{gomppernelson2012}
\begin{equation}
T_{eff}=\frac{k_BT}{\kappa}\sqrt{\gamma}
\label{eq:ulmSolidTriP0}
\end{equation}
تعریف نمود. به عنوان نتایج اولیه، مش‌های تصادفی با مدول یانگ
$Y_{2D}=2.41\times10^{-3}k_BT~[\varepsilon/l^2]$,
 ضریب خمش
$\kappa=35k_BT~[\varepsilon]$,
 و دمای 
$k_BT=2.49~[\varepsilon]$
آماده‌سازی شدند. با تغییر شعاع می‌توان رفتار افت و خیز غشا‌ را برای مقادیر مختلف  دمای موثر بررسی کرد (شکل
\ref{fig:3DAnalysis_low}).
 نقاط مربعی، اندازه‌گیری افت و خیز سطحی مش‌های با 
$N=1002$
راس را نمایش می‌دهد. به علت محدودیت تعداد رئوس، اندازه‌گیری مد‌های بزرگتر از 
$\ell>13$
برای این مش‌ها معنی ندارد. نقاط دایره‌ای پُر، داده‌ی مش‌های با 
$N=25,002$
را نشان می‌دهد. این مش‌ها برای مطالعه‌ی رفتار خمش (مد‌های بالا) استفاده شده است. با تغییر شعاع مش، نسبت انرژی کششی 
$Y_{2D}r_0^2$
به انرژی خمش
$\kappa$
تغییر می‌کند. با توجه به رابطه‌ی
\ref{eq:ulmSolidTriP0}
رفتار توانی انرژی کشسانی از مرتبه‌ی 
$\ell^0$
است و رفتار توانی انرژی خمش از مرتبه‌ی
$\ell^4$
است. با تغییر اندازه‌ی شعاع رقابت میان این دو انرژی در مقیاس لوگاریتمی شکل
\ref{fig:3DAnalysis_low}
به وضوح دیده می‌شود. همانطور که در بخش 
\ref{sec:youngMesh}
بحث شد، مدول یانگ در غشا به شکل فنر‌های هارمونیک تعریف می‌شود. در صورتی که طول‌ اولیه‌فنر‌ها در سراسر مش یکسان باشد شاهد تغییر شکل مش در اعداد فاپل فون کارمان بالا خواهیم بود. اما با تعیین طول اولیه‌ی  اضلاع مش به عنوان طول اولیه‌ی فنرها (تغییر جنس فنر‌ها در محل نقاط نقص) می‌توان با این تغییر شکل مقابله کرد و در نتیجه همانطور که در شکل 
\ref{fig:3DAnalysis_low}
مشاهده می‌کنید اثری از تغییر شکل (تحریک مُدهای ۶ و ۱۳) در این غشا‌ها دیده نمی‌شود.


\begin{figure}[htbp]
\begin{center}
\includegraphics[width=14cm]{\MemRes/Pics/3DAnalysis_resolutions}
\caption{
دامنه‌ی افت و خیز غشای جامد با درجه‌ی وضوح مختلف (تعداد نقاط روی  سطح) مش‌های تصادفی. مشاهده می‌کنیم که رفتار افت و خیز تابع تعداد نقاط مش نیست. 
}
\label{fig:3DAnalysis_resolutions}
\end{center}
\end{figure}

جهت بررسی اثر وضوح مش (تعداد رئوس مش 
$N_m$)
 اندازه‌گیری افت و خیز سطح غشا را با ثابت نگه داشتن مدول یانگ، ضریب سختی خمش، و شعاع مش را با مش‌های مختلف تصادفی تکرار کرده‌ایم. همانطور که در شکل
\ref{fig:3DAnalysis_resolutions}
مشاهده می‌شود، وضوحِ مِش اثر خود را بر رفتار مُد‌های بلند (یا رفتار خمش)  می‌گزارد. در نتیجه جهت مطالعه‌ی این رفتار باید مش با تعداد رئوس مناسب انتخاب شود. 

\begin{figure}[htbp]
\begin{center}
\includegraphics[width=12cm]{\MemRes/Pics/ResearchWeek2.pdf}
\caption{
بررسی اثر اختلاف فشار وارد بر غشای جامد. با افزایش اختلاف فشار، رفتار 
$\sim\ell^2$
در مُد‌های میانی 
$4\leq\ell\leq9$
مشخص‌تر می‌شود.
}
\label{fig:presure}
\end{center}
\end{figure}

در انتها، با قرار دادن ذرات گاز واندروالس درون غشا، صحت رابطه‌ی فشار با دامنه‌ی افت خیز نیز بررسی شده‌است. در اینجا اندیس 
$N_n$
نشان دهنده‌ی تعداد ذرات گازی‌است. در معادله‌ی 
\ref{eq:ulmSolidTri}
که در اینجا باز نویسی شده‌است،
\begin{equation}
\langle|u_{\ell,m}|^2\rangle=\frac{k_BT}{\kappa(\ell+2)^2(\ell-1)^2-pR^3\left[1+\frac{1}{2}\ell(\ell+1)\right]+Y_{2D}R^2\frac{3(\ell^2+\ell-2)}{3(\ell^2+\ell)-2}}
\end{equation}
$p$
اختلاف فشار بیرون و داخل غشا را نشان می‌دهد. در حالتی که فشار داخلی بیشتر باشد، علامت اختلاف فشار در رابطه‌ی فوق منفی‌ است. در شکل 
\ref{fig:presure}
اثر اختلاف فشار با افزایش تعداد ذرات نشان دا‌ده‌ شده‌است. با محاسبه‌ی اختلاف فشار توسط رابطه‌ی گاز واندروالس می‌توان دامنه‌ی افت و خیز غشای تحت فشار را محاسبه نمود. رفتار توانی فشار از مرتبه‌ی 
$\ell^2$
است. در صورتی که تعداد ذرات در خارج از غشا بیشتر باشده، علامت 
$p$
مثبت خواهد بود.





%\begin{figure}[htbp]
%     \centering
%     \begin{subfigure}[b]{0.4\textwidth}
%         \centering
%         \includegraphics[width=7cm]{\MemRes/Pics/snapShotstomato_disco}
%         \caption{Firts subfigure.}
%    	\label{fig:first}
%     \end{subfigure}
%%     \hfill
%     \hspace{2cm}
%     \begin{subfigure}[b]{0.4\textwidth}
%         \centering
%         \includegraphics[width=7cm]{\MemRes/Pics/snapShotliquid_disco}
%         \caption{Second subfigure.}
%    	\label{fig:second}
%     \end{subfigure}
%%     \caption{Three simple graphs}
%%     \label{fig:three graphs}
%\end{figure}

%\begin{figure}[htbp]
%\begin{center}
%\includegraphics[width=10cm]{\MemRes/Pics/snapShotBending}
%\caption{
%شدت نوسانات هماهنگ‌های کروی سطح غشا با حجم کاهیده‌ی 
%$\nu=0.977$.
%خطوط پیش‌بینی شدت‌ نوسانات برای ضریب سختی‌های مختلف طبق معادله‌ی
%\ref{eq:bendingFluctuations}
%را نمایش می‌دهد. نقاط اندازه‌گیری افت و خیز حاصل از شبیه‌سازی با ضرایب سختی مختلف را نشان می‌دهد.
%}
%\label{fig:snapShotBending}
%\end{center}
%\end{figure}
%
%
%
%\begin{figure}[htbp]
%\begin{center}
%\includegraphics[width=10cm]{\MemRes/Pics/snapShotBendingSurface}
%\caption{
%شدت نوسانات هماهنگ‌های کروی سطح غشا با حجم کاهیده‌ی 
%$\nu=0.977$.
%خطوط پیش‌بینی شدت‌ نوسانات برای ضریب سختی‌های مختلف طبق معادله‌ی
%\ref{eq:bendingFluctuations}
%را نمایش می‌دهد. نقاط اندازه‌گیری افت و خیز حاصل از شبیه‌سازی با ضرایب سختی مختلف را نشان می‌دهد.
%}
%\label{fig:snapShotBendingSurface}
%\end{center}
%\end{figure}





%\begin{figure}[htbp]
%\begin{center}
%\includegraphics[width=10cm]{\MemRes/Pics/snapShot}
%\caption{
%شدت نوسانات هماهنگ‌های کروی سطح غشا با حجم کاهیده‌ی 
%$\nu=0.977$.
%خطوط پیش‌بینی شدت‌ نوسانات برای ضریب سختی‌های مختلف طبق معادله‌ی
%\ref{eq:bendingFluctuations}
%را نمایش می‌دهد. نقاط اندازه‌گیری افت و خیز حاصل از شبیه‌سازی با ضرایب سختی مختلف را نشان می‌دهد.
%}
%\label{fig:snapShot}
%\end{center}
%\end{figure}

%\begin{figure}[htbp]
%     \centering
%     \begin{subfigure}[b]{0.4\textwidth}
%         \centering
%         \includegraphics[width=8cm]{\MemRes/Pics/2DAnalysis_low}
%         \caption{2d subfigure.}
%    	\label{fig:first}
%     \end{subfigure}
%%     \hfill
%     \hspace{1cm}
%     \begin{subfigure}[b]{0.4\textwidth}
%         \centering
%         \includegraphics[width=8cm]{\MemRes/Pics/3DAnalysis_low}
%         \caption{3d subfigure.}
%    	\label{fig:second}
%     \end{subfigure}
%%     \caption{Three simple graphs}
%%     \label{fig:three graphs}
%\end{figure}



%\begin{figure}[htbp]
%\begin{center}
%\includegraphics[width=10cm]{\MemRes/Pics/2DAnalysis_low}
%\caption{
%شدت نوسانات هماهنگ‌های کروی سطح غشا با حجم کاهیده‌ی 
%$\nu=0.977$.
%خطوط پیش‌بینی شدت‌ نوسانات برای ضریب سختی‌های مختلف طبق معادله‌ی
%\ref{eq:bendingFluctuations}
%را نمایش می‌دهد. نقاط اندازه‌گیری افت و خیز حاصل از شبیه‌سازی با ضرایب سختی مختلف را نشان می‌دهد.
%}
%\label{fig:2DAnalysis_low}
%\end{center}
%\end{figure}




%\begin{figure}[htbp]
%\begin{center}
%\includegraphics[width=10cm]{\MemRes/Pics/AmpRatio}
%\caption{
%شدت نوسانات هماهنگ‌های کروی سطح غشا با حجم کاهیده‌ی 
%$\nu=0.977$.
%خطوط پیش‌بینی شدت‌ نوسانات برای ضریب سختی‌های مختلف طبق معادله‌ی
%\ref{eq:bendingFluctuations}
%را نمایش می‌دهد. نقاط اندازه‌گیری افت و خیز حاصل از شبیه‌سازی با ضرایب سختی مختلف را نشان می‌دهد.
%}
%\label{fig:AmpRatio}
%\end{center}
%\end{figure}




