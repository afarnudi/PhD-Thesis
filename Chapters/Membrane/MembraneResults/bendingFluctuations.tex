حجم کاهیده‌ی مورد نیاز برای جا دادن افت‌ و خیز‌های سطحی برای غشایی که انرژی خمش دارد به کمک معادله‌ی 
\ref{eq:nuUndulated}
قابل محاسبه‌است. با استفاده از اصل هم پاری انرژی برای محاسبه‌ی شدت نوسانات  غشایی با ضریب سختی 
$k_BT/\kappa=1/20$
حجم کاهیده به شکل عددی
$\nu_{num}=0.97676$
محاسبه‌ می‌شود. غشایی با پتانسیل خمش گامپر-بریسنتریک و مساحت با دینامیک لانژونی شبیه‌سازی شد. در غیاب پتانسیل حجم، تغییرات حجم سیستم هزینه‌ی انرژی نداشت و غشا حجم خود را با توجه به افت و خیز سطحی تنظیم کرد. حجم کاهیده‌ی  متوسط غشاهایی که تحت این شرایط نوسان کردند 
$\nu_{sim}=0.9779\pm0.0003$
اندازه‌گیری شد. 

\begin{figure}[htbp]
\begin{center}
\includegraphics[width=10cm]{\MemRes/Pics/kappa50ulms.pdf}
\caption{
شدت نوسانات هماهنگ‌های کروی سطح غشا با حجم کاهیده‌ی 
$\nu=0.977$.
خطوط پیش‌بینی شدت‌ نوسانات برای ضریب سختی‌های مختلف طبق معادله‌ی
\ref{eq:bendingFluctuations}
را نمایش می‌دهد. نقاط اندازه‌گیری افت و خیز حاصل از شبیه‌سازی با ضرایب سختی مختلف را نشان می‌دهد.
}
\label{fig:kappaULMS}
\end{center}
\end{figure}


نتایج اندازه‌گیری اُفت و خیز‌های سطح غشا با حجم کاهیده‌ی 
$\nu=0.977$
و سختی خمش
$\kappa$
مختلف در شکل
\ref{fig:kappaULMS}
رسم شده‌است. مقادیر سختی خمش 
$10$, $20$, $30$
و
$40k_BT$
انتخاب شد (
$k_BT=2.49[\varepsilon]$
). مشاهده می‌شود که رفتار آماری غشاهای شبیه‌سازی شده با روش توزیع دینامیک مساحت همخوانی خوبی با نتایج نظری دارد.  با افزایش سختی خمش، سطح انعطاف کمتری خواهد داشت و دامنه‌ی افت و خیز کاهش می‌یابد. از آنجایی که اثر انرژی مساحت و حجم از مرتبه‌ی 
$\mathcal{O}(|u_{\ell,m}|^4)$
است، در رفتار مرتبه‌ی دوم دامنه‌ی افت و خیز تاثیر ندارد. مشاهده‌ی رفتار خمش در مُدهای بلند (عدد مد کوچک) قسمت اصلی‌است که فیزیک افت و خیز غشا سیال را از رفتار غشا‌های جامد متمایز می‌کند. 










