\begin{figure}[htbp]
\begin{center}
\includegraphics[width=\columnwidth]{\MemRes/Pics/ulm_amps_N_Phi_kappa_R.pdf}
\caption{
دامنه‌ی نوسانات مش‌های کروی بر حسب مد‌های هماهنگ کروی. در تمامی شکل‌ها مش‌ها بدون پتانسیل‌ حجم (یعنی
$P=0$
یا
$k_V=0$)
 شبیه‌سازی شده‌اند. 
(a)
غشا‌هایی با شعاع
 $r_0=1\mu m$,  $\Phi=0.037$, 
 و
$\kappa=20k_BT$
را نشان می‌دهد که با تعداد نقاط مختلف شبیه‌سازی شده‌است.  مش‌هایی که نقاط بیشتری‌ دارند می‌توانند تغییر شکل‌های ناشی از مد‌های بزرگتر (طول موج کوتاه‌تر) را بهتر نشان دهند. در نمودار
(b)
افت و خیز دامنه‌‌ی مش‌هایی نشان داده شده که 
$r_0=1\mu m$, $\Phi=0.037$,
 و
$N=1002$
یکسان دارند ولی ضریب سختی خمش آنها متفاوت است. در نمودار
(c)
نشان می‌دهیم که طیف افت و خیز غشا از اندازه‌ی مش مستقل است. 
(d)
مش‌هایی را نشان می‌دهد که 
$N=1002$
و
$\kappa=20k_BT$
یکسان دارند ولی نرمی یا 
$\Phi$
آنها متفاوت است. مقادیر زیاد 
 $\Phi$ 
رفتار مش‌ در مدهای کوچک را شبیه به یک جامد نشان می‌دهد. اثر مدول یانگ به نظر می‌رسد که از رابطه‌ی
$Y_{2D}(\Phi)=(\Phi/0.59)^2~1.8\times10^{-3}k_BT.(nm)^{-2}$
پیروی می‌کند. 
(e)
اثر ثابت مدول یانگ موثر، مش‌هایی را نشان می‌دهد که سختی
$\Phi=0.59$
یسکان دارند ولی شعاع آنها متفاوت است. 
(f)
در این نمودار داده‌ها مشابه به نمودار 
(a)
تولید شده ولی به جای مدل خمش گامپر-کرول-بریسنتر از مدل خمش یولیشر استفاده شده‌است. تفاوت واضحی میان این دو طیف مشاهده نمی‌شود. خطوط ممتد رفتار طیف را طبق معادله‌ی
\ref{eq:bendingFluctuations}
نشان می‌دهد. تمامی خطوط نقطه‌چین معادله‌ی 
\ref{eq:ulmShell}
را با ورودی‌های مشخص شده در هر نمودار نشان می‌‌دهد.
}
\label{fig:kappaULMS}
\end{center}
\end{figure}

در بخش‌های گذشته نحوه‌ی تنظیم  شبیه‌سازیِ دینامیکِ مولکولیِ پایدارِ دینامکِ مش را نشان دادیم. حال زمان به کاربردن مِش‌هایِ نرم فرا رسیده‌است. این سوال ایجاد می‌شود که آیا واقعا می‌توان رفتار سیال‌گون غشا را با استفاده از مش‌های نرم باز تولید کرد؟

اولین آزمایشِ کمی، شبیه‌سازی غشا‌های تقریبا کروی با خمش ذاتی صفر خواهد بود. این سیستم را برای انواع پارامترهای مختلف مساله، شعاع، وضوح فضایی، سختی خمش، و نرمی مِش بررسی کرده‌ایم. فیلم
{\bf SM\_Ah\_GKB\_L}
نمونه‌ای از غشاهای در حال افت و خیز با مش‌ نرم 
($\Phi\approx0.037$)
را نشان می‌دهد. نمودارهای مختلف در شکل
\ref{fig:kappaULMS}
وابستگی افت و خیز‌های سطحی بر پارامتر‌های مساله را نشان می‌دهد. تمامی نمودار‌ها بر محور‌های لوگاریتمی رسم شده‌اند. تمامی نمودار‌ها  نشان‌گر  رفتار نمایی سیال‌گون غشا
$\langle|u_{\ell,m}|^2\rangle\propto \frac{1}{\ell^4}$
 در طول‌ موج‌های بلند هستند. در ادامه، ابتدا تاثیر پارامتر‌های فیزیکی بر طیف افت وخیز غشا را بررسی می‌کنیم و سپس به تاثیر نرمیِ مِش بر شکل طیف می‌پردازیم.

در شکل
\ref{fig:kappaULMS}(a)
دامنه‌ی افت و خیز مِش‌هایی رسم شده‌است که شعاع
$r_0$,
ضریب سختی
$\kappa$,
 و نرمی یکسان
 $\Phi$
  داشته ولی تعداد نقاطشان
$N$
متفاوت است. این نمودار نشان می‌دهد که تعداد نقاط مش بر رفتار طیف در طول موج‌های بلند (که با خط ممتد مشخص شده‌است) تاثیر ندارد. همچنین، بنا به انتظار، با افزایش تعداد نقاط مش، دقت در اندازه‌گیری دامنه‌ی مد‌های بزرگ افزایش می‌یابد.

معادله‌ی 
\ref{eq:bendingFluctuations}
پیش بینی می‌شود که طیف افت و خیز موج‌های سطحی یک غشای سیال با ضریب سختی خمش غشا رابطه‌ی عکس دارد
$\langle|u_{\ell,m}|^2\rangle\propto \frac{k_BT}{\kappa}$.
 وقتی که ضریب سختی غشا بیشتر می‌شود، سطح غشا سخت‌تر شده و دامنه‌ی نوسان کاهش می‌یابد.  نمودار
\ref{fig:kappaULMS}(b)
 طیف مش‌های نرم که از 
$N=1002$ 
نقطه تشکیل شده و دارای
$\Phi=0.037$ 
را نشان می‌دهد که ضرایب سختی خمش مختلفی دارند. مش‌های شبیه‌سازی شده این اثر را به خوبی نشان می‌دهند. 

از طرف دیگر، دامنه‌ی افت و خیز مش‌های با ضریب سختی خمش یکسان مستقل از اندازه‌ی غشاست. نمودار 
\ref{fig:kappaULMS}(c)
تایید می‌کند که مش‌های با شعاع مختلف همگی رفتار یکسان دارند. در این نمودار از مش‌هایی با 
$N=252$, $496$, $1002$, 
 و
$1962$
نقطه استفاده شده که به ترتیب قطر
 $r_0=0.5\mu m$, $0.7\mu m$, $1\mu m$,
 و 
$1.5 \mu m$
دارند.


در نمودار
\ref{fig:kappaULMS}(d)
مش‌های تصادفی با تعداد نقاط
$N=1002$
و شعاع
$r_0=1 \mu m$
با نرمی مختلف شبیه‌سازی شده‌است. مش‌های نرم به خاطر وجود پتانسیل تکمیلی
$U_h$
رفتار الاستیکی محدودی به علت برخورد نقاط با رئوس دارد. این اثر برای مش‌های بسیار نرم
 ($\Phi\ll1$)
تقریبا قابل صرف نظر است. اما زمانی که 
$\Phi$
افزایش می‌یابد، رفتار طیفی‌ مش‌ها شبیه به غشای جامد می‌شود. طیف غشاهای جامد رفتار
($\sim\ell^0$)
برای مدهای کوچک (تغییر شکل‌های بزرگ) دارد و این رفتار در مد‌های بزرگ (تغییر شکل‌های کوچک) به حالت مایع گون
($\sim\ell^4$)
تبدیل می‌شود. در نمودار
\ref{fig:kappaULMS}d
طیف افت و خیز مش‌های با نرمی 
$\Phi$
بسیار کوچک شبیه به غشای مایع رفتار می‌کند ولی مش‌هایی که نرمی 
$\Phi\geq0.15$
دارند رفتارشان شبیه به غشای جامد می‌شود (رفتار از خط ممتد مشکی فاصله می‌گیرد). خط‌چین‌های رسم شده در این نمودار رفتار الاستیک غشا با فرض 
$\Delta p=0$
با مقداری حدسی برای مدول یانگ دو بعدی غشا
$Y_{2D}$
را نشان می‌دهند. خطوط نشان می‌دهد که رفتار این مش‌ها کاملا با معادله‌ی
\ref{eq:ulmShell}
سازگار است. پیشبینی می‌کنی که مدول یانگ دو بعدی مش‌‌های نرم با رابطه‌ی زیر قابل محاسبه است
\begin{equation}
Y_{2D}(\Phi)= \left(\frac{\Phi}{0.59}\right)^2~1.8\times 10^{-3}~k_BT/(Nm)^2.
\label{eq:YoungPhi}
\end{equation}

تعداد مد‌های قابل اندازه‌گیری به تعداد نقاط روی مش بستگی دارد. در نمودار
\ref{fig:kappaULMS}(e)
مش‌هایی با ضریب سختی خمش، نرمی، وضوح فضایی یکسان
$r_0^2/N$,
 ولی اندازه‌ی شعاع‌های مختلف شبیه‌سازی شده‌اند. مدول یانگ‌ با استفاده از معادله‌ی
\ref{eq:YoungPhi}
محاسبه شده و برای توصیف رفتار مش‌ها استفاده شده‌است. شعاع کوچک‌ترین مش
$0.5\mu m$ ($N=252$)
و شعاع بزرگترین مش
$1.5\mu m$ ($N=4002$)
است. نتایج فرضیه ما برای وابستگی مدول یانگ به نرمی مش
$\Phi$
را تایید کرده و روش تنظیم نرمی مش را نشان می‌دهد.

در انتها نتایج پتانسیل گامپر-کرول-بریسنتر در نمودار 
\ref{fig:kappaULMS}(a)
 را با پتانسیل یولیشر 
\cite{Julicher1996}
باز تولید کرده و در نمودار 
\ref{fig:kappaULMS}(f)
رسم نمودیم. از آنجایی که هر دو پتانسیل از نسبت وزنی بریسنتر برای نقاط مش استفاده می‌کنند به لحاظ شبیه‌سازی بسیار پایدار هستند. 























