
فرض می‌کنیم که شبکه‌ای را بررسی می‌کنیم که فاصله‌ی متوسط بین تمام نقاط به اندازه‌ی $a$ باشد. هر گونه تغییر شکل در شبکه یک نقطه از شبکه را از $r_a$ به $r_a'$ جابجا خواهد کرد. در نتیجه می‌توان انرژی کشش را به شکل زیر تعریف کرد (شکل
\ref{fig:mesh_def}).
\begin{figure}[h]
\begin{center}
\includegraphics[width=6in]{\MemDiscr/Pics/mesh_def.pages.pdf}
\caption{
تغییر شکل مش
}
\label{fig:mesh_def}
\end{center}
\end{figure}

\begin{equation}
E_s^{discrete}=\frac{1}{2}\epsilon_s\sum_{\langle a,b\rangle}\left(|r_a'-r_b'|-a\right)^2
\label{eq:stretchdiscrete}
\end{equation}
که جمع روی تمام جفت‌های $a$ و $b$ است که شامل تغییر شکل شده‌اند.  همچنین می‌توان جمع بالا را به شکل چگالی موضعی انرژی حول نقاط شبکه و جمع روی همسایگی‌ آن نقاط تعریف کرد،

\begin{equation}
\begin{aligned}
&E_s^{discrete}=\frac{1}{2}\epsilon_s\sum_aU_a\\
&U_a=\frac{1}{2}\sum_b\left(|r_a'-r_b'|-a\right)^2
\end{aligned}
\end{equation}
برای محاسبه‌ی حد پیوستگی فرض می‌کنیم که نقشه‌‌ی تغییر شکل پیوسته‌ای وجود دارد که نقاط 
$r\rightarrow r'$
که معادل نقشه‌ی گسسته‌ی شبکه‌ی ماست
$r_a\rightarrow r_a'=r_a+u_a(r_a)$
. اگر تانسور متریک این تغییر شکل به شکل زیر تعریف شده باشد،
\begin{equation}
g_{ij}=\partial_i r'\cdot\partial_jr'
\end{equation}
در نتیجه می‌توانیم تغییر شکل گسسته را به شکل زیر تخمین بزنیم،

\begin{equation}
\begin{aligned}
|r_a'-r_b'|&\approx \left[g_{ij}(r_a)r_{ab}^ir_{ab}^j\right]^{1/2}\\
&= \left[g_{ij}(r_a)r_{ab}^ir_{ab}^j\right]^{1/2}\\
&= \left\{\left[\delta_{ij}+2u_{ij}(r_a)\right]r_{ab}^ir_{ab}^j\right\}^{1/2}\\
&= a\left[1+2u_{ij}(r_a)\frac{r_{ab}^ir_{ab}^j}{a^2}\right]^\frac{1}{2}\\
&\approx a\left[1+u_{ij}(r_a)\frac{r_{ab}^ir_{ab}^j}{a^2}\right]
\label{eq:gstrain1}
\end{aligned}
\end{equation}
که در رابطه‌ی بالا تانسور متریک را با تانسور تنش جاگذاری کردیم،
$g_{ij}=\delta_{ij}+2u_{ij}$
از آنجایی که اندیس $b$ بین تمامی همسایه‌ی $a$ تعریف می‌شود و همچنین بردار فاصله‌
$r_{ab}=r_a-r_b$
روی بردار‌های
$d_\beta$
 شبکه‌ی شش ضلعی تعریف می‌شود می‌توانیم انرژی موضعی را به این ترتیب محاسبه‌ کنیم،

\begin{equation}
\begin{aligned}
U_a&=\frac{1}{2}\sum_{\beta=1}^6(u_{ij}\frac{d_\beta^id_\beta^j}{a})^2\\
&=\frac{1}{2a^2}\sum_{\beta=1}^6u_{ij}u_{kl}d_\beta^id_\beta^jd_\beta^kd_\beta^l\\
&=\frac{1}{2a^2}a^2u_{ij}u_{kl}(\delta_{ij}\delta_{kl}+\delta_{ik}\delta_{jl}+\delta_{il}\delta_{jk})\cos^2(\pi/3)\\
&=\frac{3}{8}(2u_{ij}^2+u_{kk}^2)
\label{eq:gstrain1}
\end{aligned}
\end{equation}
در نتیجه‌ حد پیوسته انرژی کشسانی را می‌توان به شکل زیر نوشت
\begin{equation}
\begin{aligned}
E_s^{discrete}=\frac{1}{2}\epsilon_s\sum_\alpha U_a&\approx\frac{1}{\sqrt3}\epsilon\int d^2rU(r)\\
&\approx\frac{\sqrt3}{8}\epsilon_s\int d^2r(2u_{ij}^2+u_{kk}^2)
\end{aligned}
\end{equation}
با مقایسه با معادله‌ی 
\ref{eq:energylame}
می‌توانیم ضرایب لم را بخوانیم
\begin{equation}
\lambda=\mu=\frac{\sqrt3}{4}\epsilon_s
\end{equation}
با داشتن ضرایب لم می‌توانیم با توجه به معادله‌ی 
\ref{eq:younglame}
مدول ۲ بعدی یانگ و ضریب پواسون را برای این شبکه محاسبه کنیم،
\begin{equation}
\begin{aligned}
Y&=\frac{4\mu(\mu+\lambda)}{2\mu+\lambda}=\frac{2}{\sqrt3}\epsilon_s\\
\nu&=\frac{\lambda}{2\mu+\lambda}=\frac{1}{3}
\end{aligned}
\end{equation}
همانطور که می‌بینیم برای مش‌های مثلثی ۶ ضلعی، مدول یانگ و نسبت پواسون به اندازه‌ی مش بستگی ندارد. محاسبات عددی
\cite{springnetworkPRE2011}
نیز این نتایج را تایید می‌کنند.




.
 
 
 
 
 
 
 
 
 
 
 
 
 
 
 