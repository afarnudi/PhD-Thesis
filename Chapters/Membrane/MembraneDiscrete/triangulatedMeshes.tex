محاسبه‌ی انرژی از طریق مطالعه‌ی افت و خیز سطح روش مناسبی برای بررسی اشکال شبه کُروی است اما برای بررسی عمومی اشکال غیر کروی مشخصات آن باید مستقیم اندازه‌گیری شود. مدل‌های پیوسته‌ی غشا را می‌توان بر روی شبکه‌های مثلثی تعریف کرد. با توزیع یکنواخت نقاط بر روی یک سطح و اتصال هر نقطه  به همسایه‌هایش می‌توان شبکه‌ی مثلثی ساخت. در صورتی که نقاط تنها به همسایه‌های نزدیکشان متصل شوند شبکه‌ی مثلثی تشکیل خواهد شد که هیچ مثلثی در آن با مثلث دیگری همپوشانی ندارد و سطح با مثلث کاملا کاشی می‌شود. شبکه‌های مثلث خواص بسیار جذابی دارند. در شبکه‌‌ی مثلثی تخت با توزیع نقاط یکنواخت، هر نقطه دقیقا ۶ همسایه دارد و آن را با مفهومی به نام درجه‌ی نقطه
\LTRfootnote{vertex degree}
بیان کرده و تعداد نقاط با درجه‌ی 
$n$
را با 
$\vartheta_n$
نمایش می‌دهیم. در مورد سطوح با هندسه‌ی بسته یا سطوح با انحنای زیاد تعداد درجات دیگر ثابت نخواهد بود. اویلر نشان داد که اختلاف تعداد  نقاط با درجات ۵ و ۷ برای سطح با هندسه‌ی بسته تابع جینوس
$\chi$
یا مشخصه‌ی اویلر سطح است
\cite{Nguyen2005PRE}
،


\begin{equation}
\vartheta_D=\vartheta_5-\vartheta_7=12(1-\chi)
\label{eq:vertexDegreeGenus}
\end{equation}
شبکه‌های مثلثی را می‌توان به ۴ گروه شبکه‌ی منظم
\LTRfootnote{ordered}
، تصادفی
\LTRfootnote{random}
، منظم دَرهَم
\LTRfootnote{fluctuating ordered}
، و تصادفی درهم
\LTRfootnote{fluctuating random}
دسته بندی کرد. شبکه‌ی منظم شبکه‌ای است با توزیع یکنواخت نقاط که دارای کمترین تعداد نقاط با درجه‌ی به غیر از ۶ است. برای یک کره، طبق رابطه‌ی 
\ref{eq:vertexDegreeGenus}
شبکه‌ی مثلثی است با ۱۲ نقطه با درجه‌ی ۵
$\vartheta_5=12$
و باقی نقاط با درجه‌ی ۶ (
$\vartheta_7=0$
). در این صورت محل نقاط با درجه‌ی ۵ در گوشه‌های یک ۲۰ وجهی منتظم
\LTRfootnote{Icosahedron}
 مماس به کره خواهد بود.
 
 شبکه‌های تصادفی نیز توزیع کم و بیش یکنواختی از نقاط دارند ولی تعداد نقاط با درجه‌ی ۵، ۶، ۷،  و بیشتر در آن بسیار متغیر است. با افزایش دقت در توزیع نقاط بر روی سطح می‌توان درجه‌ی نقاط تشکیل شده را به ۵، ۶، و ۷ محدود کرد. 
 
شبکه‌های درهم منظم یا تصادفی با تغییر توزیع نقاط در شبکه‌های منظم و تصادفی ساخته می‌شوند. توزیع نقاط تحت قیود زیر  پذیرفته می‌شود:

۱) ساختار شبکه‌ی اولیه در فرآیند تغییر مکان نقاط ثابت بماند. به عبارت دیگر توپولوژی شبکه‌ی درهم منظم با توپولوژی شبکه‌ی منظم برابر است.

۲) مثلث‌های جدید حاصل از تغییر توزیع مکان نقاط همچنان سطح را بدون همپوشانی کاشی کند.

۳) حد کمینه برای اندازه‌ی مثلث‌های تشکیل شده روی سطح قابل تعریف باشد (مثلث با مساحت صفر معنی ندارد).
