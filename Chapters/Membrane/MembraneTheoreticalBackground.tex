\setRL
\clearpage

\def \MemTB {\Mempath /MembraneTheoreticalBackground}

\section{
چکیده
}
در این فصل با مقدمه‌ای از نقش غشای زیستی در سلول شروع کرده. سپس مروری کوتاهی بر تارخچه‌ی تحقیقات انجام شده در جهت کشف و بررسی ساختار غشای سلول‌های زنده در قرن‌های ۱۹ و ۲۰ میلادی شده‌است. در مورد ساختار واحد‌های تشکیل دهنده‌ی غشا بحث شده و در نهایت در مورد چیدمان غشای هسته یا بسته‌ی هسته توضیح داده شده‌است.


\subsection{
تنش حاصل از تغییر مساحت
}
همانطور که قبل‌تر توضیح داده شد سطح غشا بسته به شرایط ترمودینامیکی محیط یک مقدار تعادلی 

دارد. هر پدیده‌ای که سطح غشا را تغییر دهد در آن تنش 
$\gamma_{st}(A)$
ایجاد خواهد کرد. باید توجه ویژه داشت که غشا ماهیت سیال‌گون دارد، در نتیجه تنش‌هایی که تنها باعث جابجایی ملکول‌ها روی سطح شود (تنش برشی
\LTRfootnote{shear}
) انرژی سطح غشا را تغییر نمی‌دهد. تنها تنش‌هایی که مساحت کل غشا را تغییر دهد با مقاومت روبرو خواهد شد. تنش غشا تا مرتبه‌ی اول در جمله‌ی 
$(A-A_0)$
به شکل زیر تعریف می‌شود،
\begin{equation}
\gamma_{st}=k_A\frac{A-A_0}{A_0}
\end{equation}
. در این معادله
$k_A$
مدول فشردگی سطحی
\LTRfootnote{area compressibility modulus}
است. بدیحی‌است که  تنش ایجاد شده باید از مقدار تنش آستانه‌ی پاره شدن غشا کمتر باشد. برای غشاهای لیپیدی مقدار تنش آستانه‌ی پاره شدن دو مرتبه‌ی بزرگی کوچک‌تر از مدول فشردگی سطحی و حدود چند
$mN/m$
است. انرژی تنش ایجاد شده معدل کار انجام شده برای تغییر سطح است،
\begin{equation}
E_{st}(A)=\int \gamma_{st}()=\frac{1}{2}k_A\frac{(A-A_0)^2}{A_0}
\label{eq:surfaceTension}
\end{equation}
مشابه به این بحث می‌توان هزینه‌ی انرژی برای تغییر حجم سیال درون غشا را نیز به صورت زیر تعریف کرد،
\begin{equation}
E_{v}(V)=\frac{1}{2}k_V\frac{(V-V_0)^2}{V_0}
\label{eq:volumeEnergy}
\end{equation}
. که در اینجا
$k_V$
مدول فشردگی حجمی
\LTRfootnote{volume compressibility modulus}
، و 
$V_0$
حجم تعادلی سیال است.








\subsection{
تنش حاصل از تغییر مساحت و حجم
}
همانطور که قبل‌تر توضیح داده شد سطح غشا بسته به شرایط ترمودینامیکی محیط یک مقدار تعادلی 

دارد. هر پدیده‌ای که سطح غشا را تغییر دهد در آن تنش 
$\gamma_{st}(A)$
ایجاد خواهد کرد. باید توجه ویژه داشت که غشا ماهیت سیال‌گون دارد، در نتیجه تنش‌هایی که تنها باعث جابجایی ملکول‌ها روی سطح شود (تنش برشی
\LTRfootnote{shear}
) انرژی سطح غشا را تغییر نمی‌دهد. تنها تنش‌هایی که مساحت کل غشا را تغییر دهد با مقاومت روبرو خواهد شد. تنش غشا تا مرتبه‌ی اول در جمله‌ی 
$(A-A_0)$
به شکل زیر تعریف می‌شود،
\begin{equation}
\gamma_{st}=k_A\frac{A-A_0}{A_0}
\end{equation}
. در این معادله
$k_A$
مدول فشردگی سطحی
\LTRfootnote{area compressibility modulus}
است. بدیحی‌است که  تنش ایجاد شده باید از مقدار تنش آستانه‌ی پاره شدن غشا کمتر باشد. برای غشاهای لیپیدی مقدار تنش آستانه‌ی پاره شدن دو مرتبه‌ی بزرگی کوچک‌تر از مدول فشردگی سطحی و حدود چند
$mN/m$
است. انرژی تنش ایجاد شده معدل کار انجام شده برای تغییر سطح است،
\begin{equation}
E_{st}(A)=\int \gamma_{st}()=\frac{1}{2}k_A\frac{(A-A_0)^2}{A_0}
\label{eq:surfaceTension}
\end{equation}
مشابه به این بحث می‌توان هزینه‌ی انرژی برای تغییر حجم سیال درون غشا را نیز به صورت زیر تعریف کرد،
\begin{equation}
E_{v}(V)=\frac{1}{2}k_V\frac{(V-V_0)^2}{V_0}
\label{eq:volumeEnergy}
\end{equation}
. که در اینجا
$k_V$
مدول فشردگی حجمی
\LTRfootnote{volume compressibility modulus}
، و 
$V_0$
حجم تعادلی سیال است.








\subsection{
تغییر شکل غشا
}
\begin{figure}[h]
\begin{center}
\includegraphics[width=4.5in]{\MemTB /Pics/polymorphism}
\caption{
تغییر شکل یک غشا حاصل تغییر دمای محیط. شکل غشا ابتدا دمبلی است 
$(D)$
سپس دو شکل میانی ستومتوسایت
$S_1$
و
$S_2$
تشکیل شده و در نهایت پس از اتصال دو بازو، دو کُره‌ی تو در تو 
$L^{sto}$
تشکیل می‌دهد 
\cite{Berndl1990EPL}.
}
\label{fig:allAtom}
\end{center}
\end{figure}

درسته که غشاها رفتار سیال‌گون دارند ولی رفتار آنها با یک قطره‌ی مایع متفاوت است. در غیاب نیروی خارجی و قیود، قطره‌ی مایع به منظور کاهش انرژی سطح مشترک آن با محیط به شکل کُره در می‌آید. برخلاف یک قطره، غشا می‌توانند اشکال مختلفی همچون دیسکوسایت
\LTRfootnote{discocyte}
، ستومَتوسایت
\LTRfootnote{stomatocyte}
، و  دَمبِلی داشته باشد. همچنین شرایط ترمودینامیکی محیط (فشار اسمزی و دما) می‌تواند باعث تغییر شکل آن شود. از آنجایی که ملکلول‌های لیپیدی حل‌شوندگی ناچیزی دارند می‌توان فرض کرد که تعداد ملکلول‌های لیپیدی هنگام تغییر شکل غشا ثابت است. علاوه بر آن فاصله‌ی تعادلی ملکلول‌های لیپیدی تابع دمای محیط است. از طرفی تغییر مساحت غشا بر اثر نیرو‌های خارجی یا قیود مختلف بیسار ناچیز است و تغییر مساحت زیاد باعث پاره شدن غشا می‌شود. در نتیجه هنگام تغییر فشار اسمزی (کاهش یا افزایش سیال داخل آن)  در دمای ثابت، مساحت غشا با تقریب بسیار خوبی ثابت است. به طور عمومی حجم غشا می‌تواند به مقدار خیلی زیادی کاهش یابد ولی هرگز نمی‌تواند از حجم یک کُره بیشتر شود. 

در صورتی که دمای محیط تغییر کند، هم مساحت غشا هم حجم سیال درون آن تغییر خواهد کرد. نسبت سطح به حجم حاصل از دمای جدید شکل تعادلی غشا را مشخص خواهد کرد. در صورتی که دمای محیط به میزان 
$\Delta T$
تغییر کند، سطح و حجم غشا از حالت تعادلی
$A_0$
و
$V_0$
به مقدار 
$\Delta A=\alpha_A\Delta T A_0$
و حجم سیال درون آن 
$\Delta V=\alpha_V\Delta T V_0$
تغییر می‌کند. برای یک غشای لیپیدی که سیال درون آن آب باشد،
$\alpha_A\approx2\times 10^{-3}/K$
و
$\alpha_V\approx2\times 10^{-4}/K$
. یک محاسبه‌ی سر انگشتی نشان می‌دهد که در صورت افزایش دمای محیط نسبت حجم به سطح غشا کاهش پیدا می‌کند. نمونه‌ای از تغییر شکل غشا حاصل از افزایش دمای محیط در شکل 

نمایش داده شده‌است.


\subsection{
حجم کاهیده
}
همانطور که در فصل اول توضیح داده شد، غشاهای زیستی نمیه‌تراوا هستند. یعنی حجم آب درون آنها توسط دمای محیط، غلظت مولکول‌های محیط، و مولکول‌های درون آن (یا به طور عمومی شرایط اسمزی محیط) مشخص می‌شود. در صورتی که اختلاف فشار اسمزی  بیرون غشا بالا باشد، غشا ممکن است مچاله شود، بر روی خود تا شود، و یا غشا‌های کوچک‌تری تشکیل دهد. مولکول‌های لیپید بسته به شرایط محیطی (مثلا دما) با چیدمان مشخصی فضا را اشغال می‌کنند. با فرض اینکه تعداد مولکول‌های غشا تغییر نکند، سطح غشا در طول عمر آن ثابت خواهد بود. از طرفی، تحت اختلاف فشار اسمزی منفی، غشا متورم خواهد شد. البته میزان تغییر سطح غشا فقط در حدود چند درصد است و در صورتی که نیاز به تغییر سطح بیشتری باشد، پاره خواهد شد. برای غشایی با مساحت
$A$
شعاع کُره‌ای معادل که همان مساحت را داشته باشد،
\begin{equation}
R_{ve}=\sqrt{\frac{A}{4\pi}}
\end{equation}
است. از آنجایی که بیشترین حجمی که غشا می‌تواند داشته باشد حجم یک کُره‌ است، حجم غشا همیشه کمتر مساوی این مقدار خواهد بود،
\begin{equation}
V\leq\frac{4\pi}{3}R_{ve}^3=\frac{4\pi}{3}\left(\frac{A}{4\pi}\right)^{\frac{3}{2}}.
\end{equation}
در نتیجه می‌توان حجم کاهیده غشا را به شکل زیر تعریف کرد،
\begin{equation}
\nu=\frac{V}{\frac{4\pi}{3}R_{ve}^3}=6\sqrt{\pi}VA^{-\frac{3}{2}},
\label{eq:reducedVolume}
\end{equation}
که همیشه کوچک‌تر از یک است. در صورتی که غشا شکل کُره‌ای بی نقص داشته باشد، حجم کاهیده برابر یک خواهد بود.



\section{
انحنای غشا
}
\subsection{
تعریف خمش
}

ضخامت غشا از مرتبه‌ی چند نانومتر است و مقیاس انحنا‌هایی که شکل کلی غشا را مشخص می‌کنند از مرتبه‌ی بزرگی میکرون بوده. در نتیجه به طور عمومی مدل کردن شکل و انحنای غشا با یک رویه‌ی ۲ بعدی کاملا مورد قبول است. تصاویر میکروسکوپی غشا مانند شکل‌های 
\ref{fig:budding}
و
\ref{fig:flucmem} 
نشان می‌دهد که غشا سطحی نسبتا صاف و پیوسته دارد. در نتیجه در مقیاس‌های میکرومتری غشا را با رویه‌ای با چنین ویژگی‌ای مدل می‌کنیم. البته که این فرض در مقیاس نانومتری پابرجا نیست. 

\begin{figure}[h]
\begin{center}
\includegraphics[width=6in]{\Mempath/Pics/Membrane_fluctuations}
\caption{
مجموعه تصاویر پست سر هم از تغییر شکل یک غشای لیپیدی را با تصویر برداری فلورسانت در بازه‌های ۵ ثانیه‌ای نشان می‌دهد. خط مقیاس سفید رنگ اندازه‌ی ۵ میکرومتر را نشان می‌دهد. 
\cite{ParthasarathyMembraneMeasurement}
}
\label{fig:flucmem}
\end{center}
\end{figure}

ملکول‌های غشا درون یک سیال غوطه‌ور است که تحت افت خیز ترمودینامیکی محیط در جهت‌ موازی سطح غشا و عمود بر آن در حال حرکت است. در نتیجه برای تعریف انحنا لازم است که سطح غشا به بخش‌های کوچک تقسیم بندی شود و با میانگین‌گیری رو  مکان ملکول‌ها انحنا را تغیین کرد. اندازه‌ی این بخش‌ها تابع شدت افت و خیز ملکول‌هاست. محاسبات حاصل از  شبیه‌ سازی دینامیک ملکلولی برای غشایی که ملکول‌های یکسان داد حدود ۱.۵ برابر ضخامت غشا گزارش شده است
\cite{Goetz1998}
. از آنجایی که برای یک غشای ساده با ضخامت 
$4nm$
خمش هر بخش حاصل از رفتار دست جمعی حدود 
$100$
ملکول لیپیدی خواهد بود. در مقیاس بزرگ هر کدام از این بخش‌ها یک نقطه بر روی رویه‌ی غشا هستند. ما می‌توانیم برای هر نقطه روی غشا یک صغحه‌ی مماس و یک صفحه‌ی عمود بر مماس تعریف کنیم. صفحه‌ی عمود بر سطح با رویه‌ی غشا فصل مشترکی به شکر یک خط دارد.  (مانند شکل 
\ref{fig:normalPlaneIntersection}
).
\begin{figure}[h]
\begin{center}
\includegraphics[width=6in]{\MemTB/Pics/NormalPlane}
\caption{
خم حاصل از فصل مشترک صفحه‌ی عمود بر سطح غشا در نقطه‌ی 
P
را نشان می‌دهد.
}
\label{fig:normalPlaneIntersection}
\end{center}
\end{figure}
برای خم تشکیل شده از فصل مشترک می‌توان خمش
$C$
تعریف کرد به صورتی که اگر انحنا در جهت بردار عمود بر سطح باشد (
$\cap$
) آنرا با علامت مثبت و در حالتی که در جهت مخالف باشد (
$\cup$
) آنرا با علامت منفی نشان می‌دهیم. می‌توان فرض کرد که خط مشترک قسمتی از یک دایره  است و خمش این خط عکس شعاع دایره خواهد بود. یک انتخاب برای صفحه‌ی مماس بر سطح وجود ندارد ولی صفحه‌ی عمود بر سطح می‌تواند در جهت‌های مختلف تعریف شود که خمش‌های مختلفی را تعریف کند. اگر تمام خمش‌های ممکن در یک نقطه‌ را با تغییر جهت صفحه‌ی متعامد اندازه‌گیری کنیم، اندازه‌گیری ما در یک بازه‌ای محدود به مقادیر کمینه و بیشینه‌ی خمش در آن نقطه قرار خواهد گرفت،
$C_{min}$
و
$C_{max}$
. مقادیر کمینه و بیشینه خمش به خمش‌های اصلی سطح معرف هستند که با 
$C_1$
و
$C_2$
نمایش داده می‌شوند. خمش‌های اصلی همچنین  ویژه‌مقدار‌های تانسور خمش در آن نقطه هستند. همچنین اگر خمش‌های اصلی برابر یکدیگر نباشند،
$C_1\neq C_2$ 
صفحاتی که خم‌ها با آن تعریف می‌شوند حتما عمود بر هم خواهند بود. از آنجایی که ملکول‌های سطح غشا حرکت پخشی می‌کنند  شکل غشا باید بر اساس تعاریفی باشد که تحت تغییر روش پارامتریزه
\LTRfootnote{paprameterisation}  
 کردن سطح ناوردا باشد. خمش‌های اصلی سطح چنین ویژگی دارند. خمش میانگین در هر نقطه‌ بر روی سطح به صورت 
\begin{equation}
M=\frac{1}{2}(C_1+C_2)
\label{eq:meanCurv}
\end{equation}
و خمش گاووسی به صورت
\begin{equation}
G=C_1C_2
\label{eq:gaussianCurv}
\end{equation}
 تعریف کرد. خمش میانگین معادل رَد
\LTRfootnote{trace} 
 تانسور خمش و خمش گاووسی برابر با دترمینان این تانسور است. همچنین می‌توان روابط بالا را بازنویسی کرد و خمش‌های اصلی را بر حسب خمش میانگین و خمش گاووسی محاسبه کرد،

\begin{figure}[h]
\begin{center}
\includegraphics[width=6in]{\MemTB/Pics/curvatureSign}
\caption{
قرارداد برای تعیین علامت خمش میانگین. با فرض اینکه بالا محیط خارج غشا و پایین داخل غشا را مشخص کند، بردار نرمال غشا جهت (بردار قرمز رنگ) رو به بالا خواهد داشت. در شکل الف) خمش میانگین مثبت هنگامی که هر دو خمش اصلی برآمدگی در جهت بردار نرمال داشته باشند. ب) برای قسمت تخت خمش میانگین صفر، ج) خمش منفی هنگامی که  برآمدگی به سمت داخل غشا باشد، د) در نقاط زین اسبی خمش‌های اصلی علامت‌های مخالف یکدیگر دارند و در این صورت خمش میانگین مقدار کمی خواهد داشت.
}
\label{fig:curvatureSign}
\end{center}
\end{figure}

\begin{equation}
\begin{aligned}
C_1&=M-\sqrt{M^2-G}\\
C_2&=M+\sqrt{M^2-G}.
\label{eq:gaussianCurv}
\end{aligned}
\end{equation}
که در روابط بالا، از آنجایی که 
$M^2\geq G$
\cite{Seifert1991}
 هر دو مقدار همیشه حقیقی هستند. مقدار خمش میانگین، 
 $M$،
 تحت تمامی تبدیل‌های دستگاه مختصات که دترمینان ژاکوبین آن مثبت باشد (جهت بردار عمود بر سطح را تغییر ندهد) تغییر نخواهد کرد. به طور مثال اگر یک سطح با پارامتر‌های 
 $(s^1,s^2)$
 تعریف شده باشد، تبدیلی که پارامتر‌ها را با یکدیگر تعویض کند، 
 $(s^{-1}\equiv s^2,s^{-2}\equiv s^1)$
 تبدیلی است که جهت نرمال سطح را تغییر می‌دهد که در نتیجه علامت خمش میانگین را تغییر می‌دهد. هرچند که چنین تبدیل‌هایی در فیزیک بسیار مهم هستند زیرا که انتخاب دستگاه مختصات بر مشخصات برخی خواص فیزیکی نباید تاثیرگذار باشد، ولی در مورد خمش باید تعریف مشخصی برای محاسبات وجود داشته باشد تا بتوان میان محیط داخل و خارج غشا تمییز قائل بشویم. با توجه به رابطه‌ی
 \ref{eq:meanCurvature}
 علامت خمش میانگین در هر نقطه روی غشا تابع مقادیر خمش‌های اصلی در آن نقطه‌ است. شکل 
 \ref{fig:curvatureSign}
 حالت‌های مختلف که بر علامت خمش میانگین تاثیر می‌گذارد را نشان می‌دهد. 

به طور کلی، برای سطح تخت خمش میانگین صفر است، در صورتی که برآمدگی خمش به سمت داخل غشا باشد، خمش میانگین منفی و در صورتی که برآمدگی به سمت بیرون غشا باشد، خمش میانگین مثبت خواهد بود. در نقاط زین اسبی خمش‌های اصلی علامت‌های مخالف یکدیگر دارند و در نتیجه مقدار خمش میانگین بسیار کوچک خواهد بود. مهم است که اشاره شود که خمش میانگین ابزار مناسبی برای اندازه‌گیری تاثیر نقاط زین اسبی نیست زیراکه مقدار آن با سطح تخت اختلاف چندانی ندارد. برای اندازه‌گیری تاثیر نقاط زین اسبی، خمش گاووسی ابزار مناسبی است.
\begin{figure}[h]
\begin{center}
\includegraphics[width=6in]{\MemTB/Pics/simpleMembraneShapes}
\caption{
شکل‌های ساده‌ی غشا که در تمام نقاط روی سطح خمش میانگین ثابتی دارند. شکل الف، کُره‌ای به شعاع 
$R_{sp}$
با خمش میانگین 
$M=\pm 1/R_{sp}$
ب، استوانه‌ای با شعاع 
$R_{cy}$
با خمش میانگین
$M=\pm 1/2R_{cy}$
و در نهایت ج، کتانوید با خمش میان صفر. علامت خمش میانگین برای کُره و استوانه به این بستگی دارد که محیط بیرون، سیال خارج از غشا تعریف شود یا سیال داخل.
}
\label{fig:simpleMembraneShapes}
\end{center}
\end{figure}
به طور عمومی خمش میانگین یک کمیت موضعی است و در نقاط مختلف روی سطح غشا تغییر می‌کند. اما برخی اَشکال ساده در تمامی نقاط روی سطح خود یک مقدار ثابت خمش میان‌گین دارند. برای مثال خمش میانگین یک غشای تخت صفر است. برای مثال‌های بیشتر با شکل 
\ref{fig:simpleMembraneShapes}
توجه کنید. خمش میانگین یک کُره با شعاع
$R_{sp}$
برابر با 
$C=1/R_{sp}$
هنگامی که لایه‌ی خارجی آن با محیط بیرون غشا در ارتباط است و 
$C=-1/R_{sp}$
زمانی که لایه‌ی داخلی آن با محیط بیرون در ارتباط است. همچنین استوانه‌‌ای با شعاع 
$R_{cy}$
دارای خمش میانگین
$C=\pm1/2R_{cy}$
که علامت آن تابع تعریف جهت بردار عمود خواهد بود. یک شکل ساده‌ی جالب، کتانوید
\LTRfootnote{catanoid} 
 است که تمام نقاط روی سطح آن از نقاط زین اسبی تشکیل شده و در نتیجه خمش میانگین همه جا خمش میانگین آن صفر است.





 
 
 
 
 




\subsection{
خمش ذاتی
}
تا به اینجا فرض شده که تک لایه‌های تشکیل دهنده‌ی غشا تمایلی به خم شدن در جهت مشخصی ندارند. شکل تعادلی موضعی چنین غشایی یک سطح تخت خواهد بود. کمتر غشایی در طبیعت با چنین تقارنی مشاهده می‌شود ولی دلیل مطالعه‌ی چنین سیستم از نظر مدل‌سازی بسیار پر بهره است چرا که تمام ویژگی‌های الاستیک آن توسط پارامتر سختی خمش
$\kappa$
تعیین می‌شود که مقیاس خوبی برای انرژی یک غشاست. برای غشاهای فسفولیپیدی در دمای اتاق مرتبه‌ی انرژی سختی خمش حدود
$10^{19}J$،
یا
$20k_BT$
است. سختی خمش برای غشاهای از جنس دیگر ممکن است تا حدود یک مرتبه‌ی بزرگی متفاوت باشد. به دلیل اختلاف در ترکیبات تک لایه‌های سازنده‌ی غشاهای زیستی
\cite{Meer2008}
، غشاهای دولایه معمولا نامتقارن هستند. معروف‌ترین مثال غشاهای تشکیل شده از گنگلیوساید 
$GM1$
\LTRfootnote{ganglioside GM1}  
است که از نوع گلایکولیپید‌هاست
\LTRfootnote{glycolopids}  
و در غشاهای سلول‌های عصبی پستانداران به طور فراوان یافت می‌شود. 
$GM1$
نقش لنگر را برای مواد سمی، باکتری‌ها، و ویروس‌ها بازی می‌کند
\cite{Ewers2010}
. نحوه‌ی تغییر خمش موضعی با تغییر غلظت 
$GM1$
به طور مفصل به شکل تجربی
\cite{Bhatia2018, Raktim2018}
و هم به شکل شبیه‌سازی
\cite{Raktim2018, Sreekumari2018}
مطالعه شده‌است. همچنین جهت‌گیری پروتئین‌های درون غشا، چسبیدن پروتئین‌های محیط بر سطح غشا معمولا خمش را تغییر می‌دهد. در نتیجه اگر ذرات زیادی به سطح غشا بچسبند، خمش ذاتی در سطح غشا،
$C_0$
، ایجاد خواهد شد
\cite{Lipowsky2002}
که تابع تعداد ذراتی‌ است که به تک‌لایه‌های مختلف غشا متصل شده باشند
\cite{Breidenich2000}
. مقدار خمش ذاتی بسیار متغییر است و از مقیاس موضعی 
$1/10nm$
تا مقیاس خود غشا
$1/50\mu m$
گزارش شده است.


\subsection{
نظریه‌ی مدل خمش ذاتی
\label{sec:spontaneousCurvatureModel}
}
در این بخش به بنا کردن چهارچوب نظریه‌ای پرداخته می‌شود که نقش کلیدی در فهم شکل‌های غشا‌های غول‌آسا داشته است. این نظریه بر اساس انرژی الاستیک خمش غشا ایجاد شده‌است. همچنین در این نظریه حل‌شوندگی بسیار کم مولکول‌های لیپیدی و تاثیر اختلاف فشار اسمزی به شکل قید روی سطح و حجم غشا در نظر گرفته شده است. درواقع جذابیت این نظریه در توصیف تبادل بین انرژی‌هایی است با منشا موضعی و سراسری. از طرفی شکل انحنای غشا بر اساس انرژی‌های ناشی از خمش‌های موضعی (خمش میانگین و خمش گاووسی) تعیین می‌شود و از طرف دیگر با فرض اینکه غشا تغییر توپولوژیکی نداشته باشد (با غشای دیگری جوش نخورد، تکه‌ای از غشا به شکل جوانه از آن جدا نشود، یا حفره‌ای در آن ایجاد نشود) سطح و حجم ثابتی خواهد داشت که بر شکل نهایی که غشا می‌تواند به خود بگیرد تاثیر بسیار مشخصی دارد. ارتباط میان پدیده‌های موضعی و سراسری با دو کمیت به نام تنش مکانیکی\LTRfootnote{mechanical tension}  
$\gamma_{ten}$
و اختلاف فشار
$\Delta P$
برقرار می‌شود.

نظریه‌ی مدل خمش ذاتی\LTRfootnote{spontaneous curvature model}  
یک غشا را با دو مشخصه‌ی هندسی، سطح غشا 
$A$
و حجم آن 
$V$
، به همراه دو مشخصه‌ی مادی\LTRfootnote{material property}  
سختی خمش غشا
$\kappa$
و خمش ذاتی 
$C_0$
آن توصیف می‌کند. مدل خمش ذاتی بر اساس انرژی خمش بر حسب توان‌های خمش‌های اصلی پایه‌گزاری شده و تا زمانی که خمش نسبت به عکس ضخامت غشا کوچک باشند پابرجاست.
انرژی انحنای غشایی که شکل 
$S$
را دارد 
$\mathcal{E}_{cu}\{S\}$
است که بر حسب انتگرال سطحی چگالی موضعی انرژی 
$\varepsilon_{cu}(S)$
تعریف می‌شود،
\begin{equation}
\mathcal{E}_{cu}\{S\}=\int dA~\varepsilon_{cu}(S).
\end{equation}
فرض می‌کنیم چگالی انرژی موضعی تنها تابع خمش‌های اصلی 
$C_1$
و
$C_2$
است. همچنین اگر در هر نقطه بر روی سطح غشا دستگاه مختصات را 
$\pi/2$
در جهات بردار عمود برسطح دوارن دهیم، انرژی خمش نباید تغییر کند،
$\varepsilon_{cu}(C_1,C_2)=\varepsilon_{cu}(C_2,C_1)$.
بسط انرژی تا جمله‌ی توان دوم به شکل حدی با معادله‌ی زیر برابر خواهد بود،
\begin{equation}
\varepsilon_{cu}(C_1,C_2)\approx a_0+a_1(C_1+C_2)+a_2(C_1^2+C_2^2) + a_3 C_1C2.
\end{equation}
این بسط را می‌توان بر حسب خمش میانگین و خمش ذاتی به شکل زیر باز نویسی کرد،
\begin{equation}
\varepsilon_{cu}\approx 2\kappa(M-C_0)^2+\kappa_GG.
\end{equation}
با جایگذاری در انتگرال سطحی شکل انرژی انحنای هلفریش
\LTRfootnote{Helfrich}
\cite{Helfrich1973}
بدست می‌آید،
\begin{equation}
E_{cu}=\int dA\left[\frac{1}{2}\kappa(C_1+C_2-2C_0)^2+\kappa_GC_1C_2\right],
\label{eq:HelfrichCurvatureEnergy}
\end{equation}
که به دو انتگرال انرژی خمش 
\begin{equation}
E_{b}=\frac{1}{2}\kappa\int dA (C_1+C_2-2C_0)^2
\label{eq:HelfrichBendingEnergy}
\end{equation}
و انرژی خمش گاووسی
\begin{equation}
E_{G}=\kappa_G\int dA C_1C_2
\label{eq:HelfrichGaussianEnergy}
\end{equation}
تقسیم می‌شود.
در صورتی که غشا هندسه‌ی بسته داشته باشد و هیچ لبه‌ی آزادی در آن نباشد (مثلا یک شکل کُروی داشته باشد و به شکل یک صفحه‌ی آزاد در محیط نباشد) قضیه‌ی گاووس-بونت\LTRfootnote{Gauss-Bonnet theorem}
\cite{NelsonBook2004}
در هندسه‌ی دیفرانیسیلی جواب انتگرال معادله‌ی
\ref{eq:HelfrichGaussianEnergy}
 تابع مشخصه‌ی اویلری رویه\LTRfootnote{Euler characteristic of the surface}
 یا جینوس\LTRfootnote{genus}
خواهد بود.
 \begin{equation}
E_{G}=\kappa_G\int dA C_1C_2=2\pi\kappa_G(2-2g).
\label{eq:GaussianBonnet}
\end{equation}
 
 جینوس سطح تعداد سوراخ‌هایا تعداد دسته‌های یک شکل را می‌شمارد.
مثلا یک کُره هیچ سوراخ یا دسته‌ای ندارد و جینوس آن صفر است. در صورتی که یک شکل چنبره\LTRfootnote{toroid}
یک سوراخ یا دسته دارد و جینوس آن یک است
$g=1$
. همچنین رویه‌ای که سطح یک لیوان دسته‌دار را می‌پوشاند جینوس برابر با یک خواهد داشت (به مثال‌های شکل 
\ref{fig:genus012}
توجه کنید).
\begin{figure}[t]
\begin{center}
\includegraphics[width=\columnwidth]{\MemTB/Pics/genus}
\caption{
به ترتیب از چپ به راست یک کُره، چنبره، و دو چنبره متصل به هم را مشاهده می‌کند که به ترتیب شکلی با 
$0$, $1$,
 و 
$2$
سوراخ/دسته که همچنین مقدار جینوس این سطوح را تعیین می‌کند.
}
\label{fig:genus012}
\end{center}
\end{figure}
برای مثال انرژی انحنای یک کُره به شعاع
$R$
را با استفاده از معادلات
\ref{eq:HelfrichBendingEnergy}
و
\ref{eq:HelfrichGaussianEnergy}
محاسبه می‌کنیم. با فرض اینکه بردار عمود بر سطح کره در جهت خارج کُره تعریف شده باشد، از انجایی که شعاع‌ کره در تمام نقاط سطح آن ثابت است خمش در سراسر سطح با یک مقدار
$C_1=C_2=1/R$
تعیین می‌شود. تنها فرض باقی مانده تعیین خمش ذاتی شکل است. در صورتی که فرض شود حالت تعادلی رویه‌ای که کُره را تشکیل داده سطح تخت باشد
$C_0=0$
انرژی آن
 \begin{equation}
 \begin{aligned}
E_{b}&=\frac{1}{2}\kappa\int R^2d\Omega (\frac{2}{R}-0)^2=8\pi\kappa\\
E_{G}&=\kappa_G\int R^2d\Omega \frac{1}{R^2}=4\pi\kappa_G=2\pi\kappa_G(2-2g)=2\pi\kappa_G(2-0) \\
E_{cu}&=8\pi\kappa+4\pi\kappa_G.
\end{aligned}
\end{equation}
در صورتی که خمش ذاتی آن،
$C_0=1/R$
باشد انرژی انحنای آن
 \begin{equation}
 \begin{aligned}
E_{b}&=\frac{1}{2}\kappa\int R^2d\Omega (\frac{2}{R}-\frac{2}{R})^2=0\\
E_{G}&=2\pi\kappa_G(2-2g)=4\pi\kappa_G \\
E_{cu}&=4\pi\kappa_G
\end{aligned}
\end{equation}
است. توجه کنید که انرژی انحنای محاسبه شده تابع شعاع کُره نیست. به طور عمومی انرژی محاسبه شده با مدل خمش ذاتی از مقیاس شکل مستقل است. در نتیجه مدول خمشی
$\kappa$
نیز تابع مقیاس سیستم نخواهد بود. این مفهوم بسیار متفاوت از تعاریف رایج از مدول خمشی برای پلیمر‌هاست و باید توجه ویژه‌ای به این نکته بشود. از آنجایی که در مطالعه‌ی شکل غشا معمولا اختلاف انرژی بین اشکال اهمیت دارد و (به خصوص برای محاسبه‌ی نیرو) تا زمانی که غشا تغییر توپولوژیکی نداشته باشد، از محاسبه‌ی مقدار ثابت انرژی خمش گاووسی چشم پوشی می‌شود. 







































\section{
انرژی کشش الاستیک
}
\setRL
%\pagenumbering{arabic} 

بعضی غشاها تنها از یک غشای دو‌-لایه‌ی لیپیدی تشکیل نشده‌اند. مانند غشای هسته‌ی سلول‌های پستانداران که معمولا از دو غشای لیپیدی دو لایه متصل به یک شبکه‌ی پلیمری دو بعدی الاستیک تشکیل شده‌است. در این بخش به مدل‌سازی انرژی شبکه‌های پلیمری الاستیک می‌پردازیم.
%\subsection{
%انرژی کشش در سطح
%}
اگر فرض کنیم جابجایی روی یک عنصر سطحی حاصل از کشیده‌ یا فشرده شدن سطح با بردار 
$u$
توصیف شود، با فرض خطی بودن عکس العمل ماده، انرژی پتانسیل حاصل از تغییر شکل سطح را می‌توان با معادله‌ی زیر بررسی کنیم.

\begin{equation}
E_{stretching}=\frac{1}{2}Y_{2D}A\varepsilon^2
\end{equation}
که اینجا 
$Y_2D$
مدول دو بعدی یانگ،
$A$
سطح عنصر در حالت کشیده نشده، و
$\varepsilon$
تانسور کرنش است. تانسور کرنش برای سطح دو بعدی به شکل زیر تعریف می‌شود:
\begin{equation}
\varepsilon_{ij} = \frac{1}{2}(u_{ij}+u_{ji})
\end{equation}


در نظریه‌ی الاستیک سطح هر تغییر شکل با یک میدان بردار جابجایی 
$u(r)=(u_1,u_2)$
نشان داده می‌شود نقطه‌ی 
$r(x,y)$
را به نقطه‌ی 
$r+u$
نگاشت می‌کند. اگر در شبکه نقص وجود نداشته باشد این نگاشت یک به یک خواهد بود. در صورتی که فرض کنیم که ماده مورد مطالعه یکنواخت و همسانگرد است، برای جابجایی‌های کوچک (رژیم خطی) قانون هوک را به شکل توان دوم تانسور کرنش نوشت
\LTRfootnote{Cauchy, 1822; Lam ́e, 1852}
،
\begin{equation}
E_s=\frac{1}{2}\int d^2r(2\mu u_{ij}^2+\lambda u_{kk}^2)
\label{eq:energylame}
\end{equation}
که در اینجا $\lambda$
و $\mu$
ثابت‌های لم
\LTRfootnote{Lamé Coefficients}
است. ما می‌دانیم که تانسور کرنش به شکل زیر تعریف می‌شود،
\begin{equation}
u_{ij}=\frac{1}{2}(\partial_i u_j+\partial_j u_i+\partial_i u_k\partial_j u_k)
\end{equation}
اما برای جابجایی کوچک از جمله‌ی غیر خطی صرف نظر می‌کنیم و تانسور کرنش را به این شکل تعریف می‌کنیم.
\begin{equation}
u_{ij}=\frac{1}{2}(\partial_i u_j+\partial_j u_i)
\label{eq:simplestrain}
\end{equation}
می‌توانیم  از انرژی کششی گرادیان بگیریم و مقدار کمینه‌ی آن را بررسی کنیم، در نتیجه
\begin{equation}
\begin{aligned}
&\partial_i\sigma_{ij}=0\\
&\sigma_{ij}=2\mu u_{ij}+\lambda u_{kk}\delta_{ij}
\label{eq:stress}
\end{aligned}
\end{equation}
که در این معادله 
$\sigma_{ij}$
تانسور تنش است. معادله‌ی 
\ref{eq:stress}
را به تنهایی می‌توان حل کرد ولی از آنجایی که دیورژانس تنش صفر است معمول است که این معادله را به شکل یک پتانسیل اسکالر بنویسیم،
\begin{equation}
\sigma_{xx}=\frac{\partial^2\chi}{\partial y^2},\quad\sigma_{yy}=\frac{\partial^2\chi}{\partial x^2},\quad\sigma_{xy}=\frac{\partial^2\chi}{\partial_x\partial_y} 
\end{equation}
انتخاب‌های خیلی زیادی می‌توانند معادله‌ی بالا را ارضاء خواهد کرد، ولی جواب‌هایی که به لحاظ فیزیک قابل قبول هستند باید بتوانند رابطه‌ی بین میدان جابجایی و 
$\chi$
را رعایت کنند،
\begin{equation}
\begin{aligned}
\frac{1}{2}(\partial_iu_j+\partial_ju_i)&=u_{ij}\\
&=\frac{1+\nu}{Y}\sigma_{ij}-\frac{\nu}{Y}\sigma_{ll}\sigma_{ij}\\
&=\frac{1+\nu}{Y}\epsilon_{im}\epsilon_{jn}\partial_{m}\partial_{n}\chi-\frac{\nu}{Y}\nabla^2\chi\delta_{ij}
\label{eq:constraint}
\end{aligned}
\end{equation}
در اینجا $Y$
و $\nu$
به ترتیب مدول ۲ بعدی یانگ
\LTRfootnote{2D Young Modulus}
 و نسبت پواسون
\LTRfootnote{Poisson ratio}
است که بر حسب ضرایب لم به شکل زیر بیان می‌شوند،
\begin{equation}
\begin{aligned}
Y&=\frac{4\mu(\mu+\lambda)}{2\mu+\lambda}\\
\nu&=\frac{\lambda}{2\mu+\lambda}
\label{eq:younglame}
\end{aligned}
\end{equation}

 
 
 
 
 
 











