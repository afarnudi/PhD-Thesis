\thispagestyle{empty}
\noindent
\centerline{\textbf{\large{چکیده}}} \\
در فصل اول این پژوهش مروری بر مطالعات اخیر افراد در مورد ساختار اسکلت سلولی و عوامل دخیل در انتقال نیرو در سلول‌های دارای هسته شده است. فیزیک رفتار مواد ویسکوالاستیک با جزئیات نیز بررسی شده‌. در ادامه‌ی این فصل برخی مطالعات منتخب معرفی شده‌ که در آن از مدل‌های ویسکوالاستیکی برای  مدل‌سازی نتایج آزمایشگاهی رفتار بخش‌های مختلف سلول استفاده شده‌است. در نهایت ساختار کروماتین در هسته‌ی سلول و ماتریس اتصالات کروماتین‌ها درون هسته نیز بررسی شده‌است. همچنین برخی از مدل‌هایی که در ۲-۳ سال گذشته برای توجیه این ساختار و ماتریس اتصالات انتشار یافته نیز مطالعه شده‌است. \\
در فصل دوم طرح پژوهشی توضیح داده شده و در فصل آخر منابع مورد استفاده نامبرده شده‌است.
\\
\\
\textbf{کلمات کلیدی:}
کروماتین، ماتریس اتصالات، ویسکوالاستیک، هسته سلول، شبیه‌سازی دینامیک ملکولی
\textit{}