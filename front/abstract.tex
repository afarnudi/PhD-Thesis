
% -------------------------------------------------------
%  Abstract
% -------------------------------------------------------


\شروع{وسط‌چین}
\مهم{چکیده}
\پایان{وسط‌چین}
%\بدون‌تورفتگی
طبیعت از دولایه‌های لیپیدی (غشاها) برای ساخت دیوار سلولی و همچنین برای بسته بندی ملکول‌ها و اعضای زیر مجموعه‌ی سلول استفاده می‌کند. ضخامت غشا‌ها معمولا در مقایسه با مقیاسی که در آن مطالعه می‌شوند بسیار کوچک است. در نتیجه، می‌توان غشا را با صفحه‌ای دو بعدی مدل سازی کرد که انرژی کششی، بُرشی، و انحنا دارد. از آنجایی که نحوه‌ی گسسته‌سازی مدل‌های الاستیک و خمشی بر شبکه‌های مثلثی در گذشته به دقت مطالعه شده‌است، استفاده از این این مِش‌ها برای شبیه‌سازی غشا‌ها بسیار مورد توجه قرار گرفته‌است.

%\بدون‌تورفتگی
غشای لیپیدی سیال است و مدول بُرشی ندارند و مدلسازی آن‌ بر شبکه‌های تشکیل شده از گوی و فنر دشوار است. الگوریتم مثلث بندی دینامیکی روشی است که با تغییر شکل دائم شبکه‌، مِش بدون مدول بُرشی ایجاد می‌کند. در سال ۱۹۹۶، گامپر و کرول با استفاده از این الگوریتم و با محاسبه‌ی انرژی خمش بر شبکه‌هایِ مثلثی توانستند غشا‌های سیال‌گون را شبیه‌سازی کنند. از آنجایی که در روش مثلث‌ بندی دینامیک، مِش دائم در حال تغییر است این شبیه‌سازی به روش مانتی کارلو پیاده سازی می‌شود. روش مانتی‌کارلو برای اندازه‌گیری آماری بسیار مناسب بوده ولی نمی‌توان با آن دینامیک غشا را مطالعه کرد. 

مدل‌های پیوسته مانند هامیلتونی هلفریش به طور فراگیر برای مطالعه‌ی غشاهای سیال استفاده می شود. در این رساله روشی  سازگار با شبیه‌سازی دینامیک ملکلوی ابداع شده که در آن از مش‌های دو بعدی برای تصیف دینمایک غشا استفاده می‌شود. در این روش حرکت درون صفحه‌ای نقاط مش فقط تحت قید بسیار ضعیفی با حرکت برون صفحه‌ایشان جفت شده‌اند. ما نشان می‌دهیم که روش گسسته‌سازی انحنای یولیشر به لحاظ پایداری محاسباتی بسیار مستحکم‌تر از روش فراگیر شده‌ی گامپر و کرول است. همچنین نشان می‌دهیم که  مش‌های نرمی که  با شبیه‌سازی دینامیک مش تولید شده‌اند رفتار مشخصه‌ی غشا‌های سیال‌گون 
$\ell^{-4}$
برای طول موج‌های بلند را به خوبی نشان می‌دهد. در نهایت دینامیک دامنه‌ی موج‌های سطحی غشا را زمانی که نقاط مش در رژیم‌های نیوتنی، لانژون، و براونی حرکت می‌کنند، مطالعه کردیم. نشان می‌دهیم که اتلاف ناشی از حرکت درون صفحه‌ای نقاط مش برای مطالعه‌ی غشا در اعداد رنولدز پایین بسیار مناسب است و می‌توان از این روش در ترکیب با موتور‌های حل هیدرودینامیک استفاده کرد.


%شبیه‌سازی‌های دینامیک ملکولی بهترین روش برای شبیه‌سازی اتمی یا درشت دانه‌ی غشا‌هاست. همچنین در این روش شبیه‌سازی می‌توان مدل‌های مختلف درشت‌دانه‌ی اجزای سلول را با یکدیگر ترکیب کرد تا رفتار سلول‌های زنده را مدل‌سازی کرد. در سال ۱۹۹۶ روش دومی نیز برای محاسبه‌ی خمش بر شبکه‌های مثلثی توسط یولیشر ابداع شد. ما نشان می‌دهیم که روش یولیشر برای محاسبه‌ی خمش روی مش‌های با توزیع رئوس تصادفی به لحاظ دینامیکی بسیار پایدار است. همچنین تغییراتی برای مدل گامپر و کرول پیشنهاد کردیم که آن را نیز برای شبیه‌سازی دینامیک ملکولی پایدار می‌سازد. حاصل این کار ایجاد یک روش جدید برای شبیه‌سازی رفتار سیال‌گون غشاست  که با شبیه‌سازی دینامیک ملکولی کاملا سازگار است. نام این روش توزیع دینامیک مساحت انتخاب شده‌است. می‌توان با استفاده از این روش از تمام مزیت‌های شبیه‌سازی دینامیک ملکولی مانند قابلیت مطالعه‌ی دینامیک و افت و خیز سامانه‌ها تحت شرایط هیدرودینامیکی مختلف و استفاده از بسته‌های شبیه‌سازی دینامیک ملکولی آماده بهره برد. 



\پرش‌بلند
\بدون‌تورفتگی \مهم{کلیدواژه‌ها}: 
غشا، مش مثلثی، دینامیک ملکولی، مدل‌های پیوسته
\صفحه‌جدید
