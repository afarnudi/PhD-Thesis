
% -------------------------------------------------------
%  Abstract
% -------------------------------------------------------


\شروع{وسط‌چین}
\مهم{چکیده}
\پایان{وسط‌چین}
%\بدون‌تورفتگی
طبیعت از دولایه‌ لیپیدی (غشا) برای ساخت دیوار سلولی و همچنین بسته بندی مولکول‌ها و اعضای زیر مجموعه‌ی سلول استفاده می‌کند. ضخامت غشا‌ نسبت به اندازه‌ی آن، بسیار کوچک است. در نتیجه، هنگام مطالعه‌ی نظری، می‌توان غشا را با رویه‌ای دو بعدی دارای انرژی کششی، بُرشی، و انحنا مدل‌سازی کرد. برای شبیه‌سازی غشا به روش‌های مانتی‌ کارلو یا دینامیک مولکولی نیاز به گسسته‌سازی رویه است. نحوه‌ی گسسته‌سازی مدل‌های الاستیک و خمشی بر مِش (شبکه‌) مثلثی در گذشته به دقت مطالعه شده‌است و اکنون به عنوان شناخته‌ترین روش گسسته‌سازی رویه‌ استفاده می‌شود.

%\بدون‌تورفتگی
غشای لیپیدی، یک سیال دو بعدی است و در نتیجه مدول بُرشی ندارند. ‌الگوریتم مثلث بندی دینامیکی روشی است که با تغییر شکل دائم اتصالت میان نقاط مِش، تنش‌های ناشی زا تغییر شکل مِش را  از میان می‌برد. در سال ۱۹۹۶، گامپر و کرول با استفاده از این الگوریتم و با محاسبه‌ی انرژی خمش بر شبکه‌هایِ مثلثی توانستند غشا‌های سیال‌گون را شبیه‌سازی کنند
\cite{gompper1996}.
 از آنجایی که در روش مثلث‌ بندی دینامیک، مِش دائم در حال تغییر است این شبیه‌سازی به روش مانتی کارلو پیاده سازی می‌شود. روش مانتی‌کارلو برای اندازه‌گیری آماری بسیار مناسب بوده ولی نمی‌تواند دینامیک غشا را نشان دهد.

هامیلتونی هلفریش به طور فراگیر برای مطالعه‌ی غشاهای سیال استفاده شده‌است. در این رساله روشی  سازگار با شبیه‌سازی دینامیک مولکولی ابداع شده‌است که در آن از مِش‌های دو بعدی برای توصیف دینامیک غشا استفاده شده‌است. در این روش نقاط مِش تقریبا آزادانه بر روی رویه‌ی غشا حرکت کرده (حرکت درون-صفحه‌ای) و تقریبا مستقل از حرکت برون-صفحه‌ای عمل می‌کنند. در این رساله نشان دادیم که روش گسسته‌سازی انحنای یولیشر 
\cite{Julicher1996}
به لحاظ پایداری محاسباتی بسیار مستحکم‌تر از روش فراگیر شده‌ی گامپر و کرول است
\cite{gompper1996}.
همچنین نشان دادیم که مِش‌های «نرم» با روش شبیه‌سازی دینامیکِ مِش، رفتار مشخصه‌ی غشا‌های سیال‌گون 
$\ell^{-4}$
برای طول موج‌های بلند را به خوبی نشان می‌دهد. در نهایت دینامیک دامنه‌ی موج‌های سطحی غشا، در رژیم‌های نیوتنی، لانژون، و براونی، مطالعه شد. نشان داده‌ شد که اتلاف ناشی از حرکت درون-صفحه‌ای نقاطِ مِش برای مطالعه‌ی غشا در رژیم اعداد رنولدز پایین بسیار مناسب است و پیش‌بینی می‌شود که از این روش در ترکیب با موتور‌های هیدرودینامیکی می‌توان استفاده کرد.


%شبیه‌سازی‌های دینامیک مولکولی بهترین روش برای شبیه‌سازی اتمی یا درشت دانه‌ی غشا‌هاست. همچنین در این روش شبیه‌سازی می‌توان مدل‌های مختلف درشت‌دانه‌ی اجزای سلول را با یکدیگر ترکیب کرد تا رفتار سلول‌های زنده را مدل‌سازی کرد. در سال ۱۹۹۶ روش دومی نیز برای محاسبه‌ی خمش بر شبکه‌های مثلثی توسط یولیشر ابداع شد. ما نشان می‌دهیم که روش یولیشر برای محاسبه‌ی خمش روی مش‌های با توزیع رئوس تصادفی به لحاظ دینامیکی بسیار پایدار است. همچنین تغییراتی برای مدل گامپر و کرول پیشنهاد کردیم که آن را نیز برای شبیه‌سازی دینامیک مولکولی پایدار می‌سازد. حاصل این کار ایجاد یک روش جدید برای شبیه‌سازی رفتار سیال‌گون غشاست  که با شبیه‌سازی دینامیک مولکولی کاملا سازگار است. نام این روش توزیع دینامیک مساحت انتخاب شده‌است. می‌توان با استفاده از این روش از تمام مزیت‌های شبیه‌سازی دینامیک مولکولی مانند قابلیت مطالعه‌ی دینامیک و افت و خیز سامانه‌ها تحت شرایط هیدرودینامیکی مختلف و استفاده از بسته‌های شبیه‌سازی دینامیک مولکولی آماده بهره برد. 



\پرش‌بلند
\بدون‌تورفتگی \مهم{کلیدواژه‌ها}: 
غشا، مش مثلثی، دینامیک مولکولی، مدل‌های پیوسته
\صفحه‌جدید
