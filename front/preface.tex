
% -------------------------------------------------------
%  Abstract
% -------------------------------------------------------


\شروع{وسط‌چین}
\مهم{پیش گفتار}
\پایان{وسط‌چین}

در این پایان نامه روش جدیدی برای پیاده‌سازی مدل‌های پیوسته‌ی غشا بر شبکه‌های مثلثی ابداع شده‌است که قابل اجرا در شبیه‌سازی‌های دینامیک مولکولی است. جهت معرفی این روش ساختار زیر در نظر گرفته شده‌است:

در فصل ۱ تاریخچه‌ی کوتاهی راجع به غشا‌ها و ساز و کار ساختار مولکولی آن  ارائه شده‌است. سپس مدل‌های پیوسته‌ی رایج برای مطالعه‌ی غشا در فصل ۲ معرفی شده‌اند. این مدل‌ها شامل توصیف انرژی مکانیکی لازم  برای تغییر مساحت و حجم غشا، انرژی انحنای سطح، و در صورتی که خاصیت الاستیک داشته باشد، انرژی کششی آن است.

در فصل ۳ ابزار اصلی مطالعه‌ی رفتار فیزیک غشا، یعنی اندازه‌گیری افت و خیز سطح آن با جزئیات محاسبه  شده‌است.  شدت افت و خیز انرژی‌های سطح، حجم، انحنا، و کشش برای غشا‌های تقریبا کُروی  محاسبه شده‌است. با مطالعه‌ی این بخش می‌توان تفاوت رفتار افت و خیز سطح غشا‌های سیال‌گون و غشا‌های ترکیبی که خاصیت الاستیکی دارند را مقایسه‌ کرد. شدت‌ افت خیز سطح، قابل اندازه‌گیری در آزمایشگاه نیز هست. همچنین، روش محاسبه‌ی مقیاس زمانی افت و خیز‌ها نیز در این فصل ارائه شده که با استفاده از آن می‌توان مدل‌ غشا را کاملا کالیبره کرد.

روش‌های معرفی شده در فصل ۳ مناسب مطالعه‌ی غشا‌های کُروی‌است. برای بررسی رفتار غشا‌هایی غیر کُروی ناچاریم انرژی‌ غشا را مستقیم با روش‌های گسسته‌سازی اندازه‌گیری کنیم. در فصل ۴ روش‌های فراگیر برای مطالعه‌ی عددی غشا‌ها معرفی شده‌است. در فصل ۵ با انتخاب مِش‌های مثلثی به عنوان روش اصلی گسسته‌سازی، نحوه‌ی تغییر معادلات پیوسته برای انرژی‌های سطح، حجم، انحنا، و کشش کاملا بررسی شده‌است.

در فصل ۶ روش جدیدی برای پیاده‌سازی معادلات گسسته در شبیه‌سازی دینامیک مولکولی معرفی شده‌است. این روش با مش‌های جدیدی سر و کار دارد که  نحوه‌ی تهیه‌ی آن‌ها در ابتدای فصل کامل توضیح داده شده‌است. سپس پیاده‌سازی معادلات پیوسته برای شبیه‌سازی‌های دینامیک مولکولی محاسبه‌ شده‌است. در این رساله از موتور محاسباتی 
OpenMM
برای اجرای شبیه‌سازی‌های دینامیک مولکولی استفاده شده‌است که قادر به محاسبه‌ی نیرو با استفاده از عبارات انرژی پتانسیل است. در نتیجه روش محاسبه‌ی نیرو در این رساله ذکر نشده‌است.

در فصل نتایج (فصل ۷) صحت اندازه‌گیری‌های مستقیم انرژی‌های سطح، حجم، و انحنا بر مش‌های جدید بررسی شده‌است. پس از تایید محاسبات استاتیک، چشم انداز انرژی برای  مش‌های «نرم» مطالعه شده‌است. در اینجا ناپایداری‌های دینامیکی روش‌های اندازه‌گیری خمش کاملا بررسی شده و تغییرات لازم برای پایدار سازی معادلات ارائه شده‌است. با اندازه‌گیری افت و خیز سطح و شکل‌ ظاهری غشا‌های حاصل از شبیه‌سازی، صحت  روش دینامیک مش بررسی شد. همچنین به عنوان کاربرد اولیه‌ نحوه‌ی شبیه‌سازی گلبول قرمز بحث شده‌است. در نهایت این فصل با مطالعه‌ی دینامیک سطح غشا به سرانجام رسیده‌است.

در آخر، در فصل ۸، در مورد یافته‌های حاصل از این پژوهش بحث کرده و خلاصه‌ای از نتایج را ارائه داده‌ایم. این رساله  با پیشنهاد کار برای مطالعات آینده به سرانجام می‌رسد.



\صفحه‌جدید
